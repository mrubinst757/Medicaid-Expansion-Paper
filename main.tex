% Template for the submission to:
%   The Annals of Applied Statistics    [AOAS]
%
%%%%%%%%%%%%%%%%%%%%%%%%%%%%%%%%%%%%%%%%%%%%%%
%% In this template, the places where you   %%
%% need to fill in your information are     %%
%% indicated by '???'.                      %%
%%                                          %%
%% Please do not use \input{...} to include %%
%% other tex files. Submit your LaTeX       %%
%% manuscript as one .tex document.         %%
%%%%%%%%%%%%%%%%%%%%%%%%%%%%%%%%%%%%%%%%%%%%%%

\documentclass[article]{imsart}

%% Packages
\RequirePackage{amsthm,amsmath,amsfonts,amssymb,centernot,float,import,makeidx,subfiles}
\RequirePackage{natbib}
%\RequirePackage[colorlinks,citecolor=blue,urlcolor=blue]{hyperref}
\RequirePackage{graphicx}% uncomment this for including figures

\startlocaldefs
%%%%%%%%%%%%%%%%%%%%%%%%%%%%%%%%%%%%%%%%%%%%%%
%%                                          %%
%% Uncomment next line to change            %%
%% the type of equation numbering           %%
%%                                          %%
%%%%%%%%%%%%%%%%%%%%%%%%%%%%%%%%%%%%%%%%%%%%%%
%\numberwithin{equation}{section}
%%%%%%%%%%%%%%%%%%%%%%%%%%%%%%%%%%%%%%%%%%%%%%
%%                                          %%
%% For Axiom, Claim, Corollary, Hypothezis, %%
%% Lemma, Theorem, Proposition              %%
%% use \theoremstyle{plain}                 %%
%%                                          %%
%%%%%%%%%%%%%%%%%%%%%%%%%%%%%%%%%%%%%%%%%%%%%%
%%%%%%%%%%%%%%%%%%%%%%%%%%%%%%%%%%%%%%%%%%%%%%
\theoremstyle{plain}
\newtheorem{axiom}{Axiom}
\newtheorem{claim}[axiom]{Claim}
\newtheorem{theorem}{Theorem}[section]
\newtheorem{lemma}[theorem]{Lemma}
\newtheorem{proposition}{Proposition}
\newcommand{\matr}[1]{\mathbf{#1}} % undergraduate algebra version
\newcommand{\mathbbm}[1]{\text{\usefont{U}{bbm}{m}{n}#1}} 
%%%%%%%%%%%%%%%%%%%%%%%%%%%%%%%%%%%%%%%%%%%%%%
%%                                          %%
%% For Assumption, Definition, Example,     %%
%% Notation, Property, Remark, Fact         %%
%% use \theoremstyle{remark}                %%
%%                                          %%
%%%%%%%%%%%%%%%%%%%%%%%%%%%%%%%%%%%%%%%%%%%%%%
\theoremstyle{remark}
\newtheorem{remark}{remark}
%%%%%%%%%%%%%%%%%%%%%%%%%%%%%%%%%%%%%%%%%%%%
%\theoremstyle{plain}
%\newtheorem{???}{???}
%\newtheorem*{???}{???}
%\newtheorem{???}{???}[???]
%\newtheorem{???}[???]{???}
%%%%%%%%%%%%%%%%%%%%%%%%%%%%%%%%%%%%%%%%%%%%%%
%%                                          %%
%% For Assumption, Definition, Example,     %%
%% Notation, Property, Remark, Fact         %%
%% use \theoremstyle{remark}                %%
%%                                          %%
%%%%%%%%%%%%%%%%%%%%%%%%%%%%%%%%%%%%%%%%%%%%%%
%\theoremstyle{remark}
%\newtheorem{???}{???}
%\newtheorem*{???}{???}
%\newtheorem{???}{???}[???]
%\newtheorem{???}[???]{???}
%%%%%%%%%%%%%%%%%%%%%%%%%%%%%%%%%%%%%%%%%%%%%%
%% Please put your definitions here:        %%
%%%%%%%%%%%%%%%%%%%%%%%%%%%%%%%%%%%%%%%%%%%%%%

\endlocaldefs

\begin{document}

\begin{frontmatter}
%%%%%%%%%%%%%%%%%%%%%%%%%%%%%%%%%%%%%%%%%%%%%%
%%                                          %%
%% Enter the title of your article here     %%
%%                                          %%
%%%%%%%%%%%%%%%%%%%%%%%%%%%%%%%%%%%%%%%%%%%%%%
\title{The Effect of Medicaid Expansion on Non-Elderly Adult Uninsurance Rates Among States that did not Expand Medicaid}
%\title{A sample article title with some additional note\thanksref{T1}}
\runtitle{}
%\thankstext{T1}{A sample of additional note to the title.}

\begin{aug}
%%%%%%%%%%%%%%%%%%%%%%%%%%%%%%%%%%%%%%%%%%%%%%
%%Only one address is permitted per author. %%
%%Only division, organization and e-mail is %%
%%included in the address.                  %%
%%Additional information can be included in %%
%%the Acknowledgments section if necessary. %%
%%%%%%%%%%%%%%%%%%%%%%%%%%%%%%%%%%%%%%%%%%%%%%
\author[A]{\fnms{Max} \snm{Rubinstein}\ead[label=e1]{mrubinst@andrew.cmu.edu; amelia@andrew.cmu.edu}} and
\author[A]{\fnms{Amelia} \snm{Haviland}}
%%%%%%%%%%%%%%%%%%%%%%%%%%%%%%%%%%%%%%%%%%%%%%
%% Addresses                                %%
%%%%%%%%%%%%%%%%%%%%%%%%%%%%%%%%%%%%%%%%%%%%%%
\address[A]{Carnegie Mellon University, Heinz College and Department of Statistics and Data Science \printead{e1}}

\end{aug}

\begin{flushleft}
We estimate the effect of Medicaid expansion on the adult uninsurance rate in states that did not expand Medicaid in 2014. Using data from the American Community Survey (ACS), we estimate this effect - the treatment effect on the controls (ETC) - by re-weighting expansion regions to approximately balance the covariates from non-expansion regions using an extension of the stable balancing weights objective function (\cite{zubizarreta2015stable}). We contribute to the balancing weights literature by accounting for hierarchical data structure and covariate measurement error when calculating our weights, and to the synthetic controls literature (see, e.g. \cite{abadie2010synthetic}) by outlining a set of assumptions that identifies the ETC using time-series cross-sectional data. We estimate that Medicaid expansion would have changed the uninsurance rate by -2.33 (-3.49, -1.16) percentage points. These results are smaller in absolute magnitude than existing estimates of the treatment effect on the treated (ETT), though may not be directly comparable due to the study design, target population, and level of analysis. Regardless, we caution against making inferences about the ETC using estimates of the ETT, and emphasize the need to directly estimate the appropriate counterfactual when they are the quantity of interest.
\end{flushleft}


\begin{keyword}
\kwd{Synthetic controls}
\kwd{balancing weights}
\kwd{medicaid expansion}
\kwd{measurement error}
\kwd{hierarchical data}
\kwd{regression to the mean}
\end{keyword}

\end{frontmatter}
%%%%%%%%%%%%%%%%%%%%%%%%%%%%%%%%%%%%%%%%%%%%%%
%% Please use \tableofcontents for articles %%
%% with 50 pages and more                   %%
%%%%%%%%%%%%%%%%%%%%%%%%%%%%%%%%%%%%%%%%%%%%%%
%\tableofcontents

%%%%%%%%%%%%%%%%%%%%%%%%%%%%%%%%%%%%%%%%%%%%%%
%%%% Main text entry area:

\section{Introduction}
The 2010 Affordable Care Act (ACA) required states to expand their Medicaid eligibility requirements by 2014 to offer coverage to all adults with incomes at or below 138 percent of the federal poverty level (FPL). The United States Supreme Court ruled this requirement unconstitutional in 2012, allowing states to decide whether to expand Medicaid coverage. In 2014, twenty-six states and the District of Columbia expanded their Medicaid programs. From 2015 through 2020 an additional twelve states elected to expand their Medicaid programs. More recently, Oklahoma and Missouri voted to expand their programs in July 2021 (though the Missouri Republican party is currently attempting to block funding for the program\footnote{https://www.kansascity.com/news/politics-government/article250170945.html}). Following the passage of the American Rescue Plan in March 2021, Republican state legislatures in other traditionally conservative states, including Alabama, North Carolina, and Wyoming are also reportedly considering expanding their programs.\footnote{https://www.nbcnews.com/politics/politics-news/changed-hearts-minds-biden-s-funding-offer-shifts-medicaid-expansion-n1262229} The effects of Medicaid expansion on various outcomes, including uninsurance rates, mortality rates, and emergency department use, have been widely studied, primarily by using the initial expansions in 2014 and 2015 to divide expansion states into ``treated'' states and non-expansion states as ``control'' states. 

Medicaid enrollment is not automatic, and Medicaid take-up rates have historically have varied across states. This variation is partly a function of state discretion in administering programs: for example, program outreach, citizenship verification policies, and application processes differ across states (\cite{courtemanche2017early}). While states expanded their eligibility requirements, the number that actually enrolled in Medicaid afterwards is some random number of the total number of eligible individuals. Understanding how Medicaid eligibility expansion actually affected the number of uninsured individuals is therefore an important effect. Existing studies have estimated that Medicaid expansion reduced the number of uninsured adults between three and six percentage points among states that expanded Medicaid. These estimates differed depending on the data used, specific target population, study design, and level of analysis (see, e.g., \cite{kaestner2017effects}, \cite{courtemanche2017early}, \cite{frean2017premium}). However, none of these studies have directly estimated the treatment effect on the controls (ETC). 

We therefore propose to study the effect of 2014 Medicaid expansion on adult uninsurance rates among states that did not expand Medicaid. Moreover, we use approximate balancing weights to estimate this effect (\cite{wang2017minimal}). Approximate balancing weights are a popular estimation method in causal inference that grew out of the propensity score weighting literature. Rather than iteratively modeling the propensity score until the inverse probability weights achieve a desired level of balance (the so-called ``propensity score tautology'' \cite{imai2014covariate}), recent papers propose using optimization methods to generate weights that enforce covariate balance between the treated and control units (see, e.g., \cite{imai2014covariate}, \cite{zubizarreta2015stable}). From an applied perspective, there are at least four benefits of this approach: first, it does not require iterating propensity score models to generate satisfactory weights. Second, these methods do not use outcomes in the modeling stage. This mitigates the risk of cherry-picking model specifications. Third, these methods often constrain the weights to be positive, preventing extrapolation from the data and reducing model dependence \cite{zubizarreta2015stable}. Finally, the estimates are more interpretable: by making the comparison group explicit, it is easy to communicate exactly which units contributed to the counterfactual estimate.

To date most proposed methods in the balancing weights (though not synthetic controls) literature require the following two assumptions: (1) the covariates are measured without error, and (2) the observations are independent. Our approach accounts for both of these problems directly and provides a general method that can easily be used for other applications. In our application we estimate our covariates using annual American Community Survey (ACS) microdata aggregated to the consistent public use microdata area (CPUMA) level. The sampling variability in the covariate estimates is a form of measurement error that may bias our effect estimates. We therefore use regression calibration techniques to reduce the bias from the estimation error of our covariates \cite{gleser1992importance}. Moreover, CPUMAs are regions that nest within states. A common assumption in the applied literature is that regions within states contain dependencies that can worsen the efficiency of standard estimation procedures (see, e.g., \cite{cameron2015practitioner}). Using the assumed correlation structure outlined in \cite{kloek1981ols}, we also provide a modification of the stable balancing weights objective (\cite{zubizarreta2015stable}) to account for potential state-level dependencies in the outcomes. Our approach therefore builds on the existing methods by providing a method that accounts for measurement error in our covariates and a hierarchical data structure.

Our approach also relates to the ``synthetic controls'' literature. Synthetic control is a popular balancing weights approach frequently used in the applied economics literature when studying the effects of policy changes using region-level time series cross-sectional data. The typical estimand in the synthetic controls literature is the treatment effect on the treated (ETT) (\cite{abadie2010synthetic}). By contrast, we consider the problem of estimating the ETC. This treatment effect is often an estimand of substantive interest. For example, \cite{miller2019medicaid} used their estimates of the ETT to predict that had states that did not expand Medicaid done so, they would have seen 15,000 fewer deaths during their study period. However, we cannot in general assume that estimates of the ETT should provide good counterfactual estimates of the ETC. More recently, \cite{born2020lockdowns} estimated the effect a lockdown would have had on Sweden on COVID-19 cases and deaths. While \cite{born2020lockdowns} directly estimate the relevant counterfactual, they follow the traditional synthetic controls approach without noting that they are estimating a different treatment effect than synthetic controls were designed to predict. Due to potential heterogeneous treatment effects is important to estimate the causal estimand of substantive interest, as well as to clarify the assumptions used to estimate this effect. A third contribution of our paper is therefore to clarify one set of assumptions to appropriately use ``synthetic controls'' to estimate the ETC. A key takeaway is that we caution against using conducting variable selection or determining relative covariate importance using pre-treatment outcomes alone. 

The remainder of this paper has the following structure. Section 2 provides an overview of the data and defines the study period, covariates, outcome, and treatment. Section 3 discusses our methods, beginning by defining our target estimand, and then outlining our identification, estimation, and inferential procedures. Section 4 presents our results, sensitivity analyses, and investigation of treatment effect heterogeneity. Section 5 contains a discussion of the policy relevance of our findings, and Section 6 contains a brief summary. The Appendices contain additional materials, including proofs, summary statistics, and additional results.

\section{Data}
In this section we provide an overview of our data source, the covariates, the outcome, and the treatment assignment.

\subsection{Data Source}

Our primary data source is the annual household and person public use microdata files from the American Community Survey (ACS) from 2011 through 2014. The ACS is an annual survey of approximately three million individuals across the United States. The public use microdata files include information on individuals in geographic areas greater than 65,000 people. The smallest geographic unit contained in these data are public-use microdata areas (PUMAs), arbitrary boundaries that nest within states but not within counties or other more commonly used geographic units. One limitation of these data is a 2012 change in the PUMA boundaries, which do not overlap well with the previous boundaries. As a result, the smallest possible geographic areas that nest both PUMA coding systems are known as consistent PUMAs (CPUMAs). The United States contains 1,075 total CPUMAs, with states ranging from having one CPUMA (South Dakota, Montana, and Idaho) to 123 CPUMAs (New York). Our primary dataset (discussed in Section~\ref{sssec:txassign} contained 929 CPUMAs among 46 states. The average total number of sampled individuals per CPUMA across the four years is 1,001; the minimum number of people sampled was 334 and the maximum is 23,990. Importantly, this survey is a repeated cross-section rather than a longitudinal dataset of individuals over time.

\subsection{Study period}

We begin our analysis in 2011 following \cite{courtemanche2017early}, who note that several other aspects of the ACA were implemented in 2010 -- including the provision allowing for dependent coverage until age 26 and the elimination of co-payments for preventative care -- and likely induced differential shocks across states. We also restrict our post-treatment period to 2014: several additional states expanded Medicaid in 2015, including Indiana, Michigan, and Pennsylvania. However, these states did not expand Medicaid contemporaneously with the 2014 ACA provisions. Without additional assumptions, this second-year expansion cannot help us estimate the effect of the 2014 expansion. 

\subsection{Covariates}

We use the underlying individual-level ACS survey data and accompanying survey weights to aggregate the data at the CPUMA level. We choose our covariates to approximately align with those considered in \cite{courtemanche2017early} and that are likely to be potential confounders. Because we are ultimately interested in calculating rates, these variables include both the numerator and denominator counts.

Using the ACS survey weights, we first estimate: the total non-elderly adult population for each year 2011-2014; the total labor force population (among non-elderly adults) for each year 2011-2013; and the total number of households averaged from 2011-2013. We also construct an average of the total non-elderly adult population from 2011-2013. These are our denominator variables. For our numerator counts, we estimate the total number of: females; whites; people of Hispanic ethnicity; people born outside of the United States; citizens; people with disabilities; married individuals; people with less than a high school education, high school degrees, some college, or college graduates or higher; people living under 138 percent of the FPL, between 139 and 299 percent, 300 and 499 percent, more than 500 percent, and who did not respond to the income survey question; people aged 19-29, 30-39, 40-49, 50-64; households with one, two, or three or more children, and households that did not respond about the number of children. We average these estimated counts using the ACS survey weights from 2011-2013. For each individual year from 2011-2013, we estimate the total number of people who were unemployed and uninsured at the time of the survey (calculated among all non-elderly adults and all non-elderly adults within the labor force, respectively). We divide the numerator counts by the corresponding denominator counts to estimate the percentage in each category. For the demographics, these include the average number of non-elderly adults from 2011-2013. For the time-varying variables, we use the corresponding year (where uninsurance rates are calculated as a fraction of the labor force rather than the non-elderly adult population). We also calculate the average non-elderly adult population growth and the average number of households to adults across 2011-2013. 

In addition to the ACS microdata, we use 2010 Census data to calculate the approximate percentage of people living within an ``urban'' area for each CPUMA. Finally, we include three state-level covariates reflecting the partisan composition of each state's government in 2013 using data from the National Conference of State Legislatures (NCLS). Specifically, we generate an indicator for states with a Republican governor, an indicator for states with Republican control over the lower legislative chamber, and an indicator for states with Republican control over both chambers of the legislature and the governorship.\footnote{Nebraska is the only state with a unicameral legislature; moreover, the legislature is technically non-partisan. We nevertheless classified them as having Republican control of the legislature.} 

\subsection{Outcome}

Our outcome of interest is the non-elderly adult uninsurance rate in 2014, which we denote using $Y$. While take-up among the Medicaid-eligible population is a more natural outcome, we choose the non-elderly adult uninsurance rate for two reasons, one theoretic and one practical. First, Medicaid eligibility in the post-period is likely endogenous: Medicaid expansion may affect an individual's income and poverty levels, which in general define Medicaid eligibility. A second reason is to align our study with others to compare our results with the existing literature, and this is the outcome that \cite{courtemanche2017early} use. One drawback of using this outcome is that the simultaneous adoption of other ACA provisions by all states in 2014 more clearly affects this rate in a way that a more targeted group might not be.

\subsection{Treatment assignment} \label{sssec:txassign}

While some states expanded Medicaid in 2014 and other states did not, assigning a binary treatment status simplifies a more complex reality. There are three reasons to be cautious about this simplification. First, states differed substantially in their Medicaid coverage policies prior to 2014: with perfect data we might consider Medicaid expansion as a continuous treatment with values proportional to the number of newly eligible individuals. The challenge, however, is correctly identifying newly eligible individuals in the data (see \cite{frean2017premium}, who attempt to address this). Second, \cite{frean2017premium} note that five states (California, Connecticut, Minnesota, New Jersey, and Washington) and the District of Columbia adopted partial limited Medicaid expansions prior to 2014. \footnote{\cite{kaestner2017effects} and \cite{courtemanche2017early} also consider Arizona, Colorado, Hawaii, Illinois, Iowa, Maryland, and Oregon to have had early expansions.} Lastly, timing is an issue: among the states that expanded Medicaid in 2014, Michigan's expansion did not go into effect until April 2014, while New Hampshire's expansion did not occur until September 2014.

Our primary analysis excludes New York, Vermont, Massachusetts, Delaware, and the District of Columbia from our pool of expansion regions, because these regions had comparable Medicaid coverage policies prior to 2014 (\cite{kaestner2017effects}). We also exclude New Hampshire because it did not expand Medicaid until September 2014. While Michigan expanded Medicaid in April 2014, we leave this state in our pool of treated states. We consider the remaining expansion states as ``treated'' and the non-expansion states as ``control'' states. We later consider the sensitivity of our results to these classifications by removing the early expansion states noted by \cite{frean2017premium}. Our final dataset contains aggregated statistics for all of the above variables for 925 CPUMAs in our non-expansion and our pool of expansion states. There are 414 CPUMAs among 24 non-expansion states and 511 CPUMAs among 21 expansion states. When we exclude the early expansion states for sensitivity analyses, we are left with 296 CPUMAs across 17 expansion states.

\section{Methods}\label{sec:methods}

In this section we present our causal estimand, identifying assumptions, estimation strategy, and inferential procedure.

\subsection{Estimand}

Our goal is to estimate the average effect 2014 Medicaid expansion would have had on the non-elderly adult uninsurance rate in states that did not expand Medicaid. Let $A$ indicate treatment assignment, $c$ index a CPUMA, $s$ index the state, and $t$ index the time period. Let $n_1$ be the number of treated CPUMAs, $n_0$ be the number of control CPUMAs, and $n$ be the total number of CPUMAs. Similarly, let and $m = m_1 + m_0$ states (with $m_1$ and $m_0$ defined analogously). Each state has $p_s$ CPUMAs. $A_s=0$ indicates untreated states and $A_s=1$ indicates treated states, when this notation is a superscript if it indicates the potential outcome. Since we are only interested in the counterfactual at time $T = 2014$, we simplify notation by removing this variable and the subscript and write our formal estimand as:

\begin{equation}
\psi = \psi^1 - \psi^0 = n_0^{-1}\sum_{s, c: A_s = 0} Y_{sc}^{A_s = 1} - Y_{sc}^{A_s = 0} 
\end{equation}

The challenge is that we do not observe the counterfactual outcomes for non-expansion CPUMAs had they been in states that expanded their Medicaid programs. We therefore require causal assumptions to tie this counterfactual quantity to our observed data.\footnote{The 2014 Medicaid expansion occurred simultaneously with the implementation of several other major ACA provisions, including (but not limited to) the creation of the ACA-marketplace exchanges, the individual mandate, health insurance subsidies, and community-rating and guaranteed issue of insurance plans (\cite{courtemanche2017early}). Almost all states broadly implemented these reforms beginning January 2014. Conceptually we think of the other ACA components as a state-level treatment ($R$) separate from Medicaid expansion ($A$). Therefore, our total estimated effect may also include interactions between these policy changes; however, we do not attempt to separately identify these effects. Because the ACA implementation and Medicaid expansion may vary over time, we do not try to generalize these results beyond 2014.} 

\subsection{Identification}

The following causal assumptions are necessary (though insufficient) to identify our target parameter from our observed data: the stable unit treatment value assumption (SUTVA), no unmeasured confounding, and no anticipatory treatment effects. We explain these assumptions in detail and their consequences below. We additionally invoke several parametric assumptions to help us identify our causal parameter given the measurement error in our covariates. These assumptions in total are sufficient to identify our causal estimand.

SUTVA has two implications: first, that there is only one version of treatment; second, that $Y_{sc}^{\mathbf{a}} = Y_{sc}^{\mathbf{a}'}$ when $a_{sc} = a_{sc}'$, where $\mathbf{a}$ is the vector of all treatment assignments. We discussed potential violations of the first assumption previously when considering how to reduce Medicaid Expansion to a binary treatment classification. Our solution is to remove states with less restrictive Medicaid eligibility requirements prior to 2014 to approximately satisfy this condition. The second part of this assumption implies that the potential outcomes in each region do not depend on another region's treatment assignment. This is a standard assumption, but is often not realistic in practice. Violations are likely in our setting: for example, \cite{frean2017premium} find evidence that Medicaid expansion drove previously eligible but uninsured individuals to enroll in Medicaid in both expansion and non-expansion states. Signing the potential bias from this violation requires redefining the causal estimand: for example, we might consider the treatment effect on the untreated given that all states have expanded Medicaid, where the contrast is against where only the observed expansion states expanded Medicaid. If the spillover effects were equal in each region, and the magnitude of the spillovers increase with the total number of treated regions, then the true effect would be larger in absolute magnitude than the estimated effect using the observed data. We could consider other estimands or assumptions to get different predictions about the sign of the bias; however, this is beyond the scope of this paper.

We next assume that there were no anticipatory treatment effects. Letting treatment occur at time $T$, we have that for $t < T$:

\begin{align*}
Y_{sct} = Y_{sct}^0
\end{align*}

This assumption is necessary because we will condition on pre-treatment outcomes. If these outcomes were affected by the treatment before it were implemented, these covariates would be endogenous. Anticipatory treatment effects may occur if plans to expand Medicaid induce uninsured but Medicaid-eligible individuals to enroll in Medicaid prior to expansion. We do not think these violations occurred in large enough numbers to substantially affect our results. Instead, we address a more concerning version of this violation: the fact that several states allowed certain counties to expand Medicaid prior to 2014. We test the sensitivity of our results to the exclusion of these states.

Third, we assume no unmeasured confounding; that is, that at time $T$ the potential outcomes for each CPUMA are independent of the state-level treatment assignment conditional on the population-level CPUMA and state-level covariates $X_{sc}$, a $q$ dimensional vector of covariates (which includes pre-treatment outcomes):

\begin{align*}
Y_{sc}^a \perp A_{sc} \mid X_{sc}
\end{align*}

While unverifiable, we believe it is reasonable here given our rich covariate set. To be explicit, we believe that the pair of potential uninsurance rates for each CPUMA are independent of the treatment assignment conditional on the percentage of uninsured individuals in each year of the pre-treatment period, the percentage of unemployed individuals in each year of the pre-treatment period, the average population growth, the average ratio of households to non-elderly adult population, the state's political composition, the average proportion of households with one, two, or three or more children during the pre-treatment period, the average proportion of households who did not respond about their number children, and the average proportion of individuals during the pre-treatment period with given demographics noted above (age group, sex, white, Hispanic ethnicity, U.S. citizenship, foreign born, income-to-poverty group (including non-response), disability status, urban residence, and educational attainment group). 

A key problem we address in this paper is the violation of this assumption due to measurement error in our covariates. Let 
$X = (X_0, X_1)^T$ be the $n$ by $q$ matrix of true covariates (and separating the control from treated units using $X_a$). Because our covariates are estimated using the ACS data, rather than $(Y, X)$, we instead observe $(J, W)$, which consist of estimates of the true outcome and covariate values, where $W$ is structured analogously to $X$. Importantly, $Y_{sc}^a \perp A_{sc} \mid X_{sc} \centernot\implies J_{sc}^a \perp A_{sc} \mid W_{sc}$. The use of these proxies may therefore bias our estimates. We rely on several modeling assumptions to correct for this.

We first model our observed data as functions of the true values plus mean-zero Gaussian noise: $J_{sc} = Y_{sc} + \xi_{sc}$ and $W_{sc} = X_{sc} + v_{sc}$, where we assume $\xi_{sc}$ and $v_{sc}$ are independent though not identically distributed.\footnote{Our covariates are almost all ratio estimates, which will in general be biased. This bias, however, decreases quickly with the sample size (is $O(n^{-1})$). Given that our CPUMA sample sizes are all over 300, we treat these estimates as unbiased in our analysis.} We assume that these errors are uncorrelated with the true values, i.e. $\mathbb{E}\{\xi_{sc} \mid Y_{sc}\} = 0$ (and similarly for all elements of $X$). Second, we assume that $\xi_{sc}$ are uncorrelated with the errors in the covariate measurements. These assumptions and our model for the observed data are reasonable given that the measurement error in this context is sampling variability. Moreover, our outcomes are measured on a different cross-section than our covariates, so it is reasonable to assume that they are uncorrelated with the measurement errors in the covariates. 

We next assume that the true potential outcomes are linear in the true covariates $X_{sc}$. Specifically, we assume that the following model generates the potential non-elderly adult uninsurance rate under treatment $A = a$:

\begin{equation}\label{eqn:outcomemodel}
Y_{sc}^a = \alpha_a + X_{sc}^T\beta_a + \epsilon_{sc} + c_s
\end{equation}

We assume that the errors $\epsilon_{sc}$ and $c_s$ are mean-zero, independent from each other and across time (i.e., we rule out serial-correlation), and are uncorrelated with the true covariates and the treatment assignment, i.e. $\mathbb{E}\{\epsilon_{sc}c_s \mid X_{sc}, A_s\} = \mathbb{E}\{\epsilon_{sc} \mid X_{sc}, A_s\} = \mathbb{E}\{c_s \mid X_{sc}, A_s\} = 0$. We can then identify $\psi^a$ in terms of our model parameters (see Appendix A); specifically, we have that $\psi^a = \alpha_a + \bar{X}_0^T\beta_a$, where $\bar{X}_0$ is the vector of mean covariate values among the control units. Moreover, we can substitute $J_{sc}$ for $Y_{sc}$ in Equation~\ref{eqn:outcomemodel}, and add $\xi_{sc}$ to the error term without affecting identification. 

We still have the problem that we observe $W_{sc}$ instead of $X_{sc}$. Let $\eta_a = \mathbb{E}\{X_{sc} \mid W_{sc}, A_s = a\}$. By linearity, we know that

\begin{equation}
    J_{sc} = \eta_a(W_{sc})^T\beta + (X_{sc} - \eta_a(W_{sc}))^T\beta + \xi_{sc} + \epsilon_{sc} + c_s 
\end{equation}

If we knew $\eta_a$, we could estimate this model using the observed data $(J, W)$. The approach we follow here is known as ``regression calibration'' in the measurement-error literature. In particular, we assume a linear model for $\eta_a$:

\begin{align*}
\eta_a(W_{sc}) = \upsilon_a + \kappa_a^T(W_{sc} - \upsilon_a)
\end{align*}

where $\kappa_a = (\Sigma_{XX \mid A = a} + \Sigma_{vv \mid A = a})^{-1}\Sigma_{XX \mid A = a}$. The assumption motivating this model is that $(X_{sc}, v_{sc}) \stackrel{iid}\sim MVN((\upsilon_a, 0), \Sigma_a)$ and $\Sigma_a$ is a $2q$ by $2q$ block-diagonal matrix consisting of $q$ by $q$ matrices $\Sigma_{XX \mid A = a}$ and $\Sigma_{vv \mid A = a}$ and $0$ in the off-diagonals. Given sufficient auxillary data to estimate $\kappa_a$, we can then estimate $\psi$. We discuss this further below and in Appendices A and B (see also \cite{gleser1992importance}).

\subsection{Estimation}

We outline our estimation strategy first emphasizing how estimating the ETC differs from estimating the ETT with respect to variable selection under the ``synthetic controls'' framework and associated assumptions required. Second, we explain our estimation procedure, which modifies the SBW criterion to address the hierarchical data structure. This objective, which we call H-SBW, reduces the variance of our estimator under our assumption of constant variance and constant within-state correlation of model errors. Third, we connect our estimator to the regression calibration literature by generating weights that balance a linear prediction of the true covariates $\hat{\eta}_a(W_{sc})$ using the observed covariates $W_{sc}$. Fourth, we test the sensitivity of our estimator to a regression-augmented version, using ridge-regression weights following the suggestion of \cite{ben2018augmented}; this allows us to achieve better covariate balance by extrapolating beyond the support of the data. We conclude by proposing a model validation procedure that uses pre-treatment outcomes to compare the performance of our estimators on pre-treatment data.

\subsubsection{Variable selection}

We seek to generate a set of positive weights that balance the means of covariates for the treated units to the mean covariates for the control units. Assume that we observe the true covariate matrices for the treated data $X_a = (X_{a,1}, ..., X_{a, q})$, and the true outcomes $Y$. Let $\bar{X}_{0, r}$ be the mean covariate value for the $r$-th covariate in the non-expansion region. Assume there exists some $\gamma^\star \in \Gamma$ satisfying: 

\begin{equation}\label{eqn:constraint}
\Gamma = \{\gamma \in \mathbb{R}^{n_1}: \lvert \gamma^TX_{1, r} - \bar{X}_{0, r} \lvert \le \delta_r \ \ (r = 1, ..., q), \ \gamma_{sc} > 0, \sum_{s, c: A_{sc} = 1}\gamma_{sc} = 1\}
\end{equation}

for $\delta_r = 0$ for all $r = 1, ..., q$. We could then estimate $\psi$ as

\begin{equation}\label{eqn:psi}
\hat{\psi} = \sum_{s: A_s = 1}^{m_1}\sum_{c = 1}^{p_s}\gamma_{sc}^\star Y_{sc} - n_0^{-1}\sum_{s: A_s = 0}^{m_0}\sum_{c = 1}^{p_s}Y_{sc}
\end{equation}

Again assuming that the potential outcomes are a linear function of the true covariates: $\mu_a(X_{sc}) = \alpha_a + X_{sc}^T\beta_a$, the bias of our estimate of $\psi^1$ (again assuming we observed $X_{sc}$), is less than or equal to $\lvert\beta_1\rvert^T\delta = 0$ (see, e.g., \cite{zubizarreta2015stable}). The challenge is that for any given dataset we have no guarantee that any such $\gamma^\star$ exists that exactly balances the covariates. We therefore often require some method of determining which parts of the covariate distribution we wish to prioritize balancing - or which covariates to balance at all - to minimize this bias.

The synthetic controls approach chooses the $\gamma$ that minimizes the weighted L2-squared distance of the covariates using a diagonal weighting matrix $V$. $V$ is then chosen to minimize the mean-square error of the weighted difference in pre-treatment outcomes. Letting $Z_a$ be the matrix of pre-treatment outcomes for treatment group $A = a$, the synthetic controls algorithm solves the following optimization problem for a fixed $V$:

\begin{align}
\gamma(V) = \arg\min_{\tilde{\gamma}(V^\star)} = (\bar{X}_1 - X_0^T\tilde{\gamma})'V(\bar{X}_1 - X_0^T\tilde{\gamma}) 
\end{align}

This is the ``inner'' optimization. $V^\star$ is then determined in an ``outer'' optimization to minimize the imbalances in the pre-treatment outcomes $Z$:

\begin{align}
    V^\star = \arg\min_V (\bar{Z}_1 - Z_0^T\gamma(V))'(\bar{Z}_1 - Z_0^T\gamma(V))
\end{align}

In applications the covariate matrix $X_a$ may contain some elements of $Z_a$. In cases where $X_a$ contains all pre-treatment outcomes, \cite{kaul2015synthetic} has shown that the predictor weights $V^\star$ will give no weight to auxillary covariates (covariates that are not the pre-treatment outcomes), rendering these irrelevant to the model. 

While in practice $V$ is often learned on the same data as the weights, we consider the case where we use cross-validation to choose $V$, as proposed by \cite{abadie2015comparative}. Assume we can divide our pre-treatment data into a training data from periods $T = 1, ..., T - l - 1$, a validation period from periods $T - l, ..., T - 1$, and a post-treatment period at time $T$. To make this discussion more general, assume that we are evaluating a set of candidate models $\mathcal{M}$ on the validation data (where for the synthetic controls algorithm we can think of this as the set of all possible weighting matrices $V$). Let $\bar{Y}^a_{a', t}$ 
be the mean potential outcome under treatment $A = a$ for treatment group $A = a'$ at time $t$ (where $t$ occurs during the validation period). Let $\hat{\bar{Y}}^a_{a'', t}(m)$ be an estimator of that potential outcome at time $t$ using model $m$, which was trained during the training period using data from treatment group $A = a''$. Finally, let $\bar{Y}_{a'}^{a_T}$ be the post-treatment estimand, where $\hat{Y}^a_{a'', T}(m)$ is the estimator using model $m$ trained using validation period data. This learning procedure implicitly assumes that:

\begin{align*}
m^\star = \min_{m \in \mathcal{M}}\sum_{T - l}^{T-1}\|\hat{Y}^0_{0, t}(m) - \bar{Y}^0_{1, t}\| = \min_{m \in \mathcal{M}}\mathbb{E}\{\|\hat{Y}^0_{0, T}(m) - \bar{Y}^0_{1, T}\|\}
\end{align*}

In other words, we select our model using the empirical loss in the validation period as a proxy for the expected loss in the post-treatment time-period.\footnote{It is possible that multiple models in $\mathcal{M}$ either perfectly predict the pre-treatment outcomes, or predict them equally well. In this case we would require an additional criteria to choose the optimal model (see, e.g, \cite{becker2017cross}}. This makes intuitive sense in the typical synthetic controls setting where the estimand is the ETT since we observe $Y^0_{sct}$ for $t < T$. When synthetic controls are used to estimate the ETC, this strategy alone is insufficient as we never observe $Y^1_{sct}$ (or a mean-unbiased proxy) prior to treatment for any unit. We therefore cannot easily use pre-treatment outcomes to optimally select variables or determine relative covariate importance without stronger assumptions.

One such assumption is the following:

\begin{align*}\label{assumption:second}
m^\star = \min_{m \in \mathcal{M}}\sum_{T - l}^{T-1}\|\hat{Y}^0_{1, t}(m) - \hat{Y}^0_{0, t}\| = \min_{m \in \mathcal{M}}\mathbb{E}\{\|\hat{Y}^1_{1, T}(m) - \bar{Y}^1_{0, T}\|\}
\end{align*}

We call this assumption ``counterfactual risk invariance.'' In other words, we assume that the model that minimizes the validation-period risk also minimizes the post-treatment risk. We caution that this is a strong assumption for conducting any form of variable selection or covariate weighting in this setting. As a simple example, assume that we can partition $X = (R, S)$, where, for simplicity, we assume $S$ is univariate. Further assume that $Y^0_t \perp A \mid R$ for all $t = 1, ..., T$ but that $Y^1_T \perp A \mid X$. Again assume that $\mu_a$ are linear in the covariates with (time-invariant) coefficients $\beta_{a, r}$ ($r = 1, ..., q$). These assumptions imply that $\beta_{0, s} = 0$. If our goal is to predict the ETT, then we wish to use the untreated data to predict $\bar{Y}_{1, T}^0$ (or $\psi^1$). We may then conduct some variable selection procedure using our pre-treatment data, learn that covariate $S$ is unimportant, and estimate a model that downweights imbalances in $S$ (or ignores the covariate entirely) but perfectly balances the remaining covariates. This model would give an unbiased estimate of the counterfactual outcome for the treated group absent treatment in time-period $T$. However, if our goal were instead to predict $\bar{Y}^1_{0, T}$ -- the counterfactual outcome under treatment for the untreated units -- the same procedure applied to the treated data would again downweight $S$ and result in a biased estimate, with bias equal to $(\gamma^TS_1 - \bar{S}_0) \beta_{1, s}$. If $S$ is a strong predictor of treatment assignment, this could lead to substantial bias. \footnote{We conflate two issues in this discussion: the synthetic controls variable weighting algorithm and variable selection more generally. Provided $S$ is contained in $X$ and exact balancing weights exist with high probability as $n \to \infty$, then under our modeling assumptions, the synthetic controls procedure is still consistent for $\psi^1$ even if the variable weights $V$ are sub-optimal. However, if $S$ is not contained in $X$ at all, then this procedure will not balance these covariates even asymptotically provided that $\mathbb{E}\{S \mid A = 1\} \ne \mathbb{E}\{S \mid A = 0\}$. However, here we take a finite-sample perspective. Imbalances in $S$ will lead to bias, whether or not $S$ is imbalanced due to receiving low weights on the weighting matrix $V$, or omitted from the objective entirely. Moreover, these two conditions are identical in the case where the optimal weight on covariate $S$, $V_s = 0$.} 

As a practical example, we highlight the potential confounding role of Republican governance for our counterfactual estimate. Republican governance is a strong predictor of a state's decision to expand Medicaid \cite{courtemanche2017early}. Moreover, existing evidence prior to Medicaid expansion showed that Medicaid take-up rates were lower in more conservative states \cite{sommers2012understanding}. Yet when generating their synthetic control weights to estimate the ETT, \cite{courtemanche2017early} and \cite{kaestner2017effects} do not control for these factors. \footnote{\cite{courtemanche2017early} does control for Republican governor in their regression model and they find that it is a statistically significant predictor of 2013 uninsurance rates. One reason they may not control for this in the synthetic control model is practical: it is challenging to balance this covariate using control data without extrapolating from the data.} However, it is clear that if take-up rates depend on governance, we may expect this to be a strong confounder of $Y^1$ and hence confound the ETC, even if arguably it is not a confounder of $Y^0$ (and hence not a confounder for the ETT).

We demonstrate this in our application by conducting a variable importance analysis. Specifically, we remove the balance constraints from the Republican governance indicators and examine how our estimates of $\hat{\psi}^1$ change. Letting $\hat{\psi}^1_s$ be the estimate when removing the Republican governance indicators (or more generally, the covariate matrix $S$ where $X = (R, S)$). We subtract our original point estimate $\hat{\psi}^1_0$ from $\hat{\psi}^1_s$ to generate the difference $\hat{\Delta}^1$. This difference tells us about the direction of the bias our estimate of $\hat{\psi}^1$ would incur when we do attempt to constrain the imbalance in covariate $S$. Our hypothesis implies that we should expect $\hat{\Delta}_s^1 < 0$: that is, keeping all other covariates (roughly) fixed, we expect the predicted uninsurance rate will decrease when as the level of Republican governance decreases. In addition to the Republican governance indicators, we also examine four other covariate groups: pre-treatment uninsurance rates and pre-treatment unemployment rates, and three sets of different demographic indicators, which we detail in Appendix E.\footnote{We caution that our results do not imply that Republican governance is not an important confounder of $Y^0_{1, T}$ since we do not analyze this directly.} 

Overall we emphasize that predicting the outcome under treatment is different, and perhaps more challenging, than predicting the outcome absent treatment. The former requires understanding which covariates matter most to predicting treatment response, which we cannot as naturally learn from pre-treatment outcomes. We instead rely on our prior knowledge and modeling assumptions to choose which covariates to balance and which covariates to prioritize in this balancing. We point out that modeling the ETC requires greater justification of the covariates used to predict treatment response than for the ETT, and that using the standard synthetic controls variable weighting procedure is unlikely to be optimal for this purpose.\footnote{Our analysis assumes no unmeasured confounding and a linear model for $\mu_a$. By contrast, synthetic controls are frequently motivated by a linear factor model for $\mu_0$. \cite{abadie2010synthetic} and \cite{ferman2016revisiting} outline conditions where this method is consistent as the number of pre-treatment outcomes goes to infinity, in particular because the method balances the unobserved factor loadings. Analogous to our analysis, if we assume $\mu_{a, T}$ both follow a linear factor model, identification of the ETC requires that the unobserved factor loadings that confound $Y^1_T$ are the same that confound $Y^0_t$. Under this assumption, we might be able to show that the synthetic control estimator is consistent in this setting. However, the tuning procedure to determine the predictor weights may again be sub-optimal from a finite-sample bias perspective, depending again on how the covariates (or unobserved factors) that are most predictive of treatment response vary between treatment groups and on their associations with the potential outcomes.}

Given these challenges, we therefore use a variation of SBW to estimate the ETC.\footnote{Specifically, we use a modified implementation of Noah Griefer's ``optweight'' package in R, available on github.com/mrubinst757} SBW minimizes the variance of the weights subject to user-specified balance constraints. The primary advantages of H-SBW over synthetic controls in this setting are that it gives the user finer control over the desired levels of covariate balance, allowing the user to navigate a bias-variance tradeoff with respect to balance and the variability of the weights. Specifically, SBW weights solve the following minimization:

\begin{equation}
\gamma = \arg\min_{\tilde{\gamma} \in \Gamma} \quad \sum_{s: A_s = 1}^{m_1}\sum_{c = 1}^{p_s} \tilde{\gamma}_{sc}^2  
\end{equation}

where $\Gamma$ is defined in Equation~\ref{eqn:constraint}. We can then estimate $\psi$ using Equation~\ref{eqn:psi}, substituting $J_{sc}$ for $Y_{sc}$ and plugging in the weights $\gamma$. By contrast, the synthetic controls algorithm will minimize the weighted L2 distance between the treated and control units in the criterion; this algorithm in general may lead to lower imbalances, but the balance tradeoffs are difficult to control, and the resulting weights may be more extreme. Finally, the algorithm, as formulated in \cite{abadie2010synthetic} and presented above, may not have a unique solution (but see \cite{ben2018augmented}, \cite{becker2017cross}).

For our primary estimates we lean heavily on assumptions to justify our choice of $\delta$. We use a priori domain knowledge about which covariates are most likely to be important predictors of treatment response when setting $\delta$, but also choose $\delta$ to avoid generating overly extreme weights. For our application, we constrain $\delta$ to be 0.05 percentage points (out of 100) for pre-treatment outcomes, 0.15 percentage points for pre-treatment unemployment rates, and 25 percentage points for the Republican governance indicators. We believe these covariates are most likely to predict treatment response. While we believe that Republican governance is an important covariate to balance, we are unable to reduce the constraints further given the support of the data. For the remaining covariates, we let $\delta$ be 0.5 percentage points for average population growth and household to adult ratio, 1 percentage point for female, Hispanic ethnicity, white race, age category, disability, and number of children category; 2 percentage points for urban, citizenship, education category, income-to-poverty category, student, and foreign-born, again choosing these constraints with respect to both feasibility and extreme weight concerns. 

\subsubsection{H-SBW objective}

The motivation of the SBW criterion is to produce the minimum variance weights for a fixed $\delta$. This produces the minimum variance estimator within the constraint set if, for example, the errors in the outcome model are independent and identically distributed \cite{zubizarreta2015stable}. In our setting we allow for possible state-level dependencies, potentially reducing the efficiency of the SBW estimator. To address this possibility, we add the tuning parameter $\rho \in [0, 1]$ in the objective below. Assuming a constant variance across units for each error component, $\rho$ represents a constant (and known) within-state correlation of the errors. 

\begin{equation}\label{eqn:objective}
\gamma = \arg\min_{\tilde{\gamma} \in \Gamma} \quad \sum_{s: A_s = 1}^{m_1}(\sum_{c = 1}^{p_s} \tilde{\gamma}_{sc}^2 + \sum_{c \ne d}\rho \tilde{\gamma}_{sc}\tilde{\gamma}_{sd})\\
\end{equation}

For $\delta \to \infty$, this objective yields the solution:

\begin{equation}\label{eqn:sbwsol}
\gamma_{sc} \propto \frac{1}{(p_s - 1)\rho + 1}
\end{equation}

Setting $\rho = 0$ returns the SBW solution: $\gamma_{sc} \propto 1$. By contrast, when setting $\rho = 1$, we see that $\gamma_{sc} \propto \frac{1}{p_s}$. In other words, as we increase $\rho$, this objective downweight CPUMAs in states with large numbers of CPUMAs and upweight CPUMAs in states with small numbers of CPUMAs (all while assigning each CPUMA within a state equal weight). Moreover, when $p_s$ is constant across states this objective will also the SBW solution. In short, as we increase $\rho$, the objective will attempt to more uniformly disperse weights across states.

As a brief illustration, we simulate $n = 800$ observations in $m = 40$ regions each with $p_s = 20$ units, and draw $X_{sc} \sim N(\mu_s, 1)$, for $\mu \in \{0, 1\}$ (drawn from a Bernoulli with equal probability); $A_s \sim Bern(expit(\bar{X}_s))$, where $\bar{X}_s$ is the mean covariate value in group $s$. We then generate weights to balance the control to the treated group mean. We run H-SBW variants setting $\rho = 0$ (which is equivalent to SBW), $\rho = 0.5$, and $\rho = 0.99$ all while keeping $\delta$ fixed at zero. Figure 1 shows the weights summed to the group level for all control regions. The color of each set of bars is the overall variance of the weights. We can see that the weights in general are uniform across units for SBW (the variance is lower); however, the weights are not uniformly dispersed across regions (the sum of weights within each region is not even). As we increase $\rho$, the weights disperse more evenly across regions. Despite the increase in the variance of the weights, these weights are optimal under our assumed covariance structure. 

\begin{figure}
\begin{center}
    \includegraphics[scale=0.5]{01_Plots/proofofconcept.png}
    \caption{Comparison of SBW and H-SBW: within group sum of weights}
    \label{oatepref}
\end{center}
\end{figure}

The particular covariance structure we assume is identical to the one proposed by \cite{kloek1981ols}. In Appendix A we show that this objective produces the minimum variance estimator under the constraint set for this correlation structure. We note that theoretically we could incorporate any other assumed covariance structure into this objective, though the number of tuning parameters might change. Broadly speaking, we can think of H-SBW being to SBW what generalized least squares (GLS) is to ordinary least squares (OLS): both SBW and OLS can produce unbiased estimates of model parameters; however, H-SBW and GLS can improve the efficiency of our estimates under different assumed correlation structures of the outcome errors.

\subsubsection{Measurement error}

A second advancement in our estimation procedure comes in our balance constraints: rather than balancing on the observed covariate values $W_{sc}$, we instead balance on the imputed covariate estimates $\hat{\eta}_1(W_{sc})$ (we refer to these as the ``adjusted covariates''). This procedure attempts to correct for the estimation error in these CPUMA-level covariates that may bias our estimate of $\psi^1$. In Appendix A, we consider the super-population target $\psi^{1, sp} = \mathbb{E}\{Y^1 \mid A = 0\}$ and show that under the classical errors-in-variables model, the bias for the SBW estimator that balances on the observed covariates $W$ and sets $\delta = 0$ is equivalent to the bias of a linear combination of coefficient estimates from the OLS-based regression estimator. Specifically, the bias for either estimator is:

\begin{equation}
\mathbb{E}\{\hat{\psi}^{1} - \psi^{1, sp}\} = (\upsilon_0 - \upsilon_1)^T(\kappa - I_d)\beta_1
\end{equation}

The intuition for this result is as follows: exact balancing weights implicitly estimate $\beta_1$ on a subset of the data where we have sufficient covariate overlap. We can therefore think of SBW as returning a solution to some weighted-least squares problem. Assuming that the outcome model holds across all of the data, WLS and OLS are estimating the same $\beta_1$; therefore, the bias that effects the least squares solution will have the same effect on the WLS, and therefore SBW, solution. In Appendix A, Proposition 2, we show that if we had access to $\eta_1$, we can obtain an unbiased estimate of $\psi^{1, sp}$ by reweighting $\eta_1(W_{sc})$.\footnote{In Remark 3, we observe that this estimator will have finite-sample bias conditional on $W$ and viewing $X$ as fixed for the finite-sample parameter we are targeting.} Of course, in practice we do not know $\eta_1$ but must instead estimate it using auxillary data. In Appendix A, Proposition 3, we show that we can obtain a consistent of $\psi^{1, sp}$ when balancing on an estimate of $\eta_1$ using auxillary data. 

The key in our application is to estimate $\eta_a$: at a high-level, we use the ACS micro-data replicate survey weights to estimate the covariance matrix of CPUMA sampling-variability $\Sigma_{vv, sc}$. Using our observed data to estimate $\Sigma_{WW \mid A = a}$ and $\bar{W}$, we combine these estimates to generate an estimate of $\eta_a$. This is a technique that comes from the regression-calibration literature (see, e.g., \cite{gleser1992importance}). We also consider an adjustment procedure that further accounts for the differential measurement error due to the highly variable sample sizes used to calculate each covariate. This procedure allows our adjustment to differentially adjust covariate values depending on the sample-sizes involved in the adjustment procedure. We refer to the first procedure the ``homogeneous adjustment'' and the second procedure the ``heterogeneous adjustment'' because the adjustment is constant for all units in the first case, but varies by unit in the second case. Further details about these procedures are available in Appendix B.

This is the first application we are aware of to use regression calibration in the context of balancing weights to address the problem of measurement error. We emphasize two critical assumptions for using this procedure in our context: (1) the outcome model is linear in the true covariates; and (2) the measurement error in the outcome is uncorrelated with the measurement error in the covariates. The first assumption is strong, though often used in practice. The second assumption is reasonable, because our outcomes are estimated from a different cross-section than our covariates. 

\subsubsection{Bias-correction for imbalances}

Because we are unable to reduce the balance constraints to our preferred level without generating very extreme weights, following the recent literature on synthetic controls, we test the sensitivity of our results to the imbalances in the observed (or adjusted) covariates using ridge-regression augmented weights \cite{ben2018augmented}. Letting $\hat{X}_1$ be the matrix of adjusted covariates, and $\gamma^{hsbw}$ be our H-SBW weights, we consider the regression-augmented weights:

\begin{equation}
\gamma^{aug} = \gamma^{hsbw} + (\gamma^{hsbw}\hat{X}_1 - \bar{W}_0)^T(\hat{X}_1^T\Omega^{-1}\hat{X}_1 + \lambda I_q)^{-1}\hat{X}_1^T\Omega^{-1}
\end{equation}

where $\Omega$ is a block diagonal matrix with diagonal entries equal to one and the within-group off diagonals equal to $\rho$. We choose $\lambda$ so that the remaining imbalances all fall within 0.5 percentage points. The cost of this procedure is that we must extrapolate off the support of the data, and therefore rely more heavily on our outcome modeling assumptions. We refer to \cite{ben2018augmented} for more details about this procedure. In our results we consider estimators using SBW ($\rho = 0$), H-SBW ($\rho = 1/6$), and ridge-augmented versions of SBW and H-SBW that we call BC-SBW and BC-HSBW. 

\subsection{Model validation}

We argued above that we cannot easily use pre-treatment outcomes to conduct variable selection or to learn about the relative importance of covariates. The challenge with using pre-treatment data in this setting is that it is not obvious why the best model of $\bar{Y}_{0, t}^0$ $(t = T-l,..., T-1)$ should also be the best model, or even a good model, of $\bar{Y}^1_{0,T}$ (or $\psi^1$). However, for fixed covariates and targeted levels of imbalance $\delta$, we can use the heuristic that a good model of $\bar{Y}^1_{0,T}$ should also be a good model of $\bar{Y}_{0, t}^0$ to compare our models. We can also assume that when comparing two models, one with uniformly better covariate balance than the other, the model with uniformly better covariate balance should have lower bias both for $\bar{Y}_{0, T}^1$ and $\bar{Y}_{0, t}^0$ if our model assumptions are correct. We justify these comparisons by again assuming that $\mu_{a, t}$ are linear in $X$ for all time periods $t$. This is a stronger assumption than we require to estimate the ETC, which only requires that $\mu_{1, T}$ is linear in $X$. However, with this stronger assumption we can use pre-treatment data to at least heuristically compare our models. While we caution that these comparisons may not indicate the best model of $\bar{Y}^1_{0,T}$, we can still use these comparisons to see which models we may trust less.

We therefore rerun our procedures on pre-treatment data to compare the performance of our models for a fixed level of imbalances $\delta$. In particular, we train our model on 2009-2011 data to predict 2012 outcomes, and 2010-2012 data to predict 2013 outcomes. We limit to one-year prediction error since our estimand is only one-year forward. We then examine the performance of the H-SBW versus SBW estimators, which only vary with respect to the tuning parameter $\rho$, the bias-corrected versions, and the covariate adjustment procedure used to determine the weights. 

We expect that the estimators trained on the adjusted data should perform better than the estimators trained on the unadjusted data. If our outcome model is correct, these estimators should achieve better balance across the true covariates and therefore have lower bias than the estimators trained on the unadjusted data. We assume that any difference between the performance of the two adjustments is due to improved balance on the (unobserved) true covariates. We therefore use the performance on this data to select which adjustment we prefer in our final results. Conditional on the adjustment, we assume that if the bias-corrected estimators all have uniformly better covariate balance than the uncorrected estimators, these estimators should also perform better than the uncorrected estimators. However, if the assumed outcome models are incorrect, these estimators may suffer from extrapolation bias and perform worse despite achieving better covariate balance. Finally, we expect in general that H-SBW and SBW should have similar performance if we knew the true covariate values. However, in Appendix A, we show that for fixed (unobserved) $X$, these weighting estimators have a finite sample bias. The squared-bias term reduces with the square of the weights, suggesting that SBW may have less bias than H-SBW. On the other hand, H-SBW should have less variability given within-state dependencies in the outcomes. From an MSE perspective it is unclear which should be optimal, and given only two-years of pre-treatment data we do not believe we can select which model is better.

\subsection{Inference}

We consider $W$ to be fixed (and $X$ as fixed unknown parameters), and we consider inference over repeated samples from some super-population of CPUMAs with a state-level dependency structure.\footnote{Alternatively, viewing the potential outcomes as fixed and treatment assignment as random, we could consider inference over the randomization distribution of treatment at the state-level.} While placebo tests are frequently used in the synthetic controls literature for inference, we view these as qualitative statistical tests (see, e.g., \cite{arkhangelsky2019synthetic}) and instead use the leave-one-state-out jackknife to estimate the variance of $\hat{\psi}^1$ (\cite{cameron2015practitioner}). Specifically, we exclude each state and re-calculate the weights holding our targeted mean fixed at $\bar{W}_0$.\footnote{When our preferred initial choice of $\delta$ does not converge, we gradually reduce the constraints until it does.} We compute this estimator in two ways: first, we condition on our covariate adjustment $\hat{\eta}_1$. This is our preferred estimator; however, it does not account for the randomness in $\hat{\eta}_1$. We therefore also conduct a second procedure where we re-estimate $\hat{\eta}_1$ for each state omitted in the jackknife procedure and provide these results in the Appendix.

To estimate $Var(\hat{\psi}^0 \mid X, W)$ we use an auxillary regression model and use the CR-2 standard error adjustment (using the ``clubSandwich'' package in R) to estimate the variance of the linear combination $\bar{W}_0^T\hat{\beta}_0$. We can estimate this quantity using the original (unadjusted) data given that $\mathbb{E}\{\bar{W}_0^T\hat{\beta}_0\} = \psi^0$ (since the regression line runs through the point $(\bar{W}_0, \bar{J}_0)$, which are unbiased estimates of $(\bar{X}_0, \bar{Y}_0)$). Our total estimate $\hat{Var}(\hat{\psi})$ is simply the sum of these two variance estimates. We use the standard normal quantiles to generate confidence intervals. 

\section{Results}

This section presents our results. The first sub-section contains summary statistics regarding the variability of six key covariates pre- and post- adjustment to address measurement error. The second sub-section contains covariate balance diagnostics. The third sub-section contains our primary ETC results, presenting our estimates of the treatment effect and a series of sensitivity analyses. The final sub-section contains our analysis of covariate importance and investigation into treatment effect heterogeneity.

\subsection{Covariate adjustment}

We first examine the effects of our covariate adjustments on the variance of our pre-treatment outcomes and pre-treatment unemployment rates among the expansion states. We most heavily prioritize balancing these covariates, but they are also among the least precisely estimated (all of our other covariates average over three years of data, rather than just one). Table~\ref{tab:adjust1} displays the variance of each covariate on the unadjusted and adjusted datasets. We see that both the homogeneous and heterogeneous adjustment procedures reduce the variability in the data by comparable amounts. Intuitively, the adjustment to address measurement error reduces the likelihood that our balancing weights will over-fit to noise in the data. These results are consistent across most of our other covariates.

\begin{table}[ht]
\caption{Sample variance on unadjusted and adjusted datasets, expansion states}
\label{tab:adjust1}
\begin{tabular}{lrrr}
  \hline
Variable & No adjustment & Heterogeneous & Homogeneous \\ 
  \hline
Uninsured Pct 2011 & 8.35 & 8.04 & 8.05 \\ 
  Uninsured Pct 2012 & 8.20 & 7.89 & 7.90 \\ 
  Uninsured Pct 2013 & 8.09 & 7.78 & 7.79 \\ 
  Unemployed Pct 2011 & 3.66 & 3.25 & 3.27 \\ 
  Unemployed Pct 2012 & 3.72 & 3.38 & 3.38 \\ 
  Unemployed Pct 2013 & 3.20 & 2.88 & 2.87 \\ 
   \hline
\end{tabular}
\end{table}

\subsection{Covariate balance}

Figure~\ref{fig:loveplotc1} displays the reduction of imbalances using our H-SBW weights. This plot only displays covariates with greater than one percentage point difference between the targeted mean in the expansion region and the mean values in the non-expansion region prior to weighting, and the reweighted treatment values use our preferred covariate adjustment $\hat{\eta}_1(W_{sc})$. Before applying our weights, we see that there are substantial imbalances in the Republican governance indicators, as well as pre-treatment uninsurance and unemployment rates. Our weights reduce these differences; however, some remain, particularly among the Republican governance indicators. A complete balance table is available in Appendix D, Table~\ref{tab:baltab1}. 

\begin{figure}[H]
\begin{center}
    \caption{Balance plot, primary dataset}
    \label{fig:loveplotc1}
    \includegraphics[scale=0.5]{01_Plots/balance-plot-etuc1.png}
\end{center}
\end{figure}

We also compare the H-SBW estimator to the SBW estimates in Figure~\ref{fig:sbwvhsbw1}. We see that H-SBW more evenly distributes the weights across states relative to SBW, particularly by reducing the amount of weight given to CPUMAs in Ohio. 

\begin{figure}[H]
\begin{center}
    \caption{H-SBW versus SBW, weights summed by state, primary dataset}
    \label{fig:sbwvhsbw1}
    \includegraphics[scale=0.6]{01_Plots/weights-by-state-sbw-hsbw-c1.png}
\end{center}
\end{figure}

We then use ridge-regression augmentation to extrapolate from the data in order to reduce all imbalances within 0.5 percentage points. Figure~\ref{fig:statewghts} shows the total weights summed across states for two estimators: H-SBW and BC-HSBW. This figure sums the negative weights separately from the positive weights to show the extent of the extrapolation. Figure~\ref{fig:statewghts} illustrates that BC-HSBW extrapolates somewhat heavily in order to achieve the desired level of balance, particularly for CPUMAs in California. 

\begin{figure}[H]
\begin{center}
    \caption{H-SBW versus BC-HSBW, weights summed by state, primary dataset}
    \label{fig:statewghts}
    \includegraphics[scale=0.6]{01_Plots/weights-by-state-hsbw-c1.png}
\end{center}
\end{figure}

Finally, we examine whether the H-SBW weights generated using the unadjusted data balance the adjusted covariates. While this balance measure does not reflect the ``true'' imbalances, this comparison does give some indication of whether the unadjusted weights are overfitting to noisy covariate measurements. Table~\ref{tab:balcomp} compares the imbalances among our pre-treatment outcomes and uninsurance rates using H-SBW weights generated on our unadjusted dataset applied to the adjusted (homogeneous) dataset. The ``Unweighted Difference'' column represents the raw difference in means, while the ``Weighted Diff'' column reflects the weighted difference that we calculate on the unadjusted dataset. The ``Homogeneous Diff'' column displays the weighted imbalance when applying the H-SBW weights to the dataset using the homogeneous adjustment, and likewise for ``Heterogeneous Diff.'' We see that the weighted pre-treatment outcomes are approximately one percentage point lower than we desired in the two years prior to treatment using the heterogeneous adjustment, and -0.2 percentage points lower (on average) using the homogeneous adjustment. Given the high degree of expected correlation between pre-treatment and post-treatment outcomes, we may expect the estimator of $\psi^1$ trained on the unadjusted data to have some downward bias.

\begin{table}[ht]
\caption{Balance comparison: unadjusted weights on adjusted data}
\label{tab:balcomp}
\begin{tabular}{lrrrr}
  \hline
Variables & Unweighted Diff & Weighted Diff (none) & Homogeneous Diff & Heterogeneous Diff\\ 
  \hline
Uninsured Pct 2011 & -3.09 & -0.05 & -0.11 & 0.92 \\ 
  Uninsured Pct 2012 & -2.99 & -0.05 & -0.21 & -1.06 \\ 
  Uninsured Pct 2013 & -3.00 & -0.05 & -0.38 & -0.93 \\
   \hline
\end{tabular}
\end{table}

\subsubsection{Model validation}

We compare the performance of our models by repeating the covariate adjustments and calculating our procedure on 2009-2011 ACS data to predict 2012 outcomes, and similarly for 2010-2012 data to predict 2013 outcomes for the treated states. Table~\ref{tab:pretxpred} see that the estimators trained on the covariate adjusted data have substantially better performance than the unadjusted data. Moreover, the estimators trained on the homogeneous adjustment seem to do slightly better than the ones that model the heterogeneity; we therefore prioritize presenting results on the homogeneous adjustment. We find that SBW tends to have slightly lower RMSE than H-SBW. However, the results are quite similar, as we expected. Finally, we see that the bias corrected estimators tend to perform worse in this application. This may indicate that the extrapolation bias outweighs the cost of reducing the covariate imbalances. While this does not necessarily imply this worse performance should hold for the model of $Y^1_T$, it does caution against these results. We see this as a function of our models as best reflecting an approximation: we expect in general that assuming the linear outcome models approximately holds on the support of the data where we have sufficient covariate overlap, but we these models may lead us astray when our weights extrapolate excessively from the data. The worst performing estimators are the bias-corrected estimators trained on the unadjusted data.

\begin{table}[ht]
\caption{Estimator pre-treatment outcome prediction error}
\label{tab:pretxpred}
\begin{tabular}{llrrr}
  \hline
Sigma estimate & Estimator & 2012 error & 2013 error & RMSE \\ 
  \hline
Homogeneous & SBW & -0.18 & -0.22 & 0.20 \\ 
  Homogeneous & H-SBW & -0.24 & -0.21 & 0.23 \\ 
  Heterogeneous & SBW & -0.25 & -0.30 & 0.27 \\ 
  Heterogeneous & H-SBW & -0.32 & -0.39 & 0.36 \\ 
  Homogeneous & BC-SBW & -0.42 & -0.35 & 0.39 \\ 
  Heterogeneous & BC-SBW & -0.45 & -0.39 & 0.42 \\ 
  None & SBW & -0.50 & -0.61 & 0.56 \\ 
  None & H-SBW & -0.52 & -0.61 & 0.57 \\ 
  Homogeneous & BC-HSBW & -0.53 & -0.62 & 0.58 \\ 
  Heterogeneous & BC-HSBW & -0.53 & -0.72 & 0.63 \\ 
  None & BC-SBW & -0.82 & -0.93 & 0.88 \\ 
  None & BC-HSBW & -0.93 & -0.99 & 0.96 \\ 
   \hline
\end{tabular}
\end{table}

Finally, we find a consistent negative bias across all of our estimators: that is, all of our models tend to under-predict the true uninsurance rate among the non-expansion states the subsequent year. If we believe that the sign of this bias will also affect our estimates of $\bar{Y}^1$, then we should expect our treatment effect estimates to have downward bias (which would bias our estimates away from zero). That is, the true treatment effect may be smaller (closer to zero) in absolute magnitude than the estimated treatment effect. If we believe the bias will also have comparable magnitude, these results suggests that the true treatment effect will be approximately 0.2 to 0.3 percentage points closer to zero than the estimated effect.

\subsection{Primary Results}

Using H-SBW we estimate an effect of -2.33 (-3.49, -1.16) percentage points. The SBW results are almost identical with -2.35 (-3.65, -1.06) percentage points. Compared to the unadjusted data we see very similar estimates at -2.34 (-2.85, -1.82) percentage points for H-SBW and -2.39 (-2.95, -1.83) percentage points for SBW. We see that H-SBW reduces the confidence intervals relative to SBW. We also observe that using the adjusted covariate set increases the width of the estimated confidence intervals. This increase in variability is expected because the adjustment procedure generally reduces the variability in the data, as we saw in Table~\ref{tab:adjust1}, thereby requiring that the balancing weights also increase in variability to achieve the desired level of balance. Importantly, this variance estimate conditions on the covariate adjustment, and does not take into account the randomness in this procedure, and may therefore understate the true uncertainty. When we recalculate the entire adjustment procedure, we find that the confidence intervals are of a comparable magnitude. The results are available in Appendix E.

When we add the bias-correction, the absolute magnitude of the point estimate decreases: we estimate -2.05 (-3.32, -0.79) percentage points for BC-HSBW and -2.00 (-2.98, -1.01) percentage points for BC-SBW. Interestingly, in contrast to our validation tests, where the bias-corrected estimators tended to predict lower uninsurance rates than the other estimators, here the bias-correction predicts higher uninsurance rates for $\psi^1$. We also see that in contrast to the H-SBW and SBW estimators, the confidence interval for BC-SBW is narrower than for BC-HSBW. Figure~\ref{fig:estimators} presents all of our estimates. All adjusted estimates were closer to zero than the unadjusted estimates, though the point estimates from the SBW and H-SBW were estimators were virtually identical. We briefly note that the heterogeneous adjustments were all closer to zero than the unadjusted estimates. Complete results are available in Appendix E.

\begin{figure}[H]
\begin{center}
    \caption{Primary point estimates and confidence intervals}
    \label{fig:estimators}
    \includegraphics[scale=0.6]{01_Plots/point-estimates-c1.png}
\end{center}
\end{figure}

Lastly, we examine the robustness of our point estimates to the removal of individual states (these are the same point estimates used to calculate our confidence intervals). Figure~\ref{fig:loostateplot} shows how the point estimates for each estimator changes for both the (homogeneous) adjusted and unadjusted datasets. We see similar results in either case: removing Ohio tends to move the point estimates farther from zero, though the change is larger on the adjusted dataset. By contrast, removing North Dakota, Kentucky, or California tends to move the estimates closer to zero. 

\begin{figure}[H]
\begin{center}
    \caption{Estimator sensitivity to states}
    \label{fig:loostateplot}
    \includegraphics[scale=0.6]{01_Plots/loostate-sensitivityc1-state-uu-i.png}
\end{center}
\end{figure}

\subsection{Sensitivity analysis} \label{sssec:sensitivity}

We now consider the sensitivity of our analysis with respect to no anticipatory treatment effects. Several states had partial limited expansions prior to 2014. Following \cite{frean2017premium}, these states are California, Connecticut, Minnesota, New Jersey, and Washington. We rerun our analyses excluding CPUMAs from all five of these states. We have no a priori expectation about how removing these states might affect our estimates: on the one hand, states that expanded early might have a smaller treatment effect after 2014 because they already enrolled newly eligible individuals. On the other hand, if these states were also more motivated to enroll people in Medicaid, they might have experienced larger post-expansion coverage gains. Figure~\ref{fig:weightsbystatec2} in Appendix D displays the H-SBW weights summed by state alongside BC-HSBW, which extrapolates to reduce the imbalances. 

Excluding early expansion states, we estimate an effect of -2.09 (-2.85, -1.33) percentage points using H-SBW weights and -2.05 (-2.75, -1.35) percentage points for SBW. These estimates are quite similar to though slightly closer to zero than the the primary estimates. We also see that the differential between these estimates and the unadjusted estimates is slightly larger: -2.28  (-2.82, -1.74) percentage points for H-SBW and -2.21 (-2.71, -1.72) percentage points for SBW. 

When we add the bias-correction the point estimates again move closer to zero: -1.94 (-2.96, -0.92) percentage points for BC-HSBW and -1.99 (-3.00, -0.99) percentage points for BC-SBW. Overall our primary results are relatively robust to the exclusion of these states, so we conclude that potential violations of this causal assumption are not a large factor.

\subsection{Covariate importance}

We also investigate our hypothesis that factors associated with Governance are associated with treatment response. As discussed above, we first remove the balance constraints on the Republican governance indicators and estimate $\hat{\psi}^1_v$, and then subtract our original ETC point estimate from this quantity to generate $\hat{\Delta}_v^1$. Because this quantity does not reflect a clear population target, instead of confidence intervals, we present the minimum and maximum leave-one-state-out values in parentheses next to the original estimate.

For the H-SBW estimator we calculate $\hat{\Delta}^1$ equal to -0.69 (min = -0.83, max = -0.42) and equal to -0.79 (min = -0.90, max = -0.66) on our unadjusted dataset. In other words, our primary estimated treatment effect moved 0.78 percentage points further away from zero when we excluded the Republican governance indicators. This reflects a 33 percent decrease in our point estimate, a not-unsubstantial difference. Moreover, all of these estimates were less than zero, regardless of whether we conditioned on the covariate adjustment or not, regardless of whether we remove the early expansion states or not, and when removing each state. Across all specifications that we ran the minimum change we calculated was -1.34 and the maximum was -0.36. Additional distributional results across all leave-one-state-out estimates are available in Appendix E, Table~\ref{tab:rdiffc1}. Figure~\ref{fig:repub} displays our estimates of $\hat{\Delta}_v^1$ on our primary dataset and removing early expansion states (conditional on the covariate adjustment). 

We also consider four other covariate sets. We find that our estimates are most sensitive to controlling for pre-treatment outcomes and unemployment rates. This is not unexpected: all else equal, the expansion region had much lower pre-treatment uninsurance rates. If we do not control for these covariates, the comparable region will likely have lower pre-treament uninsurance rates, causing the estimated counterfactual to be closer to zero. The effect estimates were less sensitive to the removal of other covariate groups, and all point estimates are available in Appendix E, Table~\ref{tab:ptests}.

\begin{figure}[H]
\begin{center}
    \caption{Removing Republican Governance Indicators}
    \label{fig:repub}
    \includegraphics[scale=0.6]{01_Plots/repub-diff-all-estimators.png}
\end{center}
\end{figure}

These results highlight the importance of Republican governance in our counterfactual outcome model of $Y^1$. If the models specified by \cite{kaestner2017effects} and \cite{courtemanche2017early} are correct (that is, they correctly omit Republican governance from their balancing weights for estimating $\bar{Y}^0_{1, T}$), this would suggest treatment effect heterogeneity with respect to Republican governance. Moreover, because the expansion-state region is much more Democratic than the non-expansion region, this heterogeneity could potentially drive differences between the ETC and the ETT.

Since this is a policy question of some interest, we directly investigate this by estimating the outcome model on the full data with treatment assignment interacted with each covariate \footnote{For this analysis we calculate separate covariate adjustments on the untreated data. The summary statistics for this adjustment are available in Appendix D.}. We then examine how the estimated treatment effect would change if we decreased the interaction between treatment assignment and each Republican governance indicator -- Republican governor, Republican lower legislature control, and Republican total control -- by 50 percentage points (the original variables are either 0 or 100 and are measured at the state level). This linear combination of coefficients estimates how the treatment effect would change for any given collection of states against a set that is identical except for being 50 percentage points lower, on average, across the Republican governance indicators. We find that the linear combination is positive (0.21 percentage points) and statistically significant at the 5 percent level on the unadjusted dataset. In contrast to our previous results, this would indicate that the estimated treatment effect may be larger among Republican governed areas. However, this finding is not robust to any other specification that we run. Ultimately we interpret these results as providing no evidence of treatment effect heterogeneity with respect to Republican governance. The full results are available in Appendix E, Table~\ref{tab:hte}.

\section{Discussion}
We estimate that had states that did not expand Medicaid in 2014 instead expanded their programs, they would have seen a -2.33 (-3.49, -1.16) percentage point change in the adult uninsurance rate. Existing estimates place the ETT between -3 and -6 percentage points. These estimates vary depending on the targeted sub-population of interest, the data used, the level of modeling (individuals or regions), and the modeling approach (see, e.g., \cite{courtemanche2017early}, \cite{kaestner2017effects}, \cite{frean2017premium}). We find that our estimate of the ETC is closer to zero than these ETT estimates. This difference may be a function of these different modeling strategies, or it may suggest that the ETC is smaller in absolute magnitude than the ETC. Regardless, due to the potential for effect heterogeneity, we emphasize the importance of directly estimating the targeted counterfactual of interest (e.g. the ETT or ETC), and being explicit about the assumptions used to estimate these quantities. We now consider our methodological contributions, study limitations, and we conclude by considering the policy implications of these findings.

\subsection{Methodological considerations}

Our study makes several methodological contributions to the literature on synthetic controls and balancing weights. First, we clarify some of the assumptions required to extend the synthetic controls literature to estimate the treatment effect on the controls. The key challenge is that we need to predict treatment response rather than the outcome absent treatment. We argue that we cannot use pre-treatment outcomes to conduct variable selection or optimally determine relative covariate importance without strong assumptions about the relationship between the counterfactual outcome models. In brief, estimating the ETC is an arguably more difficult problem because it requires a priori understanding of which covariates likely predict treatment response. We emphasize there may exist covariates that are not strong confounders of the outcome absent treatment, but that are important confounders of the outcome under treatment. Using pre-treatment outcomes to conduct variable selection or determine relative variable importance can therefore result in sub-optimal treatment effect estimates. As an example, we consider the role of Republican governance in our application: \cite{kaestner2017effects} and \cite{courtemanche2017early} do not balance on these factors when generating their synthetic control weights in their estimates of the effect of Medicaid expansion on uninsurance rates among treated states. By contrast we show that failing to control for this factor in our models leads to substantially larger treatment effect estimates, indicating their confounding role in our counterfactual outcome model.\footnote{We caution that in actuality these covariates may also be confounders of the outcome absent treatment; we do not directly investigate this.} While perhaps obvious, these points do not seem to have yet been appreciated in the applied literature. For example, \cite{born2020lockdowns} recently used synthetic controls to estimate Sweden's COVID cases and deaths had they instituted a lockdown. The authors balance on pre-treatment infections, urbanization rate, and population size, and argue that the treatment is effectively random conditional on these covariates. While this assumption may be true, the authors then use the standard synthetic controls algorithm to generate their weights, using pre-treatment outcomes $Y^0$ to determine the best model of the post-treatment counterfactua of $Y^1$. We argue that these types of procedures may be sub-optimal in this setting, and for our application use our prior knowledge about the problem to prioritize covariate balance instead.

Second, our estimation procedure introduces and illustrates the H-SBW objective, which can improve upon the SBW objective when using hierarchical data. Assuming the errors in the outcome model follow the covariance structure posited by \cite{kloek1981ols}, H-SBW produces a lower variance estimator by more evenly dispersing weights across states. The assumption underlying the particular structure of our objective is that our model errors have constant variance and constant within-state correlation $\rho$. However, our procedure requires assuming the covariance structure and $\rho$ in advance. We choose $\rho = 1/6$ for this application; however, it would be interesting to identify a data-driven approach to choose this tuning parameter (or perhaps for the covariance structure in general). 

Third, our estimation procedure accounts for measurement error in our covariates. We modify the constraint set to balance on a linear approximation to the true covariate values by adapting regression-calibration techniques (\cite{gleser1992importance}) to the balancing weights context. In Table~\ref{tab:balcomp} we show that the weights calculated on the unadjusted dataset fail to achieve the desired level of covariate balance on the adjusted dataset. Specifically, we find that the weighted pre-treatment outcomes may be lower than we wanted, which we speculate may bias our treatment effect downward. When we compared our estimates using the adjusted covariates to the unadjusted covariates, we find that our point estimates decrease (although often only slightly) in absolute magnitude. Essentially, when we generate weights on the unadjusted data to estimate the 2014 counterfactual outcome, they are likely fitting to noise. This causes the observed level of balance to appear better than it truly is. Meanwhile, the re-weighted region may suffer from regression to the mean in the post-treatment period, making our treatment effect estimates appear larger in absolute magnitude than the truth. Once we adjust for the measurement error, our point estimates decrease in absolute magnitude (see also \cite{daw2018matching}, who discuss this phenomenon in more detail in the context of difference-in-differences designs). Overall, our study provides a roadmap for future studies that may wish to correct for potential measurement error while using balancing weights. 

One direction for further work is to calibrate this procedure to determine an optimal bias-variance tradeoff with respect to the measurement error. It is possible that the procedure we implemented was sub-optimal with respect to the mean-square error of our estimator. In particular, the bias induced by the measurement error decreases with square root of the sample size used to calculate each CPUMAs covariate values, the minimum of which were over three hundred. Meanwhile, the variance of our counterfactual estimate should decrease with the square root of the number of treated states (of which there are 21). From a theoretical perspective, the variance is of a larger order than the bias; moreover, adjusting for the bias will further increase the variance of the estimator. These concerns are consistent with our observed results: we find that the change in our point estimates from the unadjusted data to the adjusted data are of smaller absolute magnitude than our variance estimate on our point estimate on the unadjusted data. Moreover, once we adjust for the measurement error, our confidence intervals increase more widely than the point estimates change.

\subsection{Limitations}

Our study is not without methodological limitations. We first caution that we required strong modeling assumptions throughout. In particular, we require SUTVA, no anticipatory treatment effects, no unmeasured confounding conditional on the true covariates, and several parametric assumptions about both the outcome and measurement error models. We were able to address some concerns about possible violations of these assumptions. For example, our results were qualitatively similar whether we excluded possible ``early expansion states,'' or used different weighting strategies (including relaxing the positivity restrictions and changing the tuning parameter $\rho$). We also examined two versions of our covariate adjustment and found similar results with either. However, we do not attempt to address concerns about SUTVA violations, particularly the impact of spillovers across regions. And while we believe that no unmeasured confounding is reasonable for this problem, we did not conduct any sensitivity analyses with respect to this assumption.

A second limitation pertains to interpreting our results against the existing literature. As we noted above, prior studies differ with respect to the data used, the targeted population of interest, the modeling choices, and unit level of analysis. To attempt to make our study comparable with existing work, we follow the covariates, study period, and data used most closely by \cite{courtemanche2017early}, who calculate an ETT estimate of -3.1 percentage points. However, we note two key differences between our studies. First, \cite{courtemanche2017early} modeled individual-level data, while we model CPUMA-level aggregates. Second, we exclude several states from our expansion pool while \cite{courtemanche2017early} include all states and DC in their analysis. As a result we do not make any formal statistical claims about the differences between our estimates, or between our estimates and any other specific paper, as they could also be a function of any of these other differences in our study. Relatedly, we caution against committing the ``ecological fallacy:'' specifically, we cannot directly infer individual-level behavior from the ecological correlations in our study without much stronger assumptions (see, e.g., \cite{subramanian2009revisiting}).

\subsection{Policy considerations}

We find that our point estimates for the ETC are  somewhat smaller in absolute magnitude than existing estimates of the ETT. While we make no formal statistical claims about these differences, this finding nevertheless highlights the importance of caution when using estimates of the ETT to make inferences about the ETC. Because almost every outcome of interest is mediated through increasing the number of insured individuals, if the ETC is in fact different than the ETT, then projecting findings from an estimate of the ETT to the ETC may lead to inaccurate inference. For example, \cite{miller2019medicaid} study the effect of Medicaid expansion on mortality. Using their estimate of the ETT they project that had all states expanded Medicaid, 15,600 deaths would have been avoided during their study's time-period. If we believe that this number increases monotonically with the number of uninsured individuals, this estimate may be an overestimate if the ETC is less than the ETT, or an underestimate if the ETC is greater than the ETT. Directly estimating the ETC can therefore also help us better model policy relevant downstream effects mediated through decreasing the uninsurance rate. 

Medicaid expansion is still an ongoing policy debate in the United States. Following the passage of the American Rescue Plan, state legislatures in Wyoming, Alabama, and North Carolina are reportedly considering expanding their programs. Our study estimates the effect of Medicaid expansion on adult uninsurance rates; however, this effect is only interesting because Medicaid enrollment is not automatic for eligible individuals. Different state policies may therefore make it easier or harder to enroll in Medicaid. We again emphasize that if the goal of Medicaid expansion is to increase insurance access for low-income adults, state policy-makers also make wish to make it easier to enroll in Medicaid. 

\section{Conclusion}

This is the first study we are aware of that directly estimates the foregone coverage expansions of Medicaid expansion on states that did not expand Medicaid in 2014. Our estimation approach contributes to the methodological literature on synthetic controls by outlining a set of identifying assumptions to estimate the ETC rather than the ETT, and to the balancing weights literature by using an estimation procedure that account for hierarchical data structure and covariates measured with error. We estimate that had states that did not expand their Medicaid eligibility requirements in 2014 done so, they would have seen a -2.33 (-3.49, -1.16) percentage point change in their uninsurance rate. This point estimate is closer to zero than existing estimates of the ETT, which range between -3 and -6 percentage points (\cite{frean2017premium}). From a practical standpoint, we caution against using using existing estimates of the ETT to make inferences about the ETC. From a policy standpoint, if the goal of Medicaid expansion is to increase access to insurance for low-income adults, state and federal policy-makers may wish to consider policies that make Medicaid enrollment easier if not automatic in addition to eligibility changes.

\section*{Acknowledgements}

The authors gratefully acknowledge invaluable advice and comments from Zachary Branson, Dave Choi, Edward Kennedy, Brian Kovak, Akshaya Jha, Lowell Taylor, and Jose Zubizaretta.

\begin{supplement}
Analysis programs and supporting materials are available online at github.com/mrubinst757/ medicaid-expansion
\end{supplement}

\bibliographystyle{imsart-nameyear} % Style BST file
\bibliography{research.bib}       % Bibliography file (usually '*.bib')

\begin{appendix}

\section{Proofs}

We show three results regarding the bias of the SBW estimator under the classical errors-in-variables model. First, we show that the bias of the SBW estimator that sets $\delta = 0$ (i.e. exactly balances the means of the covariates) is equal to the bias of the OLS estimator. Second, we show that if the observed covariate values can be replaced by their conditional expectations $\eta$ given the noisy observations, then the SBW estimator will be unbiased. Finally, we consider the case where we can estimate $\eta$ using auxillary data to estimate the covariance matrix of the error terms and show that the SBW estimator is consistent if we replace $\eta$ by an estimate $\hat{\eta}$ in the constraint set. In contrast to our application, we take the perspective throughout that $X$ is random, noting that this estimator still has desirable properties even when we view $X$ is fixed (see, e.g., \cite{gleser1992importance}).

Consider a dataset that consists of $i = 1, ..., n$ units where we observe the outcomes $Y_i$ and a treatment assignment indicator $A_i \in \{0, 1\}$. Let $n_a$ be the number of units in each treatment group, i.e, $n_a = \sum_{i = 1}^n \mathbbm{1}(A_i = a)$. Let $X_i \in \mathbb{R}^q$ be a vector of covariate values where $X_i \mid A_i = a \stackrel{iid}\sim MVN(\upsilon_a, \Sigma_{XX})$. 

Our target parameter is $\psi = \mathbb{E}\{Y_i^1 \mid A_i = 0\}$.\footnote{Note that we switch to the super-population target because we view $X$ as random.} We assume unconfoundedness ($Y_i^a \perp A_i \mid X_i$), consistency ($Y_i^a = Y_i \mid A_i = a$), and that $\mathbb{E}\{Y_i \mid X_i, A_i = a\} = \alpha_a + X_i^T\beta_a$. For simplicity we also assume throughout that there exist positive weights $\gamma$ that exactly balance the covariates, i.e. $\sum_{i: A_i = 1}\gamma_iX_{i, r} = \bar{X}_{0, r}$ for all $r = 1, ..., q$ and $\sum_{i: A_i = 1}\gamma_i = 1$, $\gamma_i > 0$. 

We begin by establishing the following identity:

\begin{equation}
\psi = \mu_y + (\upsilon_0 - \upsilon_1)^T\beta_1
\end{equation}

where $\mu_y = \mathbb{E}\{Y_i \mid A_i = 1\}$ and $\upsilon_a = \mathbb{E}\{X_i \mid A_i = a\}$.

\begin{proof}

Using our causal and modeling assumptions we have that:

\begin{align*}
\mathbb{E}\{Y_i^1 \mid X_i, A_i = 0\} &= \mathbb{E}\{Y_i^1 \mid X_i, A_i = 1\} \\
&= \mathbb{E}\{Y_i \mid X_i, A_i = 1\} \\
&= \alpha_1 + X_i^T\beta_1 \\
&= \mu_y + (X_i - \upsilon_1)^T\beta \\
&\implies \psi = \mu_y + (\upsilon_0 - \upsilon_1)^T\beta_1
\end{align*}

where the first equality follows from unconfoundedness, the second equality from consistency, the third from our parametric modeling assumptions, and the fourth by definition of $\alpha$. The final implication comes from plugging $\upsilon_0$ in place of $X_i$.

\end{proof}

We now outline the classical errors-in-variables model. Let $Y_i(X_i, a) = \alpha_a + X_i^T\beta_a + \epsilon_i$ where $\epsilon_i \stackrel{iid}\sim N(0, \sigma^2)$. We consider the case where we observe $W_i$, a vector of mean-unbiased proxies for the true (unobserved) covariate vector $X_i$; i.e., $W_i = X_i + v_i$, where $v_i \stackrel{iid}\sim MVN(0, \Sigma_{vv})$. Consider the model:

\begin{equation}
(\epsilon_i, v_i) \sim MVN((0, 0), \begin{pmatrix} 
\sigma^2 & 0 \\ 
0 & \Sigma_{vv}  
\end{pmatrix}
\end{equation}

In other words the error in the outcome model is uncorrelated with the error in the covariates. Let $u_i = (\epsilon_i, v_i)$. We further assume that the covariates are uncorrelated with any of the error terms:

\begin{equation}
(X_i, u_i) \mid A_i = a \sim MVN((\upsilon_a, 0), \begin{pmatrix} 
\Sigma_{XX} & 0 \\ 
0 & \Sigma_{uu}  
\end{pmatrix}
\end{equation}

Then we see that $(X_i, W_i) \mid A_i = a \stackrel{iid}{\sim} MVN((\upsilon_a, \upsilon_a), \Sigma)$\footnote{In contrast to our application, we instead assume here that $\Sigma_{WW \mid A = 1} = \Sigma_{WW \mid A = 0} = \Sigma_{WW}$ and $\Sigma_{XX \mid A = 1} = \Sigma_{XX \mid A = 0} = \Sigma_{XX}$. This is not a necessary assumption, but helps simplify notation.} where 

$$
\Sigma = \begin{pmatrix} 
\Sigma_{XX} & \Sigma_{XX} \\ 
\Sigma_{XX} & \Sigma_{WW}  
\end{pmatrix}
$$ 

and $\Sigma_{WW} = \Sigma_{XX} + \Sigma_{vv}$. Let $\kappa = \Sigma_{WW}^{-1}\Sigma_{XX}$. By the normality of the joint distribution of $X_i$ and $W_i$, we also know that

\begin{equation}
\mathbb{E}\{X_i \mid W_i, A_i = a\} = \upsilon_a + \kappa^T(W_i - \upsilon_a)
\end{equation}

We consider estimating $\psi$ in this setting. Let 

$$
\hat{\psi}^{reg} = \bar{Y}_1 + (\bar{W}_0 - \bar{W}_1)^T\hat{\beta}_1
$$ 

where $\hat{\beta}_1$ is the OLS estimator of $\beta_1$, $\bar{W}_a = n_a^{-1}\sum_{i:A_i = a} W_i$ and $\bar{Y}_a$ is defined analogously. Let 

$$
\hat{\psi}^{sbw} = \sum_{i: A_i = a} \gamma_i Y_i
$$ 

be the SBW estimator, where

\begin{align*}
\gamma &= \arg\min_{\theta} \sum_{i: A_i = 1}\theta_i^2 \text{ such that } W_{1, r}^T\theta = \bar{W}_{0, r} \ \ \ (r = 1, ..., q) \\
&\theta_i > 0, \sum_{i: A_i = 1}\theta_i = 1
\end{align*}

where $W_{a, r}$ is the $n_a$ dimensional (column) vector associated with covariate $r$.

\begin{proposition}\label{cl1}
The bias of $\hat{\psi}^{reg}$ is equal to the bias of $\hat{\psi}^{sbw}$; specifically, 

$$
\mathbb{E}\{\psi^{reg} - \psi\} = \mathbb{E}\{\psi^{sbw} - \psi\} = (\upsilon_0 - \upsilon_1)^T(\kappa - I_d)\beta
$$
\end{proposition}

\begin{proof}
Recall that $\mathbb{E}(\hat{\beta}_1) = \kappa\beta_1$ (c.l. \cite{gleser1992importance}). Consider the error of $\hat{\psi}^{reg}$: 

\begin{align*}
    \hat{\psi}^{reg} - \psi &= \bar{Y}_1 + (\bar{W}_0 - \bar{W}_1)^T\hat{\beta}_1 - (\mu_y + (\upsilon_0 - \upsilon_1)^T\beta_1) \\
    &= \underbrace{(\bar{Y}_1 - \mu_y)}_{T_1} + \underbrace{(\bar{W}_0 - \upsilon_0)^T\hat{\beta}_1}_{T_2} - \underbrace{(\bar{W}_1 - \upsilon_1)^T\hat{\beta}_1}_{T_3} + \underbrace{(\upsilon_0 - \upsilon_1)^T(\hat{\beta}_1 - \beta_1)}_{T_4} \\
    \implies \mathbb{E}\{\hat{\psi}^{reg} - \psi\} &= (\upsilon_0 - \upsilon_1)^T(\kappa - I_d)\beta_1
\end{align*}

The first equality holds by definition and the second by rearranging terms. The second equality consists of four terms. $T_1$ is simply the estimation error from a sample average, which has expectation zero. $T_2$ is the product of the estimation error from a sample average and $\hat{\beta}$; this also has expectation zero because $\bar{X}_0$ is estimated on a different part of the sample than $\hat{\beta}$, so these errors are independent. $T_3$ also has expectation zero because $\bar{W}_1^T\hat{\beta}_1 = \bar{Y}_1$, which has expectation $\mu_y$, and $\upsilon_1^T\hat{\beta}_1$ also has expectation $\mu_y$. We are then left with $T_4$; we substitute $\mathbb{E}{\hat{\beta}} = \kappa\beta$ to get the final result. 

We now derive the bias of $\hat{\psi}^{sbw}$:

\begin{align*}
    \hat{\psi}^{sbw} - \psi &= \sum_{i: A_i = 1}\gamma_iY_i - (\alpha_1 + \upsilon_0^T\beta_1) \\
    &= \sum_{i: A_i = 1} \gamma_i(\alpha_1 + X_i^T\beta_1 + \epsilon_i) - (\alpha_1 + \bar{W}_0^T\beta_1 + (\upsilon_0 - \bar{W}_0)^T\beta_1) \\
    &= \sum_{i: A_i = 1} (\gamma_i(W_i - v_i)^T\beta_1 + \gamma_i\epsilon_i) - \bar{W}_0^T\beta_1 + (\upsilon_0 - \bar{W}_0)^T\beta_1 \\
    &= \underbrace{-\sum_{i: A_i = 1}\gamma_iv_i^T\beta_1}_{T_1} + \underbrace{\sum_{i: A_i = 1}\gamma_i\epsilon_i}_{T_2}  + \underbrace{(\upsilon_0 - \bar{W}_0)^T\beta_1}_{T_3}
\end{align*}

Conditioning on $W_i$, we can take expectations over $X_i$, and see that $T_2$ has expectation zero (noting that the weights, conditional on $W_i$, are independent of these errors). $T_3$ is simply the scaled sum of mean zero estimation error and therefore has expectation zero. We conclude by considering $T_1$ and again take expectations over $X_i$ conditional on $W_i$: 

\begin{align*}
    \sum_{i: A_i = 1} \gamma_i\mathbb{E}\{X_i - W_i \mid W_i\}^T\beta_1 &= \sum_{i: A_i = 1} \gamma_i (\upsilon_1 + \kappa^T(W_i - \upsilon_1))^T\beta_1 - \sum_{i: A_i = 1}\gamma_i W_i^T\beta_1 \\
    &= (\upsilon_1 + \kappa^T(\bar{W}_0 - \upsilon_1))^T\beta_1 - \bar{W}_0^T\beta_1 \\
    &= (\kappa^T(\bar{W}_0 - \upsilon_1))^T\beta_1 - (\bar{W}_0 - \upsilon_1)^T\beta_1  \\
    &= (\bar{W}_0 - \upsilon_1)^T(\kappa - I_d)\beta_1 \\
    &= (\upsilon_0 - \upsilon_1)^T(\kappa - I_d)\beta_1 + (\bar{W}_0 - \upsilon_0)^T(\kappa - I_d)\beta_1 \\
    \implies \mathbb{E}\{\hat{\psi}^{sbw} - \psi\} &= (\upsilon_0 - \upsilon_1)^T(\kappa - I_d)\beta_1
\end{align*}

The final line holds because in the second to last line, we can take expectation over $W_i$ and see that the second term is a scaled sum of mean zero estimation error and has expectation zero. 
\end{proof}

Let $\hat{\psi}^{sbw}(\eta)$ be the SBW estimator that reweights $\eta_1(W_i)$ rather than $W_i$; i.e. $\hat{\psi}^{sbw}(\eta) = \sum_{i: A_i = 1}\gamma_i^\star Y_i$ where 

\begin{align*}
\gamma^\star = &\arg\min_{\theta} \sum_{i: A_i = 1}\theta_i^2 \text{ such that } \eta_1(W_{1, r})^T\theta = \upsilon_{0, r} \ \ \ (r = 1, ..., q) \\
&\theta_i > 0, \sum_{i: A_i = 1} \theta_i = 1
\end{align*}

\begin{proposition}
The estimator $\hat{\psi}^{sbw}(\eta)$ is unbiased, i.e.
$\mathbb{E}\{\hat{\psi}^{sbw}(\eta)\} = \psi$
\end{proposition}

\begin{proof}

By linearity we know that

\begin{align*}
Y_i = \alpha_1 + \eta_1(W_i)^T\beta_1 + (X_i - \eta_1(W_i))^T\beta_1
\end{align*}

We then have that:

\begin{align*}
    \hat{\psi}^{sbw}(\eta) - \psi &= \sum_{i: A_i = 1}\gamma_i^\star Y_i - (\alpha_1 + \upsilon_0^T\beta_1) \\
    &= \sum_{i: A_i = 1}\gamma_i^\star\alpha_1 + \sum_{i: A_i = 1}\gamma_i^\star\eta_1(W_i)^T\beta_1 + \sum_{i: A_i = 1}\gamma_i^\star(X_i - \eta_1(W_i))^T\beta_1 - (\alpha + \upsilon_0^T\beta_1) \\
    &= \sum_{i: A_i = 1}\gamma_i^\star(X_i - \eta_1(W_i))^T\beta_1
\end{align*}

Conditional on $W$, the weights are fixed and $X_i - \eta_1(W_i)$ has expectation zero; therefore, the estimator is unbiased.

\end{proof}
\begin{remark}

While we have assumed no model error, this estimator still has variance, conditional on $W_i$, equal to

$$
\sum_{i: A_i = 1} \gamma_i^{\star^2}\beta^T Cov(X_i \mid W_i)\beta
$$

where, assuming that $(X_i, W_i)$ are jointly normally distributed, $Cov(X_i \mid W_i) = \Sigma_{XX} - \Sigma_{XX}\Sigma_{WW}^{-1}\Sigma_{XX}$. Therefore, the variance of this estimator is higher than if we knew the true $X_i$ unless $\Sigma_{WW} = \Sigma_{XX}$ (i.e. we observe $X_i$). 

\end{remark}

\begin{remark}
Assuming there is a model error simply leads to the additional term $\sum_{i: A_i = 1}\gamma_i\epsilon_i$. This again has expectation zero, because the weights remain independent of the error in the outcome model, and adds a term to the total variance (conditional on $W_i$) equal to

$$
\sigma^2\sum_{i: A_i = 1}\gamma_i^{\star^2}
$$
\end{remark}

\begin{remark}
If we view both $W$ and $X$ as fixed and substitute $\bar{X}_a$ for $\upsilon_a$ (again assuming this quantity is known), the remaining term contributes to a finite-sample bias rather than variance. Moreover, the squared-bias term (equivalent to the MSE when $\sigma^2 = 0$) will decrease with the variance of the weights.
\end{remark}

Proposition 2 assumes that we know $\eta_a$ and $\upsilon_0$; however, in practice we estimate it from the data. Moreover, the estimation of $\hat{\eta}_1$ typically involves both the observed dataset and auxillary data, which we have not yet specified here. We now consider a simple version of this case.

Recall that $\eta_a(W_i) = \upsilon_a + \kappa^T(W_i - \upsilon_a)$, where $\kappa = \Sigma_{WW}^{-1}\Sigma_{XX}$. We can easily estimate $\upsilon_a$ consistently using $\bar{W}_a$; the challenge is estimating $\kappa$. 

Following \cite{gleser1992importance}, consider the setting where we observe $T$ independent vectors of measurements $W_i^\star$ from known values $X_i^\star$, so that $v_i^\star = W_i^\star - X_i^\star$. Assume that $\Sigma_{v^\star v^\star} = \Sigma_{vv}$. We can then estimate $\hat{\Sigma}_{vv} = \frac{1}{T}\sum_{i=1}^T(W_i^\star - X_i^\star)'(W_i^\star - X_i^\star)$. We can then estimate

$$
\hat{\kappa} = (n\hat{\Sigma}_{WW})^{-1}(n(\hat{\Sigma}_{WW} - \hat{\Sigma}_{vv}))
$$

assuming $n(\hat{\Sigma}_{WW} - \hat{\Sigma}_{vv})$ is positive semi-definite, and where $\hat{\Sigma}_{WW} = \frac{1}{n}\sum_{i=1}^n (W_i - \bar{W})(W_i - \bar{W})'$. 

\begin{proposition}
Let $\tilde{\gamma}$ be the weights that solve the SBW objective with $\sum_{i: A_i = 1}\tilde{\gamma}_i\hat{\eta}_1(W_{i, r}) = \bar{W}_{0, r}$ for all $r = 1, ..., q$. The estimator $\sum_{i: A_i = 1}\tilde{\gamma}_iY_i \to \psi$ as $n \to \infty$, $T \to \infty$.
\end{proposition}

\begin{proof}

Rewriting the error expression from Lemma A.2 leads to the additional terms $\sum_{i: A_i = 1}\tilde{\gamma}_i(\eta_1(W_i) - \hat{\eta_1}(W_i))^T\beta_1$ and $(\upsilon_0 - \bar{W}_0)^T\beta_1$. The second term is scaled estimation error and has expectation zero. It therefore suffices to show that 

$$
\sum_{i: A_i = 1}\tilde{\gamma}_i(\eta_1(W_i) - \hat{\eta_1}(W_i))^T\beta_1 \to 0
$$

By the weak law of large numbers, $\hat{\Sigma}_{WW} - \hat{\Sigma}_{vv} \to \Sigma_{XX}$ as $n \to \infty$, $T \to \infty$; similarly $\hat{\Sigma}_{WW} \to \Sigma_{WW}$ as $n \to \infty$. By the continuous mapping theorem $\hat{\kappa} \to \kappa$. Since $\bar{W}_0 \to \upsilon_0$, we have that $\hat{\eta} \to \eta$. 

Let $\gamma^\star$ be the weights defined in Lemma A.3. We can then rewrite the error term above as

\begin{align*}
\sum_{i: A_i = 1}(\tilde{\gamma}_i - \gamma_i^\star)(\hat{\eta}_1(W_i) - \eta(W_i))^T\beta_1 + \sum_{i: A_i = 1}\gamma_i^\star(\hat{\eta}_1(W_i) - \eta_1(W_i))^T\beta_1
\end{align*}

As $n \to \infty$, $T \to \infty$, $\hat{\eta}_1 \to \eta_1$, and therefore by definition of the SBW objective, $\tilde{\gamma}_i \to \gamma_i^\star$. These two terms therefore both converge in probability to zero. 

\end{proof}

Our estimation of $\eta_a$ follows a somewhat different procedure which we outline in Appendix B.

\subsection{H-SBW Objective}

We show that the H-SBW estimator minimizes the variance of any estimator within the constraint set given a constant within-group correlation structure. Here we assume that the true covariates are measured. Consider the outcome model:

\begin{equation}
    Y_{sc}(X_{sc}, a) = \alpha_a + X_{sc}^T\beta_a + c_s + \epsilon_{sc}
\end{equation}

where $\mathbb{E}\{c_s \mid X_{sc}, A_{sc}\} = \mathbb{E}\{\epsilon_{sc} \mid X_{sc}, A_{sc}\} = \mathbb{E}\{c_s\epsilon_{sc} \mid X_{sc}, A_{sc}\} = 0$, and $Var(\epsilon_{sc}) = \sigma^2$ and $Var(c_s) = \eta^2$. Let $\rho = \frac{\eta^2}{\sigma^2 + \eta^2}$.

The conditional variance of any weighting estimator $\sum_{sc: A_{sc} = 1}\gamma_{sc}Y_{sc}$ using weights $\gamma$ where $\gamma^T1 = n_t$ is then equal to:

\begin{align}
    Var(n_t^{-1}\sum_{sc: A_s = 1}\gamma_{sc}Y_{sc} \mid X_{sc}, A_{sc}) &= n_t^{-2}\sum_{s: A_s = 1}^{m_1}(\sum_{c = 1}^{p_s}\gamma_{sc}^2(\sigma^2 + \eta^2) + \sum_{c \ne d}\gamma_{sc}\gamma_{sd}\eta^2) \\
    &\propto \sum_{s: A_s = 1}(\sum_{c = 1}^{p_s}(\gamma_{sc}^2 + \sum_{c \ne d}\rho \gamma_{sc}\gamma_{sd})
\end{align}

where the second line follows by dividing by $\eta^2 + \sigma^2$. This suggests the H-SBW objective, which minimizes the (conditional) variance of this estimator for known $\rho$.


\section{Adjustment Details}

We now discuss how we estimate $\eta_a$ in our data. We begin by estimating the unit-level covariance matrices $\Sigma_{vv, sc}$, the sampling variability for each CPUMA by using the individual replicate survey weights to generate $b = 80$ additional CPUMA-level datasets. We then estimate $\Sigma_{vv, sc}$:

\begin{equation}
\hat{\Sigma}_{vv, sc} = \frac{4}{80}\sum_{b=1}^{80}(W_{sc}^B - \bar{W}_{sc})(W_{b, sc} - \bar{W}_{sc})'
\end{equation}

where the $4$ in the numerator comes from the process used to generate the replicate survey weights and $\bar{W}_{sc}$ is the value calculated in the original dataset. 

Unlike in our proofs above, we do not assume that $\Sigma_{XX \mid A = a}$ or $\Sigma_{vv \mid A = a}$ are equal for $A_i \in \{0, 1\}$. Instead we let $\hat{\Sigma}_{WW \mid A = a} = n_a^{-1}\sum_{s, c: A_s = a} (W_{sc} - \bar{W})(W_{sc} - \bar{W})^T$. This estimator is calculated on the original observed dataset. We then estimate $\Sigma_{XX \mid A = a}$ using:

\begin{equation}
\hat{\Sigma}_{XX \mid A = a} = \hat{\Sigma}_{WW \mid A = a} - n_a^{-1}\sum_{s, c: A_s = a} \hat{\Sigma}_{vv, sc}
\end{equation}

Define

\begin{equation}
\hat{\kappa}_a = \hat{\Sigma}_{WW \mid A = a}^{-1}\hat{\Sigma}_{XX \mid A = a}
\end{equation}

Notice that $\hat{\kappa}$ is a matrix of estimated coefficients of a linear regressions of the (unobserved) matrix $X_{sc}$ on (observed) matrix $W_{sc}$. We can then estimate $X_{sc}$ as

\begin{equation}
\hat{\eta}_a(W_{sc}) = \bar{W}_a + \hat{\kappa}_a^T(W_{sc} - \bar{W}_a)
\end{equation}

where $\bar{W}_a = n_a^{-1}\sum_{s, c: A_s = a} W_{sc}$. This approximately aligns with the adjustments suggested by \cite{carroll2006measurement} and \cite{gleser1992importance}. 

In our setting we have additional access to information about a substantial source of heterogeneity in the measurement error: in particular, regions with large populations are estimated quite precisely, while regions with small populations are estimated much less precisely. This is because the survey is a one-percent sample across all regions. Moreover, for a given CPUMA, some covariates are measured using three years of data, and others only one. However, using the conventional regression calibration approach will adjust precisely estimated covariates to imprecisely estimated covariates in a similar way. 

Our preferred estimators therefore use an alternative approach where we model an individual-level $\Sigma_{vv, sc}$ as a function of the sample sizes used to estimate each covariate. In particular, let $s_{sc}$ be the q-dimensional (row) vector of the sample sizes used to estimate each covariate value for a given CPUMA. Let $\matr{S}_{sc} = \sqrt{s_{sc}}\sqrt{s_{sc}}^T$. We assume that $\sqrt{s_{sc}}v_{sc}^T \sim N(0, \Sigma_{vv}^a)$. We then know that $\Sigma_{vv, sc}^m = \Sigma_{vv}^a \oslash \matr{S}_{sc}$. We add the superscript $m$ to distinguish that is a modeled covariance matrix.

To estimate these matrices, we pool our initial estimates of the CPUMA-level covariance matrices ($\hat{\Sigma}_{vv, sc}$) to generate $\hat{\Sigma}_{vv}^a = n_a^{-1}\sum_{s, c: A_s = a} \matr{S}_{sc} \circ \Sigma_{vv, sc}$. We estimate $\hat{\Sigma}_{vv, sc}^m = \hat{\Sigma}_{vv}^a \oslash \matr{S}_{sc}$. From this we estimate $\hat{\Sigma}_{XX \mid A = a} = \hat{\Sigma}_{WW \mid A = a} - n_a^{-1}\sum_{s, c: A_s = a}\hat{\Sigma}_{vv, sc}^a$. Finally, we calculate $\hat{\kappa}_{sc} = (\hat{\Sigma}_{XX \mid A = a} + \hat{\Sigma}_{vv, sc}^m)^{-1}\hat{\Sigma}_{XX \mid A = a}$, which we use to estimate $\hat{\eta}_a(W_{sc})$. 

We prefer this adjustment because it accounts for CPUMA-level variability in the measurement error. Specifically, this adjustment should more greatly affect outlying values of imprecisely estimated covariates, while leaving precisely estimated covariates closer to their observed value in the dataset. Moreover, we are able to use the full efficiency of using all units in the modeling, so our CPUMA-level $\hat{\kappa}$ estimates are using the full data. The disadvantage is that, as shown in Appendix~\ref{sec:appendixsumstat}, this adjustment, compared to the conventional approach, appears more likely to lead to extreme values that fall outside of the support of the original data. A second cost is that this aggregation models all differences as a function of the sample sizes, and averages over any potential heterogeneity due to heteroskedasticity (i.e. the measurement error covariance matrix changes depending on the true value of $X_{sc}$). In practice we find that either adjustment yields quite similar results (see Appendix~\ref{ssec:allresults}).



\section{Summary Statistics}
\label{sec:appendixsumstat}

Table~\ref{tab:summarytab1} and Table~\ref{tab:summarytab2} display univariate summary statistics from the primary dataset. We display the mean, interquartile range, and the range (as defined by the maximum value minus the minimum value) for the unadjusted dataset (sigma\_zero), the preferred adjustment (sigma\_uu\_i), and the secondary adjustment (sigma\_uu\_avg). We see that in general the covariate adjustments reduce the variability in the data. However, we also see that the range sometimes increases drastically, particularly for the untreated dataset. Table~\ref{tab:extreme1} displays the overall number of CPUMAs whose adjusted values fall outside of the range of the unadjusted data. Note that for our ETT estimates, we only use the adjustments on the treatment data, while we use both for the OATE estimates. 
We also calculate these adjustments when leaving out each state one at a time from the primary dataset to calculate our secondary variance estimates; these results are available on request. 

\begin{table}[ht]
\centering
    \caption{Univariate summary statistics (1)}
    \label{tab:summarytab1}
\begin{tabular}{rllll}
  \hline
Treatment & Variable & sigma\_zero & sigma\_uu\_i & sigma\_uu\_avg \\ 
  \hline
0 & age\_cat2\_19\_29\_pct & (24.8, 5.5, 48.7) & (24.8, 5.4, 47.5) & (24.8, 5.4, 47.8) \\ 
  1 & age\_cat2\_19\_29\_pct & (24.5, 6, 30.9) & (24.5, 5.9, 29) & (24.5, 5.9, 29) \\ 
  0 & age\_cat2\_30\_39\_pct & (20.6, 2.6, 18.4) & (20.6, 2.5, 17.3) & (20.6, 2.5, 17.2) \\ 
  1 & age\_cat2\_30\_39\_pct & (20.9, 3.4, 20.9) & (20.9, 3.1, 19.1) & (20.9, 3.2, 19.4) \\ 
  0 & age\_cat2\_40\_49\_pct & (22, 2.5, 19.5) & (22, 2.4, 33.3) & (22, 2.3, 18.5) \\ 
  1 & age\_cat2\_40\_49\_pct & (22.2, 2.5, 15.4) & (22.2, 2.3, 13.8) & (22.2, 2.3, 13.8) \\ 
  0 & avg\_adult\_hh\_ratio & (139.7, 18.7, 156.8) & (139.7, 18.9, 153.7) & (139.7, 18.9, 154.2) \\ 
  1 & avg\_adult\_hh\_ratio & (150.8, 27.1, 174.3) & (150.8, 27, 173.8) & (150.8, 27.1, 173.3) \\ 
  0 & citizenship\_pct & (93.6, 5.9, 39.8) & (93.6, 5.9, 39.7) & (93.6, 5.9, 39.3) \\ 
  1 & citizenship\_pct & (90, 11.8, 57.1) & (90, 11.8, 55.4) & (90, 11.7, 55.7) \\ 
  0 & disability\_pct & (11.9, 5, 22.1) & (11.9, 4.8, 21.1) & (11.9, 4.8, 20.7) \\ 
  1 & disability\_pct & (10.5, 5.3, 28.6) & (10.4, 5.3, 26.7) & (10.5, 5.4, 27.2) \\ 
  0 & educ\_hs\_degree\_pct & (29.6, 8.8, 42.7) & (29.6, 8.8, 40.7) & (29.6, 8.8, 41.2) \\ 
  1 & educ\_hs\_degree\_pct & (26.3, 10.6, 43.2) & (26.3, 10.5, 41.8) & (26.3, 10.6, 41.9) \\ 
  0 & educ\_less\_than\_hs\_pct & (11.8, 7.1, 30) & (11.8, 6.9, 30.2) & (11.8, 7, 30.2) \\ 
  1 & educ\_less\_than\_hs\_pct & (11.4, 7.6, 45.3) & (11.4, 7.4, 45) & (11.4, 7.3, 44.6) \\ 
  0 & educ\_some\_college\_pct & (33.8, 5.6, 42.9) & (33.8, 5.1, 45.7) & (33.8, 5.1, 42.3) \\ 
  1 & educ\_some\_college\_pct & (33.5, 7.8, 34.2) & (33.5, 7.4, 32.8) & (33.5, 7.4, 32.9) \\ 
  0 & female\_pct & (50.5, 1.6, 15.8) & (50.5, 1.5, 14.9) & (50.5, 1.5, 15.1) \\ 
  1 & female\_pct & (50.1, 1.6, 15.4) & (50.1, 1.4, 14.2) & (50.1, 1.5, 14.2) \\ 
  0 & foreign\_born\_pct & (10.5, 8.8, 63.2) & (10.5, 8.9, 62.9) & (10.5, 8.9, 62.7) \\ 
  1 & foreign\_born\_pct & (18, 22.2, 76) & (18, 22.2, 75.2) & (18, 22.2, 75.4) \\ 
  0 & hins\_unins\_pct\_2011 & (22.7, 10.5, 51.4) & (22.3, 10, 757.8) & (22.7, 9.2, 50.1) \\ 
  1 & hins\_unins\_pct\_2011 & (19.6, 11.2, 59) & (19.6, 10.8, 51.8) & (19.6, 10.8, 52.4) \\ 
  0 & hins\_unins\_pct\_2012 & (22.4, 9.3, 49.5) & (22.8, 9.4, 468.8) & (22.4, 9.2, 47.8) \\ 
  1 & hins\_unins\_pct\_2012 & (19.4, 9.8, 50.6) & (19.4, 10, 49.7) & (19.4, 10.1, 50.1) \\ 
  0 & hins\_unins\_pct\_2013 & (22, 10.1, 50.1) & (22.3, 9.4, 323.6) & (22, 9.3, 49.2) \\ 
  1 & hins\_unins\_pct\_2013 & (19, 11.1, 49.9) & (19, 10.2, 48.1) & (19, 10.4, 48.6) \\ 
  0 & hispanic\_pct & (11.4, 10.2, 93.9) & (11.5, 10.2, 93.8) & (11.4, 10.2, 93.8) \\ 
  1 & hispanic\_pct & (15.8, 17.7, 97.2) & (15.8, 17.7, 97) & (15.8, 17.7, 97) \\ 
  0 & inc\_pov2\_138\_pct & (22, 8.8, 42.8) & (22, 8.8, 41) & (22, 8.7, 41.8) \\ 
  1 & inc\_pov2\_138\_pct & (19.9, 11.9, 45.6) & (19.9, 11.7, 44.8) & (19.9, 11.7, 43.9) \\ 
  0 & inc\_pov2\_139\_299\_pct & (27.4, 5.1, 27.9) & (27.2, 4.9, 206.1) & (27.4, 4.8, 26.7) \\ 
  1 & inc\_pov2\_139\_299\_pct & (24.9, 8.4, 34.2) & (24.9, 7.8, 34.2) & (24.9, 7.8, 34.1) \\ 
  0 & inc\_pov2\_300\_499\_pct & (24.2, 4.8, 21) & (24.4, 4.8, 317.6) & (24.2, 4.5, 18.4) \\ 
  1 & inc\_pov2\_300\_499\_pct & (23.6, 5.5, 23) & (23.6, 4.9, 22.2) & (23.6, 4.8, 22.2) \\ 
  0 & inc\_pov2\_500\_plus\_pct & (23.7, 9.9, 51.9) & (23.6, 9.8, 118.8) & (23.7, 9.7, 52.1) \\ 
  1 & inc\_pov2\_500\_plus\_pct & (29.3, 18.4, 69.1) & (29.3, 18.5, 68) & (29.3, 18.4, 68) \\ 
  0 & married\_pct & (51.5, 9.1, 59.7) & (51.6, 8.9, 75.5) & (51.5, 9.1, 59.5) \\ 
  1 & married\_pct & (50.7, 9.4, 45.1) & (50.8, 9, 44.2) & (50.7, 9.1, 44.3) \\ 
  0 & missing\_children\_pct & (13.7, 7.4, 36.2) & (13.8, 7.3, 36.2) & (13.7, 7.3, 35.6) \\ 
  1 & missing\_children\_pct & (10.5, 6.6, 41) & (10.5, 6.6, 40.8) & (10.5, 6.6, 40.5) \\ 
  0 & one\_child\_pct & (10.4, 2.5, 13) & (10.3, 2.1, 159) & (10.4, 2.1, 12.3) \\ 
  1 & one\_child\_pct & (11.1, 3.1, 14.3) & (11.1, 2.8, 12.3) & (11.1, 2.9, 12.5) \\ 
  0 & pop\_growth & (100.3, 1.7, 14.3) & (100.1, 1.2, 267.3) & (100.3, 1, 5) \\ 
  1 & pop\_growth & (100.3, 1.9, 13.7) & (100.3, 1.2, 6.2) & (100.3, 1.2, 6.5) \\ 
  0 & race\_white\_pct & (77.8, 22.2, 90.5) & (77.8, 22.1, 90.2) & (77.8, 22.1, 90.4) \\ 
  1 & race\_white\_pct & (73.9, 25.4, 91.9) & (73.9, 25.5, 91.8) & (73.9, 25.5, 91.8) \\ 
  0 & repub\_gov & (95.9, 0, 100) & (95.9, 0, 100) & (95.9, 0, 100) \\ 
  1 & repub\_gov & (30.9, 100, 100) & (30.9, 100, 100) & (30.9, 100, 100) \\
  0 & repub\_lower\_control & (98.8, 0, 100) & (98.8, 0, 100) & (98.8, 0, 100) \\ 
  1 & repub\_lower\_control & (23.9, 0, 100) & (23.9, 0, 100) & (23.9, 0, 100) \\ 
  0 & repub\_total\_control & (94.7, 0, 100) & (94.7, 0, 100) & (94.7, 0, 100) \\ 
  1 & repub\_total\_control & (21, 0, 100) & (21, 0, 100) & (21, 0, 100) \\ 
   \hline
\end{tabular}
\end{table}

\begin{table}[ht]
\centering
    \caption{Univariate summary statistics (2)}
    \label{tab:summarytab2}
\begin{tabular}{rllll}
  \hline
Treat & Variable & sigma\_zero & sigma\_uu\_i & sigma\_uu\_avg \\ 
  \hline
  0 & student\_pct & (11.5, 3.9, 41) & (11.5, 3.9, 53.8) & (11.5, 3.8, 41.3) \\ 
  1 & student\_pct & (11.7, 3.4, 29.5) & (11.7, 3.5, 28.1) & (11.7, 3.5, 28) \\ 
  0 & three\_plus\_child\_pct & (5.2, 2.2, 25.3) & (5.2, 2.1, 42.9) & (5.2, 2.2, 23.2) \\ 
  1 & three\_plus\_child\_pct & (5.2, 2, 14.1) & (5.2, 1.7, 13.5) & (5.2, 1.7, 13.3) \\ 
  0 & two\_child\_pct & (8.9, 2.8, 15) & (9, 2.5, 122.8) & (8.9, 2.5, 15.1) \\ 
  1 & two\_child\_pct & (9.7, 3.5, 15) & (9.7, 3.2, 13.4) & (9.7, 3.2, 13.6) \\ 
  0 & unemployed\_pct\_2011 & (9.4, 4.6, 21.7) & (9.4, 4, 23.2) & (9.4, 4, 18.9) \\ 
  1 & unemployed\_pct\_2011 & (10.2, 4.6, 25.5) & (10.2, 3.9, 23.9) & (10.2, 4, 22.6) \\ 
  0 & unemployed\_pct\_2012 & (8.8, 4.2, 22.4) & (8.6, 3.6, 180.8) & (8.8, 3.6, 18.6) \\ 
  1 & unemployed\_pct\_2012 & (9.4, 4.5, 28.3) & (9.4, 4.3, 23.6) & (9.4, 4.3, 23.6) \\ 
  0 & unemployed\_pct\_2013 & (8, 3.7, 19) & (8.1, 3.4, 146.7) & (8, 3.4, 17.6) \\ 
  1 & unemployed\_pct\_2013 & (8.4, 3.7, 23.4) & (8.4, 3.5, 20.2) & (8.4, 3.5, 20.6) \\ 
  0 & urban\_pct & (0.7, 0.4, 0.9) & (0.7, 0.4, 0.9) & (0.7, 0.4, 0.9) \\ 
  1 & urban\_pct & (0.8, 0.3, 0.9) & (0.8, 0.3, 0.9) & (0.8, 0.3, 0.9) \\ 
  \hline
\end{tabular}
\end{table}

Table~\ref{tab:extreme1} displays the frequency that the adjusted covariates fell outside of the support of the unadjusted dataset for the (control, treatment) groups. We see that that frequency is higher for our preferred adjustment (sigma\_uu\_i) than the secondary adjustment (sigma\_uu\_avg), though the counts are low in either case. The counts are also slightly higher in the control group than the treated group; this may be a function of the smaller number of CPUMAs for the control versus the treated group $(n_0 = 414; n_1 = 515)$. 

\begin{table}[ht]
\centering
    \caption{Extreme covariate adjustments (control, treated)}
    \label{tab:extreme1}
\begin{tabular}{lll}
  \hline
Variables & sigma\_uu\_i & sigma\_uu\_avg \\ 
  \hline
age\_cat2\_19\_29\_pct & (0, 0) & (0, 0) \\ 
  age\_cat2\_30\_39\_pct & (0, 0) & (0, 0) \\ 
  age\_cat2\_40\_49\_pct & (2, 0) & (0, 0) \\ 
  avg\_adult\_hh\_ratio & (0, 0) & (0, 0) \\ 
  citizenship\_pct & (2, 0) & (2, 0) \\ 
  disability\_pct & (1, 2) & (0, 2) \\ 
  educ\_hs\_degree\_pct & (0, 0) & (0, 0) \\ 
  educ\_less\_than\_hs\_pct & (1, 1) & (2, 1) \\ 
  educ\_some\_college\_pct & (1, 0) & (0, 0) \\ 
  female\_pct & (0, 0) & (0, 0) \\ 
  foreign\_born\_pct & (0, 1) & (0, 1) \\ 
  hins\_unins\_pct\_2011 & (4, 0) & (0, 0) \\ 
  hins\_unins\_pct\_2012 & (3, 0) & (0, 1) \\ 
  hins\_unins\_pct\_2013 & (4, 0) & (1, 1) \\ 
  hispanic\_pct & (1, 0) & (1, 0) \\ 
  inc\_pov2\_138\_pct & (2, 1) & (2, 1) \\ 
  inc\_pov2\_139\_299\_pct & (4, 1) & (1, 1) \\ 
  inc\_pov2\_300\_499\_pct & (3, 0) & (0, 0) \\ 
  inc\_pov2\_500\_plus\_pct & (4, 0) & (2, 0) \\ 
  married\_pct & (2, 0) & (1, 0) \\ 
  missing\_children\_pct & (2, 1) & (1, 1) \\ 
  one\_child\_pct & (4, 0) & (0, 0) \\ 
  pop\_growth & (3, 0) & (0, 0) \\ 
  race\_white\_pct & (1, 1) & (1, 1) \\ 
  student\_pct & (2, 0) & (1, 0) \\ 
  three\_plus\_child\_pct & (3, 2) & (2, 1) \\ 
  two\_child\_pct & (4, 0) & (1, 1) \\ 
  unemployed\_pct\_2011 & (1, 0) & (0, 0) \\ 
  unemployed\_pct\_2012 & (2, 0) & (0, 0) \\ 
  unemployed\_pct\_2013 & (3, 0) & (0, 0) \\ 
   \hline
\end{tabular}
\end{table}

Figure~\ref{fig:corrmatrix} displays the Pearson's correlation coefficients for the bivariate relationships between the covariates on the unadjusted dataset. The relationships are quite similar when using the adjusted datasets.

\begin{figure}[]
\begin{center}
    \caption{Correlation matrix (unadjusted covariates)}
    \label{fig:corrmatrix}
    \includegraphics[scale=0.6]{01_Plots/correlation-plot-c1-sigma-zero.png}
\end{center}
\end{figure}


\section{Weight Diagnostics}
\label{ssec:balancetables}

Table~\ref{tab:baltab1} displays the differences between the weighted mean covariate values of the expansion region and the mean of the non-expansion region for our primary dataset and with the early expansion states excluded (calculated using our covariate adjustments). The weights presented here are from the H-SBW estimator. The values under each column of ``Primary'' and ``Early Excluded'' are in the following format: (unweighted difference, weighted difference). Additional results are available on request.

\begin{table}[ht]
\centering
    \caption{Balance Table}
    \label{tab:baltab1}
\begin{tabular}{lll}
  \hline
Variables & Preferred & Early Excluded \\ 
  \hline
age\_cat2\_19\_29\_pct & (-0.35, -0.2) & (-0.62, -0.07) \\ 
  age\_cat2\_30\_39\_pct & (0.33, -0.16) & (-0.11, -0.04) \\ 
  age\_cat2\_40\_49\_pct & (0.22, -0.41) & (0.04, -0.50) \\ 
  avg\_adult\_hh\_ratio & (11.09, -0.40) & (3.18, -0.18) \\ 
  citizenship\_pct & (-3.56, -1.93) & (-0.20, -1.80) \\ 
  disability\_pct & (-1.46, 0.57) & (-0.21, 0.73) \\ 
  educ\_hs\_degree\_pct & (-3.35, 0.99) & (-1.05, 1.14) \\ 
  educ\_less\_than\_hs\_pct & (-0.40, 1.09) & (-1.27, 1.27) \\ 
  educ\_some\_college\_pct & (-0.30, 0.69) & (0.44, 0.53) \\ 
  female\_pct & (-0.34, -0.87) & (-0.25, -1.00) \\ 
  foreign\_born\_pct & (7.51, 1.99) & (0.97, 1.99) \\ 
  hins\_unins\_pct\_2011 & (-2.64, 0.48) & (-3.09, 0.45) \\ 
  hins\_unins\_pct\_2012 & (-3.37, -0.42) & (-3.9, -0.42) \\ 
  hins\_unins\_pct\_2013 & (-3.27, -0.32) & (-3.77, -0.32) \\ 
  hispanic\_pct & (4.36, 1.00) & (-1.44, 1.00) \\ 
  inc\_pov2\_138\_pct & (-2.08, 0.93) & (-1.36, 0.60) \\ 
  inc\_pov2\_139\_299\_pct & (-2.34, 0.91) & (-1.45, 0.86) \\ 
  inc\_pov2\_300\_499\_pct & (-0.8, -0.39) & (0.05, -0.34) \\ 
  inc\_pov2\_500\_plus\_pct & (5.69, -1.91) & (3.06, -1.91) \\ 
  married\_pct & (-0.80, -0.38) & (-0.34, -0.60) \\ 
  missing\_children\_pct & (-3.21, -0.36) & (-1.93, -0.08) \\ 
  one\_child\_pct & (0.80, -0.22) & (0.21, -0.25) \\ 
  pop\_growth & (0.11, -0.10) & (-0.08, -0.15) \\ 
  race\_white\_pct & (-3.87, 1.00) & (0.31, 1.00) \\ 
  repub\_gov & (-65.02, -22.06) & (-55.02, -21.97) \\ 
  repub\_lower\_control & (-74.91, -22.23) & (-57.24, -20.66) \\ 
  repub\_total\_control & (-73.72, -25.00) & (-58.2, -25) \\ 
  student\_pct & (0.29, -0.41) & (0.12, -0.38) \\ 
  three\_plus\_child\_pct & (0.02, -0.14) & (-0.17, -0.15) \\ 
  two\_child\_pct & (0.66, -0.47) & (0.08, -0.55) \\ 
  unemployed\_pct\_2011 & (0.80, 0.15) & (0.68, 0.15) \\ 
  unemployed\_pct\_2012 & (0.75, 0) & (0.58, 0.00) \\ 
  unemployed\_pct\_2013 & (0.29, -0.25) & (0.10, -0.25) \\ 
  urban\_pct & (0.08, -0.07) & (0.03, -0.06) \\ 
   \hline
\end{tabular}
\end{table}

Table~\ref{tab:oatedist1} displays the difference in means from the overlap region from the control and treated regions on the primary dataset and with the early expansion states excluded. The numbers are displayed as (absolute distance from control region, absolute distance from treated region). These distances are computed using our adjusted datasets. 

\begin{table}[ht]
\centering
    \caption{Overlap region distance from control, treatment regions (1)}
    \label{tab:oatedist1}
\begin{tabular}{lll}
  \hline
Variable & Preferred & Early expansion excluded \\ 
  \hline
age\_cat2\_19\_29\_pct & (-1.39, -1.03) & (-1.34, -0.71) \\ 
  age\_cat2\_30\_39\_pct & (-0.79, -1.12) & (-0.82, -0.72) \\ 
  age\_cat2\_40\_49\_pct & (0.24, 0.02) & (0.2, 0.16) \\ 
  avg\_adult\_hh\_ratio & (-3.72, -14.81) & (-4.07, -7.25) \\ 
  citizenship\_pct & (1.7, 5.26) & (1.97, 2.17) \\ 
  disability\_pct & (0.29, 1.75) & (0.38, 0.59) \\ 
  educ\_hs\_degree\_pct & (1.14, 4.49) & (1.22, 2.27) \\ 
  educ\_less\_than\_hs\_pct & (-1.67, -1.27) & (-1.58, -0.31) \\ 
  educ\_some\_college\_pct & (0.47, 0.78) & (0.75, 0.31) \\ 
  female\_pct & (-0.11, 0.24) & (-0.16, 0.09) \\ 
  foreign\_born\_pct & (-2.23, -9.73) & (-2.8, -3.77) \\ 
  hins\_unins\_pct\_2011 & (-2.89, -0.24) & (-2.89, 0.19) \\ 
  hins\_unins\_pct\_2012 & (-3.68, -0.31) & (-3.68, 0.22) \\ 
  hins\_unins\_pct\_2013 & (-3.82, -0.55) & (-3.84, -0.07) \\ 
  hispanic\_pct & (-4.48, -8.84) & (-4.57, -3.13) \\ 
  inc\_pov2\_138\_pct & (-1.44, 0.64) & (-1.31, 0.05) \\ 
  inc\_pov2\_139\_299\_pct & (-0.18, 2.16) & (0, 1.45) \\ 
  inc\_pov2\_300\_499\_pct & (0.88, 1.68) & (0.96, 0.91) \\ 
  inc\_pov2\_500\_plus\_pct & (0.98, -4.71) & (0.58, -2.49) \\ 
  married\_pct & (1.6, 2.4) & (1.84, 2.18) \\ 
  missing\_children\_pct & (-0.72, 2.5) & (-0.62, 1.31) \\ 
  one\_child\_pct & (-0.27, -1.07) & (-0.32, -0.53) \\ 
  pop\_growth & (-0.27, -0.39) & (-0.29, -0.2) \\ 
  race\_white\_pct & (4.48, 8.35) & (5.15, 4.84) \\ 
  repub\_gov & (-11.88, 53.14) & (-11.94, 43.07) \\ 
  repub\_lower\_control & (-5.7, 69.21) & (-4.31, 52.92) \\ 
  repub\_total\_control & (-17.58, 56.14) & (-16.26, 41.94) \\ 
  student\_pct & (-0.5, -0.79) & (-0.5, -0.62) \\ 
  three\_plus\_child\_pct & (-0.17, -0.19) & (-0.1, 0.07) \\ 
  two\_child\_pct & (-0.15, -0.81) & (-0.16, -0.24) \\ 
  unemployed\_pct\_2011 & (0.09, -0.71) & (0.08, -0.6) \\ 
  unemployed\_pct\_2012 & (0.05, -0.69) & (0.02, -0.55) \\ 
  unemployed\_pct\_2013 & (-0.42, -0.71) & (-0.44, -0.54) \\ 
  urban\_pct & (-0.05, -0.13) & (-0.05, -0.08) \\ 
   \hline
\end{tabular}
\end{table}

Table ~\ref{tab:oatestateweightsc1} displays the sum of CPUMA-level weights within each state for the OATE region, using the preferred covariate adjustment (``sigma\_uu\_i''), as well as no adjustment (``sigma\_zero'') and the secondary adjustment (``sigma\_uu\_avg''). We display all states where the total sum of weights within states is greater than one for any of the covariate sets. The total sum of weights for each set is standardized to sum to 100 for each set of weights. Table ~\ref{tab:oatestateweightsc2} displays the same results with the early expansion states excluded.

\begin{table}[ht]
\centering
\caption{OATE weights summed by state by covariate adjustment, primary dataset}
\label{tab:oatestateweightsc1}
\begin{tabular}{lrrrr}
  \hline
State & Treatment & sigma\_uu\_i & sigma\_uu\_avg & sigma\_zero \\ 
  \hline
OH & 1 & 34.41 & 34.51 & 36.75 \\ 
  MI & 1 & 26.76 & 27.27 & 27.97 \\ 
  AR & 1 & 15.99 & 15.87 & 12.28 \\ 
  PA & 0 & 15.51 & 16.94 & 18.95 \\ 
  MO & 0 & 15.49 & 15.27 & 11.66 \\ 
  WI & 0 & 13.37 & 13.39 & 13.66 \\ 
  FL & 0 & 11.40 & 12.15 & 11.22 \\ 
  AZ & 1 & 9.00 & 8.40 & 8.96 \\ 
  ME & 0 & 6.90 & 6.42 & 6.17 \\ 
  TN & 0 & 6.30 & 6.83 & 5.77 \\ 
  IN & 0 & 4.67 & 4.45 & 5.56 \\ 
  IA & 1 & 4.62 & 4.78 & 5.73 \\ 
  NC & 0 & 4.50 & 4.12 & 4.69 \\ 
  NJ & 1 & 4.16 & 4.14 & 4.57 \\ 
  NE & 0 & 3.62 & 4.39 & 3.84 \\ 
  KS & 0 & 3.29 & 2.39 & 2.82 \\ 
  VA & 0 & 3.03 & 2.18 & 1.79 \\ 
  AL & 0 & 2.73 & 2.20 & 2.31 \\ 
  UT & 0 & 2.34 & 1.83 & 1.82 \\ 
  ND & 1 & 2.32 & 2.76 & 2.13 \\ 
  TX & 0 & 2.27 & 1.91 & 2.85 \\ 
  NV & 1 & 1.52 & 0.75 & 0.71 \\ 
  NM & 1 & 1.22 & 1.54 & 0.90 \\ 
  OK & 0 & 1.21 & 1.52 & 0.81 \\ 
  AK & 0 & 0.77 & 1.37 & 1.55 \\ 
  \hline
\end{tabular}
\end{table}

\begin{table}[ht]
\centering
\caption{OATE weights summed by state by covariate adjustment, early expansion excluded}
\label{tab:oatestateweightsc2}
\begin{tabular}{lrrrr}
  \hline
State & Treatment & sigma\_uu\_i & sigma\_uu\_avg & sigma\_zero \\ 
  \hline
OH & 1 & 34.66 & 34.81 & 36.92 \\ 
  MI & 1 & 27.65 & 28.05 & 28.38 \\ 
  AR & 1 & 16.05 & 16.08 & 12.57 \\ 
  MO & 0 & 15.61 & 15.43 & 11.89 \\ 
  PA & 0 & 14.70 & 15.87 & 17.58 \\ 
  WI & 0 & 14.58 & 14.43 & 14.52 \\ 
  FL & 0 & 10.78 & 11.28 & 10.51 \\ 
  AZ & 1 & 9.06 & 8.50 & 8.92 \\ 
  TN & 0 & 6.50 & 7.31 & 5.85 \\ 
  ME & 0 & 5.52 & 5.06 & 4.92 \\ 
  IA & 1 & 4.78 & 5.02 & 6.15 \\ 
  IN & 0 & 4.67 & 4.41 & 5.97 \\ 
  NC & 0 & 4.63 & 4.25 & 4.93 \\ 
  NE & 0 & 4.03 & 4.83 & 4.33 \\ 
  KS & 0 & 3.82 & 2.88 & 3.08 \\ 
  NV & 1 & 3.69 & 2.74 & 2.97 \\ 
  AL & 0 & 3.15 & 2.55 & 2.52 \\ 
  UT & 0 & 2.73 & 2.14 & 2.15 \\ 
  TX & 0 & 2.56 & 2.28 & 3.17 \\ 
  ND & 1 & 2.28 & 2.48 & 2.14 \\ 
  VA & 0 & 2.01 & 1.40 & 1.46 \\ 
  NM & 1 & 1.84 & 2.32 & 1.95 \\ 
  OK & 0 & 1.29 & 1.70 & 0.82 \\ 
  AK & 0 & 0.75 & 1.30 & 1.35 \\ 
  SC & 0 & 0.57 & 0.54 & 1.10 \\ 
  MT & 0 & 0.44 & 0.64 & 0.68 \\ 
  WY & 0 & 0.39 & 0.26 & 0.65 \\ 
  MS & 0 & 0.38 & 0.49 & 0.79 \\ 
  LA & 0 & 0.32 & 0.36 & 0.43 \\ 
  GA & 0 & 0.28 & 0.30 & 0.70 \\ 
  ID & 0 & 0.17 & 0.11 & 0.25 \\ 
  SD & 0 & 0.11 & 0.19 & 0.33 \\ 
  HI & 1 & 0.00 & 0.00 & 0.00 \\ 
  MD & 1 & 0.00 & 0.00 & 0.00 \\ 
  KY & 1 & 0.00 & 0.00 & 0.00 \\ 
  IL & 1 & 0.00 & 0.00 & 0.00 \\ 
  RI & 1 & 0.00 & 0.00 & 0.00 \\ 
  NH & 1 & 0.00 & 0.00 & 0.00 \\ 
  CO & 1 & 0.00 & 0.00 & 0.00 \\ 
  WV & 1 & 0.00 & 0.00 & 0.00 \\ 
  OR & 1 & 0.00 & 0.00 & 0.00 \\ 
   \hline
\end{tabular}
\end{table}


\section{Additional Results}\label{app:allresults}

Table~\ref{tab:pretxpredfull} presents the full model validation results that also include estimators using the Oaxaca-Blinder OLS and GLS weights (see, e.g, \cite{kline2011oaxaca}), with the rows ordered by RMSE of the prediction errors. We find in general that these estimators perform quite poorly, again showing that the cost of extrapolation is quite high in our application. We see some particularly extreme numbers for the error using the heterogeneous adjustment in 2012: one reason for this may be because in practice that the heterogeneous adjustment is more likely to impute values that fall outside of the support of the data. In some cases we have found that these imputations are quite extreme. The estimators that extrapolate from the data are then more likely to fit to these values, leading to poor estimator performance overall. This result makes the heterogeneous adjustment potentially less desirable, particularly for estimators that allow extrapolation.

\begin{table}[ht]
\caption{Estimator pre-treatment outcome prediction error}\label{tab:pretxpredfull}
\begin{tabular}{llrrr}
  \hline
Sigma estimate & Estimator & 2011 error & 2012 error & RMSE \\ 
  \hline
Homogeneous & SBW & -0.18 & -0.22 & 0.20 \\ 
  Homogeneous & H-SBW & -0.24 & -0.21 & 0.23 \\ 
  Heterogeneous & SBW & -0.25 & -0.30 & 0.27 \\ 
  Heterogeneous & H-SBW & -0.32 & -0.39 & 0.36 \\ 
  Homogeneous & BC-SBW & -0.42 & -0.35 & 0.39 \\ 
  Heterogeneous & BC-SBW & -0.45 & -0.39 & 0.42 \\ 
  None & SBW & -0.50 & -0.61 & 0.56 \\ 
  None & H-SBW & -0.52 & -0.61 & 0.57 \\ 
  Homogeneous & BC-HSBW & -0.53 & -0.62 & 0.58 \\ 
  Heterogeneous & BC-HSBW & -0.53 & -0.72 & 0.63 \\ 
  None & BC-SBW & -0.82 & -0.93 & 0.88 \\ 
  None & OLS & -0.91 & -0.84 & 0.88 \\ 
  None & GLS & -0.87 & -0.91 & 0.89 \\ 
  None & BC-HSBW & -0.93 & -0.99 & 0.96 \\ 
  Homogeneous & OLS & -1.75 & -1.21 & 1.50 \\ 
  Homogeneous & GLS & -1.76 & -1.45 & 1.61 \\ 
  Heterogeneous & GLS & -1.18 & -17.60 & 12.47 \\ 
  Heterogeneous & OLS & -0.85 & -22.90 & 16.21 \\ 
   \hline
\end{tabular}
\end{table}


Table~\ref{tab:confintmain} displays the point estimates and confidence intervals from all estimators, including estimates calculated on a second version of the adjusted data where we model a separate $\kappa$ for all values (the ``heterogeneous adjustment''). ``Homogeneous'' corresponds to our preferred covariate adjustment, and are the results presented in the main paper.

\begin{table}[h!]
\centering
\begin{threeparttable}
\caption{Point estimates and confidence intervals: primary dataset}
\label{tab:confintmain}
\begin{tabular}{llrl}
  \hline
Weight type & Adjustment & Psihat & 95\% CI \\ 
  \hline
H-SBW & Homogeneous & -2.33 & (-3.47, -1.19) \\ 
  H-SBW & Heterogeneous & -2.24 & (-3.4, -1.08) \\ 
  H-SBW & None & -2.34 & (-2.85, -1.82) \\ 
  BC-HSBW & Homogeneous & -2.05 & (-3.22, -0.87) \\ 
  BC-HSBW & Heterogeneous & -1.98 & (-3.13, -0.83) \\ 
  BC-HSBW & None & -2.22 & (-2.87, -1.56) \\ 
  SBW & Homogeneous & -2.35 & (-3.67, -1.03) \\ 
  SBW & Heterogeneous & -2.28 & (-3.5, -1.05) \\ 
  SBW & None & -2.39 & (-2.95, -1.83) \\ 
  BC-SBW & Homogeneous & -2.07 & (-3.07, -1.06) \\ 
  BC-SBW & Heterogeneous & -2.00 & (-3, -0.99) \\ 
  BC-SBW & None & -2.19 & (-2.9, -1.49) \\ 
   \hline
\end{tabular}
    \begin{tablenotes}
      \item[] Note: Confidence intervals are estimated using the leave-one-state-out jackknife and the standard normal quantiles.
    \end{tablenotes}
\end{threeparttable}
\end{table}

Table~\ref{tab:confintmainc2} presents the same results when excluding the early expansion states.

\begin{table}[h!]
\centering
\begin{threeparttable}
\caption{Point estimates and confidence intervals: early expansion excluded}
\label{tab:confintmainc2}
\begin{tabular}{llrl}
  \hline
Weight type & Adjustment & Psihat & 95\% CI \\ 
  \hline
H-SBW & Homogeneous & -2.09 & (-3.15, -1.03) \\ 
  H-SBW & Heterogeneous & -2.06 & (-3.26, -0.87) \\ 
  H-SBW & None & -2.28 & (-2.82, -1.74) \\ 
  BC-HSBW & Homogeneous & -1.94 & (-3.17, -0.72) \\ 
  BC-HSBW & Heterogeneous & -1.93 & (-3.42, -0.45) \\ 
  BC-HSBW & None & -2.22 & (-3.07, -1.38) \\ 
  SBW & Homogeneous & -2.05 & (-3.10, -1.00) \\ 
  SBW & Heterogeneous & -2.03 & (-3.24, -0.82) \\ 
  SBW & None & -2.21 & (-2.71, -1.72) \\ 
  BC-SBW & Homogeneous & -1.99 & (-3.22, -0.77) \\ 
  BC-SBW & Heterogeneous & -2.00 & (-3.52, -0.47) \\ 
  BC-SBW & None & -2.23 & (-3.05, -1.40) \\ 
   \hline
\end{tabular}
    \begin{tablenotes}
      \item[] Note: Confidence intervals are estimated using the leave-one-state-out jackknife and the standard normal quantiles.
    \end{tablenotes}
\end{threeparttable}
\end{table}

Figure~\ref{fig:loostateplot} displays the change in our estimator when removing each state for all four of our estimators on the adjusted dataset (``homogeneous'') and the unadjusted dataset (``none''). These results condition on the covariate adjustment, but are similar when recalculating the entire adjustment procedure. Additional results are available on request.

\begin{figure}[H]
\begin{center}
    \caption{Leave-one-state-out point estimates minus primary estimate}
    \label{fig:loostateplot}
    \includegraphics[scale=0.6]{01_Plots/loostate-sensitivityc1-proc-uu-avg.png}
\end{center}
\subcaption{Colors reflect the magnitude of the difference in the estimates when subtracting the original estimate from the estimate that excludes the specified state. The values in the right panel are on the unadjusted data.}
\end{figure}

\clearpage


\end{appendix}

\end{document}
