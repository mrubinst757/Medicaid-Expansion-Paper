We estimate the effect of Medicaid expansion on the adult uninsurance rate in states that did not expand Medicaid in 2014. Using data from the American Community Survey (ACS), we estimate this effect - the treatment effect on the controls (ETC) - by re-weighting expansion regions to approximately balance the covariates from non-expansion regions using an extension of the stable balancing weights objective function (\cite{zubizarreta2015stable}). We contribute to the balancing weights literature by accounting for hierarchical data structure and covariate measurement error when calculating our weights, and to the synthetic controls literature (see, e.g. \cite{abadie2010synthetic}) by outlining assumptions required to estimate the ETC using time-series cross-sectional data. We estimate that Medicaid expansion would have changed the uninsurance rate by -2.33 percentage points (-3.49, -1.16). These results are smaller in absolute magnitude than existing estimates of the treatment effect on the treated (ETT), though may not be directly comparable due to the study design, target population, and level of analysis. Regardless, we caution against making inferences about the ETC using estimates of the ETT, and emphasize the need to directly estimate these counterfactuals when they are the quantity of interest.

