In this section we provide an overview of our data source, the covariates, the outcome, and the treatment assignment.

\subsection{Data Source}

Our primary data source is the annual household and person public use microdata files from the American Community Survey (ACS) from 2011 through 2014. The ACS is an annual survey of approximately three million individuals across the United States. The public use microdata files include information on individuals in geographic areas greater than 65,000 people. The smallest geographic unit contained in these data are public-use microdata areas (PUMAs), arbitrary boundaries that nest within states but not within counties or other more commonly used geographic units. One limitation of these data is a 2012 change in the PUMA boundaries, which do not overlap well with the previous boundaries. As a result, the smallest possible geographic areas that nest both PUMA coding systems are known as consistent PUMAs (CPUMAs). The United States contains 1,075 total CPUMAs, with states ranging from having one CPUMA (South Dakota, Montana, and Idaho) to 123 CPUMAs (New York). Our primary dataset (discussed in Section~\ref{sssec:txassign} contained 929 CPUMAs among 46 states. The average total number of sampled individuals per CPUMA across the four years is 1,001; the minimum number of people sampled was 334 and the maximum is 23,990. Importantly, this survey is a repeated cross-section rather than a longitudinal dataset of individuals over time.

\subsection{Study period}

We begin our analysis in 2011 following \cite{courtemanche2017early}, who note that several other aspects of the ACA were implemented in 2010 -- including the provision allowing for dependent coverage until age 26 and the elimination of co-payments for preventative care -- and likely induced differential shocks across states. We also restrict our post-treatment period to 2014: several additional states expanded Medicaid in 2015, including Indiana, Michigan, and Pennsylvania. However, these states did not expand Medicaid contemporaneously with the 2014 ACA provisions. Without additional assumptions, this second-year expansion cannot help us estimate the effect of the 2014 expansion. 

\subsection{Covariates}

We use the underlying individual-level ACS survey data and accompanying survey weights to aggregate the data at the CPUMA level. We choose our covariates to approximately align with those considered in \cite{courtemanche2017early} and that are likely to be potential confounders. Because we are ultimately interested in calculating rates, these variables include both the numerator and denominator counts.

Using the ACS survey weights, we first estimate: the total non-elderly adult population for each year 2011-2014; the total labor force population (among non-elderly adults) for each year 2011-2013; and the total number of households averaged from 2011-2013. We also construct an average of the total non-elderly adult population from 2011-2013. These are our denominator variables. For our numerator counts, we estimate the total number of: females; whites; people of Hispanic ethnicity; people born outside of the United States; citizens; people with disabilities; married individuals; people with less than a high school education, high school degrees, some college, or college graduates or higher; people living under 138 percent of the FPL, between 139 and 299 percent, 300 and 499 percent, more than 500 percent, and who did not respond to the income survey question; people aged 19-29, 30-39, 40-49, 50-64; households with one, two, or three or more children, and households that did not respond about the number of children.\footnote{Number of children and income to poverty ratio were the only two variables with missing data in the underlying microdata.} We average these estimated counts using the ACS survey weights from 2011-2013. For each individual year from 2011-2013, we estimate the total number of people who were unemployed and uninsured at the time of the survey (calculated among all non-elderly adults and all non-elderly adults within the labor force, respectively). We divide the numerator counts by the corresponding denominator counts to estimate the percentage in each category. For the demographics, these include the average number of non-elderly adults from 2011-2013. For the time-varying variables, we use the corresponding year (where uninsurance rates are calculated as a fraction of the labor force rather than the non-elderly adult population). We also calculate the average non-elderly adult population growth and the average number of households to adults across 2011-2013. 

In addition to the ACS microdata, we use 2010 Census data to calculate the approximate percentage of people living within an ``urban'' area for each CPUMA. Finally, we include three state-level covariates reflecting the partisan composition of each state's government in 2013 using data from the National Conference of State Legislatures (NCLS). Specifically, we generate an indicator for states with a Republican governor, an indicator for states with Republican control over the lower legislative chamber, and an indicator for states with Republican control over both chambers of the legislature and the governorship.\footnote{Nebraska is the only state with a unicameral legislature; moreover, the legislature is technically non-partisan. We nevertheless classified them as having Republican control of the legislature.} 

\subsection{Outcome}

Our outcome of interest is the non-elderly adult uninsurance rate in 2014, which we denote using $Y$. While take-up among the Medicaid-eligible population is a more natural outcome, we choose the non-elderly adult uninsurance rate for two reasons, one theoretic and one practical. First, Medicaid eligibility in the post-period is likely endogenous: Medicaid expansion may affect an individual's income and poverty levels, which in general define Medicaid eligibility. A second reason is to align our study with others to compare our results with the existing literature, and this is the outcome that \cite{courtemanche2017early} use. One drawback of using this outcome is that the simultaneous adoption of other ACA provisions by all states in 2014 more clearly affects this rate in a way that a more targeted group might not be.

\subsection{Treatment assignment} \label{sssec:txassign}

While some states expanded Medicaid in 2014 and other states did not, assigning a binary treatment status simplifies a more complex reality. There are three reasons to be cautious about this simplification. First, states differed substantially in their Medicaid coverage policies prior to 2014: with perfect data we might consider Medicaid expansion as a continuous treatment with values proportional to the number of newly eligible individuals. The challenge, however, is correctly identifying newly eligible individuals in the data (see \cite{frean2017premium}, who attempt to address this). Second, \cite{frean2017premium} note that five states (California, Connecticut, Minnesota, New Jersey, and Washington) and the District of Columbia adopted partial limited Medicaid expansions prior to 2014. \footnote{\cite{kaestner2017effects} and \cite{courtemanche2017early} also consider Arizona, Colorado, Hawaii, Illinois, Iowa, Maryland, and Oregon to have had early expansions.} Lastly, timing is an issue: among the states that expanded Medicaid in 2014, Michigan's expansion did not go into effect until April 2014, while New Hampshire's expansion did not occur until September 2014.

Our primary analysis excludes New York, Vermont, Massachusetts, Delaware, and the District of Columbia from our pool of expansion regions, because these regions had comparable Medicaid coverage policies prior to 2014 (\cite{kaestner2017effects}). We also exclude New Hampshire because it did not expand Medicaid until September 2014. While Michigan expanded Medicaid in April 2014, we leave this state in our pool of treated states. We consider the remaining expansion states as ``treated'' and the non-expansion states as ``control'' states. We later consider the sensitivity of our results to these classifications by removing the early expansion states noted by \cite{frean2017premium}. Our final dataset contains aggregated statistics for all of the above variables for 925 CPUMAs in our non-expansion and our pool of expansion states. There are 414 CPUMAs among 24 non-expansion states and 511 CPUMAs among 21 expansion states. When we exclude the early expansion states for sensitivity analyses, we are left with 296 CPUMAs across 17 expansion states.