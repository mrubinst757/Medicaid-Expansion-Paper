% Template for the submission to:
%   The Annals of Applied Statistics    [AOAS]
%
%%%%%%%%%%%%%%%%%%%%%%%%%%%%%%%%%%%%%%%%%%%%%%
%% In this template, the places where you   %%
%% need to fill in your information are     %%
%% indicated by '???'.                      %%
%%                                          %%
%% Please do not use \input{...} to include %%
%% other tex files. Submit your LaTeX       %%
%% manuscript as one .tex document.         %%
%%%%%%%%%%%%%%%%%%%%%%%%%%%%%%%%%%%%%%%%%%%%%%

\documentclass[aoas]{imsart}

%% Packages
\RequirePackage{amsthm,amsmath,amsfonts,amssymb,centernot,float,import,makeidx,subfiles}
\RequirePackage{natbib}
%\RequirePackage[colorlinks,citecolor=blue,urlcolor=blue]{hyperref}
\RequirePackage{graphicx}% uncomment this for including figures
\usepackage{hyperref}
\usepackage{lscape}
\usepackage{subcaption}
\usepackage{threeparttable}
\usepackage[nokeyprefix]{refstyle}
\usepackage{varioref}
\startlocaldefs
%%%%%%%%%%%%%%%%%%%%%%%%%%%%%%%%%%%%%%%%%%%%%%
%%                                          %%
%% Uncomment next line to change            %%
%% the type of equation numbering           %%
%%                                          %%
%%%%%%%%%%%%%%%%%%%%%%%%%%%%%%%%%%%%%%%%%%%%%%
%\numberwithin{equation}{section}
%%%%%%%%%%%%%%%%%%%%%%%%%%%%%%%%%%%%%%%%%%%%%%
%%                                          %%
%% For Axiom, Claim, Corollary, Hypothezis, %%
%% Lemma, Theorem, Proposition              %%
%% use \theoremstyle{plain}                 %%
%%                                          %%
%%%%%%%%%%%%%%%%%%%%%%%%%%%%%%%%%%%%%%%%%%%%%%
%%%%%%%%%%%%%%%%%%%%%%%%%%%%%%%%%%%%%%%%%%%%%%
\theoremstyle{plain}
\newtheorem{axiom}{Axiom}
\newtheorem{claim}[axiom]{Claim}
\newtheorem{theorem}{Theorem}[section]
\newtheorem{lemma}[theorem]{Lemma}
\newtheorem{proposition}{Proposition}
\newcommand{\matr}[1]{\mathbf{#1}} % undergraduate algebra version
\newcommand{\mathbbm}[1]{\text{\usefont{U}{bbm}{m}{n}#1}} 
%%%%%%%%%%%%%%%%%%%%%%%%%%%%%%%%%%%%%%%%%%%%%%
%%                                          %%
%% For Assumption, Definition, Example,     %%
%% Notation, Property, Remark, Fact         %%
%% use \theoremstyle{remark}                %%
%%                                          %%
%%%%%%%%%%%%%%%%%%%%%%%%%%%%%%%%%%%%%%%%%%%%%%
\theoremstyle{remark}
\newtheorem{remark}{remark}
%%%%%%%%%%%%%%%%%%%%%%%%%%%%%%%%%%%%%%%%%%%%
%\theoremstyle{plain}
%\newtheorem{???}{???}
%\newtheorem*{???}{???}
%\newtheorem{???}{???}[???]
%\newtheorem{???}[???]{???}
%%%%%%%%%%%%%%%%%%%%%%%%%%%%%%%%%%%%%%%%%%%%%%
%%                                          %%
%% For Assumption, Definition, Example,     %%
%% Notation, Property, Remark, Fact         %%
%% use \theoremstyle{remark}                %%
%%                                          %%
%%%%%%%%%%%%%%%%%%%%%%%%%%%%%%%%%%%%%%%%%%%%%%
%\theoremstyle{remark}
%\newtheorem{???}{???}
%\newtheorem*{???}{???}
%\newtheorem{???}{???}[???]
%\newtheorem{???}[???]{???}
%%%%%%%%%%%%%%%%%%%%%%%%%%%%%%%%%%%%%%%%%%%%%%
%% Please put your definitions here:        %%
%%%%%%%%%%%%%%%%%%%%%%%%%%%%%%%%%%%%%%%%%%%%%%
\endlocaldefs

% reference external document
\makeatletter
\newcommand*{\addFileDependency}[1]{
  \typeout{(#1)}
  \@addtofilelist{#1}
  \IfFileExists{#1}{}{\typeout{No file #1.}}
}
\makeatother

\newcommand*{\myexternaldocument}[1]{
    \externaldocument{#1}
    \addFileDependency{#1.tex}
    \addFileDependency{#1.aux}
}
%%% END HELPER CODE

% put all the external documents here!

\begin{document}

\begin{frontmatter}
%%%%%%%%%%%%%%%%%%%%%%%%%%%%%%%%%%%%%%%%%%%%%%
%%                                          %%
%% Enter the title of your article here     %%
%%                                          %%
%%%%%%%%%%%%%%%%%%%%%%%%%%%%%%%%%%%%%%%%%%%%%%
\title{The Effect of Medicaid Expansion on Non-Elderly Adult Uninsurance Rates Among States that did not Expand Medicaid}
%\title{A sample article title with some additional note\thanksref{T1}}
\runtitle{Medicaid Expansion}
%\thankstext{T1}{A sample of additional note to the title.}

\begin{aug}
%%%%%%%%%%%%%%%%%%%%%%%%%%%%%%%%%%%%%%%%%%%%%%
%%Only one address is permitted per author. %%
%%Only division, organization and e-mail is %%
%%included in the address.                  %%
%%Additional information can be included in %%
%%the Acknowledgments section if necessary. %%
%%%%%%%%%%%%%%%%%%%%%%%%%%%%%%%%%%%%%%%%%%%%%%
\author[A]{\fnms{Max} \snm{Rubinstein}\ead[label=e1]{mrubinst@andrew.cmu.edu; amelia@andrew.cmu.edu}} and
\author[A]{\fnms{Amelia} \snm{Haviland}}
%%%%%%%%%%%%%%%%%%%%%%%%%%%%%%%%%%%%%%%%%%%%%%
%% Addresses                                %%
%%%%%%%%%%%%%%%%%%%%%%%%%%%%%%%%%%%%%%%%%%%%%%
\address[A]{Carnegie Mellon University, Heinz College and Department of Statistics and Data Science \printead{e1}}

\end{aug}

\begin{flushleft}
The literature on balancing weights often assumes that the covariates are measured without error and the popular Stable Balancing Weights objective (\cite{zubizarreta2015stable}) is known to produce the minimum variance estimator (conditional on the covariates) when the errors in the outcome model are independent and identically distributed. We consider an application where our covariates are measured with mean-zero random noise and our outcome model has errors that may be correlated within states. Our proposed approach follows the regression-calibration literature (see, e.g., \cite{gleser1992importance}) and requires access to auxillary data to obtain an estimate of the variability of the measurement error. Using this information, we adjust our observed data and use these adjusted covariates to estimate our weights. We also propose a modification to the SBW objective that improves the efficiency of the weights under an assuming the model errors follow some assumed correlation structure. We then apply this method to estimate the effect of Medicaid expansion on the adult uninsurance rate in states that did not expand Medicaid as if they had expanded Medicaid in 2014 using region-level data estimated using the American Communities Survey. We conclude that Medicaid expansion would have caused a -2.33 (-3.49, -1.16) percentage point change in the adult uninsurance rate among these states. 
\end{flushleft}


\begin{keyword}
\kwd{Balancing weights}
\kwd{synthetic controls}
\kwd{Medicaid expansion}
\kwd{measurement error}
\kwd{hierarchical data}
\kwd{regression to the mean}
\end{keyword}

\end{frontmatter}
%%%%%%%%%%%%%%%%%%%%%%%%%%%%%%%%%%%%%%%%%%%%%%
%% Please use \tableofcontents for articles %%
%% with 50 pages and more                   %%
%%%%%%%%%%%%%%%%%%%%%%%%%%%%%%%%%%%%%%%%%%%%%%
%\tableofcontents

%%%%%%%%%%%%%%%%%%%%%%%%%%%%%%%%%%%%%%%%%%%%%%
%%%% Main text entry area:

\section{Introduction}

The 2010 Affordable Care Act (ACA) required states to expand their Medicaid eligibility requirements by 2014 to offer coverage to all adults with incomes at or below 138 percent of the federal poverty level (FPL). The United States Supreme Court ruled this requirement unconstitutional in 2012, allowing states to decide whether to expand Medicaid coverage. In 2014, twenty-six states and the District of Columbia expanded their Medicaid programs. From 2015 through 2020 an additional twelve states elected to expand their Medicaid programs. More recently, Oklahoma and Missouri voted to expand their programs in July 2021 \footnote{https://www.kansascity.com/news/politics-government/article250170945.html}. Following the passage of the American Rescue Plan in March 2021, Republican state legislatures in other traditionally conservative states, including Alabama, North Carolina, and Wyoming are also reportedly considering expanding their programs.\footnote{https://www.nbcnews.com/politics/politics-news/changed-hearts-minds-biden-s-funding-offer-shifts-medicaid-expansion-n1262229} The effects of Medicaid expansion on various outcomes, including uninsurance rates, mortality rates, and emergency department use, have been widely studied, primarily by using the initial expansions in 2014 and 2015 to divide expansion states into ``treated'' states and non-expansion states as ``control'' states. 

Medicaid enrollment is not automatic, and Medicaid take-up rates have historically varied across states. This variation is partly a function of state discretion in administering programs: for example, program outreach, citizenship verification policies, and application processes differ across states (\cite{courtemanche2017early}). When states expanded their eligibility requirements, the number that actually enrolled in Medicaid afterwards is random among the eligible individuals. Understanding how Medicaid eligibility expansion actually affected the number of uninsured individuals is an important effect. Existing studies have estimated that Medicaid expansion reduced the uninsurance rate between three and six percentage points among states that expanded Medicaid. These estimates differed depending on the data used, specific target population, study design, and level of analysis (see, e.g., \cite{kaestner2017effects}, \cite{courtemanche2017early}, \cite{frean2017premium}). However, none of these studies have directly estimated what the treatment's effect would have been on the controls (ETC). 

We study the effect of 2014 Medicaid expansion on adult uninsurance rates among states that did not expand Medicaid, using approximate balancing weights to estimate this effect (\cite{wang2017minimal}). Approximate balancing weights are a popular estimation method in causal inference that grew out of the propensity score weighting literature. Rather than iteratively modeling the propensity score until the inverse probability weights achieve a desired level of balance (the so-called ``propensity score tautology'' \cite{imai2014covariate}), recent papers propose using optimization methods to generate weights that enforce covariate balance between the treated and control units (see, e.g., \cite{hainmueller2012entropy}, \cite{imai2014covariate}, \cite{zubizarreta2015stable}). From an applied perspective, there are at least four benefits of this approach: first, it does not require iterating propensity score models to generate satisfactory weights. Second, these methods do not use outcomes in the modeling stage, mitigating the risk of cherry-picking model specifications. Third, these methods can constrain the weights to prevent extrapolation from the data, reducing model dependence \cite{zubizarreta2015stable}. Finally, the estimates are more interpretable: by making the comparison group explicit, it is easy to communicate exactly which units contributed to the counterfactual estimate.

To date most proposed methods in the balancing weights literature requires the following two assumptions: (1) the covariates are measured without error, and (2) the observations are independent. Our approach relaxes these assumptions and provides a general method that can be used for other applications given sufficient auxillary data. In our application we estimate our covariates using annual American Community Survey (ACS) microdata aggregated to the consistent public use microdata area (CPUMA) level. The sampling variability in the covariate estimates is a form of measurement error that may bias our effect estimates. We use regression calibration techniques to reduce the bias from the estimation error of our covariates \cite{gleser1992importance}. Moreover, CPUMAs are regions that nest within states. A common assumption in the applied literature is that regions within states contain dependencies that can worsen the efficiency of standard estimation procedures (see, e.g., \cite{cameron2015practitioner}). Using the assumed correlation structure outlined in \cite{kloek1981ols}, we modify the stable balancing weights objective (\cite{zubizarreta2015stable}) to account for state-level dependencies in the outcomes.\footnote{This approach can accommodate other assumed correlation structures as well.}

Our approach also relates to the ``synthetic controls'' literature (see, e.g., \cite{abadie2010synthetic}). Synthetic controls are a popular balancing weights approach frequently used in the applied economics literature to estimate treatment effects on the treated (ETT) for region-level policy changes when using time series cross sectional data. Our application uses a similar data structure; however, we instead consider the problem of estimating the ETC. In contrast to much of the synthetic controls literature, which assumes that the potential outcomes absent treatment follow a linear factor model, we assume no unmeasured confounding and a linear outcome models of our potential outcomes. Our approach therefore illustrates a way to identify and estimate the ETC using similar data.

Section 2 provides an overview of the data and defines the study period, covariates, outcome, and treatment. Section 3 discusses our methods, beginning by defining our target estimand, and then outlining our identification, estimation, and inferential procedures. Section 4 presents our results. Section 5 contains a discussion of the policy relevance of our findings, and Section 6 contains a brief summary. The Appendices contain additional materials, including proofs, summary statistics, and additional results.

\section{Policy Problem and Data}

\subsection{Policy Problem Statement}

We seek to predict how uninsurance rates would have changed in states that did not expand Medicaid in 2014 had they expanded Medicaid. We believe that this effect - the ETC - may differ from the ETT. One reason is because every state had different coverage policies prior to 2014, and non-expansion states tended to have less generous policies than expansion states: in short, ``Medicaid expansion'' represents a set of treatments of varying intensities that are distributed unevenly across expansion and non-expansion states. Averaged over the non-expansion states, which had higher uninsurance rates prior to Medicaid expansion, we might expect the ``average effect'' to be larger in absolute magnitude than among the expansion states, where ``Medicaid expansion'' on average reflected less drastic policy changes.

We limit our analysis to states where we believe all policy changes were approximately equal (see Section~\ref{sssec:txassign}). Even if this were strictly true, we still may expect the ETT to differ from the ETC. For example, all states that were entirely controlled by the Democratic Party at the executive and legislative levels expanded their Medicaid programs, while only states where the Republican Party controlled at least part of the state government failed to expand their programs. Prior to the 2014 Medicaid expansion, \cite{sommers2012understanding} found that conservative governance was associated with lower Medicaid take-up rates. This could reflect differences in program implementation, which may serve as effect modifiers for comparable policy changes.\footnote{Interestingly, \cite{sommers2012understanding} also find that the association between conservative governance and lower take-up rates prior to 2014 existed even after controlling for a variety of factors pertaining to state-level policy administration decisions. They posit that this may reflect cultural conservatism: people in conservative states are more likely to view enrollment in social welfare programs negatively, and therefore be less likely to enroll.} If true, such factors would work attenuate the effects of Medicaid expansion among non-expansion states relative to expansion states. 

Targeting the ETC is also interesting in it's own right: to the extent the goal of studying Medicaid expansion is to understand the foregone benefits (or potential harms) of Medicaid, estimating the ETC is an important quantity of interest. Authors have previously made claims about the ETC directly estimating it: for example, \cite{miller2019medicaid} use their estimates of the ETT to predict that had non-expansion states expanded Medicaid, they would have seen 15,000 fewer deaths during their study period. Moreover, because many downstream effects of Medicaid expansion are plausibly monotonic in the number of newly insured (including mortality), we study the effects on the non-elderly adult uninsurance rate. We therefore contribute to the literature by directly targeting this causal estimand.

\subsection{Data Source and Study Period}\label{ssec:data}

Our primary data source is the annual household and person public use microdata files from the American Community Survey (ACS) from 2011 through 2014. The ACS is an annual cross-sectional survey of approximately three million individuals across the United States. The public use microdata files include information on individuals in geographic areas greater than 65,000 people. The smallest geographic unit contained in these data are public-use microdata areas (PUMAs), arbitrary boundaries that nest within states but not within counties or other more commonly used geographic units. One limitation of these data is a 2012 change in the PUMA boundaries, which do not overlap well with the previous boundaries. As a result, the smallest possible geographic areas that nest both PUMA coding systems are known as consistent PUMAs (CPUMAs). The United States contains 1,075 total CPUMAs, with states ranging from having one CPUMA (South Dakota, Montana, and Idaho) to 123 CPUMAs (New York). Our primary dataset (discussed further in Section~\ref{sssec:txassign}) contains 929 CPUMAs among 46 states. The average total number of sampled individuals per CPUMA across the four years is 1,001; the minimum number of people sampled was 334 and the maximum is 23,990.

We begin our study period in 2011 following \cite{courtemanche2017early}, who note that several other aspects of the ACA were implemented in 2010 -- including the provision allowing for dependent coverage until age 26 and the elimination of co-payments for preventative care -- and likely induced differential shocks across states. We also restrict our post-treatment period to 2014 because several additional states expanded Medicaid in 2015, including Indiana, Michigan, and Pennsylvania. However, these states did not expand Medicaid contemporaneously with the 2014 ACA provisions. Without strong assumptions, these second-year expansion states cannot help us estimate the effect of the 2014 expansion. 

\subsection{Treatment assignment} \label{sssec:txassign}

As noted previously, assigning the concept of ``Medicaid expansion'' to a binary treatment simplifies a more complex reality. There are at least three reasons to be cautious about this simplification. First, states differed substantially in their Medicaid coverage policies prior to 2014. Given perfect data we might ideally consider Medicaid expansion as a continuous treatment with values proportional to the number of newly eligible individuals. The challenge, however, is correctly identifying newly eligible individuals in the data (see \cite{frean2017premium}, who attempt to address this). Second, \cite{frean2017premium} note that five states (California, Connecticut, Minnesota, New Jersey, and Washington) and the District of Columbia adopted partial limited Medicaid expansions prior to 2014. The ``2014 expansion'' therefore actually occurred in part prior to 2014 for several states. \footnote{\cite{kaestner2017effects} and \cite{courtemanche2017early} also consider Arizona, Colorado, Hawaii, Illinois, Iowa, Maryland, and Oregon to have had early expansions.} Similarly, timing is an issue even within 2014: among the states that expanded Medicaid in 2014, Michigan's expansion did not go into effect until April 2014, while New Hampshire's expansion did not occur until September 2014.

Our primary analysis excludes New York, Vermont, Massachusetts, Delaware, and the District of Columbia from our pool of expansion states because these states had comparable Medicaid coverage policies prior to 2014 (\cite{kaestner2017effects}). We also exclude New Hampshire because it did not expand Medicaid until September 2014. While Michigan expanded Medicaid in April 2014, we leave this state in our pool of ``treated'' states. We consider the remaining expansion states as ``treated'' and the non-expansion states (including those that later expanded Medicaid) as ``control'' states. We later consider the sensitivity of our results to these classifications by removing the early expansion states indicated by \cite{frean2017premium}. Our final dataset contains aggregated statistics for all of the above variables for 925 CPUMAs in our non-expansion and our pool of expansion states. There are 414 CPUMAs among 24 non-expansion states and 511 CPUMAs among 21 expansion states. When we exclude the early expansion states for sensitivity analyses, we are left with 296 CPUMAs across 17 expansion states. We provide a complete list of states by classification in Appendix AA.

\subsection{Outcome}

Our outcome is the non-elderly adult uninsurance rate in 2014, which we denote using $Y$. While take-up among the Medicaid-eligible population is a more natural outcome, we choose the non-elderly adult uninsurance rate for two reasons, one theoretic and one practical. First, Medicaid eligibility in the post-period is likely endogenous: Medicaid expansion may affect an individual's income and poverty levels, which in general define Medicaid eligibility. Second, we can better compare our results with the existing literature, including \cite{courtemanche2017early}, who also use this outcome. One drawback is that the simultaneous adoption of other ACA provisions by all states in 2014 also affect this outcome. However, we only attempt to estimate the effect of Medicaid expansion in 2014 in the context of this changing policy environment. We discuss this further in Section~\ref{ssec:estimand} and Section~\ref{ssec:identification}. 

\subsection{Covariates}

We use the underlying individual-level ACS survey data and accompanying survey weights to aggregate the data at the CPUMA level. We choose our covariates to approximately align with those considered in \cite{courtemanche2017early} and that are likely to be potential confounders. Because we are ultimately interested in calculating rates, these variables include both the numerator and denominator counts. 

Using the ACS microdata we estimate: the total non-elderly adult population for each year 2011-2014; the total labor force population (among non-elderly adults) for each year 2011-2013; and the total number of households averaged from 2011-2013. We also construct an average of the total non-elderly adult population from 2011-2013. These are our denominator variables. For our numerator counts, we estimate the total number of: females; whites; people of Hispanic ethnicity; people born outside of the United States; citizens; people with disabilities; married individuals; people with less than a high school education, high school degrees, some college, or college graduates or higher; people living under 138 percent of the FPL, between 139 and 299 percent, 300 and 499 percent, more than 500 percent, and who did not respond to the income survey question; people aged 19-29, 30-39, 40-49, 50-64; households with one, two, or three or more children, and households that did not respond about the number of children. We average these estimated counts from 2011-2013. For each individual year from 2011-2013, we then estimate the total number of people who were unemployed and uninsured at the time of the survey (calculated among all non-elderly adults and all non-elderly adults within the labor force, respectively). We divide the numerator totals by the corresponding denominator totals to estimate the percentage in each category. For the demographics, these include the average number of non-elderly adults from 2011-2013. For the time-varying variables, we use the corresponding year (where uninsurance rates are calculated as a fraction of the labor force rather than the non-elderly adult population). We also calculate the average non-elderly adult population growth and the average number of households to adults across 2011-2013. 

In addition to the ACS microdata, we use 2010 Census data to calculate the approximate percentage of people living within an ``urban'' area for each CPUMA. Finally, we include three state-level covariates reflecting the partisan composition of each state's government in 2013 using data obtained from the National Conference of State Legislatures (NCLS). Specifically, we generate an indicator for states with a Republican governor, an indicator for states with Republican control over the lower legislative chamber, and an indicator for states with Republican control over both chambers of the legislature and the governorship.\footnote{Nebraska is the only state with a unicameral legislature and the legislature is technically non-partisan. We nevertheless classified them as having Republican control of the legislature for this analysis.} 

\section{Methods}\label{sec:methods}

In this section we present our causal estimand, identifying assumptions, estimation strategy, and inferential procedure.

\subsection{Estimand} \label{ssec:estimand}

We let $c$ index CPUMAs, $s$ index states, and $t$ index time (by year) with $T = 2014$. Let $n_1$ be the number of treated CPUMAs, $n_0$ be the number of control CPUMAs, and $n$ be the total number of CPUMAs. Similarly, let and $m = m_1 + m_0$ states (with $m_1$ and $m_0$ defined analogously). Each state has $p_s$ CPUMAs. Letting $A_{st}$ indicate whether a state expanded Medicaid in year $t$ ($A_{st} = 1$) or not ($A_{st} = 0$), we use potential outcomes notation (see, e.g., \cite{rubin2005causal}) to denote a CPUMA's uninsurance rate under Medicaid expansion in year $t$ -- $Y_{sct}^{A_{st} = 1}$ -- and without Medicaid expansion -- $Y_{sct}^{A_{st} = 0}$. We define the causal estimand:

\begin{equation}
\psi = \bar{Y}_{0, T}^1 - \bar{Y}_{0, T}^0 = n_0^{-1}\sum_{scT: A_{sT} = 0} \mathbb{E}\{Y_{scT}^{A_{sT} = 1} - Y_{scT}^{A_{sT} = 0} \mid X_{scT}\}
\end{equation}

This represents the expected treatment effect on non-expansion states conditioning on the observed covariate distribution of the non-expansion states (see, e.g., \cite{imbens2004nonparametric}). The challenge is that we do not observe the counterfactual outcomes for non-expansion CPUMAs had they been in states that expanded their Medicaid programs. We therefore require causal assumptions to identify this counterfactual quantity using our observed data.\footnote{As noted previously, the 2014 Medicaid expansion occurred simultaneously with the implementation of several other major ACA provisions, including (but not limited to) the creation of the ACA-marketplace exchanges, the individual mandate, health insurance subsidies, and community-rating and guaranteed issue of insurance plans (\cite{courtemanche2017early}). Almost all states broadly implemented these reforms beginning January 2014. Conceptually we think of the other ACA components as a state-level treatment ($R$) separate from Medicaid expansion ($A$). Our total estimated effect may also include interactions between these policy changes; however, we do not attempt to separately identify these effects. Without further assumptions -- including that these effects are additive and that the year-one effects are constant across time -- we cannot generalize these results beyond 2014.} 

\subsection{Identification} \label{ssec:identification}

We appeal to the following causal assumptions to identify $\psi$ from our observed data: the stable unit treatment value assumption (SUTVA), no unmeasured confounding given the true covariates and outcome values, and no anticipatory treatment effects. We additionally invoke parametric assumptions to model the measurement error and to express our estimand in terms of parameters from a linear model. We conclude by appealing to ideas from the ``regression-calibration'' literature (\cite{gleser1992importance}) to ensure that identification of our target estimand is possible given auxillary data on the measurement error covariance matrix.

We first assume the SUTVA at the region level. Assuming the SUTVA has two implications for our analysis: first, that there is only one version of treatment; second, that each unit's potential outcome only depends on it's treatment assignment. We discussed potential violations of the first implication previously when considering how to reduce Medicaid Expansion to a binary treatment. The second implication could be violated if one CPUMA's expansion decision affected uninsurance rates in another CPUMA (see, e.g., \cite{frean2017premium}). On the other hand, our assumption does allow for interference among individuals living within CPUMAs. Our assumption is therefore weaker than assuming no interference among any individuals at all. Further addressing this issue is beyond the scope of this paper.

Second, we assume no anticipatory treatment effects. This implies that for any time period $t < T$, a CPUMA's observed outcome $Y_{sct}$ is equal to its potential outcome absent treatment $Y_{sct}^0$. This assumption is necessary because we will use pre-treatment outcomes as covariates. If these outcomes were affected by a state's decision to expand Medicaid prior to 2014, controlling for these covariates could bias our results. This assumption is violated in our study because several states allowed specific counties to expand Medicaid prior to 2014. We later test the sensitivity of our results to the exclusion of these states.

Third, we assume no unmeasured confounding. Specifically, we posit that in 2014 the potential outcomes for each CPUMA are marginally independent of the state-level treatment assignment conditional on CPUMA and state-level covariates $X_{scT}$, a $q$ dimensional vector of covariates. This vector includes both time-varying pre-treatment covariates, including pre-treatment outcomes, and covariates averaged across 2011-2013, such as demographic characteristics, that we discuss in Section~\ref{ssec:data}.\footnote{To be precise, letting $R$ be the other ACA provisions that were implemented across all states, we also assume that $Y^{A = a, R = 1}_{scT} \perp A_{scT} \mid X_{scT}$ (and that $Y_{scT}^{A, R = 1} = A_{scT}Y_{scT} + (1 - A_{scT})Y_{scT}$). This assumption implies that in the context of this changing policy environment (indicated by $R = 1$), Medicaid expansion has the same expected effect in expansion and non-expansion CPUMAs with covariate values $X = x_0$.}

\begin{equation}
Y_{scT}^a \perp A_{scT} \mid X_{scT}
\end{equation}

We believe this assumption is reasonable given our rich covariate set informed by prior literature. 

Finally, we assume that the potential outcomes are linear in the true covariates with within-state equi-correlated errors:

\begin{equation}\label{eqn:linmod}
Y_{scT}^a = \alpha_{a, T} + X_{scT}^T\beta_{a, T} + \epsilon_{scT} + c_{sT}
\end{equation}
%
where the errors $\epsilon_{scT}$ and $c_{sT}$ are mean-zero and independent from each other. For example, $\epsilon_{scT}$ captures time-specific idiosyncracies at the local level, possibly due to the local policy environment. By contrast $c_{sT}$ captures time-specific idiosyncracies at the state-level that are shared across CPUMAs within a state due to the common policy environment. However, we emphasize that our assumption of no unmeasured confounding implies that these errors are uncorrelated with the true covariates and the treatment assignment. Because our covariates include pre-treatment outcomes, this assumption also implies that these errors are uncorrelated over time. For ease of notation, we hereafter omit the dependence of these quantities on $t$ for the remainder of this paper except when needed for clarification.

Under these assumption we can rewrite our causal estimand in terms of the model parameters:

\begin{equation}\label{eqn:outcome}
\bar{Y}_0^a = \alpha_a + \bar{X}_0^T\beta_a    
\end{equation}

If we observed $(Y, X)$ the data would identify $(\alpha_a, \beta_a)$, and therefore $\psi$. However, instead of $(Y, X)$, we observe the noisy measurements $(J, W)$. Importantly, $Y_{sc}^a \perp A_{sc} \mid X_{sc} \centernot\implies J_{sc}^a \perp A_{sc} \mid W_{sc}$. We therefore rely on the following measurement error model to identify our estimand.

First, we assume that our observed covariates and outcomes are equal to the true values plus mean-zero independent (though not necessarily identically distributed) Gaussian noise. We believe this is reasonable because measurement error in our context is sampling variability. We further assume that the measurement errors in our outcomes is uncorrelated with the measurement error in our covariates. We also believe this is reasonable because our outcomes are measured on a different cross-sectional survey than our covariates. 

\begin{equation}\label{eqn:msrmenterror}
(J_{sc}, W_{sc}) \sim N((Y_{sc}, X_{sc}), \begin{pmatrix}
\xi^2_{sc} & 0 \\ 0 & \Sigma_{\nu\nu, sc} \end{pmatrix})
\end{equation}

Our primary data is then $O = (J, W, A)$; that is, a matrix of noisy measurements of the covariates and outcome, and a vector of the true treatment assignments.\footnote{Our covariates are almost all ratio estimates, which will in general be biased. This bias, however, decreases quickly with the sample size (is $O(n^{-1})$). Given that our CPUMA sample sizes are all over 300, we treat these estimates as unbiased in our analysis.} Under this model we can conclude that $\bar{J}_0$ serves as an unbiased estimate of $\bar{Y}_0^0$. The challenge remains identifying and estimating $\bar{Y}_0^1$.

The technical conditions for the identifiability of $\bar{Y}_0^1$ depends on the identifiability of $(\alpha_1, \beta_1)$, which in turn depend on underlying distributions of covariates, model errors, and measurement errors (see, e.g., \cite{cheng1999statistical}). We therefore instead leverage our access to auxillary data and use the ``regression calibration'' approach (see, e.g., \cite{carroll2006measurement}) to ensure identification. At a high-level, regression calibration assumes access to auxillary data ihat can be used to estimate the measurement error covariance matrix $\Sigma_{\nu\nu}$ (assuming that $\Sigma_{\nu\nu, sc}$ is constant; alternatively, we can think of this as the limiting distribution of $n^{-1}\Sigma_{\nu\nu, sc}$). Using this information and the observed data $W$, we generate a function $\eta$ to adjust the observed covariates $W$. We then run our estimation procedure on thw adjusted data $\eta(W)$. 

To be precise, we define our function $\eta_a$ as a linear model of the expected value of $X$ given $(W, A = a)$: 

\begin{equation}\label{eqn:regcal}
\eta_1(W_{sc}) = \mathbb{E}\{X_{sc} \mid W_{sc}, A_{s} = 1\} = \upsilon_1 + \kappa_1 (W_{sc} - \upsilon_1)
\end{equation}
%
where $\upsilon_1 = \mathbb{E}\{X \mid A = 1\}$, $\kappa_1 = \Sigma_{XX \mid A = 1}\Sigma_{WW \mid A = 1}^{-1}$, $\Sigma_{XX \mid A = 1} = \mathbb{E}\{X_{sc}X_{sc}^T \mid A = 1\}$ and $\Sigma_{vv} = \mathbb{E}\{v_{sc}v_{sc}^T\}$, $\Sigma_{WW \mid A = 1} = \Sigma_{XX \mid A = 1} + \Sigma_{vv} = \mathbb{E}\{W_{sc}W_{sc}^T\}$.\footnote{We caution that our estimand is conditional on $X$ while we motivate this procedure under the assumption that $X$ is random. We can redefine these terms as the limiting distribution as $n \to \infty$ of their empirical counterparts, under the assumption that these limits exist.}

By linearity, we then can rewrite our outcome model in terms of $\eta_1$:

\begin{equation}
    J_{sc}^1 = \alpha_1 + \eta_{A_{sc}}(W_{sc})^T\beta_1 + (X_{sc} - \eta_{A_{sc}}(W_{sc}))^T\beta_1 + \xi_{sc} + \epsilon_{sc} + c_s 
\end{equation}
%
However, $\eta_1$ is not estimable given our observed data; for example, our observed data does not identify $\Sigma_{vv}$. We instead assume that we have auxillary data that identifies this quantity. Under these models and assumptions, we the have sufficient data to identify $\psi$. We now discuss our estimation procedure.

\subsection{Estimation}

We propose to use approximate balancing weights to estimate $\bar{Y}_0^1$. We first review approximate balancing weights and the SBW objective proposed by \cite{zubizarreta2015stable}. These methods typically assume that the covariates are measured without error. However, we show that under the classical-errors-in-variables model the SBW estimate of the super-population parameter $\psi_0^1 = \mathbb{E}\{Y^1 \mid A = 0\}$ has the same bias as the OLS estimate. 

We therefore modify SBW for our setting. The broad steps are: first, we estimate a linear model of the true covariates given the observed covariates, leveraging the ACS microdata replicate survey weights for this procedure. We consider three covariate sets: (a) the original covariates with no adjustment; (b) a homogeneous adjustment that uses the same linear model for all CPUMAs; and (c) a heterogeneous adjustment that uses a different model for each CPUMA, accounting for the differential precision for each CPUMA's covariate measurements. Second, we estimate weights that balance the expansion states' adjusted covariates to the mean of the observed non-expansion states' covariates within a chosen vector of thresholds $\delta$. To generate our weights we use both the SBW objective, which minimizes the variance of the weights, and our proposed modification, which we call H-SBW. In contrast to SBW, H-SBW minimizes the variance of the estimator assuming positive equi-correlation between CPUMAs within states (we note that we could also use this approach with other assumed correlation structures). Third, due to remaining imbalances in the covariates, some of which are quite large, we follow the suggestion of \cite{ben2018augmented}, and test the sensitivity of our results to ridge-regression augmented weights. This procedure allows us to achieve better covariate balance by extrapolating beyond the support of the data. Fourth, we use pre-treatment data to compare the performance of our proposed methods on observed outcomes under an arbitrary data generating process. We conclude by estimating the counterfactual mean $\bar{Y}^1_0$ with each set of weights.

\subsection{Stable balancing weights}

\cite{zubizarreta2015stable} propose to estimate the set of weights $\gamma$ that minimizes the following objective:

\begin{equation}\label{eqn:objective}
\gamma = \arg\min_{\tilde{\gamma} \in \Gamma} \sum_{sc: A_{sc} = 1} \tilde{\gamma}_{sc}^2
\end{equation}

\begin{equation}\label{eqn:constraint}
\Gamma(Z) = \{\tilde{\gamma}: \lvert \tilde{\gamma}^T b(Z_{1, r}) - \bar{b(Z_{0, r})} \lvert \le \delta; \tilde{\gamma}_{sc} > 0, \sum_{sc: A_{sc} = 1}\tilde{\gamma}_{sc} = 1\}
\end{equation}
%
for a q-dimensional vector $\delta$. After finding a set of weights satisfying this constraint, we can then estimate $\bar{Y}_0^1$ as

\begin{equation}\label{eqn:psi}
\hat{Y}_0^1 = \sum_{s: A_s = 1}^{m_1}\sum_{c = 1}^{p_s}\gamma_{sc}^\star Y_{sc} - n_0^{-1}\sum_{s: A_s = 0}^{m_0}\sum_{c = 1}^{p_s}Y_{sc}
\end{equation}

In the case where $b$ is the identity function and the outcomes are linear in $Z$, the bias of $\bar{Y}^1_0$ is less than or equal to $\lvert\beta_1\rvert^T\delta$, and therefore equal to zero if $\delta = 0$ is feasible (see, e.g., \cite{zubizarreta2015stable}). Moreover, $\hat{Y}_0^1$ produces the minimum variance estimator -- conditional on $Z$ -- assuming that the errors in the outcome model are independent and identically distributed. For the remainder of our discussion we take $b$ to be the identity.

\subsubsection{Measurement error}\label{ssec:methodsmsrment}

The first advancement in our estimation strategy comes in our balance constraint set $\Gamma(Z)$. Rather than taking $Z = W$, we let $Z = \hat{\eta}_1(W)$. This corrects for the bias induced by the estimation error from the observed covariates. Considering super-population target $\psi_0^1$, naively optimizing over $\Gamma(W)$ leads to the following bound on the expected imbalance on the true covariates $X$:

\begin{align*}
\mathbb{E}\{\sum_{sc}\gamma_{sc}X_{sc} - \upsilon_0\} \le \lvert(\upsilon_0 - \upsilon_1)^T(\kappa_1 - I_d)\lvert + \delta
\end{align*}
%
Assuming a linear outcome model, this leads to the following bound on the bias of the estimator:

\begin{align*}
\mathbb{E}\{\hat{\psi}^{1}_0 - \psi^1_0\} = \lvert(\upsilon_0 - \upsilon_1)^T(\kappa_1 - I_d)\beta_1\lvert + \delta^T\lvert\beta\lvert
\end{align*}
%
Notice that when $\delta = 0$ this is equivalent to the bias of a linear combination of coefficient estimates from the OLS-based regression estimator. We prove this result in Appendix A, and also show that if we have access to a consistent estimate of $\eta(W)$, we can obtain an unbiased estimate of $\psi_0^1$ by optimizing over the constraint set $\Gamma(\hat{\eta}(W))$.

The intuition is as follows: exact balancing weights implicitly estimate $\beta_1$ on a subset of the data where we have sufficient covariate overlap. We can think of SBW as returning a solution to some weighted-least squares problem. Assuming that the outcome model holds across all of the data, WLS and OLS are estimating the same $\beta_1$; consequently, the bias that effects the least squares solution will have the same effect on the WLS, and therefore SBW, solution. 

To estimate $\eta_1$ for our application we use the ACS microdata's set of 80 replicate survey weights to construct 80 additional CPUMA-level datasets. We then take the empirical variance of these estimates for each CPUMA among the expansion states we derive $\hat{\Sigma}_{\nu\nu, sc}$, which we then average to create $\hat{\Sigma}_{\nu\nu}$. We use the empirical covariance matrix on the original dataset $\hat{\Sigma}_{WW}$ and subtract $\hat{\Sigma}_{\nu\nu}$ to estimate $\hat{\Sigma}_{XX}$. We generate $\hat{\eta}_1(W_1)$ using the empirical analogues of Equation~\ref{eqn:regcal}. 

This approximately aligns with the adjustments suggested by \cite{carroll2006measurement} and \cite{gleser1992importance}. However, this adjustment assumes the measurement error variances are constant.\footnote{Additionally, we define our target estimand conditional on $X$, while regression-calibration is motivated by random-X theory. The adjustment procedure can be motivated in either case (see, e.g., \cite{gleser1992importance}). For example, we can define the covariance matrix $\Sigma_{XX}$ as the limit of the sample covariance matrix $\Sigma_{XX}(n)$ as $n\to\infty$.} In our application we have access to information about a substantial source of heterogeneity in the measurement error. Specifically, we know that regions with large populations are estimated quite precisely, while regions with small populations are estimated much less precisely. This motivates an alternative adjustment that we refer to as the ``heterogeneous adjustment,'' that allows us to generate a CPUMA-specific covariate adjustment. We defer the details on this procedure to Appendix B.

This is the first application we are aware of to apply regression calibration in the context of balancing weights to address measurement error. However, this method requires access to knowledge about $\Sigma_{\nu\nu}$. We use survey microdata to identify this parameter for our application. Alternatively, region-level datasets often contain region-level variance estimates; if one is willing to assume $\Sigma_{vv}$ is diagonal, one could leverage this information to use this approach. If no auxillary data is available, one could also consider $\Sigma_{vv}$ to be a sensitivity parameter and conduct estimates over a range of possible values (see, e.g. to SMALL ET AL). 
Regardless, we emphasize two critical assumptions for using this procedure in our context: (1) the outcome model is linear in the true covariates (i.e. $b$ as defined in Equation~\ref{eqn:constraint} is the identity); and (2) the measurement error in the outcome is uncorrelated with the measurement error in the covariates. The first assumption is strong, though commonly used.\footnote{Regression calibration techniques are used sometimes in the context of generalized linear models, when the bias induced by the error is thought to be small (see, e.g., CITE).} The second assumption is reasonable in our setting, because the outcomes are estimated from a different cross-section than the covariates. 

\subsubsection{H-SBW objective}\label{sssec:hsbw}

The next challenge is finding ``the best'' set of weights within $\Gamma(\eta_1(W))$. Unlike the setting outlined in \cite{zubizarreta2015stable}, our application likely has state-level dependencies which may reduce the efficiency of the SBW estimator. We therefore add the tuning parameter $\rho \in [0, 1)$ to penalize the within-state cross product of the weights, as detailed in Equation~\ref{eqn:objective}, representing a constant within-state correlation of the errors.

\begin{equation}\label{eqn:objective}
\gamma = \arg\min_{\tilde{\gamma} \in \Gamma} \quad \sum_{s: A_s = 1}^{m_1}(\sum_{c = 1}^{p_s} \tilde{\gamma}_{sc}^2 + \sum_{c \ne d}\rho \tilde{\gamma}_{sc}\tilde{\gamma}_{sd})\\
\end{equation}

To build intuition about this objective, for $\delta \to \infty$, the following solution is attained:

\begin{equation}\label{eqn:sbwsol}
\gamma_{sc} \propto \frac{1}{(p_s - 1)\rho + 1}
\end{equation}

Setting $\rho = 0$ returns the SBW solution: $\gamma_{sc} \propto 1$. When setting $\rho = 1$, we get $\gamma_{sc} \propto \frac{1}{p_s}$. In other words, as we increase $\rho$, this objective downweights CPUMAs in states with large numbers of CPUMAs and upweights CPUMAs in states with small numbers of CPUMAs (assigning each CPUMA within a state equal weight). In short, as we increase $\rho$, the objective will attempt to more uniformly disperse weights across states. 

In Appendix A, Proposition X, we show that the H-SBW objective produces the minimum conditional variance estimator under the constraint set for this correlation structure when $X$ is known. In theory we could incorporate any assumed correlation structure into this objective in a similar fashion; however, the number of tuning parameters might change. Broadly speaking, we can think of H-SBW being to SBW what generalized least squares (GLS) is to ordinary least squares (OLS): both SBW and OLS can produce unbiased estimates of model parameters, but H-SBW and GLS can improve the efficiency of an estimator under different assumptions about the correlation structure of the outcomes. 

An important caveat emerges in the context of measurement error. Assuming that $X$ also have positive within state correlations (i.e. $Cov(X_{sc}, X_{sd}) = \Sigma_{SS}$, even if all of our models are correct, Lemma~\ref{lemma:lemma3} in Appendix A informs us that the H-SBW objective will produce the following expected imbalance in $X$:

\begin{equation}
    \mathbb{E}\{\sum_{sc}\gamma_{sc}X_{sc} - \upsilon_0\} = f(\Sigma_{XX}, \Sigma_{SS}, \kappa, \rho)
\end{equation}

Notice that there is no expected imbalance if one of the following conditions hold: (1) the Xs are uncorrelated; (2) $\rho = 0$; (3) there is no measurement error. We prove this result in Appendix A and also show that this implies that the weights in the constrained set $\Gamma(\eta_1(W_{sc}))$ will in general also not balance the true $X$. The intuition is that when $\rho > 0$ the optimal weights to depend on the estimate of the within-state cross-product $\eta_1(W_{sc})\eta_1(W_{sc})$. However, this estimate is biased by a factor of $\kappa$, causing an imbalance on the true $X_{sc}$.

Regardless, the H-SBW estimator still may have lower MSE than the SBW estimator depending on the distribution of the data and the relative magnitude of the error terms. In our simulation study available in Appendix X, we find that these imbalances are typically small when $\rho$ is small and/or the measurement errors are small. One potential solution to this problem is to use an estimate $\mathbb{E}\{X_{sc} \mid W, A_s = 1\}$ that accounts for the correlation structure of $X$. However, we do not pursue this avenue for this application.

\subsubsection{Hyperparameter selection}

We seek to make $\delta$ as small as possible; practical guidance in the literature is to reduce the standardized mean differences to be less than 0.1 (CITE). In our application, all of our covariates are measured on the same scale. Additionally, because some of these covariates have very small variances (for example, percent female), we instead target the percentage point differences. We can then estimate $\psi$ using Equation~\ref{eqn:psi}, substituting $J_{sc}$ for $Y_{sc}$ and plugging in the weights $\gamma$.

We choose $\delta$ using domain knowledge about which covariates are most likely to be important predictors of treatment. Specifically, we know that pre-treatment outcomes are often strong predictors of post-treatment outcomes, so we constrain $\delta$ to be 0.05 percentage points (out of 100) for pre-treatment outcomes. Because health insurance is often tied to employment, we also prioritize balancing pre-treatment uninsurance rates, seeking to reduce imbalances below 0.15 percentage points. On the opposite side of the spectrum, we constrain the Republican governance indicators to fall within 25 percentage points. While we believe that Republican governance is important to balance, given the support of the data we are unable to reduce the constraints further without generating extreme weights. We detail the remaining constraints in Appendix AA. 

We consider $\rho \in \{0, 1/6\}$. As noted above, the first choice is equivalent to the SBW objective, while the second assumes constant within-state equicorrelation of $1/6$. We choose this number to be small in order to limit the bias induced by H-SBW in the context of measurement error.

Data driven procedures for hyperparameter selection are also possible with access to pseudo-outcomes. For example, one could use feasible GLS using the pre-treatment outcomes and covariates to estimate $\rho$. Data-driven procedures for $\delta$ are also possible. For example, if a long pre-treatment period is available, \cite{abadie2015synthetic} propose tuning their weights with respect to covariate balance using a ``training'' and ``validation'' period.\footnote{This procedure is better motivated heuristically when estimating the ETT than the ETC: using the control units to predict the treated unit's pre-treatment outcomes, we may believe that the $\delta$ that minimizes the RMSE of $\hat{Y}^0_{1, t}$ for $t < T$ minimizes the RMSE of $\hat{Y}^0_{1, T}$. However, using the treated units to predict the control units pre-treatment outcomes, the $\delta$ that minimizes the RMSE in $\hat{Y}^0_{0, t}$ less plausibly minimizes the bias of $\hat{Y}^1_{0, T}$. This would be a particular worry if we believe there are strong confounders of $Y^1$ that only weakly confound $Y^0$.}

\subsubsection{Bias-correction for imbalances}

In practice we are unable to reduce the balance constraints to an ideal level without generating very extreme weights. We therefore test the sensitivity of our results to the imbalances in the observed (or adjusted) covariates using ridge-regression augmented weights (proposed by \cite{ben2018augmented}). Letting $\hat{X}_1$ be the matrix of adjusted covariates, and $\gamma^{hsbw}$ be our H-SBW weights, we define these weights as:

\begin{equation}
\gamma^{aug} = \gamma^{hsbw} + (\gamma^{hsbw}\hat{X}_1 - \bar{W}_0)^T(\hat{X}_1^T\Omega^{-1}\hat{X}_1 + \lambda I_q)^{-1}\hat{X}_1^T\Omega^{-1}
\end{equation}

where $\Omega$ is a block diagonal matrix with diagonal entries equal to one and the within-group off diagonals equal to $\rho$. We choose $\lambda$ so that all imbalances fall within 0.5 percentage points. The cost of this procedure is that we must extrapolate beyond the support of the data, increasing the model dependence of our estimates. We refer to \cite{ben2018augmented} for more details about this procedure. For our results we consider estimators using SBW weights ($\rho = 0$), H-SBW weights ($\rho = 1/6$), and their ridge-augmented versions that we respectively call BC-SBW and BC-HSBW. 

\subsection{Model validation}

For our validation study we rerun our procedures on pre-treatment data to compare the performance of our models for a fixed $\delta$. In particular, we train our model on 2009-2011 data to predict 2012 outcomes, and 2010-2012 data to predict 2013 outcomes. We limit to one-year prediction error since our estimand is only one-year forward. We examine the performance of H-SBW against SBW, which vary with respect to the tuning parameter $\rho$, the bias-corrected versions, and the covariate adjustment procedure used in the balancing constraints. 

We expect that the estimators trained on the adjusted data should perform better than the estimators trained on the unadjusted data. If our outcome and measurement error models are correct\footnote{For this study we also require these models hold during the pre-treatment period.}, these estimators should achieve better balance across the true covariates and therefore have lower bias than the estimators estimated on the unadjusted data. Assuming our variance model is correct, either of our adjusted covariate sets should have similar bias; however, the heterogeneous adjustment should produce lower variance estimates. Of course in practice our models may be misspecified, and we therefore use this validation study to choose which adjustment we prefer for our final results. Conditional on the adjustment set, we assume that if the bias-corrected estimators all obtain uniformly better covariate balance than the uncorrected estimators, these estimators should also perform better than the uncorrected estimators. However, if the assumed outcome models are incorrect, these estimators may suffer from extrapolation bias and perform worse despite achieving better balance. Finally, we expect SBW to have similar performance to H-SBW, though H-SBW may have increased bias at the cost of reduced variability. 

\subsection{Inference}

We use the leave-one-state-out jackknife to estimate the variance of $\hat{Y}_0^1$ (see, e.g., \cite{cameron2015practitioner}). Specifically, we take the pool of expansion states and generate a list of datasets that exclude each state. For each dataset in this list we calculate the weights and the leave-one-state-out estimate $\bar{Y}^1_{0, -m}$. Throughout all iterations we hold our targeted mean fixed at $\bar{W}_0$.\footnote{That is, we treat $\bar{W}_0$ as identical to $\bar{X}_0$, ignoring the variability in the estimate. Note that the variability in this estimate is of smaller order than the variability in $\hat{Y}_0^1$, since the former does not depend on the number of states but instead the number of CPUMAs (and the sample size used to estimate each CPUMA-level covariate). Moreover, we treat $\bar{X}_0$ as fixed because our estimand is conditional on the observed covariate distribution of the non-expansion states.} When generating these estimates, if our preferred initial choice of $\delta$ does not converge, we gradually reduce the constraints until we can obtain a solution. We compute this variance estimate in two ways: first, we condition on our covariate adjustment $\hat{\eta}_1$. This is our preferred estimator; however, it does not account for the randomness in $\hat{\eta}_1$. We conduct a second procedure where for each dataset we also re-estimate $\hat{\eta}_1$ before estimating the weights. These results are available in Appendix E. We then estimate the variance:

\begin{equation}
    \hat{Var}(\hat{Y}_0^1) = \frac{M - 1}{M}\sum_{m = 1}^M(\hat{Y}^1_{0,-m} - \bar{Y}^1_{0, m})^2
\end{equation}

where $\bar{Y}_{0, m} = \frac{1}{M}\sum_{i=1}^M\hat{Y}^1_{0, -m}$. In other settings the jackknife has been shown to be a conservative approximation of the bootstrap (see, e.g., \cite{efron1981jackknife}). In a simulation study mirroring our setting (available in Appendix G), we also find that this procedure is conservative.

To obtain an estimate the variance of $\bar{Y}_0^0$ we run an auxillary regression model on the non-expansion states and use the CR-2 standard error adjustment to estimate the variance of the linear combination $\bar{W}_0^T\hat{\beta}_0$. Notice that we have no need to adjust the non-expansion state data to estimate this quantity: a linear regression line always contains the point $(\bar{W}_0, \bar{J}_0)$, which are unbiased estimates of $(\bar{X}_0, \bar{Y}_0)$. Therefore, $\mathbb{E}_W\{\bar{W}_0^T\hat{\beta}_0 \mid X\} = \bar{Y}_0^0$. Our final variance estimate $\hat{Var}(\hat{\psi})$ is the sum of $\hat{Var}(\hat{Y}_0^1)$ and $\hat{Var}(\hat{Y}_0^0)$ (though the latter is in general much smaller than the former). We use standard normal quantiles to generate confidence intervals. 

\section{Results}

We first present summary statistics regarding the variability of six time-varying covariates on our adjusted and unadjusted datasets. The second sub-section contains covariate balance diagnostics. The final sub-section contains our ETC estimates.

\subsection{Covariate adjustment}

Table~\ref{tab:adjust1} displays the effects of our covariate adjustment procedure on the variance of our pre-treatment outcomes among the expansion states. We most heavily prioritize balancing these covariates, but they are also among the least precisely estimated (all of our other covariates average over multiple years of data). Table~\ref{tab:adjust1} displays the variance of each covariate on the unadjusted and adjusted datasets. We see that both the homogeneous and heterogeneous adjustment procedures reduce the variability in the data by comparable amounts (see Section~\ref{ssec:methodsmsrment} for definitions of these adjustments). Intuitively, these adjustment reduce the likelihood that our balancing weights will fit to noise in the covariate measurements. These results are consistent across most of our other covariates.

\begin{table}[ht]
\caption{Sample variance on unadjusted and adjusted datasets, expansion states}
\label{tab:adjust1}
\begin{tabular}{lrrr}
  \hline
Variable & No adjustment & Heterogeneous & Homogeneous \\ 
  \hline
Uninsured Pct 2011 & 8.35 & 8.04 & 8.05 \\ 
  Uninsured Pct 2012 & 8.20 & 7.89 & 7.90 \\ 
  Uninsured Pct 2013 & 8.09 & 7.78 & 7.79 \\ 
   \hline
\end{tabular}
\end{table}

One interesting caveat, however, is that the heterogeneous adjustment is more likely to impute values that fall outside of the support of the data. In some cases these imputations are quite extreme. We present tables containing distributional information by covariate for each imputation in Appendix~\ref{sec:appendixsumstat}. This result makes the heterogeneous adjustment possibly less desirable, particularly when we allow the estimators to extrapolate from the data.

\subsection{Covariate balance}

Figure~\ref{fig:loveplotc1} displays the weighted and unweighted imbalances in our adjusted covariate set (using the homogeneous adjustment) using our H-SBW weights. Before applying our weights, we see that there are substantial imbalances in the Republican governance indicators as well as pre-treatment uninsurance rates. Our weights reduce these differences; however, some remain, particularly among the Republican governance indicators. A complete balance table is available in Appendix D, Table 4. 

\begin{figure}[H]
\begin{center}
    \caption{Balance plot, primary dataset}
    \label{fig:loveplotc1}
    \includegraphics[scale=0.5]{01_Plots/balance-plot-all-etuc1.png}
\end{center}
\end{figure}

\begin{figure}[H]
\begin{center}
    \caption{H-SBW versus BC-HSBW versus SBW, weights summed by state, primary dataset}
    \label{fig:sbwvhsbw1}
    \includegraphics[scale=0.6]{01_Plots/weights-by-state-sbw-hsbw-c1.png}
\end{center}
\end{figure}

We then augment these weights using ridge-regression. Figure~\ref{fig:sbwvhsbw1} shows the total weights summed across states for three estimators: H-SBW, BC-HSBW, and SBW. For BC-HSBW we display the negative weights separately from the positive weights to highlight the extent of the extrapolation. This figure illustrates two key points: first, that H-SBW more evenly disperses the weights across states relative to SBW; second, that BC-HSBW extrapolates somewhat heavily in order to achieve the desired level of balance, particularly for CPUMAs in California (this is likely in part because California has the most CPUMAs of any state in the dataset).

Finally, we examine whether the H-SBW weights generated using the unadjusted data balance the adjusted covariates. While this metric does not reflect the ``true'' imbalances, the comparison gives some indication of whether the unadjusted weights are overfitting to noisy covariate measurements. Table~\ref{tab:balcomp} compares the imbalances among our pre-treatment outcomes using H-SBW weights generated on our unadjusted dataset applied to the adjusted (homogeneous) dataset. The ``Unweighted Difference'' column represents the raw difference in means, while the ``Weighted Diff'' column reflects the weighted difference that we calculate on the unadjusted dataset. The ``Homogeneous Diff'' column displays the weighted imbalance when applying the H-SBW weights to the dataset using the homogeneous adjustment, and likewise for ``Heterogeneous Diff.'' The weighted pre-treatment outcomes are approximately one percentage point lower than we desired in the two years prior to treatment using the heterogeneous adjustment, and -0.2 percentage points lower (on average) using the homogeneous adjustment. Compared with the weighted difference (no adjustment) this is suggestive that the unadjusted weights are overfitting to noisy covariates and may give an overly optimistic view of balance obtained. Given the high degree of expected correlation between pre-treatment and post-treatment outcomes, we may expect the estimator of $\bar{Y}^1_0$ trained on the unadjusted data to have a downward bias.

\begin{table}[ht]
\caption{Balance comparison: weights estimated on unadjusted data applied to adjusted data}
\label{tab:balcomp}
\begin{tabular}{lrrrr}
  \hline
Variables & Unweighted Diff & Weighted Diff (none) & Homogeneous Diff & Heterogeneous Diff\\ 
  \hline
Uninsured Pct 2011 & -3.09 & -0.05 & -0.11 & 0.92 \\ 
  Uninsured Pct 2012 & -2.99 & -0.05 & -0.21 & -1.06 \\ 
  Uninsured Pct 2013 & -3.00 & -0.05 & -0.38 & -0.93 \\
   \hline
\end{tabular}
\end{table}

\subsubsection{Model validation}

We compare the performance of our models by repeating the covariate adjustments and calculating our procedure on 2009-2011 ACS data to predict 2012 outcomes, and similarly for 2010-2012 data to predict 2013 outcomes for the treated states. Table~\ref{tab:pretxpred} displays these results, with the rows ordered by RMSE of the prediction errors. Table~\ref{tab:pretxpred} shows that the estimators trained on the covariate adjusted data have substantially better performance than the unadjusted data. Moreover, the estimators trained on the homogeneous adjustment seem to do slightly better than the ones that model the heterogeneity; we present our results using the homogeneous adjustment. In these earlier years we find that SBW tends to have slightly lower RMSE than H-SBW. However, the results are quite similar, as we expected. Finally, we see that the bias corrected estimators tend to perform worse in this application. This may indicate that the extrapolation bias outweighs the cost of reducing the covariate imbalances. While this does not imply that this model will perform badly when predicting $\bar{Y}^1_{0, T}$, it does suggest caution regarding these results. We see this as a function of our models being approximations: we expect in general that assuming the linear outcome models approximately holds on the support of the data where we have sufficient covariate overlap; however, these models may lead us astray when our weights extrapolate excessively from the data. The worst performing estimators are the bias-corrected estimators trained on the unadjusted data.

\begin{table}[ht]
\caption{Estimator pre-treatment outcome prediction error}
\label{tab:pretxpred}
\begin{tabular}{llrrr}
  \hline
Sigma estimate & Estimator & 2012 error & 2013 error & RMSE \\ 
  \hline
Homogeneous & SBW & -0.18 & -0.22 & 0.20 \\ 
  Homogeneous & H-SBW & -0.24 & -0.21 & 0.23 \\ 
  Heterogeneous & SBW & -0.25 & -0.30 & 0.27 \\ 
  Heterogeneous & H-SBW & -0.32 & -0.39 & 0.36 \\ 
  Homogeneous & BC-SBW & -0.42 & -0.35 & 0.39 \\ 
  Heterogeneous & BC-SBW & -0.45 & -0.39 & 0.42 \\ 
  None & SBW & -0.50 & -0.61 & 0.56 \\ 
  None & H-SBW & -0.52 & -0.61 & 0.57 \\ 
  Homogeneous & BC-HSBW & -0.53 & -0.62 & 0.58 \\ 
  Heterogeneous & BC-HSBW & -0.53 & -0.72 & 0.63 \\ 
  None & BC-SBW & -0.82 & -0.93 & 0.88 \\ 
  None & BC-HSBW & -0.93 & -0.99 & 0.96 \\ 
   \hline
\end{tabular}
\end{table}

We find a consistent negative bias across all of our estimators: all of our models tend to under-predict the true uninsurance rate among the non-expansion states the subsequent year by between a fifth to a whole percentage point. Assuming that the sign of this bias will also affect our estimates of $\bar{Y}^1_0$, we should expect our treatment effect estimates to have slight downward bias. That is, the true treatment effect may be smaller (closer to zero) in absolute magnitude than the estimated treatment effect. 

This negative bias is not unexpected: we can think of the uninsurance rate in expansion and non-expansion regions as being drawn from separate distributions with means $(\mu_{1t}, \mu_{0t})$ where $\mu_{1t} < \mu_{0t}$. Our measurement error model for the pre-treatment outcomes ($J_{sct} = Y_{sct} + \epsilon_{sct}$) therefore suggests that balancing weights will select for positive residuals. The negative bias is therefore a form of regression-to-the-mean that occurs during the following period when $\epsilon_{scT}$ moves closer to zero. We see that using the adjusted datasets reduces, but does not eliminate, this bias. We note that this phenomenon has also been discussed in the difference-in-differences and synthetic controls literature (see, e.g., \cite{daw2018matching}). 

\subsection{Primary Results}

Using H-SBW we estimate an effect of -2.33 (-3.49, -1.16) percentage points. The SBW results are almost identical with -2.35 (-3.65, -1.06) percentage points. Compared to the unadjusted data we see very similar estimates at -2.34 (-2.85, -1.82) percentage points for H-SBW and -2.39 (-2.95, -1.83) percentage points for SBW. H-SBW reduces the confidence interval width relative to SBW on our primary dataset. By contrast using the adjusted covariate set increases the width of the estimated confidence intervals. This increase in variability is expected because the adjustment procedure generally reduces the variability in the data, as we saw in Table~\ref{tab:adjust1}, thereby requiring that the balancing weights also increase in variability to achieve the desired level of balance. Importantly, this variance estimate conditions on the covariate adjustment, and does not take into account the randomness in this procedure, and may understate the true uncertainty. When we recalculate the entire adjustment procedure, we find that the confidence intervals are of a comparable magnitude. The results are available in Appendix E.

When we add the bias-correction, the absolute magnitude of the point estimate decreases: we estimate -2.05 (-3.32, -0.79) percentage points for BC-HSBW and -2.00 (-2.98, -1.01) percentage points for BC-SBW. In contrast to our validation tests, where the bias-corrected estimators tended to predict lower uninsurance rates than the other estimators, the bias-corrected estimators predict higher uninsurance rates for $\bar{Y}^1_0$. Table~\ref{tab:mainresults} presents all of our primary estimates in the ``Estimate (95\% CI)'' column. All adjusted estimates were closer to zero than the unadjusted estimates, though the point estimates from the SBW and H-SBW were estimators were virtually identical. We briefly note that the heterogeneous adjustments were all closer to zero than the unadjusted estimates. Complete results are available in Appendix E.

\begin{table}[ht]\label{tab:mainresults}
\caption{Primary results}
\begin{tabular}{llll}
  \hline
Weight type & Adjustment & Estimate (95\% CI) & Early excluded estimate (95\% CI) \\ 
  \hline
H-SBW & Homogeneous & -2.33 (-3.49, -1.16) & -2.09 (-2.85, -1.33) \\ 
  H-SBW & None & -2.34 (-2.85, -1.82) & -2.28 (-2.82, -1.74) \\ 
  BC-HSBW & Homogeneous & -2.05 (-3.27, -0.82) & -1.94 (-2.96, -0.92) \\ 
  BC-HSBW & None & -2.22 (-2.87, -1.56) & -2.22 (-3.07, -1.38) \\ 
  SBW & Homogeneous & -2.35 (-3.65, -1.06) & -2.05 (-2.75, -1.35) \\ 
  SBW & None & -2.39 (-2.95, -1.83) & -2.21 (-2.71, -1.72) \\ 
  BC-SBW & Homogeneous & -2.07 (-3.17, -0.97) & -1.99 (-3.00, -0.99) \\ 
  BC-SBW & None & -2.19 (-2.90, -1.49) & -2.23 (-3.05, -1.40) \\ 
   \hline
\end{tabular}
\end{table}

We also consider the sensitivity of our analysis with respect to no anticipatory treatment effects. We exclude California, Connecticut, Minnesota, New Jersey, and Washington, which had partial expansions prior to 2014, and rerun our analyses. The column ``Early excluded estimate (95\% CI)'' in Table~\ref{tab:mainresults} above reflects these results. Our point estimates are similar to our primary analysis, though the numbers move slightly closer to zero. We also see that the differential between the estimates on the adjusted and unadjusted data is slightly larger: -2.28 (-2.82, -1.74) percentage points for H-SBW on the unadjusted dataset and -2.09 (-2.85, -1.33) on the adjusted data. We again find that when we add the bias-correction the point estimates again move closer to zero. Overall our primary results are relatively robust to the exclusion of these states. Additional diagnostics and results are available in Appendix D.

Lastly, we examine the robustness of our point estimates to the removal of individual states (these are the same point estimates used to calculate our confidence intervals). We find that removing Ohio tends to move the point estimates farther from zero, while removing North Dakota, Kentucky, or California tends to move the estimates closer to zero. Appendix E Figure 3 displays a heatmap showing how the estimates change for each estimator when removing each state.

\section{Discussion}

We estimate that had states that did not expand Medicaid in 2014 instead expanded their programs, they would have seen a -2.33 (-3.49, -1.16) percentage point change in the adult uninsurance rate. Existing estimates place the ETT between -3 and -6 percentage points. These estimates vary depending on the targeted sub-population of interest, the data used, the level of modeling (individuals or regions), and the modeling approach (see, e.g., \cite{courtemanche2017early}, \cite{kaestner2017effects}, \cite{frean2017premium}). Our estimate of the ETC are closer to zero than these ETT estimates.\footnote{When running an unadjusted difference-in-differences estimator on the same dataset we estimate that the ETT is -2.05 (-3.12, -0.97), where the standard errors account for clustering at the state level. We also tried using our weighting approach to estimate the ETT for a better point of comparison. However, the resulting standard error estimates are quite large (almost always over 3 percentage points), and therefore we do not present these estimates. The problem is that achieving covariate balance is much more challenging for this estimand without extensive extrapolation from the support of the data, particularly because of the absence of Democratic party governed states that did not expand Medicaid. In some sense this estimand is simply not estimable using our proposed approach.} This difference may be a function of these different modeling strategies, or it may suggest that the ETC is smaller in absolute magnitude than the ETC. Regardless, due to the potential for effect heterogeneity, we emphasize the importance of directly estimating the targeted counterfactual of interest (e.g. the ETT or ETC), and being explicit about the assumptions used to estimate these quantities. We now consider our methodological contributions, study limitations, and we conclude by considering the policy implications of these findings.

\subsection{Methodological considerations}

We provide several methodological contributions to the literature on balancing weights. First, our estimation procedure accounts for mean-zero random noise in our covariates. We modify the constraint set to balance on a linear approximation to the true covariate values by adapting regression-calibration techniques (\cite{gleser1992importance}) to the balancing weights context. In Table~\ref{tab:balcomp} we show that the weights calculated on the unadjusted dataset fail to achieve the desired level of covariate balance on the adjusted dataset. We find that the weighted imbalances in our pre-treatment outcomes may be larger than we wanted, which we speculate for this application might bias our treatment effect downward. When we compared our estimates using the adjusted covariates to the unadjusted covariates, we find that our point estimates decrease (although often only slightly) in absolute magnitude. Essentially, when we generate weights on the unadjusted data to estimate the 2014 counterfactual outcome, they are likely fitting to noise, causing the observed level of balance to appear better than it truly is. Meanwhile, the re-weighted region may suffer from regression to the mean in the post-treatment period, making our treatment effect estimates appear larger in absolute magnitude than the truth. Once we adjust for the measurement error, our point estimates decrease in absolute magnitude (see also \cite{daw2018matching}, who discuss this phenomenon in more detail in the context of difference-in-differences designs). Overall, our study provides a roadmap for future studies that may wish to correct for potential measurement error while using balancing weights. 

Second, our estimation procedure introduces and illustrates the H-SBW objective, which can improve upon the SBW objective when using hierarchical data. Assuming the errors in the outcome model follow the covariance structure posited by \cite{kloek1981ols}, H-SBW produces a lower variance estimator by more evenly dispersing weights across states. The assumption underlying the particular structure of our objective is that our model errors have constant variance and constant within-state correlation $\rho$. However, our procedure requires assuming the covariance structure and $\rho$ in advance. We choose $\rho = 1/6$ for this application. Identifying a data-driven approach to choose this tuning parameter (or perhaps for the covariance structure in general) could be a useful future contribution. 

Another direction for further work is to calibrate this procedure to determine an optimal bias-variance tradeoff with respect to the measurement error. The procedure we implemented here was likely sub-optimal with respect to the mean-square error of our estimator. In particular, the bias induced by the measurement error decreases with square root of the sample size used to calculate each CPUMA's covariate values, the minimum of which were over three hundred. Meanwhile, the variance of our counterfactual estimate should decrease with the square root of the number of treated states (of which there are 21). From a theoretical perspective, the variance is of a larger order than the bias; moreover, adjusting for the bias will further increase the variance of the estimator. These concerns are consistent with our observed results: the change in our point estimates from the unadjusted data to the adjusted data is of a smaller magnitude than our variance estimate on our point estimate on the unadjusted data. Moreover, once we adjust for the measurement error, our confidence intervals increase more widely than the point estimates change.

\subsection{Limitations}

We caution that we required strong modeling assumptions throughout this analysis. In particular, we require SUTVA, no anticipatory treatment effects, no unmeasured confounding conditional on the true covariates, and several parametric assumptions about both the outcome and measurement error models. We were able to address some concerns about possible violations of these assumptions. For example, our results were qualitatively similar whether we excluded possible ``early expansion states,'' or used different weighting strategies (including relaxing the positivity restrictions and changing the tuning parameter $\rho$). We also examined two versions of our covariate adjustment and found similar results with either. However, we do not attempt to address concerns about SUTVA violations, particularly the impact of spillovers across regions. And while we believe that no unmeasured confounding is reasonable for this problem, we did not conduct a sensitivity analysis (see, e.g., \cite{bonvini2021sensitivity}) with respect to this assumption. Finally, our approach to correcting for measurement error relies on access to auxillary information that may not be available for many applications. However, even without such information, $\Sigma_{vv}$ could also be considered a sensitivity parameter to evaluate the robustness of results to measurement error (see, e.g., CITE).

\subsection{Policy considerations}

We find that our point estimates for the ETC are somewhat smaller in absolute magnitude than existing estimates of the ETT. While we make no formal statistical claims about these differences, this finding nevertheless highlights the importance of caution when using estimates of the ETT to make inferences about the ETC. Because almost every outcome of interest is mediated through increasing the number of insured individuals, if the ETC is in fact different than the ETT, then projecting findings from an estimate of the ETT to the ETC may lead to inaccurate inference. For example, \cite{miller2019medicaid} study the effect of Medicaid expansion on mortality. Using their estimate of the ETT they project that had all states expanded Medicaid, 15,600 deaths would have been avoided during their study's time-period. If we believe that this number increases monotonically with the number of uninsured individuals, this estimate may be an overestimate if the ETC is less than the ETT, or an underestimate if the ETC is greater than the ETT. Directly estimating the ETC can help us better model policy relevant downstream effects mediated through decreasing the uninsurance rate. 

Medicaid expansion is also still an ongoing policy debate in the United States. Following the passage of the American Rescue Plan, state legislatures in Wyoming, Alabama, and North Carolina are reportedly considering expanding their programs. Our study estimates the effect of Medicaid expansion on adult uninsurance rates; however, this effect is only interesting because Medicaid enrollment is not automatic for eligible individuals. Different state policies may therefore make it easier or harder to enroll in Medicaid. We emphasize that if the goal of Medicaid expansion is to increase insurance access for low-income adults, state policy-makers also may wish to make it easier to enroll in Medicaid. 

\section{Conclusion}

This is the first study we are aware of that directly estimates the foregone coverage expansions of Medicaid expansion on states that did not expand Medicaid in 2014. Our estimation approach contributes to the methodological literature on balancing weights literature by using an estimation procedure that account for hierarchical data structure and measurement error in the covariates. We estimate that had states that did not expand their Medicaid eligibility requirements in 2014 done so, they would have seen a -2.33 (-3.49, -1.16) percentage point change in their uninsurance rate. This point estimate is closer to zero than existing estimates of the ETT, which range between -3 and -6 percentage points (\cite{frean2017premium}).\footnote{As we noted above, prior studies differ with respect to the data used, the targeted population of interest, the modeling choices, and unit level of analysis.} From a policy-analysis perspective, we caution against using using existing estimates of the ETT to make inferences about the ETC. From a policy-making standpoint, we note that if the goal of Medicaid expansion is to increase access to insurance for low-income adults, state and federal policy-makers may wish to consider policies that make Medicaid enrollment easier if not automatic.

\section*{Acknowledgements}

The authors gratefully acknowledge invaluable advice and comments from Zachary Branson, Dave Choi, Edward Kennedy, Brian Kovak, Akshaya Jha, Lowell Taylor, and Jose Zubizaretta.

\begin{supplement}
Analysis programs and supporting materials are available online at github.com /mrubinst757/medicaid-expansion. Proofs and additional results are available in the Appendix.
\end{supplement}

\bibliographystyle{imsart-nameyear} % Style BST file
\bibliography{research.bib}       % Bibliography file (usually '*.bib')

\clearpage

\appendix
\section{Proofs}

We show three results regarding the bias of the SBW estimator under the classical errors-in-variables model. First, we show that the bias of the SBW estimator that sets $\delta = 0$ (i.e. exactly balances the means of the covariates) is equal to the bias of the OLS estimator. Second, we show that if the observed covariate values can be replaced by their conditional expectations $\eta$ given the noisy observations, then the SBW estimator will be unbiased. Finally, we consider the case where we can estimate $\eta$ using auxillary data to estimate the covariance matrix of the error terms and show that the SBW estimator is consistent if we replace $\eta$ by an estimate $\hat{\eta}$ in the constraint set. In contrast to our application, we take the perspective throughout that $X$ is random, noting that this estimator still has desirable properties even when we view $X$ is fixed (see, e.g., \cite{gleser1992importance}).

Consider a dataset that consists of $i = 1, ..., n$ units where we observe the outcomes $Y_i$ and a treatment assignment indicator $A_i \in \{0, 1\}$. Let $n_a$ be the number of units in each treatment group, i.e, $n_a = \sum_{i = 1}^n \mathbbm{1}(A_i = a)$. Let $X_i \in \mathbb{R}^q$ be a vector of covariate values where $X_i \mid A_i = a \stackrel{iid}\sim MVN(\upsilon_a, \Sigma_{XX})$. 

Our target parameter is $\psi = \mathbb{E}\{Y_i^1 \mid A_i = 0\}$.\footnote{Note that we switch to the super-population target because we view $X$ as random.} We assume unconfoundedness ($Y_i^a \perp A_i \mid X_i$), consistency ($Y_i^a = Y_i \mid A_i = a$), and that $\mathbb{E}\{Y_i \mid X_i, A_i = a\} = \alpha_a + X_i^T\beta_a$. For simplicity we also assume throughout that there exist positive weights $\gamma$ that exactly balance the covariates, i.e. $\sum_{i: A_i = 1}\gamma_iX_{i, r} = \bar{X}_{0, r}$ for all $r = 1, ..., q$ and $\sum_{i: A_i = 1}\gamma_i = 1$, $\gamma_i > 0$. 

We begin by establishing the following identity:

\begin{equation}
\psi = \mu_y + (\upsilon_0 - \upsilon_1)^T\beta_1
\end{equation}

where $\mu_y = \mathbb{E}\{Y_i \mid A_i = 1\}$ and $\upsilon_a = \mathbb{E}\{X_i \mid A_i = a\}$.

\begin{proof}

Using our causal and modeling assumptions we have that:

\begin{align*}
\mathbb{E}\{Y_i^1 \mid X_i, A_i = 0\} &= \mathbb{E}\{Y_i^1 \mid X_i, A_i = 1\} \\
&= \mathbb{E}\{Y_i \mid X_i, A_i = 1\} \\
&= \alpha_1 + X_i^T\beta_1 \\
&= \mu_y + (X_i - \upsilon_1)^T\beta \\
&\implies \psi = \mu_y + (\upsilon_0 - \upsilon_1)^T\beta_1
\end{align*}

where the first equality follows from unconfoundedness, the second equality from consistency, the third from our parametric modeling assumptions, and the fourth by definition of $\alpha$. The final implication comes from plugging $\upsilon_0$ in place of $X_i$.

\end{proof}

We now outline the classical errors-in-variables model. Let $Y_i(X_i, a) = \alpha_a + X_i^T\beta_a + \epsilon_i$ where $\epsilon_i \stackrel{iid}\sim N(0, \sigma^2)$. We consider the case where we observe $W_i$, a vector of mean-unbiased proxies for the true (unobserved) covariate vector $X_i$; i.e., $W_i = X_i + v_i$, where $v_i \stackrel{iid}\sim MVN(0, \Sigma_{vv})$. Consider the model:

\begin{equation}
(\epsilon_i, v_i) \sim MVN((0, 0), \begin{pmatrix} 
\sigma^2 & 0 \\ 
0 & \Sigma_{vv}  
\end{pmatrix}
\end{equation}

In other words the error in the outcome model is uncorrelated with the error in the covariates. Let $u_i = (\epsilon_i, v_i)$. We further assume that the covariates are uncorrelated with any of the error terms:

\begin{equation}
(X_i, u_i) \mid A_i = a \sim MVN((\upsilon_a, 0), \begin{pmatrix} 
\Sigma_{XX} & 0 \\ 
0 & \Sigma_{uu}  
\end{pmatrix}
\end{equation}

Then we see that $(X_i, W_i) \mid A_i = a \stackrel{iid}{\sim} MVN((\upsilon_a, \upsilon_a), \Sigma)$\footnote{In contrast to our application, we instead assume here that $\Sigma_{WW \mid A = 1} = \Sigma_{WW \mid A = 0} = \Sigma_{WW}$ and $\Sigma_{XX \mid A = 1} = \Sigma_{XX \mid A = 0} = \Sigma_{XX}$. This is not a necessary assumption, but helps simplify notation.} where 

$$
\Sigma = \begin{pmatrix} 
\Sigma_{XX} & \Sigma_{XX} \\ 
\Sigma_{XX} & \Sigma_{WW}  
\end{pmatrix}
$$ 

and $\Sigma_{WW} = \Sigma_{XX} + \Sigma_{vv}$. Let $\kappa = \Sigma_{WW}^{-1}\Sigma_{XX}$. By the normality of the joint distribution of $X_i$ and $W_i$, we also know that

\begin{equation}
\mathbb{E}\{X_i \mid W_i, A_i = a\} = \upsilon_a + \kappa^T(W_i - \upsilon_a)
\end{equation}

We consider estimating $\psi$ in this setting. Let 

$$
\hat{\psi}^{reg} = \bar{Y}_1 + (\bar{W}_0 - \bar{W}_1)^T\hat{\beta}_1
$$ 

where $\hat{\beta}_1$ is the OLS estimator of $\beta_1$, $\bar{W}_a = n_a^{-1}\sum_{i:A_i = a} W_i$ and $\bar{Y}_a$ is defined analogously. Let 

$$
\hat{\psi}^{sbw} = \sum_{i: A_i = a} \gamma_i Y_i
$$ 

be the SBW estimator, where

\begin{align*}
\gamma &= \arg\min_{\theta} \sum_{i: A_i = 1}\theta_i^2 \text{ such that } W_{1, r}^T\theta = \bar{W}_{0, r} \ \ \ (r = 1, ..., q) \\
&\theta_i > 0, \sum_{i: A_i = 1}\theta_i = 1
\end{align*}

where $W_{a, r}$ is the $n_a$ dimensional (column) vector associated with covariate $r$.

\begin{proposition}\label{cl1}
The bias of $\hat{\psi}^{reg}$ is equal to the bias of $\hat{\psi}^{sbw}$; specifically, 

$$
\mathbb{E}\{\psi^{reg} - \psi\} = \mathbb{E}\{\psi^{sbw} - \psi\} = (\upsilon_0 - \upsilon_1)^T(\kappa - I_d)\beta
$$
\end{proposition}

\begin{proof}
Recall that $\mathbb{E}(\hat{\beta}_1) = \kappa\beta_1$ (c.l. \cite{gleser1992importance}). Consider the error of $\hat{\psi}^{reg}$: 

\begin{align*}
    \hat{\psi}^{reg} - \psi &= \bar{Y}_1 + (\bar{W}_0 - \bar{W}_1)^T\hat{\beta}_1 - (\mu_y + (\upsilon_0 - \upsilon_1)^T\beta_1) \\
    &= \underbrace{(\bar{Y}_1 - \mu_y)}_{T_1} + \underbrace{(\bar{W}_0 - \upsilon_0)^T\hat{\beta}_1}_{T_2} - \underbrace{(\bar{W}_1 - \upsilon_1)^T\hat{\beta}_1}_{T_3} + \underbrace{(\upsilon_0 - \upsilon_1)^T(\hat{\beta}_1 - \beta_1)}_{T_4} \\
    \implies \mathbb{E}\{\hat{\psi}^{reg} - \psi\} &= (\upsilon_0 - \upsilon_1)^T(\kappa - I_d)\beta_1
\end{align*}

The first equality holds by definition and the second by rearranging terms. The second equality consists of four terms. $T_1$ is simply the estimation error from a sample average, which has expectation zero. $T_2$ is the product of the estimation error from a sample average and $\hat{\beta}$; this also has expectation zero because $\bar{X}_0$ is estimated on a different part of the sample than $\hat{\beta}$, so these errors are independent. $T_3$ also has expectation zero because $\bar{W}_1^T\hat{\beta}_1 = \bar{Y}_1$, which has expectation $\mu_y$, and $\upsilon_1^T\hat{\beta}_1$ also has expectation $\mu_y$. We are then left with $T_4$; we substitute $\mathbb{E}{\hat{\beta}} = \kappa\beta$ to get the final result. 

We now derive the bias of $\hat{\psi}^{sbw}$:

\begin{align*}
    \hat{\psi}^{sbw} - \psi &= \sum_{i: A_i = 1}\gamma_iY_i - (\alpha_1 + \upsilon_0^T\beta_1) \\
    &= \sum_{i: A_i = 1} \gamma_i(\alpha_1 + X_i^T\beta_1 + \epsilon_i) - (\alpha_1 + \bar{W}_0^T\beta_1 + (\upsilon_0 - \bar{W}_0)^T\beta_1) \\
    &= \sum_{i: A_i = 1} (\gamma_i(W_i - v_i)^T\beta_1 + \gamma_i\epsilon_i) - \bar{W}_0^T\beta_1 + (\upsilon_0 - \bar{W}_0)^T\beta_1 \\
    &= \underbrace{-\sum_{i: A_i = 1}\gamma_iv_i^T\beta_1}_{T_1} + \underbrace{\sum_{i: A_i = 1}\gamma_i\epsilon_i}_{T_2}  + \underbrace{(\upsilon_0 - \bar{W}_0)^T\beta_1}_{T_3}
\end{align*}

Conditioning on $W_i$, we can take expectations over $X_i$, and see that $T_2$ has expectation zero (noting that the weights, conditional on $W_i$, are independent of these errors). $T_3$ is simply the scaled sum of mean zero estimation error and therefore has expectation zero. We conclude by considering $T_1$ and again take expectations over $X_i$ conditional on $W_i$: 

\begin{align*}
    \sum_{i: A_i = 1} \gamma_i\mathbb{E}\{X_i - W_i \mid W_i\}^T\beta_1 &= \sum_{i: A_i = 1} \gamma_i (\upsilon_1 + \kappa^T(W_i - \upsilon_1))^T\beta_1 - \sum_{i: A_i = 1}\gamma_i W_i^T\beta_1 \\
    &= (\upsilon_1 + \kappa^T(\bar{W}_0 - \upsilon_1))^T\beta_1 - \bar{W}_0^T\beta_1 \\
    &= (\kappa^T(\bar{W}_0 - \upsilon_1))^T\beta_1 - (\bar{W}_0 - \upsilon_1)^T\beta_1  \\
    &= (\bar{W}_0 - \upsilon_1)^T(\kappa - I_d)\beta_1 \\
    &= (\upsilon_0 - \upsilon_1)^T(\kappa - I_d)\beta_1 + (\bar{W}_0 - \upsilon_0)^T(\kappa - I_d)\beta_1 \\
    \implies \mathbb{E}\{\hat{\psi}^{sbw} - \psi\} &= (\upsilon_0 - \upsilon_1)^T(\kappa - I_d)\beta_1
\end{align*}

The final line holds because in the second to last line, we can take expectation over $W_i$ and see that the second term is a scaled sum of mean zero estimation error and has expectation zero. 
\end{proof}

Let $\hat{\psi}^{sbw}(\eta)$ be the SBW estimator that reweights $\eta_1(W_i)$ rather than $W_i$; i.e. $\hat{\psi}^{sbw}(\eta) = \sum_{i: A_i = 1}\gamma_i^\star Y_i$ where 

\begin{align*}
\gamma^\star = &\arg\min_{\theta} \sum_{i: A_i = 1}\theta_i^2 \text{ such that } \eta_1(W_{1, r})^T\theta = \upsilon_{0, r} \ \ \ (r = 1, ..., q) \\
&\theta_i > 0, \sum_{i: A_i = 1} \theta_i = 1
\end{align*}

\begin{proposition}
The estimator $\hat{\psi}^{sbw}(\eta)$ is unbiased, i.e.
$\mathbb{E}\{\hat{\psi}^{sbw}(\eta)\} = \psi$
\end{proposition}

\begin{proof}

By linearity we know that

\begin{align*}
Y_i = \alpha_1 + \eta_1(W_i)^T\beta_1 + (X_i - \eta_1(W_i))^T\beta_1
\end{align*}

We then have that:

\begin{align*}
    \hat{\psi}^{sbw}(\eta) - \psi &= \sum_{i: A_i = 1}\gamma_i^\star Y_i - (\alpha_1 + \upsilon_0^T\beta_1) \\
    &= \sum_{i: A_i = 1}\gamma_i^\star\alpha_1 + \sum_{i: A_i = 1}\gamma_i^\star\eta_1(W_i)^T\beta_1 + \sum_{i: A_i = 1}\gamma_i^\star(X_i - \eta_1(W_i))^T\beta_1 - (\alpha + \upsilon_0^T\beta_1) \\
    &= \sum_{i: A_i = 1}\gamma_i^\star(X_i - \eta_1(W_i))^T\beta_1
\end{align*}

Conditional on $W$, the weights are fixed and $X_i - \eta_1(W_i)$ has expectation zero; therefore, the estimator is unbiased.

\end{proof}
\begin{remark}

While we have assumed no model error, this estimator still has variance, conditional on $W_i$, equal to

$$
\sum_{i: A_i = 1} \gamma_i^{\star^2}\beta^T Cov(X_i \mid W_i)\beta
$$

where, assuming that $(X_i, W_i)$ are jointly normally distributed, $Cov(X_i \mid W_i) = \Sigma_{XX} - \Sigma_{XX}\Sigma_{WW}^{-1}\Sigma_{XX}$. Therefore, the variance of this estimator is higher than if we knew the true $X_i$ unless $\Sigma_{WW} = \Sigma_{XX}$ (i.e. we observe $X_i$). 

\end{remark}

\begin{remark}
Assuming there is a model error simply leads to the additional term $\sum_{i: A_i = 1}\gamma_i\epsilon_i$. This again has expectation zero, because the weights remain independent of the error in the outcome model, and adds a term to the total variance (conditional on $W_i$) equal to

$$
\sigma^2\sum_{i: A_i = 1}\gamma_i^{\star^2}
$$
\end{remark}

\begin{remark}
If we view both $W$ and $X$ as fixed and substitute $\bar{X}_a$ for $\upsilon_a$ (again assuming this quantity is known), the remaining term contributes to a finite-sample bias rather than variance. Moreover, the squared-bias term (equivalent to the MSE when $\sigma^2 = 0$) will decrease with the variance of the weights.
\end{remark}

Proposition 2 assumes that we know $\eta_a$ and $\upsilon_0$; however, in practice we estimate it from the data. Moreover, the estimation of $\hat{\eta}_1$ typically involves both the observed dataset and auxillary data, which we have not yet specified here. We now consider a simple version of this case.

Recall that $\eta_a(W_i) = \upsilon_a + \kappa^T(W_i - \upsilon_a)$, where $\kappa = \Sigma_{WW}^{-1}\Sigma_{XX}$. We can easily estimate $\upsilon_a$ consistently using $\bar{W}_a$; the challenge is estimating $\kappa$. 

Following \cite{gleser1992importance}, consider the setting where we observe $T$ independent vectors of measurements $W_i^\star$ from known values $X_i^\star$, so that $v_i^\star = W_i^\star - X_i^\star$. Assume that $\Sigma_{v^\star v^\star} = \Sigma_{vv}$. We can then estimate $\hat{\Sigma}_{vv} = \frac{1}{T}\sum_{i=1}^T(W_i^\star - X_i^\star)'(W_i^\star - X_i^\star)$. We can then estimate

$$
\hat{\kappa} = (n\hat{\Sigma}_{WW})^{-1}(n(\hat{\Sigma}_{WW} - \hat{\Sigma}_{vv}))
$$

assuming $n(\hat{\Sigma}_{WW} - \hat{\Sigma}_{vv})$ is positive semi-definite, and where $\hat{\Sigma}_{WW} = \frac{1}{n}\sum_{i=1}^n (W_i - \bar{W})(W_i - \bar{W})'$. 

\begin{proposition}
Let $\tilde{\gamma}$ be the weights that solve the SBW objective with $\sum_{i: A_i = 1}\tilde{\gamma}_i\hat{\eta}_1(W_{i, r}) = \bar{W}_{0, r}$ for all $r = 1, ..., q$. The estimator $\sum_{i: A_i = 1}\tilde{\gamma}_iY_i \to \psi$ as $n \to \infty$, $T \to \infty$.
\end{proposition}

\begin{proof}

Rewriting the error expression from Lemma A.2 leads to the additional terms $\sum_{i: A_i = 1}\tilde{\gamma}_i(\eta_1(W_i) - \hat{\eta_1}(W_i))^T\beta_1$ and $(\upsilon_0 - \bar{W}_0)^T\beta_1$. The second term is scaled estimation error and has expectation zero. It therefore suffices to show that 

$$
\sum_{i: A_i = 1}\tilde{\gamma}_i(\eta_1(W_i) - \hat{\eta_1}(W_i))^T\beta_1 \to 0
$$

By the weak law of large numbers, $\hat{\Sigma}_{WW} - \hat{\Sigma}_{vv} \to \Sigma_{XX}$ as $n \to \infty$, $T \to \infty$; similarly $\hat{\Sigma}_{WW} \to \Sigma_{WW}$ as $n \to \infty$. By the continuous mapping theorem $\hat{\kappa} \to \kappa$. Since $\bar{W}_0 \to \upsilon_0$, we have that $\hat{\eta} \to \eta$. 

Let $\gamma^\star$ be the weights defined in Lemma A.3. We can then rewrite the error term above as

\begin{align*}
\sum_{i: A_i = 1}(\tilde{\gamma}_i - \gamma_i^\star)(\hat{\eta}_1(W_i) - \eta(W_i))^T\beta_1 + \sum_{i: A_i = 1}\gamma_i^\star(\hat{\eta}_1(W_i) - \eta_1(W_i))^T\beta_1
\end{align*}

As $n \to \infty$, $T \to \infty$, $\hat{\eta}_1 \to \eta_1$, and therefore by definition of the SBW objective, $\tilde{\gamma}_i \to \gamma_i^\star$. These two terms therefore both converge in probability to zero. 

\end{proof}

Our estimation of $\eta_a$ follows a somewhat different procedure which we outline in Appendix B.

\subsection{H-SBW Objective}

We show that the H-SBW estimator minimizes the variance of any estimator within the constraint set given a constant within-group correlation structure. Here we assume that the true covariates are measured. Consider the outcome model:

\begin{equation}
    Y_{sc}(X_{sc}, a) = \alpha_a + X_{sc}^T\beta_a + c_s + \epsilon_{sc}
\end{equation}

where $\mathbb{E}\{c_s \mid X_{sc}, A_{sc}\} = \mathbb{E}\{\epsilon_{sc} \mid X_{sc}, A_{sc}\} = \mathbb{E}\{c_s\epsilon_{sc} \mid X_{sc}, A_{sc}\} = 0$, and $Var(\epsilon_{sc}) = \sigma^2$ and $Var(c_s) = \eta^2$. Let $\rho = \frac{\eta^2}{\sigma^2 + \eta^2}$.

The conditional variance of any weighting estimator $\sum_{sc: A_{sc} = 1}\gamma_{sc}Y_{sc}$ using weights $\gamma$ where $\gamma^T1 = n_t$ is then equal to:

\begin{align}
    Var(n_t^{-1}\sum_{sc: A_s = 1}\gamma_{sc}Y_{sc} \mid X_{sc}, A_{sc}) &= n_t^{-2}\sum_{s: A_s = 1}^{m_1}(\sum_{c = 1}^{p_s}\gamma_{sc}^2(\sigma^2 + \eta^2) + \sum_{c \ne d}\gamma_{sc}\gamma_{sd}\eta^2) \\
    &\propto \sum_{s: A_s = 1}(\sum_{c = 1}^{p_s}(\gamma_{sc}^2 + \sum_{c \ne d}\rho \gamma_{sc}\gamma_{sd})
\end{align}

where the second line follows by dividing by $\eta^2 + \sigma^2$. This suggests the H-SBW objective, which minimizes the (conditional) variance of this estimator for known $\rho$.


\section{Adjustment Details}

We now discuss how we estimate $\eta_a$ in our data. We begin by estimating the unit-level covariance matrices $\Sigma_{vv, sc}$, the sampling variability for each CPUMA by using the individual replicate survey weights to generate $b = 80$ additional CPUMA-level datasets. We then estimate $\Sigma_{vv, sc}$:

\begin{equation}
\hat{\Sigma}_{vv, sc} = \frac{4}{80}\sum_{b=1}^{80}(W_{sc}^B - \bar{W}_{sc})(W_{b, sc} - \bar{W}_{sc})'
\end{equation}

where the $4$ in the numerator comes from the process used to generate the replicate survey weights and $\bar{W}_{sc}$ is the value calculated in the original dataset. 

Unlike in our proofs above, we do not assume that $\Sigma_{XX \mid A = a}$ or $\Sigma_{vv \mid A = a}$ are equal for $A_i \in \{0, 1\}$. Instead we let $\hat{\Sigma}_{WW \mid A = a} = n_a^{-1}\sum_{s, c: A_s = a} (W_{sc} - \bar{W})(W_{sc} - \bar{W})^T$. This estimator is calculated on the original observed dataset. We then estimate $\Sigma_{XX \mid A = a}$ using:

\begin{equation}
\hat{\Sigma}_{XX \mid A = a} = \hat{\Sigma}_{WW \mid A = a} - n_a^{-1}\sum_{s, c: A_s = a} \hat{\Sigma}_{vv, sc}
\end{equation}

Define

\begin{equation}
\hat{\kappa}_a = \hat{\Sigma}_{WW \mid A = a}^{-1}\hat{\Sigma}_{XX \mid A = a}
\end{equation}

Notice that $\hat{\kappa}$ is a matrix of estimated coefficients of a linear regressions of the (unobserved) matrix $X_{sc}$ on (observed) matrix $W_{sc}$. We can then estimate $X_{sc}$ as

\begin{equation}
\hat{\eta}_a(W_{sc}) = \bar{W}_a + \hat{\kappa}_a^T(W_{sc} - \bar{W}_a)
\end{equation}

where $\bar{W}_a = n_a^{-1}\sum_{s, c: A_s = a} W_{sc}$. This approximately aligns with the adjustments suggested by \cite{carroll2006measurement} and \cite{gleser1992importance}. 

In our setting we have additional access to information about a substantial source of heterogeneity in the measurement error: in particular, regions with large populations are estimated quite precisely, while regions with small populations are estimated much less precisely. This is because the survey is a one-percent sample across all regions. Moreover, for a given CPUMA, some covariates are measured using three years of data, and others only one. However, using the conventional regression calibration approach will adjust precisely estimated covariates to imprecisely estimated covariates in a similar way. 

Our preferred estimators therefore use an alternative approach where we model an individual-level $\Sigma_{vv, sc}$ as a function of the sample sizes used to estimate each covariate. In particular, let $s_{sc}$ be the q-dimensional (row) vector of the sample sizes used to estimate each covariate value for a given CPUMA. Let $\matr{S}_{sc} = \sqrt{s_{sc}}\sqrt{s_{sc}}^T$. We assume that $\sqrt{s_{sc}}v_{sc}^T \sim N(0, \Sigma_{vv}^a)$. We then know that $\Sigma_{vv, sc}^m = \Sigma_{vv}^a \oslash \matr{S}_{sc}$. We add the superscript $m$ to distinguish that is a modeled covariance matrix.

To estimate these matrices, we pool our initial estimates of the CPUMA-level covariance matrices ($\hat{\Sigma}_{vv, sc}$) to generate $\hat{\Sigma}_{vv}^a = n_a^{-1}\sum_{s, c: A_s = a} \matr{S}_{sc} \circ \Sigma_{vv, sc}$. We estimate $\hat{\Sigma}_{vv, sc}^m = \hat{\Sigma}_{vv}^a \oslash \matr{S}_{sc}$. From this we estimate $\hat{\Sigma}_{XX \mid A = a} = \hat{\Sigma}_{WW \mid A = a} - n_a^{-1}\sum_{s, c: A_s = a}\hat{\Sigma}_{vv, sc}^a$. Finally, we calculate $\hat{\kappa}_{sc} = (\hat{\Sigma}_{XX \mid A = a} + \hat{\Sigma}_{vv, sc}^m)^{-1}\hat{\Sigma}_{XX \mid A = a}$, which we use to estimate $\hat{\eta}_a(W_{sc})$. 

We prefer this adjustment because it accounts for CPUMA-level variability in the measurement error. Specifically, this adjustment should more greatly affect outlying values of imprecisely estimated covariates, while leaving precisely estimated covariates closer to their observed value in the dataset. Moreover, we are able to use the full efficiency of using all units in the modeling, so our CPUMA-level $\hat{\kappa}$ estimates are using the full data. The disadvantage is that, as shown in Appendix~\ref{sec:appendixsumstat}, this adjustment, compared to the conventional approach, appears more likely to lead to extreme values that fall outside of the support of the original data. A second cost is that this aggregation models all differences as a function of the sample sizes, and averages over any potential heterogeneity due to heteroskedasticity (i.e. the measurement error covariance matrix changes depending on the true value of $X_{sc}$). In practice we find that either adjustment yields quite similar results (see Appendix~\ref{ssec:allresults}).



\section{Summary Statistics}
\label{sec:appendixsumstat}

Table~\ref{tab:summarytab1} and Table~\ref{tab:summarytab2} display univariate summary statistics from the primary dataset. We display the mean, interquartile range, and the range (as defined by the maximum value minus the minimum value) for the unadjusted dataset (sigma\_zero), the preferred adjustment (sigma\_uu\_i), and the secondary adjustment (sigma\_uu\_avg). We see that in general the covariate adjustments reduce the variability in the data. However, we also see that the range sometimes increases drastically, particularly for the untreated dataset. Table~\ref{tab:extreme1} displays the overall number of CPUMAs whose adjusted values fall outside of the range of the unadjusted data. Note that for our ETT estimates, we only use the adjustments on the treatment data, while we use both for the OATE estimates. 
We also calculate these adjustments when leaving out each state one at a time from the primary dataset to calculate our secondary variance estimates; these results are available on request. 

\begin{table}[ht]
\centering
    \caption{Univariate summary statistics (1)}
    \label{tab:summarytab1}
\begin{tabular}{rllll}
  \hline
Treatment & Variable & sigma\_zero & sigma\_uu\_i & sigma\_uu\_avg \\ 
  \hline
0 & age\_cat2\_19\_29\_pct & (24.8, 5.5, 48.7) & (24.8, 5.4, 47.5) & (24.8, 5.4, 47.8) \\ 
  1 & age\_cat2\_19\_29\_pct & (24.5, 6, 30.9) & (24.5, 5.9, 29) & (24.5, 5.9, 29) \\ 
  0 & age\_cat2\_30\_39\_pct & (20.6, 2.6, 18.4) & (20.6, 2.5, 17.3) & (20.6, 2.5, 17.2) \\ 
  1 & age\_cat2\_30\_39\_pct & (20.9, 3.4, 20.9) & (20.9, 3.1, 19.1) & (20.9, 3.2, 19.4) \\ 
  0 & age\_cat2\_40\_49\_pct & (22, 2.5, 19.5) & (22, 2.4, 33.3) & (22, 2.3, 18.5) \\ 
  1 & age\_cat2\_40\_49\_pct & (22.2, 2.5, 15.4) & (22.2, 2.3, 13.8) & (22.2, 2.3, 13.8) \\ 
  0 & avg\_adult\_hh\_ratio & (139.7, 18.7, 156.8) & (139.7, 18.9, 153.7) & (139.7, 18.9, 154.2) \\ 
  1 & avg\_adult\_hh\_ratio & (150.8, 27.1, 174.3) & (150.8, 27, 173.8) & (150.8, 27.1, 173.3) \\ 
  0 & citizenship\_pct & (93.6, 5.9, 39.8) & (93.6, 5.9, 39.7) & (93.6, 5.9, 39.3) \\ 
  1 & citizenship\_pct & (90, 11.8, 57.1) & (90, 11.8, 55.4) & (90, 11.7, 55.7) \\ 
  0 & disability\_pct & (11.9, 5, 22.1) & (11.9, 4.8, 21.1) & (11.9, 4.8, 20.7) \\ 
  1 & disability\_pct & (10.5, 5.3, 28.6) & (10.4, 5.3, 26.7) & (10.5, 5.4, 27.2) \\ 
  0 & educ\_hs\_degree\_pct & (29.6, 8.8, 42.7) & (29.6, 8.8, 40.7) & (29.6, 8.8, 41.2) \\ 
  1 & educ\_hs\_degree\_pct & (26.3, 10.6, 43.2) & (26.3, 10.5, 41.8) & (26.3, 10.6, 41.9) \\ 
  0 & educ\_less\_than\_hs\_pct & (11.8, 7.1, 30) & (11.8, 6.9, 30.2) & (11.8, 7, 30.2) \\ 
  1 & educ\_less\_than\_hs\_pct & (11.4, 7.6, 45.3) & (11.4, 7.4, 45) & (11.4, 7.3, 44.6) \\ 
  0 & educ\_some\_college\_pct & (33.8, 5.6, 42.9) & (33.8, 5.1, 45.7) & (33.8, 5.1, 42.3) \\ 
  1 & educ\_some\_college\_pct & (33.5, 7.8, 34.2) & (33.5, 7.4, 32.8) & (33.5, 7.4, 32.9) \\ 
  0 & female\_pct & (50.5, 1.6, 15.8) & (50.5, 1.5, 14.9) & (50.5, 1.5, 15.1) \\ 
  1 & female\_pct & (50.1, 1.6, 15.4) & (50.1, 1.4, 14.2) & (50.1, 1.5, 14.2) \\ 
  0 & foreign\_born\_pct & (10.5, 8.8, 63.2) & (10.5, 8.9, 62.9) & (10.5, 8.9, 62.7) \\ 
  1 & foreign\_born\_pct & (18, 22.2, 76) & (18, 22.2, 75.2) & (18, 22.2, 75.4) \\ 
  0 & hins\_unins\_pct\_2011 & (22.7, 10.5, 51.4) & (22.3, 10, 757.8) & (22.7, 9.2, 50.1) \\ 
  1 & hins\_unins\_pct\_2011 & (19.6, 11.2, 59) & (19.6, 10.8, 51.8) & (19.6, 10.8, 52.4) \\ 
  0 & hins\_unins\_pct\_2012 & (22.4, 9.3, 49.5) & (22.8, 9.4, 468.8) & (22.4, 9.2, 47.8) \\ 
  1 & hins\_unins\_pct\_2012 & (19.4, 9.8, 50.6) & (19.4, 10, 49.7) & (19.4, 10.1, 50.1) \\ 
  0 & hins\_unins\_pct\_2013 & (22, 10.1, 50.1) & (22.3, 9.4, 323.6) & (22, 9.3, 49.2) \\ 
  1 & hins\_unins\_pct\_2013 & (19, 11.1, 49.9) & (19, 10.2, 48.1) & (19, 10.4, 48.6) \\ 
  0 & hispanic\_pct & (11.4, 10.2, 93.9) & (11.5, 10.2, 93.8) & (11.4, 10.2, 93.8) \\ 
  1 & hispanic\_pct & (15.8, 17.7, 97.2) & (15.8, 17.7, 97) & (15.8, 17.7, 97) \\ 
  0 & inc\_pov2\_138\_pct & (22, 8.8, 42.8) & (22, 8.8, 41) & (22, 8.7, 41.8) \\ 
  1 & inc\_pov2\_138\_pct & (19.9, 11.9, 45.6) & (19.9, 11.7, 44.8) & (19.9, 11.7, 43.9) \\ 
  0 & inc\_pov2\_139\_299\_pct & (27.4, 5.1, 27.9) & (27.2, 4.9, 206.1) & (27.4, 4.8, 26.7) \\ 
  1 & inc\_pov2\_139\_299\_pct & (24.9, 8.4, 34.2) & (24.9, 7.8, 34.2) & (24.9, 7.8, 34.1) \\ 
  0 & inc\_pov2\_300\_499\_pct & (24.2, 4.8, 21) & (24.4, 4.8, 317.6) & (24.2, 4.5, 18.4) \\ 
  1 & inc\_pov2\_300\_499\_pct & (23.6, 5.5, 23) & (23.6, 4.9, 22.2) & (23.6, 4.8, 22.2) \\ 
  0 & inc\_pov2\_500\_plus\_pct & (23.7, 9.9, 51.9) & (23.6, 9.8, 118.8) & (23.7, 9.7, 52.1) \\ 
  1 & inc\_pov2\_500\_plus\_pct & (29.3, 18.4, 69.1) & (29.3, 18.5, 68) & (29.3, 18.4, 68) \\ 
  0 & married\_pct & (51.5, 9.1, 59.7) & (51.6, 8.9, 75.5) & (51.5, 9.1, 59.5) \\ 
  1 & married\_pct & (50.7, 9.4, 45.1) & (50.8, 9, 44.2) & (50.7, 9.1, 44.3) \\ 
  0 & missing\_children\_pct & (13.7, 7.4, 36.2) & (13.8, 7.3, 36.2) & (13.7, 7.3, 35.6) \\ 
  1 & missing\_children\_pct & (10.5, 6.6, 41) & (10.5, 6.6, 40.8) & (10.5, 6.6, 40.5) \\ 
  0 & one\_child\_pct & (10.4, 2.5, 13) & (10.3, 2.1, 159) & (10.4, 2.1, 12.3) \\ 
  1 & one\_child\_pct & (11.1, 3.1, 14.3) & (11.1, 2.8, 12.3) & (11.1, 2.9, 12.5) \\ 
  0 & pop\_growth & (100.3, 1.7, 14.3) & (100.1, 1.2, 267.3) & (100.3, 1, 5) \\ 
  1 & pop\_growth & (100.3, 1.9, 13.7) & (100.3, 1.2, 6.2) & (100.3, 1.2, 6.5) \\ 
  0 & race\_white\_pct & (77.8, 22.2, 90.5) & (77.8, 22.1, 90.2) & (77.8, 22.1, 90.4) \\ 
  1 & race\_white\_pct & (73.9, 25.4, 91.9) & (73.9, 25.5, 91.8) & (73.9, 25.5, 91.8) \\ 
  0 & repub\_gov & (95.9, 0, 100) & (95.9, 0, 100) & (95.9, 0, 100) \\ 
  1 & repub\_gov & (30.9, 100, 100) & (30.9, 100, 100) & (30.9, 100, 100) \\
  0 & repub\_lower\_control & (98.8, 0, 100) & (98.8, 0, 100) & (98.8, 0, 100) \\ 
  1 & repub\_lower\_control & (23.9, 0, 100) & (23.9, 0, 100) & (23.9, 0, 100) \\ 
  0 & repub\_total\_control & (94.7, 0, 100) & (94.7, 0, 100) & (94.7, 0, 100) \\ 
  1 & repub\_total\_control & (21, 0, 100) & (21, 0, 100) & (21, 0, 100) \\ 
   \hline
\end{tabular}
\end{table}

\begin{table}[ht]
\centering
    \caption{Univariate summary statistics (2)}
    \label{tab:summarytab2}
\begin{tabular}{rllll}
  \hline
Treat & Variable & sigma\_zero & sigma\_uu\_i & sigma\_uu\_avg \\ 
  \hline
  0 & student\_pct & (11.5, 3.9, 41) & (11.5, 3.9, 53.8) & (11.5, 3.8, 41.3) \\ 
  1 & student\_pct & (11.7, 3.4, 29.5) & (11.7, 3.5, 28.1) & (11.7, 3.5, 28) \\ 
  0 & three\_plus\_child\_pct & (5.2, 2.2, 25.3) & (5.2, 2.1, 42.9) & (5.2, 2.2, 23.2) \\ 
  1 & three\_plus\_child\_pct & (5.2, 2, 14.1) & (5.2, 1.7, 13.5) & (5.2, 1.7, 13.3) \\ 
  0 & two\_child\_pct & (8.9, 2.8, 15) & (9, 2.5, 122.8) & (8.9, 2.5, 15.1) \\ 
  1 & two\_child\_pct & (9.7, 3.5, 15) & (9.7, 3.2, 13.4) & (9.7, 3.2, 13.6) \\ 
  0 & unemployed\_pct\_2011 & (9.4, 4.6, 21.7) & (9.4, 4, 23.2) & (9.4, 4, 18.9) \\ 
  1 & unemployed\_pct\_2011 & (10.2, 4.6, 25.5) & (10.2, 3.9, 23.9) & (10.2, 4, 22.6) \\ 
  0 & unemployed\_pct\_2012 & (8.8, 4.2, 22.4) & (8.6, 3.6, 180.8) & (8.8, 3.6, 18.6) \\ 
  1 & unemployed\_pct\_2012 & (9.4, 4.5, 28.3) & (9.4, 4.3, 23.6) & (9.4, 4.3, 23.6) \\ 
  0 & unemployed\_pct\_2013 & (8, 3.7, 19) & (8.1, 3.4, 146.7) & (8, 3.4, 17.6) \\ 
  1 & unemployed\_pct\_2013 & (8.4, 3.7, 23.4) & (8.4, 3.5, 20.2) & (8.4, 3.5, 20.6) \\ 
  0 & urban\_pct & (0.7, 0.4, 0.9) & (0.7, 0.4, 0.9) & (0.7, 0.4, 0.9) \\ 
  1 & urban\_pct & (0.8, 0.3, 0.9) & (0.8, 0.3, 0.9) & (0.8, 0.3, 0.9) \\ 
  \hline
\end{tabular}
\end{table}

Table~\ref{tab:extreme1} displays the frequency that the adjusted covariates fell outside of the support of the unadjusted dataset for the (control, treatment) groups. We see that that frequency is higher for our preferred adjustment (sigma\_uu\_i) than the secondary adjustment (sigma\_uu\_avg), though the counts are low in either case. The counts are also slightly higher in the control group than the treated group; this may be a function of the smaller number of CPUMAs for the control versus the treated group $(n_0 = 414; n_1 = 515)$. 

\begin{table}[ht]
\centering
    \caption{Extreme covariate adjustments (control, treated)}
    \label{tab:extreme1}
\begin{tabular}{lll}
  \hline
Variables & sigma\_uu\_i & sigma\_uu\_avg \\ 
  \hline
age\_cat2\_19\_29\_pct & (0, 0) & (0, 0) \\ 
  age\_cat2\_30\_39\_pct & (0, 0) & (0, 0) \\ 
  age\_cat2\_40\_49\_pct & (2, 0) & (0, 0) \\ 
  avg\_adult\_hh\_ratio & (0, 0) & (0, 0) \\ 
  citizenship\_pct & (2, 0) & (2, 0) \\ 
  disability\_pct & (1, 2) & (0, 2) \\ 
  educ\_hs\_degree\_pct & (0, 0) & (0, 0) \\ 
  educ\_less\_than\_hs\_pct & (1, 1) & (2, 1) \\ 
  educ\_some\_college\_pct & (1, 0) & (0, 0) \\ 
  female\_pct & (0, 0) & (0, 0) \\ 
  foreign\_born\_pct & (0, 1) & (0, 1) \\ 
  hins\_unins\_pct\_2011 & (4, 0) & (0, 0) \\ 
  hins\_unins\_pct\_2012 & (3, 0) & (0, 1) \\ 
  hins\_unins\_pct\_2013 & (4, 0) & (1, 1) \\ 
  hispanic\_pct & (1, 0) & (1, 0) \\ 
  inc\_pov2\_138\_pct & (2, 1) & (2, 1) \\ 
  inc\_pov2\_139\_299\_pct & (4, 1) & (1, 1) \\ 
  inc\_pov2\_300\_499\_pct & (3, 0) & (0, 0) \\ 
  inc\_pov2\_500\_plus\_pct & (4, 0) & (2, 0) \\ 
  married\_pct & (2, 0) & (1, 0) \\ 
  missing\_children\_pct & (2, 1) & (1, 1) \\ 
  one\_child\_pct & (4, 0) & (0, 0) \\ 
  pop\_growth & (3, 0) & (0, 0) \\ 
  race\_white\_pct & (1, 1) & (1, 1) \\ 
  student\_pct & (2, 0) & (1, 0) \\ 
  three\_plus\_child\_pct & (3, 2) & (2, 1) \\ 
  two\_child\_pct & (4, 0) & (1, 1) \\ 
  unemployed\_pct\_2011 & (1, 0) & (0, 0) \\ 
  unemployed\_pct\_2012 & (2, 0) & (0, 0) \\ 
  unemployed\_pct\_2013 & (3, 0) & (0, 0) \\ 
   \hline
\end{tabular}
\end{table}

Figure~\ref{fig:corrmatrix} displays the Pearson's correlation coefficients for the bivariate relationships between the covariates on the unadjusted dataset. The relationships are quite similar when using the adjusted datasets.

\begin{figure}[]
\begin{center}
    \caption{Correlation matrix (unadjusted covariates)}
    \label{fig:corrmatrix}
    \includegraphics[scale=0.6]{01_Plots/correlation-plot-c1-sigma-zero.png}
\end{center}
\end{figure}


\section{Weight Diagnostics}
\label{ssec:balancetables}

Table~\ref{tab:baltab1} displays the differences between the weighted mean covariate values of the expansion region and the mean of the non-expansion region for our primary dataset and with the early expansion states excluded (calculated using our covariate adjustments). The weights presented here are from the H-SBW estimator. The values under each column of ``Primary'' and ``Early Excluded'' are in the following format: (unweighted difference, weighted difference). Additional results are available on request.

\begin{table}[ht]
\centering
    \caption{Balance Table}
    \label{tab:baltab1}
\begin{tabular}{lll}
  \hline
Variables & Preferred & Early Excluded \\ 
  \hline
age\_cat2\_19\_29\_pct & (-0.35, -0.2) & (-0.62, -0.07) \\ 
  age\_cat2\_30\_39\_pct & (0.33, -0.16) & (-0.11, -0.04) \\ 
  age\_cat2\_40\_49\_pct & (0.22, -0.41) & (0.04, -0.50) \\ 
  avg\_adult\_hh\_ratio & (11.09, -0.40) & (3.18, -0.18) \\ 
  citizenship\_pct & (-3.56, -1.93) & (-0.20, -1.80) \\ 
  disability\_pct & (-1.46, 0.57) & (-0.21, 0.73) \\ 
  educ\_hs\_degree\_pct & (-3.35, 0.99) & (-1.05, 1.14) \\ 
  educ\_less\_than\_hs\_pct & (-0.40, 1.09) & (-1.27, 1.27) \\ 
  educ\_some\_college\_pct & (-0.30, 0.69) & (0.44, 0.53) \\ 
  female\_pct & (-0.34, -0.87) & (-0.25, -1.00) \\ 
  foreign\_born\_pct & (7.51, 1.99) & (0.97, 1.99) \\ 
  hins\_unins\_pct\_2011 & (-2.64, 0.48) & (-3.09, 0.45) \\ 
  hins\_unins\_pct\_2012 & (-3.37, -0.42) & (-3.9, -0.42) \\ 
  hins\_unins\_pct\_2013 & (-3.27, -0.32) & (-3.77, -0.32) \\ 
  hispanic\_pct & (4.36, 1.00) & (-1.44, 1.00) \\ 
  inc\_pov2\_138\_pct & (-2.08, 0.93) & (-1.36, 0.60) \\ 
  inc\_pov2\_139\_299\_pct & (-2.34, 0.91) & (-1.45, 0.86) \\ 
  inc\_pov2\_300\_499\_pct & (-0.8, -0.39) & (0.05, -0.34) \\ 
  inc\_pov2\_500\_plus\_pct & (5.69, -1.91) & (3.06, -1.91) \\ 
  married\_pct & (-0.80, -0.38) & (-0.34, -0.60) \\ 
  missing\_children\_pct & (-3.21, -0.36) & (-1.93, -0.08) \\ 
  one\_child\_pct & (0.80, -0.22) & (0.21, -0.25) \\ 
  pop\_growth & (0.11, -0.10) & (-0.08, -0.15) \\ 
  race\_white\_pct & (-3.87, 1.00) & (0.31, 1.00) \\ 
  repub\_gov & (-65.02, -22.06) & (-55.02, -21.97) \\ 
  repub\_lower\_control & (-74.91, -22.23) & (-57.24, -20.66) \\ 
  repub\_total\_control & (-73.72, -25.00) & (-58.2, -25) \\ 
  student\_pct & (0.29, -0.41) & (0.12, -0.38) \\ 
  three\_plus\_child\_pct & (0.02, -0.14) & (-0.17, -0.15) \\ 
  two\_child\_pct & (0.66, -0.47) & (0.08, -0.55) \\ 
  unemployed\_pct\_2011 & (0.80, 0.15) & (0.68, 0.15) \\ 
  unemployed\_pct\_2012 & (0.75, 0) & (0.58, 0.00) \\ 
  unemployed\_pct\_2013 & (0.29, -0.25) & (0.10, -0.25) \\ 
  urban\_pct & (0.08, -0.07) & (0.03, -0.06) \\ 
   \hline
\end{tabular}
\end{table}

Table~\ref{tab:oatedist1} displays the difference in means from the overlap region from the control and treated regions on the primary dataset and with the early expansion states excluded. The numbers are displayed as (absolute distance from control region, absolute distance from treated region). These distances are computed using our adjusted datasets. 

\begin{table}[ht]
\centering
    \caption{Overlap region distance from control, treatment regions (1)}
    \label{tab:oatedist1}
\begin{tabular}{lll}
  \hline
Variable & Preferred & Early expansion excluded \\ 
  \hline
age\_cat2\_19\_29\_pct & (-1.39, -1.03) & (-1.34, -0.71) \\ 
  age\_cat2\_30\_39\_pct & (-0.79, -1.12) & (-0.82, -0.72) \\ 
  age\_cat2\_40\_49\_pct & (0.24, 0.02) & (0.2, 0.16) \\ 
  avg\_adult\_hh\_ratio & (-3.72, -14.81) & (-4.07, -7.25) \\ 
  citizenship\_pct & (1.7, 5.26) & (1.97, 2.17) \\ 
  disability\_pct & (0.29, 1.75) & (0.38, 0.59) \\ 
  educ\_hs\_degree\_pct & (1.14, 4.49) & (1.22, 2.27) \\ 
  educ\_less\_than\_hs\_pct & (-1.67, -1.27) & (-1.58, -0.31) \\ 
  educ\_some\_college\_pct & (0.47, 0.78) & (0.75, 0.31) \\ 
  female\_pct & (-0.11, 0.24) & (-0.16, 0.09) \\ 
  foreign\_born\_pct & (-2.23, -9.73) & (-2.8, -3.77) \\ 
  hins\_unins\_pct\_2011 & (-2.89, -0.24) & (-2.89, 0.19) \\ 
  hins\_unins\_pct\_2012 & (-3.68, -0.31) & (-3.68, 0.22) \\ 
  hins\_unins\_pct\_2013 & (-3.82, -0.55) & (-3.84, -0.07) \\ 
  hispanic\_pct & (-4.48, -8.84) & (-4.57, -3.13) \\ 
  inc\_pov2\_138\_pct & (-1.44, 0.64) & (-1.31, 0.05) \\ 
  inc\_pov2\_139\_299\_pct & (-0.18, 2.16) & (0, 1.45) \\ 
  inc\_pov2\_300\_499\_pct & (0.88, 1.68) & (0.96, 0.91) \\ 
  inc\_pov2\_500\_plus\_pct & (0.98, -4.71) & (0.58, -2.49) \\ 
  married\_pct & (1.6, 2.4) & (1.84, 2.18) \\ 
  missing\_children\_pct & (-0.72, 2.5) & (-0.62, 1.31) \\ 
  one\_child\_pct & (-0.27, -1.07) & (-0.32, -0.53) \\ 
  pop\_growth & (-0.27, -0.39) & (-0.29, -0.2) \\ 
  race\_white\_pct & (4.48, 8.35) & (5.15, 4.84) \\ 
  repub\_gov & (-11.88, 53.14) & (-11.94, 43.07) \\ 
  repub\_lower\_control & (-5.7, 69.21) & (-4.31, 52.92) \\ 
  repub\_total\_control & (-17.58, 56.14) & (-16.26, 41.94) \\ 
  student\_pct & (-0.5, -0.79) & (-0.5, -0.62) \\ 
  three\_plus\_child\_pct & (-0.17, -0.19) & (-0.1, 0.07) \\ 
  two\_child\_pct & (-0.15, -0.81) & (-0.16, -0.24) \\ 
  unemployed\_pct\_2011 & (0.09, -0.71) & (0.08, -0.6) \\ 
  unemployed\_pct\_2012 & (0.05, -0.69) & (0.02, -0.55) \\ 
  unemployed\_pct\_2013 & (-0.42, -0.71) & (-0.44, -0.54) \\ 
  urban\_pct & (-0.05, -0.13) & (-0.05, -0.08) \\ 
   \hline
\end{tabular}
\end{table}

Table ~\ref{tab:oatestateweightsc1} displays the sum of CPUMA-level weights within each state for the OATE region, using the preferred covariate adjustment (``sigma\_uu\_i''), as well as no adjustment (``sigma\_zero'') and the secondary adjustment (``sigma\_uu\_avg''). We display all states where the total sum of weights within states is greater than one for any of the covariate sets. The total sum of weights for each set is standardized to sum to 100 for each set of weights. Table ~\ref{tab:oatestateweightsc2} displays the same results with the early expansion states excluded.

\begin{table}[ht]
\centering
\caption{OATE weights summed by state by covariate adjustment, primary dataset}
\label{tab:oatestateweightsc1}
\begin{tabular}{lrrrr}
  \hline
State & Treatment & sigma\_uu\_i & sigma\_uu\_avg & sigma\_zero \\ 
  \hline
OH & 1 & 34.41 & 34.51 & 36.75 \\ 
  MI & 1 & 26.76 & 27.27 & 27.97 \\ 
  AR & 1 & 15.99 & 15.87 & 12.28 \\ 
  PA & 0 & 15.51 & 16.94 & 18.95 \\ 
  MO & 0 & 15.49 & 15.27 & 11.66 \\ 
  WI & 0 & 13.37 & 13.39 & 13.66 \\ 
  FL & 0 & 11.40 & 12.15 & 11.22 \\ 
  AZ & 1 & 9.00 & 8.40 & 8.96 \\ 
  ME & 0 & 6.90 & 6.42 & 6.17 \\ 
  TN & 0 & 6.30 & 6.83 & 5.77 \\ 
  IN & 0 & 4.67 & 4.45 & 5.56 \\ 
  IA & 1 & 4.62 & 4.78 & 5.73 \\ 
  NC & 0 & 4.50 & 4.12 & 4.69 \\ 
  NJ & 1 & 4.16 & 4.14 & 4.57 \\ 
  NE & 0 & 3.62 & 4.39 & 3.84 \\ 
  KS & 0 & 3.29 & 2.39 & 2.82 \\ 
  VA & 0 & 3.03 & 2.18 & 1.79 \\ 
  AL & 0 & 2.73 & 2.20 & 2.31 \\ 
  UT & 0 & 2.34 & 1.83 & 1.82 \\ 
  ND & 1 & 2.32 & 2.76 & 2.13 \\ 
  TX & 0 & 2.27 & 1.91 & 2.85 \\ 
  NV & 1 & 1.52 & 0.75 & 0.71 \\ 
  NM & 1 & 1.22 & 1.54 & 0.90 \\ 
  OK & 0 & 1.21 & 1.52 & 0.81 \\ 
  AK & 0 & 0.77 & 1.37 & 1.55 \\ 
  \hline
\end{tabular}
\end{table}

\begin{table}[ht]
\centering
\caption{OATE weights summed by state by covariate adjustment, early expansion excluded}
\label{tab:oatestateweightsc2}
\begin{tabular}{lrrrr}
  \hline
State & Treatment & sigma\_uu\_i & sigma\_uu\_avg & sigma\_zero \\ 
  \hline
OH & 1 & 34.66 & 34.81 & 36.92 \\ 
  MI & 1 & 27.65 & 28.05 & 28.38 \\ 
  AR & 1 & 16.05 & 16.08 & 12.57 \\ 
  MO & 0 & 15.61 & 15.43 & 11.89 \\ 
  PA & 0 & 14.70 & 15.87 & 17.58 \\ 
  WI & 0 & 14.58 & 14.43 & 14.52 \\ 
  FL & 0 & 10.78 & 11.28 & 10.51 \\ 
  AZ & 1 & 9.06 & 8.50 & 8.92 \\ 
  TN & 0 & 6.50 & 7.31 & 5.85 \\ 
  ME & 0 & 5.52 & 5.06 & 4.92 \\ 
  IA & 1 & 4.78 & 5.02 & 6.15 \\ 
  IN & 0 & 4.67 & 4.41 & 5.97 \\ 
  NC & 0 & 4.63 & 4.25 & 4.93 \\ 
  NE & 0 & 4.03 & 4.83 & 4.33 \\ 
  KS & 0 & 3.82 & 2.88 & 3.08 \\ 
  NV & 1 & 3.69 & 2.74 & 2.97 \\ 
  AL & 0 & 3.15 & 2.55 & 2.52 \\ 
  UT & 0 & 2.73 & 2.14 & 2.15 \\ 
  TX & 0 & 2.56 & 2.28 & 3.17 \\ 
  ND & 1 & 2.28 & 2.48 & 2.14 \\ 
  VA & 0 & 2.01 & 1.40 & 1.46 \\ 
  NM & 1 & 1.84 & 2.32 & 1.95 \\ 
  OK & 0 & 1.29 & 1.70 & 0.82 \\ 
  AK & 0 & 0.75 & 1.30 & 1.35 \\ 
  SC & 0 & 0.57 & 0.54 & 1.10 \\ 
  MT & 0 & 0.44 & 0.64 & 0.68 \\ 
  WY & 0 & 0.39 & 0.26 & 0.65 \\ 
  MS & 0 & 0.38 & 0.49 & 0.79 \\ 
  LA & 0 & 0.32 & 0.36 & 0.43 \\ 
  GA & 0 & 0.28 & 0.30 & 0.70 \\ 
  ID & 0 & 0.17 & 0.11 & 0.25 \\ 
  SD & 0 & 0.11 & 0.19 & 0.33 \\ 
  HI & 1 & 0.00 & 0.00 & 0.00 \\ 
  MD & 1 & 0.00 & 0.00 & 0.00 \\ 
  KY & 1 & 0.00 & 0.00 & 0.00 \\ 
  IL & 1 & 0.00 & 0.00 & 0.00 \\ 
  RI & 1 & 0.00 & 0.00 & 0.00 \\ 
  NH & 1 & 0.00 & 0.00 & 0.00 \\ 
  CO & 1 & 0.00 & 0.00 & 0.00 \\ 
  WV & 1 & 0.00 & 0.00 & 0.00 \\ 
  OR & 1 & 0.00 & 0.00 & 0.00 \\ 
   \hline
\end{tabular}
\end{table}


\section{Additional Results}\label{app:allresults}

Table~\ref{tab:pretxpredfull} presents the full model validation results that also include estimators using the Oaxaca-Blinder OLS and GLS weights (see, e.g, \cite{kline2011oaxaca}), with the rows ordered by RMSE of the prediction errors. We find in general that these estimators perform quite poorly, again showing that the cost of extrapolation is quite high in our application. We see some particularly extreme numbers for the error using the heterogeneous adjustment in 2012: one reason for this may be because in practice that the heterogeneous adjustment is more likely to impute values that fall outside of the support of the data. In some cases we have found that these imputations are quite extreme. The estimators that extrapolate from the data are then more likely to fit to these values, leading to poor estimator performance overall. This result makes the heterogeneous adjustment potentially less desirable, particularly for estimators that allow extrapolation.

\begin{table}[ht]
\caption{Estimator pre-treatment outcome prediction error}\label{tab:pretxpredfull}
\begin{tabular}{llrrr}
  \hline
Sigma estimate & Estimator & 2011 error & 2012 error & RMSE \\ 
  \hline
Homogeneous & SBW & -0.18 & -0.22 & 0.20 \\ 
  Homogeneous & H-SBW & -0.24 & -0.21 & 0.23 \\ 
  Heterogeneous & SBW & -0.25 & -0.30 & 0.27 \\ 
  Heterogeneous & H-SBW & -0.32 & -0.39 & 0.36 \\ 
  Homogeneous & BC-SBW & -0.42 & -0.35 & 0.39 \\ 
  Heterogeneous & BC-SBW & -0.45 & -0.39 & 0.42 \\ 
  None & SBW & -0.50 & -0.61 & 0.56 \\ 
  None & H-SBW & -0.52 & -0.61 & 0.57 \\ 
  Homogeneous & BC-HSBW & -0.53 & -0.62 & 0.58 \\ 
  Heterogeneous & BC-HSBW & -0.53 & -0.72 & 0.63 \\ 
  None & BC-SBW & -0.82 & -0.93 & 0.88 \\ 
  None & OLS & -0.91 & -0.84 & 0.88 \\ 
  None & GLS & -0.87 & -0.91 & 0.89 \\ 
  None & BC-HSBW & -0.93 & -0.99 & 0.96 \\ 
  Homogeneous & OLS & -1.75 & -1.21 & 1.50 \\ 
  Homogeneous & GLS & -1.76 & -1.45 & 1.61 \\ 
  Heterogeneous & GLS & -1.18 & -17.60 & 12.47 \\ 
  Heterogeneous & OLS & -0.85 & -22.90 & 16.21 \\ 
   \hline
\end{tabular}
\end{table}


Table~\ref{tab:confintmain} displays the point estimates and confidence intervals from all estimators, including estimates calculated on a second version of the adjusted data where we model a separate $\kappa$ for all values (the ``heterogeneous adjustment''). ``Homogeneous'' corresponds to our preferred covariate adjustment, and are the results presented in the main paper.

\begin{table}[h!]
\centering
\begin{threeparttable}
\caption{Point estimates and confidence intervals: primary dataset}
\label{tab:confintmain}
\begin{tabular}{llrl}
  \hline
Weight type & Adjustment & Psihat & 95\% CI \\ 
  \hline
H-SBW & Homogeneous & -2.33 & (-3.47, -1.19) \\ 
  H-SBW & Heterogeneous & -2.24 & (-3.4, -1.08) \\ 
  H-SBW & None & -2.34 & (-2.85, -1.82) \\ 
  BC-HSBW & Homogeneous & -2.05 & (-3.22, -0.87) \\ 
  BC-HSBW & Heterogeneous & -1.98 & (-3.13, -0.83) \\ 
  BC-HSBW & None & -2.22 & (-2.87, -1.56) \\ 
  SBW & Homogeneous & -2.35 & (-3.67, -1.03) \\ 
  SBW & Heterogeneous & -2.28 & (-3.5, -1.05) \\ 
  SBW & None & -2.39 & (-2.95, -1.83) \\ 
  BC-SBW & Homogeneous & -2.07 & (-3.07, -1.06) \\ 
  BC-SBW & Heterogeneous & -2.00 & (-3, -0.99) \\ 
  BC-SBW & None & -2.19 & (-2.9, -1.49) \\ 
   \hline
\end{tabular}
    \begin{tablenotes}
      \item[] Note: Confidence intervals are estimated using the leave-one-state-out jackknife and the standard normal quantiles.
    \end{tablenotes}
\end{threeparttable}
\end{table}

Table~\ref{tab:confintmainc2} presents the same results when excluding the early expansion states.

\begin{table}[h!]
\centering
\begin{threeparttable}
\caption{Point estimates and confidence intervals: early expansion excluded}
\label{tab:confintmainc2}
\begin{tabular}{llrl}
  \hline
Weight type & Adjustment & Psihat & 95\% CI \\ 
  \hline
H-SBW & Homogeneous & -2.09 & (-3.15, -1.03) \\ 
  H-SBW & Heterogeneous & -2.06 & (-3.26, -0.87) \\ 
  H-SBW & None & -2.28 & (-2.82, -1.74) \\ 
  BC-HSBW & Homogeneous & -1.94 & (-3.17, -0.72) \\ 
  BC-HSBW & Heterogeneous & -1.93 & (-3.42, -0.45) \\ 
  BC-HSBW & None & -2.22 & (-3.07, -1.38) \\ 
  SBW & Homogeneous & -2.05 & (-3.10, -1.00) \\ 
  SBW & Heterogeneous & -2.03 & (-3.24, -0.82) \\ 
  SBW & None & -2.21 & (-2.71, -1.72) \\ 
  BC-SBW & Homogeneous & -1.99 & (-3.22, -0.77) \\ 
  BC-SBW & Heterogeneous & -2.00 & (-3.52, -0.47) \\ 
  BC-SBW & None & -2.23 & (-3.05, -1.40) \\ 
   \hline
\end{tabular}
    \begin{tablenotes}
      \item[] Note: Confidence intervals are estimated using the leave-one-state-out jackknife and the standard normal quantiles.
    \end{tablenotes}
\end{threeparttable}
\end{table}

Figure~\ref{fig:loostateplot} displays the change in our estimator when removing each state for all four of our estimators on the adjusted dataset (``homogeneous'') and the unadjusted dataset (``none''). These results condition on the covariate adjustment, but are similar when recalculating the entire adjustment procedure. Additional results are available on request.

\begin{figure}[H]
\begin{center}
    \caption{Leave-one-state-out point estimates minus primary estimate}
    \label{fig:loostateplot}
    \includegraphics[scale=0.6]{01_Plots/loostate-sensitivityc1-proc-uu-avg.png}
\end{center}
\subcaption{Colors reflect the magnitude of the difference in the estimates when subtracting the original estimate from the estimate that excludes the specified state. The values in the right panel are on the unadjusted data.}
\end{figure}

\clearpage


%The synthetic controls approach chooses the $\gamma$ that minimizes the weighted L2-squared distance of the covariates using a diagonal weighting matrix $V$. $V$ is then chosen to minimize the mean-square error of the weighted difference in pre-treatment outcomes. Letting $Z_a$ be the matrix of pre-treatment outcomes for treatment group $A = a$, the synthetic controls algorithm solves the following optimization problem for a fixed $V$:

\begin{align}
\gamma(V) = \arg\min_{\tilde{\gamma}(V^\star)} = (\bar{X}_1 - X_0^T\tilde{\gamma})'V(\bar{X}_1 - X_0^T\tilde{\gamma}) 
\end{align}

This is the ``inner'' optimization. $V^\star$ is then determined in an ``outer'' optimization to minimize the imbalances in the pre-treatment outcomes $Z$:

\begin{align}
    V^\star = \arg\min_V (\bar{Z}_1 - Z_0^T\gamma(V))'(\bar{Z}_1 - Z_0^T\gamma(V))
\end{align}

In applications the covariate matrix $X_a$ may contain some elements of $Z_a$. In cases where $X_a$ contains all pre-treatment outcomes, \cite{kaul2015synthetic} has shown that the predictor weights $V^\star$ will give no weight to auxillary covariates (covariates that are not the pre-treatment outcomes), rendering these irrelevant to the model. 

While in practice $V$ is often learned on the same data as the weights, we consider the case where we use cross-validation to choose $V$, as proposed by \cite{abadie2015comparative}. Assume we can divide our pre-treatment data into a training data from periods $T = 1, ..., T - l - 1$, a validation period from periods $T - l, ..., T - 1$, and a post-treatment period at time $T$. To make this discussion more general, assume that we are evaluating a set of candidate models $\mathcal{M}$ on the validation data (where for the synthetic controls algorithm we can think of this as the set of all possible weighting matrices $V$). Let $\bar{Y}^a_{a', t}$ 
be the mean potential outcome under treatment $A = a$ for treatment group $A = a'$ at time $t$ (where $t$ occurs during the validation period). Let $\hat{\bar{Y}}^a_{a'', t}(m)$ be an estimator of that potential outcome at time $t$ using model $m$, which was trained during the training period using data from treatment group $A = a''$. Finally, let $\bar{Y}_{a'}^{a_T}$ be the post-treatment estimand, where $\hat{Y}^a_{a'', T}(m)$ is the estimator using model $m$ trained using validation period data. This learning procedure implicitly assumes that:

\begin{align*}
m^\star = \min_{m \in \mathcal{M}}\sum_{T - l}^{T-1}\|\hat{Y}^0_{0, t}(m) - \bar{Y}^0_{1, t}\| = \min_{m \in \mathcal{M}}\mathbb{E}\{\|\hat{Y}^0_{0, T}(m) - \bar{Y}^0_{1, T}\|\}
\end{align*}

In other words, we select our model using the empirical loss in the validation period as a proxy for the expected loss in the post-treatment time-period.\footnote{It is possible that multiple models in $\mathcal{M}$ either perfectly predict the pre-treatment outcomes, or predict them equally well. In this case we would require an additional criteria to choose the optimal model (see, e.g, \cite{becker2017cross}}. This makes intuitive sense in the typical synthetic controls setting where the estimand is the ETT since we observe $Y^0_{sct}$ for $t < T$. When synthetic controls are used to estimate the ETC, this strategy alone is insufficient as we never observe $Y^1_{sct}$ (or a mean-unbiased proxy) prior to treatment for any unit. We therefore cannot easily use pre-treatment outcomes to optimally select variables or determine relative covariate importance without stronger assumptions.

One such assumption is the following:

\begin{align*}\label{assumption:second}
m^\star = \min_{m \in \mathcal{M}}\sum_{T - l}^{T-1}\|\hat{Y}^0_{1, t}(m) - \hat{Y}^0_{0, t}\| = \min_{m \in \mathcal{M}}\mathbb{E}\{\|\hat{Y}^1_{1, T}(m) - \bar{Y}^1_{0, T}\|\}
\end{align*}

We call this assumption ``counterfactual risk invariance.'' In other words, we assume that the model that minimizes the validation-period risk also minimizes the post-treatment risk. We caution that this is a strong assumption for conducting any form of variable selection or covariate weighting in this setting. 

As a practical example, we highlight the potential confounding role of Republican governance for our counterfactual estimate. Republican governance is a strong predictor of a state's decision to expand Medicaid \cite{courtemanche2017early}. Moreover, existing evidence prior to Medicaid expansion showed that Medicaid take-up rates were lower in more conservative states \cite{sommers2012understanding}. However, when generating their synthetic control weights to estimate the ETT, \cite{courtemanche2017early} and \cite{kaestner2017effects} do not control for these factors. \footnote{\cite{courtemanche2017early} does control for Republican governor in their regression model and they find that it is a statistically significant predictor of 2013 uninsurance rates. One reason they may not control for this in the synthetic control model is practical: it is challenging to balance this covariate using control data without extrapolating from the data.} However, it is clear that if take-up rates depend on governance, we may expect this to be a strong confounder of $Y^1$ and hence confound the ETC, even if arguably it is not a confounder of $Y^0$ (and hence not a confounder for the ETT).

We demonstrate this in our application by conducting a variable importance analysis. Specifically, we remove the balance constraints from the Republican governance indicators and examine how our estimates of $\hat{\psi}^1$ change. Letting $\hat{\psi}^1_s$ be the estimate when removing the Republican governance indicators (or more generally, the covariate matrix $S$ where $X = (R, S)$). We subtract our original point estimate $\hat{\psi}^1_0$ from $\hat{\psi}^1_s$ to generate the difference $\hat{\Delta}^1$. This difference tells us about the direction of the bias our estimate of $\hat{\psi}^1$ would incur when we do attempt to constrain the imbalance in covariate $S$. Our hypothesis implies that we should expect $\hat{\Delta}_s^1 < 0$: that is, keeping all other covariates (roughly) fixed, we expect the predicted uninsurance rate will decrease when as the level of Republican governance decreases. In addition to the Republican governance indicators, we also examine four other covariate groups: pre-treatment uninsurance rates and pre-treatment unemployment rates, and three sets of different demographic indicators, which we detail in Appendix E.\footnote{We caution that our results do not imply that Republican governance is not an important confounder of $Y^0_{1, T}$ since we do not analyze this directly.} 

We point out that modeling the ETC requires greater justification of the covariates used to predict treatment response than for the ETT, and that using the standard synthetic controls variable weighting procedure is unlikely to be optimal for this purpose.\footnote{Our analysis assumes no unmeasured confounding and a linear model for $\mu_a$. By contrast, synthetic controls are frequently motivated by a linear factor model for $\mu_0$. \cite{abadie2010synthetic} and \cite{ferman2016revisiting} outline conditions where this method is consistent as the number of pre-treatment outcomes goes to infinity, in particular because the method balances the unobserved factor loadings. Analogous to our analysis, if we assume $\mu_{a, T}$ both follow a linear factor model, identification of the ETC requires that the unobserved factor loadings that confound $Y^1_T$ are the same that confound $Y^0_t$. Under this assumption, we might be able to show that the synthetic control estimator is consistent in this setting. However, the tuning procedure to determine the predictor weights may again be sub-optimal from a finite-sample bias perspective, depending again on how the covariates (or unobserved factors) that are most predictive of treatment response vary between treatment groups and on their associations with the potential outcomes.}

For our application, we constrain $\delta$ to be 0.05 percentage points (out of 100) for pre-treatment outcomes, 0.15 percentage points for pre-treatment unemployment rates, and 25 percentage points for the Republican governance indicators. We believe these covariates are most likely to predict treatment response. While we believe that Republican governance is an important covariate to balance, we are unable to reduce the constraints further given the support of the data. For the remaining covariates, we let $\delta$ be 0.5 percentage points for average population growth and household to adult ratio, 1 percentage point for female, Hispanic ethnicity, white race, age category, disability, and number of children category; 2 percentage points for urban, citizenship, education category, income-to-poverty category, student, and foreign-born, again choosing these constraints with respect to both feasibility and extreme weight concerns. 
\begin{figure}[H]
\begin{center}
    \caption{Estimator sensitivity to states}
    \label{fig:loostateplot}
    \includegraphics[scale=0.6]{01_Plots/loostate-sensitivityc1-state-uu-i.png}
\end{center}
\end{figure}

\subsection{Covariate importance}

We also investigate our hypothesis that factors associated with Governance are associated with treatment response. As discussed above, we first remove the balance constraints on the Republican governance indicators and estimate $\hat{\psi}^1_v$, and then subtract our original ETC point estimate from this quantity to generate $\hat{\Delta}_v^1$. Because this quantity does not reflect a clear population target, instead of confidence intervals, we present the minimum and maximum leave-one-state-out values in parentheses next to the original estimate.

For the H-SBW estimator we calculate $\hat{\Delta}^1$ equal to -0.69 (min = -0.83, max = -0.42) and equal to -0.79 (min = -0.90, max = -0.66) on our unadjusted dataset. In other words, our primary estimated treatment effect moved 0.78 percentage points further away from zero when we excluded the Republican governance indicators. This reflects a 33 percent decrease in our point estimate, a not-unsubstantial difference. Moreover, all of these estimates were less than zero, regardless of whether we conditioned on the covariate adjustment or not, regardless of whether we remove the early expansion states or not, and when removing each state. Across all specifications that we ran the minimum change we calculated was -1.34 and the maximum was -0.36. Additional distributional results across all leave-one-state-out estimates are available in Appendix E, Table~\ref{tab:rdiffc1}. Figure~\ref{fig:repub} displays our estimates of $\hat{\Delta}_v^1$ on our primary dataset and removing early expansion states (conditional on the covariate adjustment). 

We also consider four other covariate sets. We find that our estimates are most sensitive to controlling for pre-treatment outcomes and unemployment rates. This is not unexpected: all else equal, the expansion region had much lower pre-treatment uninsurance rates. If we do not control for these covariates, the comparable region will likely have lower pre-treament uninsurance rates, causing the estimated counterfactual to be closer to zero. The effect estimates were less sensitive to the removal of other covariate groups, and all point estimates are available in Appendix E, Table~\ref{tab:ptests}.

\begin{figure}[H]
\begin{center}
    \caption{Removing Republican Governance Indicators}
    \label{fig:repub}
    \includegraphics[scale=0.6]{01_Plots/repub-diff-all-estimators.png}
\end{center}
\end{figure}

These results highlight the importance of Republican governance in our counterfactual outcome model of $Y^1$. If the models specified by \cite{kaestner2017effects} and \cite{courtemanche2017early} are correct (that is, they correctly omit Republican governance from their balancing weights for estimating $\bar{Y}^0_{1, T}$), this would suggest treatment effect heterogeneity with respect to Republican governance. Moreover, because the expansion-state region is much more Democratic than the non-expansion region, this heterogeneity could potentially drive differences between the ETC and the ETT.

Since this is a policy question of some interest, we directly investigate this by estimating the outcome model on the full data with treatment assignment interacted with each covariate \footnote{For this analysis we calculate separate covariate adjustments on the untreated data. The summary statistics for this adjustment are available in Appendix D.}. We then examine how the estimated treatment effect would change if we decreased the interaction between treatment assignment and each Republican governance indicator -- Republican governor, Republican lower legislature control, and Republican total control -- by 50 percentage points (the original variables are either 0 or 100 and are measured at the state level). This linear combination of coefficients estimates how the treatment effect would change for any given collection of states against a set that is identical except for being 50 percentage points lower, on average, across the Republican governance indicators. We find that the linear combination is positive (0.21 percentage points) and statistically significant at the 5 percent level on the unadjusted dataset. In contrast to our previous results, this would indicate that the estimated treatment effect may be larger among Republican governed areas. However, this finding is not robust to any other specification that we run. Ultimately we interpret these results as providing no evidence of treatment effect heterogeneity with respect to Republican governance. The full results are available in Appendix E, Table~\ref{tab:hte}.



\end{document}
