\section{Adjustment Details}\label{app:adjustmentdetails}

We provide additional details about estimating $\tilde{X}$. We begin by estimating the unpooled unit-level covariance matrices $\Sigma_{\nu, sc}^{\text{raw}}$, the sampling variability for each CPUMA among the treated units, by using the individual replicate survey weights to generate $b = 80$ additional CPUMA-level datasets. We then compute:

\begin{equation}
\hat{\Sigma}_{\nu, sc}^{\text{raw}} = \frac{4}{80}\sum_{b=1}^{80}(W_{b, sc} - \bar{W}_{sc})(W_{b, sc} - \bar{W}_{sc})^T
\end{equation}

where the $4$ in the numerator comes from the process used to generate the replicate survey weights and $\bar{W}_{sc}$ is the vector of covariate values estimated using the original ACS weights.

We let $\hat{\Sigma}_{W} = n_1^{-1}\sum_{sc: A_s = 1} (W_{sc} - \bar{W}_1)(W_{sc} - \bar{W}_1)^T$. This estimate is calculated on the original observed dataset. We then estimate $\Sigma_{X}$ using:

\begin{equation}
\hat{\Sigma}_X = \hat{\Sigma}_W - n_1^{-1}\sum_{sc: A_{sc} = 1} \hat{\Sigma}_{\nu, sc}^{\text{raw}}
\end{equation}

Define

\begin{equation}
\hat{\kappa} = \hat{\Sigma}_W^{-1}\hat{\Sigma}_X
\end{equation}

Notice that $\hat{\kappa}$ is a matrix of estimated coefficients of a linear regressions of the (unobserved) matrix $X_{sc}$ on (observed) matrix $W_{sc}$. We can then estimate $\mathbb{E}[X_{sc} \mid W_{sc}, A_{sc} = 1]$ using: 

\begin{equation}
\hat{X}_{sc} = \hat{\mathbb{E}}[X_{sc} \mid W_{sc}, A_{sc}] = \bar{W}_1 + \hat{\kappa}^T(W_{sc} - \bar{W}_1) \forall sc: A_{sc} = 1
\end{equation}

We call this the ``homogeneous adjustment'' and note that this approximately aligns with the adjustments suggested by \cite{carroll2006measurement} and \cite{gleser1992importance}. This estimator for $\tilde{X}$ is consistent for $\mathbb{E}[X \mid W, A]$ if we assume, for example, that the measurement errors are homoskedastic (see, e.g., $\Sigma_{\nu, sc} = \Sigma_{\nu}$). However, we can theoretically weaken this assumption (see, e.g., \cite{buonaccorsi2010measurement}) to still get a consistent estimate.

To potentially improve upon this procedure, we also consider a second estimate that we call the ``heterogeneous adjustment.'' This adjustment accounts for the fact that some regions with large populations are estimated quite precisely, while regions with small populations are estimated much less precisely (additionally, for a given CPUMA, some covariates are measured using three years of data, and others only one). For this adjustment we model an individual-level $\Sigma_{\nu, sc}$ as a function of the sample sizes used to estimate each covariate. We call this $\Sigma_{\nu, sc}^m$ with the super-script $m$ denoting that this an individual-level covariance matrix that is assumed to come from a common model. 

Specifically, let $s_{sc}$ be the q-dimensional vector of the sample sizes used to estimate each covariate value for a given CPUMA. Let $\odot$ reflect the Hadamard product, and $\oslash$ reflect Hadamard division. We assume that $\sqrt{s_{sc}} \odot \nu_{sc} \mid A_{sc} = 1 \sim N(0, \Sigma_{\nu})$. Let $S_{sc} = \sqrt{s_{sc}}\sqrt{s_{sc}}^T$. We then know that $\Sigma_{\nu, sc}^m = \Sigma_{\nu} \oslash S_{sc}$.

To estimate $\Sigma_{\nu, sc}^m$, we first pool our initial estimates of the CPUMA-level covariance matrices $\hat{\Sigma}_{\nu, sc}$ to generate $\hat{\Sigma}_{\nu} = n_1^{-1}\sum_{sc: A_s = 1} \matr{S}_{sc} \circ \hat{\Sigma}_{\nu, sc}$. We estimate $\hat{\Sigma}_{\nu, sc}^m = \hat{\Sigma}_{\nu} \oslash S_{sc}$. Using the same estimate of $\hat{\Sigma}_X$ as before, we calculate $\hat{\kappa}_{sc} = (\hat{\Sigma}_{X} + \hat{\Sigma}_{\nu, sc}^m)^{-1}\hat{\Sigma}_X$, which we use to estimate the heterogeneous adjustment $\tilde{X}_{sc}$.

This adjustment accounts for CPUMA-level variability in the measurement error, and should more greatly affect outlying values of imprecisely estimated covariates, while leaving precisely estimated covariates closer to their observed value in the dataset. Moreover, unlike using the original individual-level CPUMA estimates $\hat{\Sigma}_{\nu, sc}^{\text{raw}}$, we are able to use the full efficiency of using all units in the modeling. On the other hand, this model assumes that all differences in the sampling variability are due to the sample sizes, and assumes away heterogeneity due to heteroskedasticity. In our simulation results in Appendix~\ref{app:simstudy}, we also find that this estimator leads to bias if the measurement errors are in truth homoskedastic. 