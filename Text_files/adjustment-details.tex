\section{Covariate and Adjustment Details}\label{app:adjustmentdetails}

We detail our estimation of the observed covariates $W$ and our adjusted covariates $\hat{X}$. 

\subsection{Unadjusted covariate estimates}

To estimate our CPUMA-level covariates $W$ using the ACS microdata, we require estimating both numerator and denominator counts given that we are ultimately interested in calculating rates. For each CPUMA we estimate: the total non-elderly adult population for each year 2011-2014; the total labor force population (among non-elderly adults) for each year 2011-2013; and the total number of households averaged from 2011-2013. We also construct an average of the total non-elderly adult population from 2011-2013. These are our denominator variables. For our numerator counts, we estimate the total number of: females; whites; people of Hispanic ethnicity; people born outside of the United States; citizens; people with disabilities; married individuals; students; people with less than a high school education, high school degrees, some college, or college graduates or higher; people living under 138 percent of the FPL, between 139 and 299 percent, 300 and 499 percent, more than 500 percent, and who did not respond to the income survey question; people aged 19-29, 30-39, 40-49, 50-64; households with one, two, or three or more children, and households that did not respond about the number of children. We average these estimated counts from 2011-2013. For each individual year from 2011-2013, we then estimate the total number of people who were unemployed and uninsured at the time of the survey (calculated among all non-elderly adults and all non-elderly adults within the labor force, respectively). We divide the numerator totals by the corresponding denominator totals to estimate the percentage in each category. For the demographics, these include the average number of non-elderly adults from 2011-2013. For the time-varying variables, we use the corresponding year (where uninsurance rates are calculated as a fraction of the labor force rather than the non-elderly adult population). We also calculate the average non-elderly adult population growth and the average number of households to adults across 2011-2013. All estimates are calculated using the associated set of survey weights provided in the public use microdata files.

\subsection{Covariate adjustment}

We now provide additional details about estimating $\tilde{X}$. We begin by estimating the unpooled unit-level covariance matrices $\Sigma_{\nu, sc}^{\text{raw}}$, the sampling variability for each CPUMA among the treated units, by using the individual replicate survey weights to generate $b = 80$ additional CPUMA-level datasets. We then compute:

\begin{equation}
\hat{\Sigma}_{\nu, sc}^{\text{raw}} = \frac{4}{80}\sum_{b=1}^{80}(W_{b, sc} - \bar{W}_{sc})(W_{b, sc} - \bar{W}_{sc})^T
\end{equation}
%
where the $4$ in the numerator comes from the process used to generate the replicate survey weights and $\bar{W}_{sc}$ is the vector of covariate values estimated using the original ACS weights.

We let $\hat{\Sigma}_{W} = n_1^{-1}\sum_{sc: A_s = 1} (W_{sc} - \bar{W}_1)(W_{sc} - \bar{W}_1)^T$. This estimate is calculated on the original observed dataset. We then estimate $\Sigma_{X}$ using:

\begin{equation}
\hat{\Sigma}_X = \hat{\Sigma}_W - n_1^{-1}\sum_{sc: A_{sc} = 1} \hat{\Sigma}_{\nu, sc}^{\text{raw}}
\end{equation}

Define

\begin{equation}
\hat{\kappa} = \hat{\Sigma}_W^{-1}\hat{\Sigma}_X
\end{equation}

Notice that $\hat{\kappa}$ is a matrix of estimated coefficients of a linear regressions of the (unobserved) matrix $X_{sc}$ on (observed) matrix $W_{sc}$. We can then estimate $\mathbb{E}[X_{sc} \mid W_{sc}, A_{sc} = 1]$ using: 

\begin{equation}\label{eqn:homogeneousadjustment}
\hat{X}_{sc} = \hat{\mathbb{E}}[X_{sc} \mid W_{sc}, A_{sc}] = \bar{W}_1 + \hat{\kappa}^T(W_{sc} - \bar{W}_1) \forall sc: A_{sc} = 1
\end{equation}

We call this the ``homogeneous adjustment'' and denote the corresponding set of units in \eqref{eqn:homogeneousadjustment} $\hat{X}_{A=1}^{hom}$. This adjustment approximately aligns with the adjustments suggested by \cite{carroll2006measurement} and \cite{gleser1992importance}. This estimator for $\tilde{X}$ is consistent for $\mathbb{E}[X \mid W, A]$ if we assume, for example, that the measurement errors are homoskedastic (see, e.g., $\Sigma_{\nu, sc} = \Sigma_{\nu}$).\footnote{We can theoretically weaken this assumption to still obtain a consistent estimate (see, e.g., \cite{buonaccorsi2010measurement}).}

To potentially improve upon this procedure, we also consider a second estimate that we call the ``heterogeneous adjustment.'' This adjustment accounts for the fact that some regions with large populations are estimated quite precisely, while regions with small populations are estimated much less precisely (additionally, for a given CPUMA, some covariates are measured using three years of data, and others only one). For this adjustment we model an individual-level $\Sigma_{\nu, sc}$ as a function of the sample sizes used to estimate each covariate. We call this $\Sigma_{\nu, sc}^m$ with the super-script $m$ denoting that this an individual-level covariance matrix that is assumed to come from a common model. 

Specifically, let $s_{sc}$ be the q-dimensional vector of the sample sizes used to estimate each covariate value for a given CPUMA. Let $\odot$ reflect the Hadamard product, and $\oslash$ reflect Hadamard division. We assume that $\sqrt{s_{sc}} \odot \nu_{sc} \mid A_{sc} = 1 \sim N(0, \Sigma_{\nu})$. Let $S_{sc} = \sqrt{s_{sc}}\sqrt{s_{sc}}^T$. We then know that $\Sigma_{\nu, sc}^m = \Sigma_{\nu} \oslash S_{sc}$.

To estimate $\Sigma_{\nu, sc}^m$, we first pool our initial estimates of the CPUMA-level covariance matrices $\hat{\Sigma}_{\nu, sc}$ to generate $\hat{\Sigma}_{\nu} = n_1^{-1}\sum_{sc: A_s = 1} \matr{S}_{sc} \circ \hat{\Sigma}_{\nu, sc}$. We estimate $\hat{\Sigma}_{\nu, sc}^m = \hat{\Sigma}_{\nu} \oslash S_{sc}$. Using the same estimate of $\hat{\Sigma}_X$ as before, we calculate $\hat{\kappa}_{sc} = (\hat{\Sigma}_{X} + \hat{\Sigma}_{\nu, sc}^m)^{-1}\hat{\Sigma}_X$, which we use to estimate the heterogeneous adjustment. We call the corresponding set of estimates that use this adjustment $\hat{X}_{A=1}^{het}$.

This adjustment accounts for CPUMA-level variability in the measurement error, and should more greatly affect outlying values of imprecisely estimated covariates, while leaving precisely estimated covariates closer to their observed value in the dataset. Moreover, unlike using the original individual-level CPUMA estimates $\hat{\Sigma}_{\nu, sc}^{\text{raw}}$, we are able to use the full efficiency of using all units in the modeling. On the other hand, this model assumes that all differences in the sampling variability are due to the sample sizes, and assumes away heterogeneity due to heteroskedasticity. In our simulation results in Appendix~\ref{app:simstudy}, we also find that this estimator leads to bias if the measurement errors are in truth homoskedastic. 

As a final adjustment we can also build on the homogeneous adjustment to account for the correlation structure of the data.\footnote{We use this adjustment in the simulation study in Appendix~\ref{app:simstudy}.} To motivate this adjustment we view the entire data among the treated states $X = (X_{11}, ..., X_{1p_1}, ..., X_{sp_s})$ as reflecting a single draw from the distribution $MVN(\boldsymbol{\upsilon}_1, \mathbf{\Sigma}_X)$, where $\boldsymbol{\upsilon}_1$ is a $qn_1$ dimensional vector that repeats the $q$ dimensional vector of means $\upsilon_1$ (defined previously) $n_1$ times. Conditional on $X$, $W = (W_{11}, ..., W_{1p_1}, ..., W_{sp_s})$ reflects a single draw from the distribution $N(X, \boldsymbol{\Sigma}_{\nu})$, where $\boldsymbol{\Sigma}_{\nu}$ is a diagonal matrix with $\Sigma_{\nu}$ in the diagonals. That is, we assume that the measurement errors are drawn independently across units with constant variance matrix $\Sigma_{\nu}$.

We then let $\Sigma_S$ represent the correlation of different CPUMAs within the same state, which we assume is constant across all CPUMAs and states. 

Let $\boldsymbol{\Sigma}_W = \boldsymbol{\Sigma}_X + \boldsymbol{\Sigma}_{\nu}$. $(X, W)$ are then jointly normal with common mean $\boldsymbol{\upsilon}_1$ covariance matrix $\boldsymbol{\Sigma}$: 

\begin{align*}
\boldsymbol{\Sigma} &= \begin{pmatrix}
\mathbf{\Sigma}_X & \mathbf{\Sigma}_X \\
\mathbf{\Sigma}_X & \mathbf{\Sigma}_W
\end{pmatrix} 
\end{align*}

Notice that $\mathbf{\Sigma}_X$ and $\mathbf{\Sigma}_W$ are block-diagonal matrices with $qp_s$ by $qp_s$ blocks indexed by $s$ that we denote $\Sigma_{X, s}$ and $\Sigma_{W, s}$. Each element of these matrices contains the $q$ by $q$ matrices defined previously: 

\begin{align*}
\Sigma_{X, s} &= \begin{pmatrix}
\Sigma_{X}, \Sigma_{S}, ..., \Sigma_{S} \\
\Sigma_{S}, \Sigma_{X}, ..., \Sigma_{S} \\
... \\
\Sigma_{S}, \Sigma_{S}, ..., \Sigma_{X}
\end{pmatrix};
\Sigma_{W, s} = \begin{pmatrix}
\Sigma_{W}, \Sigma_{S}, ..., \Sigma_{S} \\
\Sigma_{S}, \Sigma_{W}, ..., \Sigma_{S} \\
... \\
\Sigma_{S}, \Sigma_{S}, ..., \Sigma_{W} \\
\end{pmatrix}
\end{align*}
%
where we have defined $\Sigma_{X}$, $\Sigma_{W}$, and $\Sigma_{S}$ previously. 
Let $\boldsymbol{\kappa} = \mathbf{\Sigma_W}^{-1}\mathbf{\Sigma_X}$. We then have that by the conditional normal distribution:

\begin{align*}
    \mathbb{E}[X \mid W] &=  \boldsymbol{\upsilon}_1 + \boldsymbol{\kappa}^T(W - \boldsymbol{\upsilon}_1) \qquad \forall sc: A_{sc} = 1
\end{align*}

To estimate this quantity we need only additionally generate an estimate of $\Sigma_{S}$ in addition to the previous estimates: to do this we simply average together all the within-state cross-products $\hat{\Sigma}_{S} = n_1^{-1}\sum_{s=1}^{m_1}\sum_{c\ne d}W_{sc}W_{sd}^T$. We then replace the other quantities by their empirical counterparts to generate the adjustment set $\hat{X}_{A=1}^{cor}$.

We do not use this adjustment for our application: in practice we find that it adds substantial variability to the observed data, predicting values that frequently fall outside of the support of the original dataset. In simulations in Appendix~\ref{app:simstudy} we find that this adjustment adds quite a bit of variability to the final estimates given a $m_1 = 25$ states. 