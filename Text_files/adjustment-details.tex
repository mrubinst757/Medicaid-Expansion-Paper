\section{Adjustment Details}

In this section we provide additional details about estimating $\eta_a$, allowing separate adjustment for both the control and treated data.\footnote{We only use $\eta_0$ for an effect heterogeneity analysis in Appendix~\ref{ssec:varselection}, and we exclusively on our estimate of $\eta_1$ for our primary results.} We begin by estimating the unit-level covariance matrices $\Sigma_{vv, sc}$ (where $a$ indexes the treatment group, $s$ indexes the state, and $c$ the CPUMA), the sampling variability for each CPUMA, by using the individual replicate survey weights to generate $b = 80$ additional CPUMA-level datasets. We then estimate $\Sigma_{vv, sc}$:

\begin{equation}
\hat{\Sigma}_{vv, sc} = \frac{4}{80}\sum_{b=1}^{80}(W_{sc}^B - \bar{W}_{sc})(W_{b, sc} - \bar{W}_{sc})^T
\end{equation}

where the $4$ in the numerator comes from the process used to generate the replicate survey weights and $\bar{W}_{sc}$ is the vector of covariate values estimated using the original ACS weights.

Unlike in our proof in Section~\ref{ssec:proof}, we do not assume that $\Sigma_{XX \mid A = a}$ or $\Sigma_{vv \mid A_s = a}$ are equal across values of $a$. Instead we let $\hat{\Sigma}_{WW \mid A = a} = n_a^{-1}\sum_{sc: A_s = a} (W_{sc} - \bar{W}_a)(W_{sc} - \bar{W}_a)^T$. This estimator is calculated on the original observed dataset. We then estimate $\Sigma_{XX \mid A = a}$ using:

\begin{equation}
\hat{\Sigma}_{XX \mid A = a} = \hat{\Sigma}_{WW \mid A = a} - n_a^{-1}\sum_{sc: A_s = a} \hat{\Sigma}_{vv, sc}
\end{equation}

Define

\begin{equation}
\hat{\kappa}_a = \hat{\Sigma}_{WW \mid A = a}^{-1}\hat{\Sigma}_{XX \mid A = a}
\end{equation}

Notice that $\hat{\kappa}_a$ is a matrix of estimated coefficients of a linear regressions of the (unobserved) matrix $X_{sc}$ on (observed) matrix $W_{sc}$. We can then estimate $X_{sc}$ using $\hat{\eta}_a(W_{sc})$, where $\hat{\eta}_a(W)$ is equal to

\begin{equation}
\bar{W}_a + \hat{\kappa}_a^T(W - \bar{W}_a)
\end{equation}

where $\bar{W}_a = n_a^{-1}\sum_{sc: A_s = a} W_{sc}$. This approximately aligns with the adjustments suggested by \cite{carroll2006measurement} and \cite{gleser1992importance}. 

In our setting we have additional access to information about a substantial source of heterogeneity in the measurement error: regions with large populations are estimated quite precisely, while regions with small populations are estimated much less precisely. Additionally, for a given CPUMA, some covariates are measured using three years of data, and others only one. However, using the conventional regression calibration approach will adjust both precisely and imprecisely estimated CPUMA using a common coefficient matrix. 

We consider an alternative approach where we model an individual-level $\Sigma_{vv, sc}$ as a function of the sample sizes used to estimate each covariate. In particular, let $s_{sc}$ be the q-dimensional vector of the sample sizes used to estimate each covariate value for a given CPUMA. Let $\matr{S}_{sc} = \sqrt{s_{sc}}\sqrt{s_{sc}}^T$. We assume that $\sqrt{s_{sc}} \odot v_{sc} \sim N(0, \Sigma_{vv, a})$. We then know that $\Sigma_{vv, sc}^m = \Sigma_{vv, a} \oslash \matr{S}_{sc}$ (we add the superscript $m$ to distinguish that this individual-level covariance matrix comes from a common model).

To estimate these matrices, we pool our initial estimates of the CPUMA-level covariance matrices within each treatment group ($\hat{\Sigma}_{vv, sc}$) to generate $\hat{\Sigma}_{vv, a} = n_a^{-1}\sum_{sc: A_s = a} \matr{S}_{sc} \circ \Sigma_{vv, sc}$. We estimate $\hat{\Sigma}_{vv, sc}^m = \hat{\Sigma}_{vv, a} \oslash \matr{S}_{sc}$. From this we estimate $\hat{\Sigma}_{XX \mid A = a} = \hat{\Sigma}_{WW \mid A = a} - n_a^{-1}\sum_{sc: A_s = a}\hat{\Sigma}_{vv, sc}$. Finally, we calculate $\hat{\kappa}_{sc} = (\hat{\Sigma}_{XX \mid A = a} + \hat{\Sigma}_{vv, sc}^m)^{-1}\hat{\Sigma}_{XX \mid A = a}$, which we use to estimate $\hat{\eta}_{sc}$. We can then use these CPUMA-level covariate matrices to estimate $X_{sc}$ from $W_{sc}$.

This adjustment accounts for CPUMA-level variability in the measurement error, and should more greatly affect outlying values of imprecisely estimated covariates, while leaving precisely estimated covariates closer to their observed value in the dataset. Moreover, we are able to use the full efficiency of using all units in the modeling, so our CPUMA-level $\hat{\kappa}_{sc}$ estimates are making efficient use of the data. The disadvantage is that, as shown in Appendix~\ref{sec:appendixsumstat}, this adjustment, compared to the conventional approach, appears more likely to lead to extreme values that fall outside of the support of the original data. A second cost is that this aggregation models all differences as a function of the sample sizes, and assumes away heterogeneity due to heteroskedasticity. 
