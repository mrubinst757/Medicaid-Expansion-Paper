\section{Proofs}\label{ssec:proof}

We divide our proofs into three sections: the first two consist of propositions and the third contains the proofs of the propositions. In the first section our propositions pertain to the performance of SBW under the classical measurement error model. Our key results are that the bias of the SBW estimator is equivalent to the bias of the OLS estimator and that regression-calibration techniques can be used in this setting to obtain consistent estimators in this setting. However, these results assume that the data are gaussian. We also show that if the data are not gaussian, the OLS estimator using regression-calibration remains consistent, while the SBW estimator may be biased. In our second section we consider the properties of the H-SBW objective when the true covariates $X$ are observed. We show that if our assumed correlation structure for the outcome errors is correct, H-SBW produces the minimum conditional-on-X variance estimator within the constraint set. We also show how a generalized form of H-SBW weights relate to the implied regression weights from Generalized Least Squares (GLS). We conclude by showing that H-SBW may yield biased estimates if we do not correctly model the dependence structure of the data. Section~\ref{app:AsecIII} contains all of the proofs.

\subsection{SBW and classical measurement error}\label{app:AsecI}

We begin by showing six results regarding the bias of the OLS and SBW estimators under the classical errors-in-variables model. First, we show that without adjustment for errors-in-covariates, the bias of the SBW estimator that sets $\delta = 0$ (i.e. reweights the treated units to exactly balance the control units) is equal to the bias of the OLS estimator. Second, we show that if the observed covariate values for the treated data can be replaced by their conditional expectations $\tilde{X}$ given the noisy observations, then the SBW estimator will be unbiased and consistent. Third, we consider the case where $\tilde{X}$ must be estimated, and show that the SBW estimator is consistent if we replace $\tilde{X}$ by a consistent estimate $\hat{X}$. Finally, we remove the assumption that $X$ is gaussian, and show that while the OLS estimator remains unbiased under weaker assumptions, the SBW estimator does not, and we show a general expression for the asymptotic bias. We take the perspective throughout that $X$ is random among the treated units but fixed for the control units.

We assume that equations (\ref{eqn:unconfoundedness}) - (\ref{eqn:Xgaussian}) hold. For simplicity, we additionally assume that
\begin{equation}\label{eqn:simplifications}
\epsilon_{sc} = 0, \quad \varepsilon_s = 0,\quad \xi_{sc} = 0,\quad  \Sigma_{\nu,sc} = \Sigma_\nu, \qquad \forall s,c
\end{equation}
noting that $\xi_{sc}=0$ implies $J_{sc} = Y_{sc}$. The covariate observations of the treated units can then be seen to be i.i.d., with covariance matrix
\[ \Sigma_{W|1} = \Sigma_{X|1} + \Sigma_\nu,\]
and the conditional expectation of $X_{sc}$ given $W_{sc}$ for the treated units can be seen to equal
\[ \tilde{X}_{sc} = v_1 + \kappa^T (W_{sc} - v_1), \qquad \forall sc: A_{sc}=1,\]
where
\[ \kappa = (\Sigma_{X|1} + \Sigma_{\nu})^{-1} \Sigma_{X|1}.\]
To ease notation, we abbreviate $\Sigma_X = \Sigma_{X \mid 1}$ and similarly $ \Sigma_W = \Sigma_{W \mid 1}$. 

In Propositions \ref{cl8}, \ref{cl9}, and part of Proposition \ref{cl1}, we will remove the Gaussian covariate assumption given by \eqref{eqn:Xgaussian}. In its place, we will instead consider the weaker assumption that the empirical covariance of $X$ has a limit $S_X$,

\begin{equation}\label{eqn:limitX}
 \frac{1}{n_1} \sum_{A_{sc}=1} (X_{sc} - \bar{X}_1)(X_{sc} - \bar{X}_1)^T \rightarrow^p S_X,
\end{equation}
which implies a similar limit $S_W$ for the noisy observations $W$,

\begin{equation}\label{eqn:limitW}
 \frac{1}{n_1} \sum_{A_{sc}=1} (W_{sc} - \bar{W}_1)(W_{sc} - \bar{W}_1)^T \rightarrow^p S_W = S_X + \Sigma_{\nu},
\end{equation}
where we have used the independence of the noise terms $\nu_{sc}$, and similarly that 
\begin{equation}\label{eqn:limitWY}
 \frac{1}{n_1} \sum_{A_{sc}=1} (W_{sc} - \bar{W}_1)(Y_{sc} - \bar{Y}_1)^T \rightarrow^p S_X \beta_1,
\end{equation}
where we have additionally used the linear model for $Y_{sc}$ given by \eqref{eqn:linmod}.

% Without loss of generality, we assume that we observe $Y_{sc}$ (equivalently, the error term $\epsilon_{sc}$ can include mean zero measurement-error). We also assume that the errors in the outcome model and the measurement errors are jointly normal, independent from each other, and drawn iid, so that 

% \begin{align}\label{eqn:cevoutcomemodel}
% Y_{sc}^a = \alpha_a + X_{sc}^T\beta_a + \epsilon_{sc} & & \text{and} & & (\epsilon_{sc}, \nu_{sc}) \stackrel{\text{iid}}{\sim} \operatorname{MVN}\left(0, \left[\begin{array}{cc} \sigma_{\epsilon}^2 & 0 \\ 0 & \Sigma_{\nu} \end{array}\right] \right).
% \end{align}

We first consider estimation without adjustment for errors in covariates. 
Proposition \ref{cl1} states that the unadjusted OLS and SBW estimators have equal bias, with the bias of the OLS estimator remaining unchanged if the gaussian assumption of \eqref{eqn:gaussiannoise} is removed.

\begin{proposition}\label{cl1}
Let (\ref{eqn:unconfoundedness}) - (\ref{eqn:Xgaussian}) and (\ref{eqn:simplifications}) hold.
Let $(\hat{\alpha}, \hat{\beta})$ denote the unadjusted OLS estimator of $(\alpha_1, \beta_1)$, 
\begin{equation}\label{eqn:prop1.beta}
(\hat{\alpha}, \hat{\beta}) = \arg \min_{\alpha, \beta} \sum_{sc:A_{sc}=1} (Y_{sc} - \alpha -  W_{sc}^T\beta)^2,
\end{equation}
which induces the OLS estimator of $\psi_0^1$ given by

\begin{align*}
\hat{\psi}^{1,\textup{ols}}_0 = \bar{Y}_1 + (\bar{W}_0 - \bar{W}_1)^T\hat{\beta}_1.
\end{align*}
%
Let ${\gamma}$ denote the unadjusted SBW weights under exact balance, found by solving \eqref{eqn:SBWobjective} with constraint set $\Gamma( W_{A=1}, \bar{W}_0, 0)$, which induces the SBW estimator of $\psi_0^1$ given by

\begin{align*}
\hat{\psi}^{1,\textup{sbw}}_0 = \sum_{sc: A_{sc} = 1} {\gamma}_{sc} Y_{sc}.
\end{align*}
%
Then the estimators $\hat{\psi}^{1, \textup{ols}}_0$ and $\hat{\psi}^{1, \textup{sbw}}_0$ have equal bias, satisfying

\begin{align*}
\mathbb{E}[\hat{\psi}_0^{1,\textup{ols}}] &= \mathbb{E}[\hat{\psi}^{1, \textup{sbw}}_0]  = \psi_0^1 + (\bar{X}_0 - \upsilon_1)^T(\mathbf{\kappa} - I_q)\beta.
\end{align*}
Additionally, the bias of $\hat{\psi}_0^{1,\textup{ols}}$ is asymptotically unchanged if the gaussian covariate assumption given by \eqref{eqn:Xgaussian} is replaced by \eqref{eqn:limitX}.
\end{proposition}

To study the SBW estimator with covariate adjustment, we first consider an idealized version where $\Sigma_X$ and $\Sigma_\nu$ are known, so that $\tilde{X}_{A=1}$ is also known. Proposition \ref{cl2} shows that the resulting estimate of $\psi_0^1$ is unbiased if $\delta = 0$.

\begin{proposition}\label{cl2}
Let (\ref{eqn:unconfoundedness}) - (\ref{eqn:Xgaussian}) and (\ref{eqn:simplifications}) hold. Let $\tilde{X}_{A=1}$ equal the conditional expectation of $X_{A=1}$ given $W$,

\[ \tilde{X}_{sc} = \upsilon_1 + \kappa^T (W_{sc} - \upsilon_1), \qquad \forall sc: A_{sc} = 1,\] let $\gamma^*$ be the solution to the SBW objective defined over the constraint set $\Gamma(\tilde{X}_{A=1}, \bar{X}_0, 0)$, and let $\hat{\psi}^{1, \textup{ideal}}_0$ be the SBW estimator $\sum_{sc: A_{sc} = 1}\gamma^\star_{sc}Y_{sc}$. This estimator is unbiased for $\psi_0^1$.
\end{proposition}



Proposition \ref{prop:variance_rate} shows that the variance of this idealized SBW estimator goes to zero, implying consistency. 
\begin{proposition}\label{prop:variance_rate}
Let (\ref{eqn:unconfoundedness}) - (\ref{eqn:Xgaussian}) and (\ref{eqn:simplifications}) hold, and let $\gamma^*$ and $\hat{\psi}_0^{1, \textup{ideal}}$ be defined as in Proposition \ref{cl2}. Then the conditional variance of the estimation error is given by

\begin{align*}
\operatorname{Var}\left( \hat{\psi}_0^{1, \textup{ideal}} - \psi_0^1| W\right)  = \|\gamma^*\|^2 \cdot \beta_1^T(\Sigma_{X} - \Sigma_{X}\Sigma_{W}^{-1}\Sigma_{X})\beta_1, 
\end{align*}
with $\operatorname{Var}\left( \hat{\psi}_0^{1, \textup{ideal}} - \psi_0^1| W\right)$ and $\operatorname{Var}(\hat{\psi}_0^{1,\textup{ideal}})$ both behaving as $O_P(n_1^{-1})$ as $n_1 \rightarrow \infty$.
\end{proposition}

In practice, the idealized SBW estimator considered in Propositions \ref{cl2} and \ref{prop:variance_rate} cannot be used, as $\Sigma_X$ and $\Sigma_{\nu}$ are not known, but instead must be estimated from auxilliary data. Proposition \ref{cl3} states that if these estimates are consistent, then the resulting adjusted SBW estimator for $\psi_0^1$ is also consistent if $\delta = 0$.

\begin{proposition}\label{cl3}
Let (\ref{eqn:unconfoundedness}) - (\ref{eqn:Xgaussian}) and (\ref{eqn:simplifications}) hold. Given estimates $\hat{\Sigma}_X$ and $\hat{\Sigma}_\nu$ that are consistent for $\Sigma_X$ and $\Sigma_\nu$, let $\hat{X}_{A=1}$ be given by 
\[ \hat{X}_{sc} = \bar{W}_1 + \hat{\kappa}^T(W_{sc} - \bar{W}_1), \]
where $\hat{\kappa} = (\hat{\Sigma}_X + \hat{\Sigma}_{\nu})^{-1} \hat{\Sigma}_X$. Let $\hat{\gamma}$ be the weights that solve the SBW objective over the constraint set $\Gamma(\hat{X}_{A=1}, \bar{W}_0, 0)$, and let $\hat{\psi}^{1, \textup{adjusted}}_0 = \sum_{sc: A_{sc} = 1} \hat{\gamma}_{sc} Y_{sc}$ be the corresponding SBW estimator. This estimator is consistent for $\psi_0^1$ as $n_1 \to \infty$.
\end{proposition}

In (\ref{eqn:jackknife}) we propose a leave-one-state-out jackknife estimate of variance. Following 
\cite{efron1981jackknife}, this estimate can be decomposed a conservatively biased estimate of the variance of $\hat{\psi}_0^{1, \textup{adjusted}}$ given a sample size of $(m_1-1)$ treated states, plus a heuristic adjustment to go from sample size $(m_1-1)$ to sample size $m_1$, when treating the observations of the control states as fixed.

\begin{proposition}\label{prop:jackknife}
Let (\ref{eqn:unconfoundedness}) - (\ref{eqn:gaussiannoise}) hold, and additionally assume that $p_s$, the number of CPUMAs, is i.i.d. in the treated states. Let $\hat{\operatorname{Var}}(\hat{\psi}_0^{1, \textup{adjusted}}) = \frac{m_1-1}{m_1} \cdot \tilde{\operatorname{Var}}(\hat{\psi}_0^{1, \textup{adjusted}})$, where

\begin{equation} \label{eqn:prop.jackknife}
\tilde{\operatorname{Var}}(\hat{\psi}_0^{1, \textup{adjusted}}) = \sum_{s:A_{s}=1} (S_{(s)} - S_{(\cdot)})^2,
\end{equation}
with $S_{(s)}$ and $S_{(\cdot)}$ as defined for \eqref{eqn:jackknife}. Then $\tilde{\operatorname{Var}}$ is conservatively biased for the variance of the leave-one-state-out estimate,

\[ \mathbb{E}\left[ \tilde{\operatorname{Var}}(\hat{\psi}_0^{1, \textup{adjusted}})\right] \geq \operatorname{Var}(S_{(1)} | \bar{W}_0),\]
where $S_{(1)}$ can be seen to equal the estimator $\hat{\psi}_0^{1,\textup{adjusted}}$ under a sample size of $(m_1-1)$ treated states.

\end{proposition}

As the gaussian covariate assumption given by \eqref{eqn:Xgaussian} is strong, it would be desirable if the adjusted OLS or SBW estimators were consistent even for non-gaussian $X$. Proposition \ref{cl8} shows under mild assumptions that this is in fact true when running OLS on the adjusted covariates. 

\begin{proposition}\label{cl8}
Let (\ref{eqn:unconfoundedness}) - (\ref{eqn:gaussiannoise}), and (\ref{eqn:simplifications})- (\ref{eqn:limitX}) hold, with $S_X$ invertible. Let $(\check{\alpha}, \check{\beta})$ denote the adjusted OLS estimates of $(\alpha_1, \beta_1)$, solving

\[ \min_{\alpha,\beta} \sum_{A_{sc}=1} (Y_{sc} - \alpha - \check{X}_{sc}^T \beta)^2, \]
where $\check{X}_{sc} = \bar{W}_1 + \check{\kappa}^T(W_{sc} - \bar{W}_1)$ with $\check{\kappa} = (S_X + \Sigma_\nu)^{-1}S_X$. Then the adjusted OLS estimator of $\psi_0^1$ given by
\[ \bar{Y}_1 - (\bar{W}_0 - \bar{W}_1)^T \check{\beta},\]
remains consistent if the gaussian assumption given by \eqref{eqn:gaussiannoise} is removed.
\end{proposition}

However, the same does not hold for the adjusted SBW estimator. Proposition \ref{cl9} gives an expression for its bias when the covariates are non-gaussian. 

\begin{proposition}\label{cl9}
Let the assumptions of Proposition \ref{cl8} hold. Let $\check{\gamma}$ solve the SBW objective over the constraint set $\Gamma(\check{X}_{A=1}, \bar{W}_0, 0)$ where $\check{X}_{A=1}$ and $\check{\kappa}$ are defined as in Proposition \ref{cl8}. Let $Q$ denote the set of indices where $\check{\gamma}$ is non-zero,

\[ Q = \{sc: \check{\gamma}_{sc} > 0\},\]
with cardinality $n_Q = |Q|$, and let $\bar{W}_Q$ and $S_{W_Q}$ denote the empirical mean and covariance of $\{W_{sc}:sc \in Q\}$,

\[ \bar{W}_Q = \frac{1}{n_Q}\sum_{sc \in Q} W_{sc},\qquad S_{W_Q} = \frac{1}{n_Q} \sum_{sc \in Q} (W_{sc} - \bar{W}_Q)(W_{sc} - \bar{W}_Q)^T,\]
with $\bar{X}_Q$ the analogous empirical mean of $\{X_{sc}:sc \in Q\}$ and $S_{XW_Q}$ the empirical cross covariance,
\[ S_{XW_Q} = \frac{1}{n_Q} \sum_{sc \in Q} (X_{sc} - \bar{X}_Q)(W_{sc} - \bar{W}_Q)^T.\]
Then if the gaussian assumption given by \eqref{eqn:gaussiannoise} is removed, the adjusted SBW estimator for $\psi_0^1$ given by 

\[\sum_{A_{sc}=1} Y_{sc} \check{\gamma}_{sc},\]
may be biased for $\psi_0^1$, with estimation error given by 

\begin{align} 
\nonumber \sum_{A_{sc}=1} Y_{sc} \check{\gamma}_{sc} - \psi_0^1 & = \beta_1^T\Big[(S_{XW_Q}S_{W_Q}^{-1}S_WS_X^{-1} - I)\bar{X}_0  + (\bar{X}_Q - S_{XW_Q}S_{W_Q}^{-1} S_W S_X^{-1} \bar{X}_1) \\
& \hskip1cm {} - S_{XW_Q}S_{W_Q}^{-1}(\bar{X}_Q - \bar{X}_1)\Big](1 + o_P(1)),  \label{eqn:cl9.error}
\end{align}
which need not converge to zero unless $\bar{X}_Q \to \bar{X}_1$, $S_{XW_Q} \to S_X$, and $S_{W_Q} \to S_W$.
\end{proposition}

\begin{remark}
    While we have assumed that $\epsilon_{sc}=0$ for simplicity in our propositions, removing this assumption simply leads to the additional term $\sum_{sc: A_{sc} = 1}\gamma_{sc}\epsilon_{sc}$ in the error of the SBW estimator of $\psi_0^1$. This again has expectation zero, because the weights remain independent of the error $\epsilon_{sc}$ in the outcomes. Allowing non-zero $\epsilon_{sc}$ also adds a term to the estimator variance (conditional on $W$) equal to $\sigma^2_{\epsilon}\cdot \|\gamma^*\|^2$,    which does not change the variance bound given by Proposition \ref{prop:variance_rate}.
\end{remark}


\begin{remark}
    For the adjusted OLS estimator, in which $\beta_1$ is estimated using the adjusted covariates $\tilde{X}_{A=1}$, in practice we must estimate $\tilde{X}$ with some estimator $\hat{X}$ that relies on an estimate $\hat{\kappa}$. As long as $\hat{\kappa}$ is consistent for $\kappa$ then the OLS estimator will also be consistent by the continuous mapping theorem.
\end{remark}

\begin{remark}
To see how Proposition \ref{cl9} implies that the adjusted SBW estimate may be biased in non-gaussian settings, we observe that as the set $Q$ in Proposition \ref{cl9} will depend on the values of the covariates $X$ and observation noise $\nu$, the values of $\{X_{sc}: sc \in Q\}$ and $\{\nu_{sc}: sc \in Q\}$ may differ systematically from their population, so that $\bar{X}_Q$, $S_{XW_Q}$ and $S_{W_Q}$ may not converge to their desired counterparts. While the expression for the estimation error given by (\ref{eqn:cl9.error}) is asymptotic, an exact formula is given in  \eqref{eqn:cl9.proof3} which is very similar; the only asymptotic approximations are the convergence of $\bar{W}_1$ to $\bar{X}_1$ and $\bar{W}_0$ to $\bar{X}_0$.
\end{remark}

\subsection{Properties of H-SBW}\label{app:AsecII}

Here we consider an H-SBW setting where $\nu_{sc}=0$ so that the true covariates are observed. By \eqref{eqn:linmod}, the outcomes have CPUMA level noise terms  $\epsilon_{sc}$, and also state-level noise terms $\varepsilon_s$ that correlate the outcomes of CPUMAs in the same state. Proposition \ref{cl4} states that if $\rho$ is the within-state correlation of these error terms, the H-SBW estimator produces the minimum conditional-on-X variance estimator of $\psi_0^1$ within the constraint set.

\begin{proposition}\label{cl4}
    Consider the outcome model in ~\eqref{eqn:linmod}. Assume the errors are homoskedastic and have finite variance $\sigma^2_{\epsilon}$ and $\sigma^2_{\varepsilon}$, and let $\rho$ be the within-state correlation of the error terms. Let $\hat{\gamma}^{\textup{hsbw}}$ be the weights that solve \eqref{eqn:hsbwobjective} for known parameter $\rho$ across the constraint set $\Gamma(X_{A=1}, \bar{X}_0, \delta)$ for any $\delta$. Then the H-SBW estimator of $\psi_0^1$,

    \[\sum_{s: A_s = 1}\sum_{c=1}^{p_s}\hat{\gamma}_{sc}^{hsbw}Y_{sc}\] 
    is the minimum conditional-on-X variance estimator of $\psi_0^1$ within the constraint set $\Gamma(X_{A=1}, \bar{X}_0, \delta)$.
\end{proposition}

The SBW and H-SBW objective functions take the generic form $\gamma^T\Omega\gamma$: SBW takes $\Omega = I_n$, while H-SBW specifies an $\Omega$ that allows for positive within-state equicorrelation. Analogous versions hence exist for any assumed covariance structure $\Omega$. Proposition \ref{cl56} highlights connections between this generic form and generalized least-squares (or least-norm) problems, showing that under exact balance we can express the weights as regression weights estimated on a subset of the data. Similar results connecting regression weights to balancing weights can be found throughout the literature (see, e.g., \cite{kline2011oaxaca}, \cite{ben2021augmented}, \cite{chattopadhyay2021implied}).

\begin{proposition}\label{cl56}
Let $\gamma^*$ solve the optimization problem

\begin{equation}\label{eqn:a1.1}
 \min_\gamma \gamma^T \Omega \gamma \quad \text{ subject to } \quad  \sum_i \gamma_i Z_i = v,\ \sum_i \gamma_i = 1,\ \textup{ and } \gamma \geq 0%\gamma \in \Phi(Z, v, 0),
\end{equation}
 with $\Omega$ positive definite, and let $Q = \{i: \gamma^*_i > 0\}$ denote the indices of its non-zero entries. Then $\gamma^*$ also solves the generalized least squares problem,
  
  \begin{equation}\label{eqn:a1.2}
   \min_{\gamma}  \ \gamma^T \Omega \gamma  \quad \textup{subject to }\quad \sum_{i \in Q} \gamma_i Z_i = v,\ \sum_{i \in Q} \gamma_i = 1,\ \textup{ and }   \gamma_i = 0\  \forall\ i \not\in Q,
  \end{equation}
 and hence has non-zero entries $\gamma^*_Q = \{\gamma_i^*: i \in Q\}$ satisfying
 
 \begin{equation}\label{eqn:a1.3}
 \gamma^*_{Q} = \Omega_{Q}^{-1} (Z_{Q} - \mu)^T\left[ (Z_Q - \mu) \Omega_{Q}^{-1} (Z_Q - \mu)^T\right]^{-1} (v - \mu) + \frac{\Omega^{-1}_Q {\bf 1} }{{\bf 1}^T \Omega^{-1}_Q {\bf 1}},
 \end{equation}
where $Z_{Q}$ is the matrix whose columns are $\{Z_i: i \in Q\}$, $\Omega_Q$ is the submatrix of $\Omega$ whose rows and columns are in $Q$, ${\bf 1}$ is the column vector of ones, and $\mu$ is the vector $\frac{Z_{Q}\Omega_{Q}^{-1} {\bf 1}}{ {\bf 1}^T \Omega^{-1}_Q {\bf 1}}$. 
\end{proposition}

\begin{remark}
To lighten notation, we have used $Z_Q - \mu$ (a vector subtracted from a matrix) to mean $Z_Q - \mu{\bf 1}^T$, so that each column of $Z_{Q}$ is centered by $\mu$. 
\end{remark}

Proposition \ref{cl7} shows that if the conditional expectations can be computed for the treated units (which may be computationally difficult or require strong modeling assumptions if the data is non-gaussian, or if dependencies exist between CPUMAs), then H-SBW (which equals SBW when $\rho=0$) yields unbiased estimates. 

\begin{proposition}\label{cl7}
    Let $\tilde{X}^*$ denote the conditional expectation,
    \[\tilde{X}^*_{sc} = \mathbb{E}[X_{sc} | W, A_{sc}=1],\]
    and consider the H-SBW estimator $\hat{\psi}_0^{1, \textup{hsbw}}$ using weights $\hat{\gamma}^{hsbw}$ that solve \eqref{eqn:hsbwobjective} across $\Gamma(\tilde{X}^\star_{A=1}, \bar{X}_0, 0)$. This estimator is unbiased for $\psi_0^1$.
\end{proposition}

\begin{remark}
    In Proposition \ref{cl4}, we assumed the outcomes followed \eqref{eqn:linmod} and the constraints balanced the means of the covariates; however, we can allow for any outcome model and our balance constraints can include any function of the covariate distribution and this result still holds conditional on $X$ (though of course the estimator may be badly biased). The key assumption is that the variability in the estimates comes from the outcome model errors, which are assumed to be equicorrelated within state for known parameter $\rho$.
\end{remark}

\begin{remark}
    Assuming that $(X_{sc}, W_{sc}) \mid A_{sc} = 1$ are Gaussian but dependent, Proposition \ref{cl7} implies that if we correctly model the correlations between the CPUMAs in our regression calibration step, we can use GLS or H-SBW without inducing asymptotic bias (assuming all of our models are correct). This is similar to the approach followed in \cite{huque2014impact}, who consider parameter estimation using GLS in the context of a one-dimensional spatially-correlated covariate measured with error. We also outline in Appendix~\ref{app:adjustmentdetails} a potential adjustment when we assume an equicorrelation structure among the covariates similar to what we have assumed for the outcome, and evaluate this adjustment in simulations in Appendix~\ref{app:simstudy}. 
    
    Additionally, to be clear Proposition \ref{cl7} also implies implies that if we do not model the dependence structure, we cannot generally use the simple adjustment provided in \eqref{eqn:regcal} in combination with GLS to obtain asymptotically unbiased estimates. Intuitively this is because the implied weights from GLS depend on the covariance structure between the units, which is not correctly modeled in \eqref{eqn:regcal}. By contrast, Proposition \ref{cl5} shows that we safely can ignore such dependence in when using regression-calibration with OLS (as long as a probability limit exists for the empirical covariance matrix).
\end{remark}

\begin{remark}
    Perhaps surprisingly we find in our simulation study in Appendix~\ref{app:simstudy} that when the covariates are gaussian but dependent, the simple adjustment provided in \eqref{eqn:regcal} is sufficient to obtain an approximately unbiased estimate when using SBW. We conjecture that this is because the set $Q$ may have some limiting boundary. If true, the characterization of SBW weights as regression weights in Proposition~\ref{cl56} would imply that the SBW weight $\hat{\gamma}_{sc}^{sbw}$ is fixed conditional on input data point $W_{sc}$ asymptotically. 
    The error of the estimator can therefore decompose as a function of $(X_{sc} - \tilde{X}_{sc})$, which is independent of $\hat{\gamma}_{sc}$ given $W_{sc}$. This implies that it would suffice to balance on $\tilde{X}_{A=1}$.
\end{remark}

\subsection{Proofs}\label{app:AsecIII}

We begin by establishing the following identity for our target parameter $\psi_0^1$ defined in \eqref{eqn:psi}.

\begin{equation}\label{eqn:psi10_identity}
\psi^1_0 = \mu_y + (\bar{X}_0 - \upsilon_1)^T\beta_1
\end{equation}
%
where $\mu_y = \mathbb{E}[Y_{sc} \mid A_{sc} = 1]$ and $\upsilon_1 = \mathbb{E}[X_{sc} \mid A_{sc} = 1]$.

\begin{proof}[Proof of (\ref{eqn:psi10_identity})]
Using our causal and modeling assumptions we have that:

\begin{align*}
\mathbb{E}[Y_{sc}^1 \mid X_{sc}, A_{sc} = 0] &= \mathbb{E}[Y_{sc}^1 \mid X_{sc}, A_{sc} = 1] \\
&= \mathbb{E}[Y_{sc} \mid X_{sc}, A_{sc} = 1] \\
&= \alpha_1 + X_{sc}^T\beta_1 \\
&= \mu_y + (X_{sc} - \upsilon_1)^T\beta \\
&\implies \psi_0^1 = \mu_y + (\bar{X}_0 - \upsilon_1)^T\beta_1
\end{align*}
%
where the first equality follows from unconfoundedness, the second equality from consistency, the third from our parametric modeling assumptions, and the fourth by definition of $\alpha$. The final equation follows from averaging over the control units.
\end{proof}
%

\begin{proof}[Proof of Propositon \ref{cl1}]
It can be seen from \eqref{eqn:regcal} that for all $sc: A_{sc}=1$,

\begin{align*}
   X_{sc} &= v_1 + (W_{sc} - v_1)^T \kappa + \nu_{sc}',
\end{align*}
where $\nu_{sc}' = X_{sc} - \mathbb{E}[X_{sc}|W,A=1]$ may be viewed as an independent zero-mean noise term. Plugging into \eqref{eqn:linmod} yields 

\begin{align*}
   Y_{sc} & = \alpha_1 + v_1^T (I - \kappa)\beta_1 + W_{sc}^T \kappa \beta_1 + \epsilon_{sc}',
\end{align*}
for $\epsilon_{sc}' = \beta_1^T\nu_{sc}' + \epsilon_{sc}$. It follows that the OLS estimate $\hat{\beta}$ given by \eqref{eqn:prop1.beta} satisfies \citep{gleser1992importance},

\begin{equation}\label{eqn:prop1.0}
\mathbb{E}[\hat{\beta}|W_{A=1}] = \kappa \beta_1, \qquad \text{and} \qquad \mathbb{E}[\bar{W}_1 \hat{\beta}] = \bar{X}_1 \kappa \beta_1.
\end{equation}
To show that $\hat{\psi}_0^{1,\textup{ols}}$ and $\hat{\psi}_0^{1, \textup{sbw}}$ have identical bias, we compute their expectations:

\begin{align}
\nonumber	\mathbb{E}[\hat{\psi}_0^{1,\textup{ols}}] &= \mathbb{E}[ \bar{Y}_1 + (\bar{W}_0 - \bar{W}_1)^T \hat{\beta}] \\
	& = \bar{\mu}_y + (\bar{X}_0 - \upsilon_1)^T\kappa\beta_1 \label{eqn:prop1.1}\\
	& = \psi_0^1 + (\bar{X}_0 - \upsilon_1)^T(\kappa - I_q)\beta_1 \label{eqn:prop1.2}
\end{align}
where \eqref{eqn:prop1.1} holds by \eqref{eqn:prop1.0}, and \eqref{eqn:prop1.2} holds by \eqref{eqn:psi10_identity}. We next derive the expected value of $\hat{\psi}^{1, \textup{sbw}}$:

\begin{align}
\nonumber	\mathbb{E}[\hat{\psi}_0^{1, \textup{sbw}}] & = \mathbb{E}\left[ \sum_{A_{sc} = 1} {\gamma}_{sc} Y_{sc}\right] \\
	& = \mathbb{E}\left[ \sum_{A_{sc}=1} {\gamma}_{sc} \left(\alpha_1 + (W_{sc} - W_{sc} + X_{sc})^T \beta_1 + \epsilon_{sc}\right)\right] \label{eqn:prop1.4}\\
\nonumber	& = \mathbb{E}\left[ \alpha_1 + \sum_{A_{sc} = 1} {\gamma}_{sc} W_{sc}^T \beta_1 + \sum_{A_{sc}=1} {\gamma}_{sc} (X_{sc} - W_{sc})^T \beta_1 + \sum_{A_{sc}=1} {\gamma}_{sc} \epsilon_{sc} \right] \\
	& = \alpha_1 + \bar{X}_0^T\beta_1 + \mathbb{E}\left[ \sum_{A_{sc} = 1} {\gamma}_{sc}(X_{sc} - W_{sc})^T \beta_1\right] \label{eqn:prop1.5}\\
	& = \psi_0^1 + \mathbb{E} \left[ \sum_{A_{sc} = 1} {\gamma}_{sc}(X_{sc} - W_{sc})^T \beta_1 \right] \label{eqn:prop1.6} \\
	& = \psi_0^1 + \mathbb{E} \left[ \sum_{A_{sc} = 1} \mathbb{E}\left[ {\gamma}_{sc}(X_{sc} - W_{sc})^T \beta_1 | W \right] \right] \label{eqn:prop1.7} \\
	& = \psi_0^1 + \mathbb{E} \left[ \sum_{A_{sc} = 1}  {\gamma}_{sc} (\mathbb{E}[X_{sc}|W] - W_{sc})^T \beta_1 \right] \label{eqn:prop1.8} \\
	& = \psi_0^1 + \mathbb{E} \left[ \sum_{A_{sc} = 1}  {\gamma}_{sc} (\upsilon_1 + \kappa^T(W_{sc} - \upsilon_1) - W_{sc})^T \beta_1 \right] \label{eqn:prop1.9} \\
\nonumber	& = \psi_0^1 + \mathbb{E} \left[ \sum_{A_{sc} = 1}  {\gamma}_{sc} (W_{sc} - \upsilon_1)^T(\kappa - I)\beta_1 \right] \\
\nonumber	& = \psi_0^1 + \left(\mathbb{E}\left[\sum_{A_{sc} = 1} {\gamma}_{sc} W_{sc}\right] - \upsilon_1\right)^T(\kappa - I)\beta_1  \\
	& = \psi_0^1 + \left(\bar{X}_0 - \upsilon_1\right)^T(\kappa - I_q)\beta_1,  \label{eqn:prop1.10}
\end{align}
%
where \eqref{eqn:prop1.4} holds by the assumed linear model for $Y_{sc}$ given by  \eqref{eqn:linmod}; \eqref{eqn:prop1.5} and \eqref{eqn:prop1.10} hold because the SBW algorithm enforces that $\sum \hat{\gamma}_{sc} W_{sc} = \bar{W}_0$, which has expectation $\bar{X}_0$, and because $\epsilon_{sc}$ is zero-mean and independent of $W_{sc}$ and hence independent of $\hat{\gamma}_{sc}$; \eqref{eqn:prop1.6} holds by definition of $\psi_0^1$ and the assumed linear model in \eqref{eqn:linmod}; \eqref{eqn:prop1.7} is the tower property of expectations; \eqref{eqn:prop1.8} follows because $\gamma_{sc}$ and $W_{sc}$ are deterministic given $W$; and \eqref{eqn:prop1.9} uses the expression for the conditional expectation given by \eqref{eqn:regcal}. It can be seen that \eqref{eqn:prop1.2} and \eqref{eqn:prop1.10} are equal, and hence show that $\hat{\psi}_0^{1,\textup{ols}}$ and $\hat{\psi}_0^{1, \textup{sbw}}$ have equal bias.

It remains to show that the bias of the OLS estimator is unchanged if the gaussian assumption is relaxed so that \eqref{eqn:regcal} no longer holds. It follows from \eqref{eqn:prop1.beta} that $\hat{\beta}$ is asymptotically given by

    \begin{align*}
    \hat{\beta} &= \left(\sum_{A_{sc}=1} (W_{sc} - \bar{W}_1)(W_{sc} - \bar{W}_1)^T \right)^{-1} \left(\sum_{A_{sc}=1} (W_{sc} - \bar{W}_1)(Y_{sc} - \bar{Y}_1)^T\right) \\
     & \rightarrow^p  (S_X + \Sigma_\nu)^{-1}S_X \beta_1 = \check{\kappa} \beta_1, 
    \end{align*}
where we have used \eqref{eqn:limitW} and \eqref{eqn:limitWY}. Plugging into $\psi_0^{1,\textup{ols}}$ yields

\begin{align*}    
    \mathbb{E}[\hat{\psi}_0]^{1,\textup{ols}} = \mathbb{E}[ \bar{Y}_1 + (\bar{W}_0 - \bar{W}_1)^T \hat{\beta}] \\
    \rightarrow^p  \bar{\mu}_y + (\bar{X}_0 - \bar{X}_1)\check{\kappa} \beta_1,
\end{align*}
from which the result follows by the same steps used to show \eqref{eqn:prop1.2}.

\end{proof}


\begin{proof}[Proof of Proposition \ref{cl2}]
Assuming $\epsilon_{sc} = 0$, by linearity we know that

\begin{equation}\label{eqn:outcomerevised}
Y_{sc} = \alpha_1 + \tilde{X}_{sc}^T\beta_1 + (X_{sc} - \tilde{X}_{sc})^T\beta_1 \qquad \forall sc: A_{sc} = 1
\end{equation}

We then have that:

\begin{align}\nonumber
    \hat{\psi}_0^{1,\textup{ideal}} - \psi_0^1 &= \sum_{sc: A_{sc} = 1}\gamma_{sc}^\star Y_{sc} - (\alpha_1 + \bar{X}_0^T\beta_1) \\
    \nonumber &= \sum_{sc: A_{sc} = 1}\gamma_{sc}^\star\alpha_1 + \sum_{sc: A_{sc} = 1}\gamma_{sc}^\star\tilde{X}_{sc}^T\beta_1 \\ 
    &+ \sum_{sc: A_{sc} = 1}\gamma_{sc}^\star(X_{sc} - \tilde{X}_{sc})^T\beta_1 - (\alpha_1 + \bar{X}_0^T\beta_1) \label{eqn:outcomerevised_proof1}\\
    &= \sum_{sc: A_{sc} = 1}\gamma_{sc}^\star(X_{sc} - \tilde{X}_{sc})^T\beta_1\label{eqn:sbwregcalerror},
\end{align}
where \eqref{eqn:outcomerevised_proof1} follows from \eqref{eqn:outcomerevised}, and \eqref{eqn:sbwregcalerror} holds since $\sum \gamma_{sc}^\star = 1$ and $\sum \gamma_{sc}^\star \tilde{X}_{sc} = \bar{X}_0$. Conditioned on $W$, it can be seen that $\gamma^*$ is fixed and $X_{sc} - \tilde{X}_{sc}$ has expectation zero; therefore, \eqref{eqn:sbwregcalerror} implies that the estimator is unbiased.
\end{proof}


\begin{proof}[Proof of Proposition \ref{prop:variance_rate}] 
To derive $\operatorname{Var}\left(\hat{\psi}_0^{1,\textup{ideal}} | W\right)$, we use

\begin{align}
\operatorname{Var}\left(\hat{\psi}_0^{1,\textup{ideal}} - \psi_0^1 | W\right) &= \operatorname{Var}\left[\sum_{sc: A_{sc} = 1}\gamma_{sc}^\star(X_{sc} - \tilde{X}_{sc})^T\beta_1 \mid W\right] \label{eqn:prop:variance.1}\\
 &= \sum_{sc: A_{sc} = 1} \operatorname{Var}(\gamma_{sc}^\star(X_{sc} - \tilde{X}_{sc})^T\beta_1 \mid W) \label{eqn:prop:variance.2}\\
 &= \sum_{sc: A_{sc} = 1} \gamma_{sc}^{\star^2}\beta_1^T(\Sigma_{X} - \Sigma_{X}\Sigma_{W}^{-1}\Sigma_{X})\beta_1  \label{eqn:prop:variance.3}\\
& = \|\gamma^*\|^2 \cdot \beta_1^T(\Sigma_{X} - \Sigma_{X}\Sigma_{W}^{-1}\Sigma_{X})\beta_1,  \label{eqn:variance}
\end{align}
%
where \eqref{eqn:prop:variance.1} follows from \eqref{eqn:sbwregcalerror}, \eqref{eqn:prop:variance.2} holds because the tuples $(X_{sc}, W_{sc})$ are i.i.d, and \eqref{eqn:prop:variance.3} holds because $\gamma_{sc}^*$ is fixed given $W$ and $(X_{sc}, W_{sc})$ are jointly normal. 

To upper bound the conditional variance given by \eqref{eqn:variance}, we will construct a feasible solution $\gamma'$ to the SBW objective over the constraint set $\Gamma(\tilde{X}, \bar{X}_0, 0)$ such that $\|\gamma'\|^2 = O_P(n^{-1})$. As the optimal solution $\gamma^*$ satisfies $\|\gamma^*\|^2 \leq \|\gamma'\|^2$, the result follows.

Our construction is the following. Divide the $n_1$ treated units into $L = \lfloor n_1/n^{\text{sub}} \rfloor$ subsets of size $n^{\text{sub}}$, and a remainder subset. For the subsets $\ell=1,\ldots,L$, let $X^{(\ell)}$ denote its covariates, $\tilde{X}^{(\ell)}$ the conditional expectation $\mathbb{E}[X^{(\ell)}|W, A]$, and  $\gamma^{(\ell)}$ the solution to the SBW objective over the constraint set $\Gamma(\tilde{X}^{(\ell)}, \bar{X}_0, 0)$, with $\gamma^{(\ell)}=0$ if the constraint set is infeasible. As the units are assumed to be i.i.d., it follows that $\gamma^{(1)}, \ldots, \gamma^{(L)}$ are also i.i.d. Let $n^{\text{sub}}$ be large enough so that each $\gamma^{(\ell)}$ has positive probability of being non-zero. 

Let $L'$ denote the number of subsets whose $\gamma^{(\ell)}$ is non-zero. As each non-zero weight vector $\gamma^{(\ell)}$ is feasible for $\Gamma(\tilde{X}^{(\ell)}, \bar{X}_0, 0)$, it can be seen that the concatenated vector $\gamma' = (\gamma^{(1)}/L', \ldots, \gamma^{(L)}/L', 0)$ is feasible for $\Gamma(\tilde{X},\bar{X}_0,0)$. As the weights $\gamma^{(\ell)}$ are i.i.d, it follows that $\| \gamma'\|^2$ which equals $\frac{1}{(L')^2} \sum_\ell \|\gamma_\ell\|^2$  converges in probability to $\frac{1}{L'} \mathbb{E}\|\gamma^{(1)}\|^2 = O_P(n_1^{-1})$, proving the bound on $\operatorname{Var}\left(\hat{\psi}_0^{1,\textup{ideal}} - \psi_0^1 | W\right)$.

To show this rate also holds for $\operatorname{Var}\left(\hat{\psi}_0^{1,\textup{ideal}}\right)$, we can apply the law of total variance to $f = \hat{\psi}_0^{1,\textup{ideal}} - \psi_0^1 = \hat{\psi}_0^{1,\textup{ideal}},$
\[ \operatorname{Var}(f) = \underbrace{\mathbb{E}[\operatorname{Var}(f|W)]}_{(i)} + \underbrace{\operatorname{Var}(\mathbb{E}[f|W])}_{(ii)}, \]
observing that $\mathbb{E}[f|W] = 0$ by Proposition \ref{cl2}, so that (ii) is zero. To show that term (i) is $O(n^{-1})$, we observe that as $\operatorname{Var}(f|W) = O_P(n^{-1})$ and is bounded (since $\gamma^*$ is non-negative and sums to 1), it follows that  $\mathbb{E}[ \operatorname{Var}(f|W)]$ must be $O(n^{-1})$ as well.
\end{proof}

\begin{proof}[Proof of Proposition \ref{cl3}]
Following Proposition~\ref{cl2}, assuming $\epsilon_{sc}=0$ we can decompose the error of the estimator as follows:

\begin{align}
\nonumber    \hat{\psi}^{1,\textup{adjusted}}_0 - \psi_0^1 &= \sum_{A_{sc}=1} \hat{\gamma}_{sc} Y_{sc} - \psi_0^1 \\
    & = \sum_{A_{sc}=1} \hat{\gamma}_{sc} (\alpha_1 + X_{sc}^T \beta_1) - \psi_0^1 \label{eqn:cl3.1}\\
    \nonumber & = \sum_{A_{sc}=1} \hat{\gamma}_{sc} (\alpha_1 + \hat{X}_{sc}^T \beta_1 + (X_{sc} - \hat{X}_{sc})^T \beta_1 ) - \psi_0^1 \label{eqn:cl3.2}\\
    \nonumber & = \alpha_1 + \sum_{A_{sc}=1} \hat{\gamma}_{sc} \hat{X}_{sc}^T \beta_1 + \sum_{A_{sc}=1} \hat{\gamma}_{sc}(X_{sc} - \hat{X}_{sc})^T \beta_1  - \psi_0^1 \label{eqn:cl3.3}\\
    & = \alpha_1 + \bar{W}_0^T \beta_1 + \sum_{A_{sc}=1} \hat{\gamma}_{sc}(X_{sc} - \hat{X}_{sc})^T \beta_1  - \psi_0^1 \label{eqn:cl3.3}\\
    & = \underbrace{(\bar{X}_0 - \bar{W}_0)^T \beta_1}_{(i)} + \underbrace{\sum_{A_{sc}=1} \hat{\gamma}_{sc}(X_{sc} - \tilde{X}_{sc})^T \beta_1}_{(ii)} + \underbrace{\sum_{A_{sc}=1} \hat{\gamma}_{sc} (\tilde{X}_{sc} - \hat{X}_{sc})^T \beta_1}_{(iii)} \label{eqn:cl3.4}
\end{align}
where \eqref{eqn:cl3.1} holds by \eqref{eqn:linmod}, \eqref{eqn:cl3.3} uses that $\sum \hat{\gamma}_{sc} \hat{X}_{sc} = \bar{W}_0$, and \eqref{eqn:cl3.4} uses that $\psi_{0}^1 = \alpha_1 + \bar{X}_0^T \beta_1$.

We observe that term (i) goes to zero by the law of large numbers. To show that (ii) and (iii) converge, we observe that as  $\hat{\Sigma}_X$ and $\hat{\Sigma}_\nu$ converge, $\hat{X}$ converges to $\tilde{X}$ uniformly over all units; as $\hat{\gamma}$ is a continuous function of the constraints determined by $\hat{X}$, it follows that $\hat{\gamma}$ converges to $\gamma^*$ as well. As $\hat{\gamma} \to \gamma^*$, term (ii) goes to 0 by Propositions \ref{cl2} and \ref{prop:variance_rate}. As $\|\hat{\gamma}\|$ is bounded, term (iii) goes to zero as $\hat{X} \to \tilde{X}$.

\end{proof}

\begin{proof}[Proof of Proposition \ref{prop:jackknife}]
Let $U_s = \{(J_{sc}, W_{sc}): c = 1,\ldots,p_s\}$ denote the observed outcomes and covariates corresponding to the CPUMAs in state $s$. Under our assumptions, it holds that $U_s$ is i.i.d. for the treated states. It can also be seen that $\hat{\psi}_0^{1,\textup{adjusted}}$ is a symmetric function of the treated state observations $\{U_s: A_s=1\}$. It follows that equation (1.6) of  \cite{efron1981jackknife} can be seen to apply to our setting; as this equation is equal to (\ref{eqn:prop.jackknife}), this proves the result.
\end{proof}


\begin{proof}[Proof of Proposition \ref{cl8}]
Let $\mu$ denote 
\[ \mu = \frac{1}{n_1} \sum_{A_{sc} = 1} \check{X}_{sc},\]
so that
\[ \check{X}_{sc} - \mu = \check{\kappa}^T(W_{sc} - \bar{W}_1),\]
and hence that 
\begin{align}
 \nonumber \check{\beta} &=  \left(\sum_{A_{sc}=1} (\check{X}_{sc} - \mu)(\check{X}_{sc} - \mu)^T\right)^{-1} \sum_{A_{sc}=1} (\check{X}_{sc} - \mu)(Y_{sc} - \bar{Y}_1) \\
 \nonumber & = \left(\sum_{A_{sc}=1} \check{\kappa}^T (W_{sc} - \bar{W}_1)(W_{sc} - \bar{W}_1)^T \check{\kappa}\right)^{-1} \sum_{A_{sc}=1} \check{\kappa}^T(W_{sc} - \bar{W}_1)(Y_{sc} - \bar{Y}_1) \\
 & \to^p (\check{\kappa}^T (S_X + \Sigma_\nu) \check{\kappa})^{-1} \check{\kappa}^T S_X \beta_1  \label{eqn:cl8.1}\\
 \nonumber 
 & = \check{\kappa}^{-1} (S_X + \Sigma_{\nu})^{-1} S_X \beta_1 \\
 \nonumber 
 & = \beta_1
 \end{align}
 where \eqref{eqn:cl8.1} follows by \eqref{eqn:limitW} and \eqref{eqn:limitWY}, and the last step follows from the definition of $\check{\kappa}$. It then follows that 
 \[ \bar{Y}_1 - (\bar{W}_0 - \bar{W}_1)^T \check{\beta} \to^p \alpha_1 + \bar{X}_0^T \beta_1 = \psi_0^1,\]
 proving consistency.
\end{proof}

\begin{proof}[Proof of Proposition \ref{cl9}]

We will use Proposition \ref{cl56} which is proved later in this section. To apply it, we let $\Omega = I$, $Z = \check{X}_{A=1}$, and $v = \bar{W}_0$. Using  $\check{X}_{sc} = \bar{W}_1 + \check{\kappa}^T(W_{sc} - \bar{W}_1)$, we find that 
\begin{align}
    \nonumber \mu & = \frac{1}{n_Q} \sum_{sc \in Q} \check{X}_{sc} \\
    \label{eqn:cl9.mu} & = \bar{W}_1 + \check{\kappa}^T(\bar{W}_Q - \bar{W}_1),
\end{align}
and hence that $\check{X}_{sc} - \mu = \check{\kappa}^T(W_{sc} - \bar{W}_Q)$. Plugging into \eqref{eqn:a1.3} yields 

\begin{align}
 \nonumber \check{\gamma}_{sc} & = \frac{1}{n_Q}(W_{sc} - \bar{W}_Q)^T \check{\kappa} (\check{\kappa}^T S_{W_Q} \kappa)^{-1}(\bar{W}_0 - \mu) + \frac{1}{n_Q} \\
 \label{eqn:cl9.proof.gamma}& = \frac{1}{n_Q}(W_{sc} - \bar{W}_Q)^T S_{W_Q}^{-1} \check{\kappa}^{-T}(\bar{W}_0 - \mu) + \frac{1}{n_Q}, \qquad \forall \ sc \in Q,
\end{align}
As $Y_{sc} = \alpha_1 + \beta_1^T (\bar{X}_Q + X_{sc} - \bar{X}_Q)$ for the treated units, the SBW estimator of $\psi_0^1$ can be seen to equal
\begin{align}
    \nonumber \sum_{A_{sc}=1} Y_{sc}\check{\gamma}_{sc} & = \sum_{A_{sc}=1} (\alpha_1 + \beta_1^T \bar{X}_Q) \check{\gamma}_{sc} + \sum_{A_{sc}=1}  \beta_1^T (X_{sc} - \bar{X}_Q)\check{\gamma}_{sc} \\
 \label{eqn:cl9.proof1}    & = (\alpha_1 + \beta_1^T \bar{X}_Q)\\
    \nonumber & \hskip.5cm {} + \sum_{sc \in Q}  \beta_1^T(X_{sc} - \bar{X}_Q)\left[\frac{1}{n_Q} (W_{sc} - \bar{W}_Q)^T S_{W_Q}^{-1} \check{\kappa}^{-T}(\bar{W}_0 - \mu) +  \frac{1}{n_Q}\right] \\
\label{eqn:cl9.proof2}    & = (\alpha_1 + \beta_1^T \bar{X}_Q)\\
    \nonumber & \hskip.5cm {} +  \beta_1^TS_{XW_Q} S_{W_Q}^{-1} \check{\kappa}^{-T}(\bar{W}_0 - \mu) + \underbrace{\frac{1}{n_Q}\sum_{sc \in Q}   \beta_1^T (X_{sc} - \bar{X}_Q) }_{= 0} \\ %+ o_p(1) \\
\label{eqn:cl9.proof3}    & = \alpha_1 + \beta_1^T \bar{X}_0 - \underbrace{(\beta_1^T \bar{X}_0 -  \beta_1^TS_{XW_Q} S_{W_Q}^{-1} \check{\kappa}^{-T}\bar{W}_0)}_{(i)} \\
    \nonumber & \hskip.5cm {} + \underbrace{\beta_1^T (\bar{X}_Q - S_{XW_Q} S_{W_Q}^{-1} \check{\kappa}^{-T}\bar{W}_1)}_{(ii)} - \underbrace{\beta_1^TS_{X_Q}S_{W_Q}^{-1}(\bar{W}_Q - \bar{W}_1))}_{(iii)}\\
\label{eqn:cl9.proof4}    & \to^p \psi_0^1 - \underbrace{\beta_1^T(I - S_{XW_Q}S_{W_Q}^{-1}S_WS_X^{-1})\bar{X}_0}_{(i)} \\
    \nonumber & \hskip.5cm {} + \underbrace{\beta_1^T(\bar{X}_Q - S_{XW_Q}S_{W_Q}^{-1} S_W S_X^{-1} \bar{X}_1)}_{(ii)} - \underbrace{\beta_1^TS_{XW_Q}S_{W_Q}^{-1}(\bar{X}_Q - \bar{X}_1)}_{(iii)}, 
\end{align}
where \eqref{eqn:cl9.proof1} uses the expression for $\check{\gamma}$ given by (\ref{eqn:cl9.proof.gamma}; \eqref{eqn:cl9.proof2} follows by algebraic manipulations, and notes that $n_Q^{-1}\sum_{sc \in Q} (X_{sc} - \bar{X}_Q) = 0$; \eqref{eqn:cl9.proof3} adds and subtracts $\beta_1^T \bar{X}_0$,  substitutes for $\mu$ using (\ref{eqn:cl9.mu}), and groups the terms into (i), (ii), and (iii); and \eqref{eqn:cl9.proof4} substitutes for $\check{\kappa}$ and uses $\bar{W}_0 \to^p \bar{X}_0$ and $\bar{W}_1 \to^p \bar{X}_1$.

It can be seen that terms (i), (ii), and (iii) each go to zero if $S_{XW_Q} \to S_X$, $S_{W_Q} \to S_W$, and $\bar{X}_Q \to \bar{X}_1$, proving the result. 
\end{proof}

\begin{proof}[Proof of Proposition \ref{cl4}]
\begin{align*}
    Var[n_t^{-1}\sum_{s: A_s = 1}\sum_{c = 1}^{p_s}\gamma_{sc}Y_{sc} \mid X_{sc}, A_{sc}] &= n_t^{-2}\sum_{s: A_s = 1}[\sum_{c = 1}^{p_s}\gamma_{sc}^2(\sigma^2_{\epsilon} + \sigma^2_{\varepsilon}) + \sum_{c \ne d}\gamma_{sc}\gamma_{sd}\sigma^2_{\varepsilon}] \\
    &\propto \sum_{s: A_s = 1}[\sum_{c = 1}^{p_s}\gamma_{sc}^2 + \sum_{c \ne d}\rho \gamma_{sc}\gamma_{sd}]
\end{align*}
%
where the second line follows by dividing by $\sigma^2_{\epsilon} + \sigma^2_{\varepsilon}$. By definition of the H-SBW objective, which minimizes this function for known $\rho$, the H-SBW estimator must produce the minimum conditional-on-X variance estimator within the constraint set.
\end{proof}


\begin{proof}[Proof of Proposition \ref{cl56}]

    To show that $\gamma^*$ solves (\ref{eqn:a1.2}), we first observe that it is a feasible solution, by definition of $Q$. The result can then be proven by contradiction: if $\gamma^*$ is feasible but does not solve (\ref{eqn:a1.2}), then a feasible $\tilde{\gamma}$ must exist with lower objective value. Then for some convex combination $\gamma_\lambda = \lambda \tilde{\gamma} + (1-\lambda)\gamma^*$ with $\lambda > 0$, we can show that that $\gamma_\lambda$ is both feasible for (\ref{eqn:a1.1}), and has lower objective value than $\gamma^*$:
    \begin{enumerate}
        \item     To establish that $\gamma_\lambda$ is feasible, we observe that $\tilde{\gamma}$ is feasible for (\ref{eqn:a1.2}). This implies that if $\gamma^*_i=0$, then $\tilde{\gamma}_i=0$ as well; as a result, there exists $\lambda > 0$ such that the convex combination $\gamma_\lambda$ satisfies $\gamma_\lambda \geq 0$ and hence is feasible for (\ref{eqn:a1.1}).
    \item     To show that this $\gamma_\lambda$ has lower objective value than $\gamma^*$, we observe that if $\tilde{\gamma}$ has lower objective value than $\gamma^*$, then by strict convexity of the objective any convex combination with $\lambda > 0$ must have lower objective value than $\gamma^*$ as well.
    \end{enumerate}
    This shows that if $\gamma^*$ is not optimal for (\ref{eqn:a1.2}), then it is not optimal for (\ref{eqn:a1.1}) either. But as $\gamma^*$ is the optimal solution to (\ref{eqn:a1.1}), this is a contradiction; hence by taking the contrapositive it follows that $\gamma^*$ must solve (\ref{eqn:a1.2}).
    
To show \eqref{eqn:a1.3}, we observe that \eqref{eqn:a1.2} can be written as

\[ \min_{\gamma_Q} \gamma_Q^T \Omega_Q \gamma_Q \quad \text{subject to} \quad Z_Q \gamma_Q = v \ \text{ and } \ {\bf 1}^T \gamma_Q = 1,\] 
which can be rewritten as

\[ \min_{\gamma_Q} \gamma_Q^T \Omega_Q \gamma_Q \quad \text{subject to} \quad \left[ \begin{array}{c} Z_Q - \mu {\bf 1}^T\\ {\bf 1}^T \end{array}\right] \gamma_Q = \left[\begin{array}{c} v - \mu & 1 \end{array}\right],\] 
where we have subtracted $\mu{\bf 1}^T \gamma_Q$ (which equals $\mu$) from both sides of the constraint. This is a least norm problem, and when feasible has solution

\begin{equation}\label{eq:a1.least_norm}
 \gamma_Q^* = \Omega^{-1}_Q A^T (A\Omega^{-1}_QA^T)^{-1} b,
\end{equation}
where $A = \left[ \begin{array}{c} Z_Q - \mu {\bf 1}^T\\ {\bf 1}^T \end{array}\right]$ and $b = \left[\begin{array}{c} v - \mu & 1 \end{array}\right]$. As $(Z_{Q} - \mu{\bf 1}^T)\Omega_Q^{-1} {\bf 1} = 0$, it follows that $A\Omega^{-1}A^T$ is block diagonal

\[ A\Omega_Q^{-1}A^T = \left[\begin{array}{cc} (Z_Q - \mu)\Omega_Q^{-1} (Z_Q- \mu)^T & 0  \\ 0 & {\bf 1}^T \Omega^{-1}{\bf 1}\end{array}\right], \]
so that plugging into \eqref{eq:a1.least_norm} yields
 \begin{equation*}
 \gamma^*_{Q} = \Omega_{Q}^{-1} (Z_{Q} - \mu)^T\left[ (Z_Q - \mu) \Omega_{Q}^{-1} (Z_Q - \mu)\right]^{-1} (v - \mu) + \frac{\Omega^{-1}_Q {\bf 1} }{{\bf 1}^T \Omega^{-1}_Q {\bf 1}},
 \end{equation*}
proving the result.

\end{proof}    
    
\begin{proof}[Proof of Proposition \ref{cl7}]
   It can be seen that the derivation of \eqref{eqn:sbwregcalerror} holds for the H-SBW weights $\hat{\gamma}^{hsbw}$, and by similar steps it can be shown that 
    
    \begin{align*}
        \hat{\psi}^{1, hsbw}_0 - \psi^1_0 = \sum_{sc: A_{sc} = 1}\hat{\gamma}^{hsbw}_{sc}(X_{sc} - \tilde{X}_{sc}^\star)^T\beta_1
    \end{align*}
Conditional on $W$, $\hat{\gamma}_{sc}$ is fixed and $X_{sc} - \tilde{X}_{sc}^*$ equals $X_{sc} - \mathbb{E}[X_{sc}|W, A=1]$ which has mean zero, proving the result.
\end{proof}


