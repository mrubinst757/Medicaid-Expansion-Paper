\section{Proofs}

We show three results regarding the bias of the SBW estimator under the classical errors-in-variables model. First, we show that the bias of the SBW estimator that sets $\delta = 0$ (i.e. exactly balances the means of the covariates) is equal to the bias of the OLS estimator. Second, we show that if the observed covariate values can be replaced by their conditional expectations $\eta$ given the noisy observations, then the SBW estimator will be unbiased. Finally, we consider the case where we can estimate $\eta$ using auxillary data to estimate the covariance matrix of the error terms and show that the SBW estimator is consistent if we replace $\eta$ by an estimate $\hat{\eta}$ in the constraint set. In contrast to our application, we take the perspective throughout that $X$ is random, noting that this estimator still has desirable properties even when we view $X$ is fixed (see, e.g., \cite{gleser1992importance}).

Consider a dataset of consists of $i = 1, ..., n$ units where we observe the outcomes $Y_i$ and a treatment assignment indicator $A_i \in \{0, 1\}$. Let $n_a$ be the number of units in each treatment group, i.e, $n_a = \sum_{i = 1}^n \mathbbm{1}(A_i = a)$. Let $X_i \in \mathbb{R}^q$ be a vector of covariate values where $(X_i \mid A_i) \stackrel{iid}\sim MVN(\upsilon_a, \Sigma_{XX})$. 

Our target parameter is $\psi = \mathbb{E}\{Y_i^1 \mid A_i = 0\}$.\footnote{Note that we switch to the super-population target because we view $X$ as random.} We assume unconfoundedness ($Y_i^a \perp A_i \mid X_i$), consistency ($Y_i^a = Y_i \mid A_i = a$), and that $\mathbb{E}\{Y_i \mid X_i, A_i = a\} = \alpha_a + X_i^T\beta_a$. For simplicity we also assume throughout that there exist positive weights $\gamma$ that exactly balance the covariates, i.e. $\sum_{i: A_i = 1}\gamma_iX_{i, r} = \bar{X}_{0, r}$ for all $r = 1, ..., q$ and $\sum_{i: A_i = 1}\gamma_i = 1$, $\gamma_i > 0$. 

We begin by establishing the following identity:

\begin{equation}
\psi = \mu_y + (\upsilon_0 - \upsilon_1)^T\beta_1
\end{equation}

where $\mu_y = \mathbb{E}\{Y_i \mid A_i = 1\}$ and $\upsilon_a = \mathbb{E}\{X_i \mid A_i = a\}$.

\begin{proof}

Using our causal and modeling assumptions we have that:

\begin{align*}
\mathbb{E}\{Y_i^1 \mid X_i, A_i = 0\} &= \mathbb{E}\{Y_i^1 \mid X_i, A_i = 1\} \\
&= \mathbb{E}\{Y_i \mid X_i, A_i = 1\} \\
&= \alpha_1 + X_i^T\beta_1 \\
&= \mu_y + (X_i - \upsilon_1)^T\beta \\
&\implies \psi = \mu_y + (\upsilon_0 - \upsilon_1)^T\beta_1
\end{align*}

where the first equality follows from unconfoundedness, the second equality from consistency, the third from our parametric modeling assumptions, and the fourth by definition of $\alpha$. The final implication comes from plugging $\upsilon_0$ in place of $X_i$.

\end{proof}

We now outline the classical errors-in-variables model. Let $Y_i(X_i, a) = \alpha_a + X_i^T\beta_a + \epsilon_i$ where $\epsilon_i \stackrel{iid}\sim N(0, \sigma^2)$. We consider the case where we observe $W_i$, a vector of mean-unbiased proxies for the true (unobserved) covariate vector $X_i$; i.e., $W_i = X_i + v_i$, where $v_i \stackrel{iid}\sim MVN(0, \Sigma_{vv})$. Consider the model:

\begin{equation}
(\epsilon_i, v_i) \sim MVN((0, 0), \begin{pmatrix} 
\sigma^2 & 0 \\ 
0 & \Sigma_{vv}  
\end{pmatrix}
\end{equation}

In other words the error in the outcome model is uncorrelated with the error in the covariates. Let $u_i = (\epsilon_i, v_i)$. We further assume that the covariates are uncorrelated with any of the error terms:

\begin{equation}
(X_i, u_i) \sim MVN((\upsilon_a, 0), \begin{pmatrix} 
\Sigma_{XX} & 0 \\ 
0 & \Sigma_{uu}  
\end{pmatrix}
\end{equation}

Then we see that $(X_i, W_i) \mid A_i = a \stackrel{iid}{\sim} MVN((\upsilon_a, \upsilon_a), \Sigma)$\footnote{In contrast to our application, we instead assume here that $\Sigma_{WW \mid A = 1} = \Sigma_{WW \mid A = 0} = \Sigma_{WW}$ and $\Sigma_{XX \mid A = 1} = \Sigma_{XX \mid A = 0} = \Sigma_{XX}$. This is not a necessary assumption, but helps simplify notation.} where 

$$
\Sigma = \begin{pmatrix} 
\Sigma_{XX} & \Sigma_{XX} \\ 
\Sigma_{XX} & \Sigma_{WW}  
\end{pmatrix}
$$ 

and $\Sigma_{WW} = \Sigma_{XX} + \Sigma_{vv}$. Let $\kappa = \Sigma_{WW}^{-1}\Sigma_{XX}$. By the normality of the joint distribution of $X_i$ and $W_i$, we also know that

\begin{equation}
\mathbb{E}\{X_i \mid W_i, A_i = a\} = \upsilon_a + \kappa^T(W_i - \upsilon_a)
\end{equation}

We consider estimating $\psi$ in this setting. Let 

$$
\hat{\psi}^{reg} = \bar{Y}_1 + (\bar{W}_0 - \bar{W}_1)^T\hat{\beta}_1
$$ 

where $\hat{\beta}_1$ is the OLS estimator of $\beta_1$, $\bar{W}_a = n_a^{-1}\sum_{i:A_i = a} W_i$ and $\bar{Y}_a$ is defined analogously. Let 

$$
\hat{\psi}^{sbw} = \sum_{i: A_i = a} \gamma_i Y_i
$$ 

be the SBW estimator, where

\begin{align*}
\gamma &= \arg\min_{\theta} \sum_{i: A_i = 1}\theta_i^2 \text{ such that } W_{1, r}\theta = \bar{W}_{0, r} \ \ \ (r = 1, ..., q) \\
&\theta_i > 0, \sum_{i: A_i = 1}\theta_i = 1
\end{align*}

where $W_{a, r}$ is the $n_a$ dimensional (row) vector associated with covariate $r$.

\begin{proposition}\label{cl1}
The bias of $\hat{\psi}^{reg}$ is equal to the bias of $\hat{\psi}^{sbw}$; specifically, 

$$
\mathbb{E}\{\psi^{reg} - \psi\} = \mathbb{E}\{\psi^{sbw} - \psi\} = (\upsilon_0 - \upsilon_1)^T(\kappa - I_d)\beta
$$
\end{proposition}

\begin{proof}
Recall that $\mathbb{E}(\hat{\beta}_1) = \kappa\beta_1$ (c.l. \cite{gleser1992importance}). Consider the error of $\hat{\psi}^{reg}$: 

\begin{align*}
    \hat{\psi}^{reg} - \psi &= \bar{Y}_1 + (\bar{W}_0 - \bar{W}_1)^T\hat{\beta}_1 - (\mu_y + (\upsilon_0 - \upsilon_1)^T\beta_1) \\
    &= \underbrace{(\bar{Y}_1 - \mu_y)}_{T_1} + \underbrace{(\bar{W}_0 - \upsilon_0)^T\hat{\beta}_1}_{T_2} - \underbrace{(\bar{W}_1 - \upsilon_1)^T\hat{\beta}_1}_{T_3} + \underbrace{(\upsilon_0 - \upsilon_1)^T(\hat{\beta}_1 - \beta_1)}_{T_4} \\
    \implies \mathbb{E}\{\hat{\psi}^{reg} - \psi\} &= (\upsilon_0 - \upsilon_1)^T(\kappa - I_d)\beta_1
\end{align*}

The first equality holds by definition and the second by rearranging terms. The second equality consists of four terms. $T_1$ is simply the estimation error from a sample average, which has expectation zero. $T_2$ is the product of the estimation error from a sample average and $\hat{\beta}$; this also has expectation zero because $\bar{X}_0$ is estimated on a different part of the sample than $\hat{\beta}$, so these errors are independent. $T_3$ also has expectation zero because $\bar{W}_1^T\hat{\beta}_1 = \bar{Y}_1$, which has expectation $\mu_y$, and $\upsilon_1^T\hat{\beta}_1$ also has expectation $\mu_y$. We are then left with $T_4$; we substitute $\mathbb{E}{\hat{\beta}} = \kappa\beta$ to get the final result. 

We now derive the bias of $\hat{\psi}^{sbw}$:

\begin{align*}
    \hat{\psi}^{sbw} - \psi &= \sum_{i: A_i = 1}\gamma_iY_i - (\alpha + \upsilon_0^T\beta) \\
    &= \sum_{i: A_i = 1} \gamma_i(\alpha + X_i^T\beta + \epsilon_i) - (\alpha + \bar{W}_0^T\beta + (\upsilon_0 - \bar{W}_0)^T\beta) \\
    &= \sum_{i: A_i = 1} (\gamma_i(W_i - v_i)^T\beta + \gamma_i\epsilon_i) - \bar{W}_0^T\beta + (\upsilon_0 - \bar{W}_0)^T\beta \\
    &= \underbrace{-\sum_{i: A_i = 1}\gamma_iv_i^T\beta}_{T_1} + \underbrace{\sum_{i: A_i = 1}\gamma_i\epsilon_i}_{T_2}  + \underbrace{(\upsilon_0 - \bar{W}_0)^T\beta}_{T_3}
\end{align*}

Conditioning on $W_i$, we can take expectations over $X_i$, and see that $T_2$ has expectation zero (noting that the weights, conditional on $W_i$, are independent of these errors). $T_3$ is simply the scaled sum of mean zero estimation error and therefore has expectation zero. We conclude by considering $T_1$ and again take expectations over $X_i$ conditional on $W_i$: 

\begin{align*}
    \sum_{i: A_i = 1} \gamma_i\mathbb{E}\{X_i - W_i \mid W_i\}^T\beta &= \sum_{i: A_i = 1} \gamma_i (\upsilon_1 + \kappa^T(W_i - \upsilon_1))^T\beta - \sum_{i: A_i = 1}\gamma_i W_i^T\beta \\
    &= (\upsilon_1 + \kappa^T(\bar{W}_0 - \upsilon_1))^T\beta - \bar{W}_0^T\beta \\
    &= (\kappa^T(\bar{W}_0 - \upsilon_1))^T\beta - (\bar{W}_0 - \upsilon_1)^T\beta  \\
    &= (\bar{W}_0 - \upsilon_1)^T(\kappa - I_d)\beta \\
    &= (\upsilon_0 - \upsilon_1)^T(\kappa - I_d)\beta + (\bar{W}_0 - \upsilon_0)^T(\kappa - I_d)\beta \\
    \implies \mathbb{E}\{\hat{\psi}^{sbw} - \psi\} &= (\upsilon_0 - \upsilon_1)^T(\kappa - I_d)\beta
\end{align*}

The final line holds because in the second to last line, we can take expectation over $W_i$ and see that the second term is a scaled sum of mean zero estimation error and has expectation zero. 
\end{proof}

Let $\hat{\psi}^{sbw}(\eta)$ be the SBW estimator that reweights $\eta_1(W_i)$ rather than $W_i$; i.e. $\hat{\psi}^{sbw}(\eta) = \sum_{i: A_i = 1}\gamma_i^\star Y_i$ where 

\begin{align*}
\gamma^\star = &\arg\min_{\theta} \sum_{i: A_i = 1}\theta_i^2 \text{ such that } \eta_1(W_{1, r})\theta = \upsilon_{0, r} \ \ \ (r = 1, ..., q) \\
&\theta_i > 0, \sum_{i: A_i = 1} \theta_i = 1
\end{align*}

\begin{proposition}
The estimator $\hat{\psi}^{sbw}(\eta)$ is unbiased, i.e.
$\mathbb{E}\{\hat{\psi}^{sbw}(\eta)\} = \psi$
\end{proposition}

\begin{proof}

By linearity we know that

\begin{align*}
Y_i = \alpha + \eta_a(W_i)^T\beta_a + (X_i - \eta_a(W_i))^T\beta_a
\end{align*}

We then have that:

\begin{align*}
    \hat{\psi}^{sbw}(\eta) - \psi &= \sum_{i: A_i = 1}\gamma_i^\star Y_i - (\alpha + \upsilon_0^T\beta) \\
    &= \sum_{i: A_i = 1}\gamma_i^\star\alpha + \sum_{i: A_i = 1}\gamma_i^\star\eta_1(W_i)^T\beta_1 + \sum_{i: A_i = 1}\gamma_i^\star(X_i - \eta_1(W_i))^T\beta_1 - (\alpha + \upsilon_0^T\beta_1) \\
    &= \sum_{i: A_i = 1}\gamma_i^\star(X_i - \eta_1(W_i))^T\beta_1
\end{align*}

Conditional on $W$, the weights are fixed and $X_i - \eta_1(W_i)$ has expectation zero; therefore, the estimator is unbiased.

\end{proof}
\begin{remark}

While we have assumed no model error, this estimator still has variance, conditional on $W_i$, equal to

$$
\sum_{i: A_i = 1} \gamma_i^\star^2\beta^TCov(X_i \mid W_i)\beta
$$

where, assuming that $(X_i, W_i)$ are jointly normally distributed, $Cov(X_i \mid W_i) = \Sigma_{XX} - \Sigma_{XX}\Sigma_{WW}^{-1}\Sigma_{XX}$. Therefore, the variance of this estimator is higher than if we knew the true $X_i$ unless $\Sigma_{WW} = \Sigma_{XX}$ (i.e. we observe $X_i$). 

\end{remark}

\begin{remark}
Assuming there is a model error simply leads to the additional term $\sum_{i: A_i = 1}\gamma_i\epsilon_i$. This again has expectation zero, because the weights remain independent of the error in the outcome model, and adds a term to the total variance (conditional on $W_i$) equal to

$$
\sigma^2\sum_{i: A_i = 1}\gamma_i^\star^2
$$
\end{remark}

Proposition 2 assumes that we know $\eta_a$ and $\upsilon_0$; however, in practice we estimate it from the data. Moreover, the estimation of $\hat{\eta}_1$ typically involves both the observed dataset and auxillary data, which we have not yet specified here. We now consider a simple version of this case.

Recall that $\eta_a(W_i) = \upsilon_a + \kappa^T(W_i - \upsilon_a)$, where $\kappa = \Sigma_{WW}^{-1}\Sigma_{XX}$. We can easily estimate $\upsilon_a$ consistently using $\bar{W}_a$; the challenge is estimating $\kappa$. 

Following \cite{gleser1992importance}, consider the setting where we observe $T$ independent vectors of measurements $W_i^\star$ from known values $X_i^\star$, so that $v_i^\star = W_i^\star - X_i^\star$. Assume that $\Sigma_{v^\star v^\star} = \Sigma_{vv}$. We can then estimate $\hat{\Sigma}_{vv} = \frac{1}{T}\sum_{i=1}^T(W_i^\star - X_i^\star)'(W_i^\star - X_i^\star)$. We can then estimate

$$
\hat{\kappa} = (n\hat{\Sigma}_{WW})^{-1}(n(\hat{\Sigma}_{WW} - \hat{\Sigma}_{vv}))
$$

assuming $n(\hat{\Sigma}_{WW} - \hat{\Sigma}_{vv})$ is positive semi-definite, and where $\hat{\Sigma}_{WW} = \frac{1}{n}\sum_{i=1}^n (W_i - \bar{W})(W_i - \bar{W})'$. 

\begin{proposition}
Let $\tilde{\gamma}$ be the weights that solve the SBW objective with $\sum_{i: A_i = 1}\tilde{\gamma}_i\hat{\eta}_1(W_{i, r}) = \bar{W}_{0, r}$ for all $r = 1, ..., q$. The estimator $\sum_{i: A_i = 1}\tilde{\gamma}_iY_i \to \psi$ as $n \to \infty$, $T \to \infty$.
\end{proposition}

\begin{proof}

Rewriting the error expression from Lemma A.2 leads to the additional terms $\sum_{i: A_i = 1}\tilde{\gamma}_i(\eta_1(W_i) - \hat{\eta_1}(W_i))^T\beta$ and $(\upsilon_0 - \bar{W}_0)^T\beta$. The second term is scaled estimation error and has expectation zero. It therefore suffices to show that 

$$
\sum_{i: A_i = 1}\tilde{\gamma}_i(\eta_1(W_i) - \hat{\eta_1}(W_i))^T\beta \to 0
$$

By the weak law of large numbers, $\hat{\Sigma}_{WW} - \hat{\Sigma}_{vv} \to \Sigma_{XX}$ as $n \to \infty$, $T \to \infty$; similarly $\hat{\Sigma}_{WW} \to \Sigma_{WW}$ as $n \to \infty$. By the continuous mapping theorem $\hat{\kappa} \to \kappa$. Since $\bar{W}_0 \to \upsilon_0$, we have that $\hat{\eta} \to \eta$. 

Let $\gamma^\star$ be the weights defined in Lemma A.3. We can then rewrite the error term above as

\begin{align*}
\sum_{i: A_i = 1}(\tilde{\gamma}_i - \gamma_i^\star)(\hat{\eta}_1(W_i) - \eta(W_i))^T\beta + \sum_{i: A_i = 1}\gamma_i^\star(\hat{\eta}_1(W_i) - \eta_1(W_i))^T\beta
\end{align*}

As $n \to \infty$, $T \to \infty$, $\hat{\eta}_1 \to \eta_1$, and therefore by definition of the SBW objective, $\tilde{\gamma}_i \to \gamma_i^\star$. These two terms therefore both converge in probability to zero. 

\end{proof}

Our estimation of $\eta_a$ follows a somewhat different procedure which we outline in Appendix B.

\subsection{H-SBW Objective}

We show that the H-SBW estimator minimizes the variance of any estimator within the constraint set given a constant within-group correlation structure.

Consider the outcome model:

\begin{equation}
    Y_{sc} = X_{sc}^T\beta + c_s + \epsilon_{sc}
\end{equation}

where $\mathbb{E}\{c_s \mid X_{sc}\} = \mathbb{E}\{\epsilon_{sc} \mid X_{sc}\} = \mathbb{E}\{c_s\epsilon_{sc} \mid X_{sc}\} = 0$, and $Var(\epsilon_{sc}) = \sigma^2$ and $Var(c_s) = \eta^2$. Let $\rho = \frac{\eta^2}{\sigma^2 + \eta^2}$.

Then consider the variance of any weighting estimator using weights $\gamma$ in the form of $Y\gamma$ where $\gamma^T1 = n_t$.

\begin{align}
    Var(n_t^{-1}\sum_{sc: A_s = 1}\gamma_{sc}Y_{sc} \mid X_{sc}, A_{sc}) &= n_t^{-2}\sum_{s: A_s = 1}^{m_1}(\sum_{c = 1}^{p_s}\gamma_{sc}^2(\sigma^2 + \eta^2) + \sum_{c \ne d}\gamma_{sc}\gamma{sd}\eta^2) \\
    &\prop\to n_t^{-2}\sum_{s: A_s = 1}(\sum_{c = 1}^{p_s}(\gamma_{sc}^2 + \sum_{c \ne d}\rho \gamma_{sc}\gamma{sd})
\end{align}

where the second line follows by dividing by $\eta^2 + \sigma^2$. This suggests the H-SBW objective, which minimizes the (conditional) variance of this estimator for known $\rho$.
