\section{Additional Results}\label{app:allresults}

Table~\ref{tab:pretxpredfull} presents the full model validation results that also include estimators using the Oaxaca-Blinder OLS and GLS weights (see, e.g, \cite{kline2011oaxaca}), with the rows ordered by RMSE of the prediction errors. We find in general that these estimators perform quite poorly, again showing that the cost of extrapolation is quite high in our application. We see some particularly extreme numbers for the error using the heterogeneous adjustment in 2012: one reason for this may be because in practice that the heterogeneous adjustment is more likely to impute values that fall outside of the support of the data. In some cases we have found that these imputations are quite extreme. The estimators that extrapolate from the data are then more likely to fit to these values, leading to poor estimator performance overall. This result makes the heterogeneous adjustment potentially less desirable, particularly for estimators that allow extrapolation.

\begin{table}[ht]
\caption{Estimator pre-treatment outcome prediction error}\label{tab:pretxpredfull}
\begin{tabular}{llrrr}
  \hline
Sigma estimate & Estimator & 2011 error & 2012 error & RMSE \\ 
  \hline
Homogeneous & SBW & -0.18 & -0.22 & 0.20 \\ 
  Homogeneous & H-SBW & -0.24 & -0.21 & 0.23 \\ 
  Heterogeneous & SBW & -0.25 & -0.30 & 0.27 \\ 
  Heterogeneous & H-SBW & -0.32 & -0.39 & 0.36 \\ 
  Homogeneous & BC-SBW & -0.42 & -0.35 & 0.39 \\ 
  Heterogeneous & BC-SBW & -0.45 & -0.39 & 0.42 \\ 
  None & SBW & -0.50 & -0.61 & 0.56 \\ 
  None & H-SBW & -0.52 & -0.61 & 0.57 \\ 
  Homogeneous & BC-HSBW & -0.53 & -0.62 & 0.58 \\ 
  Heterogeneous & BC-HSBW & -0.53 & -0.72 & 0.63 \\ 
  None & BC-SBW & -0.82 & -0.93 & 0.88 \\ 
  None & OLS & -0.91 & -0.84 & 0.88 \\ 
  None & GLS & -0.87 & -0.91 & 0.89 \\ 
  None & BC-HSBW & -0.93 & -0.99 & 0.96 \\ 
  Homogeneous & OLS & -1.75 & -1.21 & 1.50 \\ 
  Homogeneous & GLS & -1.76 & -1.45 & 1.61 \\ 
  Heterogeneous & GLS & -1.18 & -17.60 & 12.47 \\ 
  Heterogeneous & OLS & -0.85 & -22.90 & 16.21 \\ 
   \hline
\end{tabular}
\end{table}


Table~\ref{tab:confintmain} displays the point estimates and confidence intervals from all estimators, including estimates calculated on a second version of the adjusted data where we model a separate $\kappa$ for all values (the ``heterogeneous adjustment''). ``Homogeneous'' corresponds to our preferred covariate adjustment, and are the results presented in the main paper.

\begin{table}[h!]
\centering
\begin{threeparttable}
\caption{Point estimates and confidence intervals: primary dataset}
\label{tab:confintmain}
\begin{tabular}{llrl}
  \hline
Weight type & Adjustment & Psihat & 95\% CI \\ 
  \hline
H-SBW & Homogeneous & -2.33 & (-3.47, -1.19) \\ 
  H-SBW & Heterogeneous & -2.24 & (-3.4, -1.08) \\ 
  H-SBW & None & -2.34 & (-2.85, -1.82) \\ 
  BC-HSBW & Homogeneous & -2.05 & (-3.22, -0.87) \\ 
  BC-HSBW & Heterogeneous & -1.98 & (-3.13, -0.83) \\ 
  BC-HSBW & None & -2.22 & (-2.87, -1.56) \\ 
  SBW & Homogeneous & -2.35 & (-3.67, -1.03) \\ 
  SBW & Heterogeneous & -2.28 & (-3.5, -1.05) \\ 
  SBW & None & -2.39 & (-2.95, -1.83) \\ 
  BC-SBW & Homogeneous & -2.07 & (-3.07, -1.06) \\ 
  BC-SBW & Heterogeneous & -2.00 & (-3, -0.99) \\ 
  BC-SBW & None & -2.19 & (-2.9, -1.49) \\ 
   \hline
\end{tabular}
    \begin{tablenotes}
      \item[] Note: Confidence intervals are estimated using the leave-one-state-out jackknife and the standard normal quantiles.
    \end{tablenotes}
\end{threeparttable}
\end{table}

Table~\ref{tab:confintmainc2} presents the same results when excluding the early expansion states.

\begin{table}[h!]
\centering
\begin{threeparttable}
\caption{Point estimates and confidence intervals: early expansion excluded}
\label{tab:confintmainc2}
\begin{tabular}{llrl}
  \hline
Weight type & Adjustment & Psihat & 95\% CI \\ 
  \hline
H-SBW & Homogeneous & -2.09 & (-3.15, -1.03) \\ 
  H-SBW & Heterogeneous & -2.06 & (-3.26, -0.87) \\ 
  H-SBW & None & -2.28 & (-2.82, -1.74) \\ 
  BC-HSBW & Homogeneous & -1.94 & (-3.17, -0.72) \\ 
  BC-HSBW & Heterogeneous & -1.93 & (-3.42, -0.45) \\ 
  BC-HSBW & None & -2.22 & (-3.07, -1.38) \\ 
  SBW & Homogeneous & -2.05 & (-3.10, -1.00) \\ 
  SBW & Heterogeneous & -2.03 & (-3.24, -0.82) \\ 
  SBW & None & -2.21 & (-2.71, -1.72) \\ 
  BC-SBW & Homogeneous & -1.99 & (-3.22, -0.77) \\ 
  BC-SBW & Heterogeneous & -2.00 & (-3.52, -0.47) \\ 
  BC-SBW & None & -2.23 & (-3.05, -1.40) \\ 
   \hline
\end{tabular}
    \begin{tablenotes}
      \item[] Note: Confidence intervals are estimated using the leave-one-state-out jackknife and the standard normal quantiles.
    \end{tablenotes}
\end{threeparttable}
\end{table}

Figure~\ref{fig:loostateplot} displays the change in our estimator when removing each state for all four of our estimators on the adjusted dataset (``homogeneous'') and the unadjusted dataset (``none''). These results condition on the covariate adjustment, but are similar when recalculating the entire adjustment procedure. Additional results are available on request.

\begin{figure}[H]
\begin{center}
    \caption{Leave-one-state-out point estimates minus primary estimate}
    \label{fig:loostateplot}
    \includegraphics[scale=0.6]{01_Plots/loostate-sensitivityc1-proc-uu-avg.png}
\end{center}
\subcaption{Colors reflect the magnitude of the difference in the estimates when subtracting the original estimate from the estimate that excludes the specified state. The values in the right panel are on the unadjusted data.}
\end{figure}

\clearpage
