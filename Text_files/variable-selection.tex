\section{Variable selection}\label{ssec:varselection}

\subsection{Synthetic controls variable weighting algorithm}

In this section we give further details about the variable weighting procedure of the synthetic controls algorithm, which we allude to in the primary paper. We also details about the types of tuning procedures that have been proposed using pre-treatment data, and again argue that these procedures may be sub-optimal when predicting the ETC. This is not meant to be a comprehensive treatment of these issues, but rather to more formally illustrate the problem of using pre-treatment data to conduct variable selection or variable weighting when estimating the ETC. For further details on the specifics of these ideas, we refer to \cite{abadie2010synthetic}, \cite{abadie2015comparative}, \cite{kaul2015synthetic}.

The synthetic controls algorithm chooses the weights $\gamma$ that minimizes the weighted L2-squared distance of the covariates using a diagonal weighting matrix $V$. $V$ is then chosen to minimize the mean-square error of the weighted difference in pre-treatment outcomes. Letting $Z_a$ be the matrix of pre-treatment outcomes for treatment group $A = a$, and $X_a$ being a covariate matrix defined similarly, the synthetic controls algorithm solves the following optimization problem for a fixed $V$:

\begin{align}
\gamma(V) = \arg\min_{\tilde{\gamma}(V^\star)} = (\bar{X}_1 - X_0^T\tilde{\gamma})'V(\bar{X}_1 - X_0^T\tilde{\gamma}) 
\end{align}

This is the ``inner'' optimization. $V^\star$ is then determined in an ``outer'' optimization to minimize the imbalances in the pre-treatment outcomes $Z$:

\begin{align}
    V^\star = \arg\min_V (\bar{Z}_1 - Z_0^T\gamma(V))'(\bar{Z}_1 - Z_0^T\gamma(V))
\end{align}

In applications the covariate matrix $X_a$ may contain some elements of $Z_a$. In cases where $X_a$ contains all pre-treatment outcomes, \cite{kaul2015synthetic} has shown that the predictor weights $V^\star$ will give no weight to auxillary covariates (covariates that are not the pre-treatment outcomes), rendering these irrelevant to the model. 

While in practice $V$ is often learned on the same data as the weights, \cite{abadie2010synthetic} also propose to use cross-validation to choose $V$. For this procedure, we assume we have enough data that we can divide our pre-treatment data into a training data from periods $T = 1, ..., T - l - 1$, a validation period from periods $T - l, ..., T - 1$, and a post-treatment period at time $T$. To make this discussion more general, assume that we are evaluating a set of candidate models $\mathcal{M}$ on the validation data (where for the synthetic controls algorithm we can think of this as the set of all possible weighting matrices $V$). Let $\bar{Y}^a_{a', t}$ be the expected potential outcome under treatment $A = a$ for treatment group $A = a'$ at time $t$ (where $t$ occurs during the validation period). Let $\hat{Y}^a_{a'', t}(m)$ be an estimator of that potential outcome at time $t$ using model $m$, trained during the training period using data from treatment group $A = a''$. Finally, let $\bar{Y}_{a'}^{a, T}$ be the post-treatment target estimand, where $\hat{Y}^a_{a'', T}(m)$ is the estimator using model $m$ learned during the training period and fit to the validation period data. This model selection procedure implicitly assumes that:

\begin{align*}
m^\star = \min_{m \in \mathcal{M}}\sum_{T - l}^{T-1}\|\hat{Y}^0_{0, t}(m) - \bar{Y}^0_{1, t}\| = \min_{m \in \mathcal{M}}\mathbb{E}\{\|\hat{Y}^0_{0, T}(m) - \bar{Y}^0_{1, T}\|\}
\end{align*}

In other words, this procedure selects the model (or weighting matrix $V$) using the empirical loss in the validation period as a proxy for the expected loss in the post-treatment time-period.\footnote{It is possible that multiple models in $\mathcal{M}$ either perfectly predict the pre-treatment outcomes, or predict them equally well. In this case we would require an additional criteria to choose the optimal model (see, e.g, \cite{becker2017cross}).} This has some intuitive appeal in the typical synthetic controls setting where the estimand is the ETT since we observe $Y^0_{sct}$ for $t < T$. 

However, when synthetic controls are used to estimate the ETC, this idea is more troubling because we never observe $Y^1_{sct}$ (or a mean-unbiased proxy) prior to treatment for any unit. If we wish to use pre-treatment outcomes to optimally select variables or determine relative covariate importance, we would need strong assumptions, for example:

\begin{align*}\label{assumption:second}
m^\star = \min_{m \in \mathcal{M}}\sum_{T - l}^{T-1}\|\hat{Y}^0_{1, t}(m) - \hat{Y}^0_{0, t}\| = \min_{m \in \mathcal{M}}\mathbb{E}\{\|\hat{Y}^1_{1, T}(m) - \bar{Y}^1_{0, T}\|\}
\end{align*}

We call this assumption ``counterfactual risk invariance.'' In other words, we assume that the model that minimizes the validation-period risk also minimizes the post-treatment risk for the opposite counterfactual. This is quite a strong assumption, and may not be well-justified depending on the nature of the problem. 

We briefly note that we conflate variable selection with the synthetic controls' variable weighting procedure for the purposes of this exposition. Regardless, from a finite-sample perspective the synthetic controls estimator will be biased should the weights fail to balance a relevant confounder regardless of whether the imbalance were due to a sub-optimal weighting matrix $V$ or a selection procedure that failed to include the variable. We also note that synthetic controls are frequently motivated by a linear factor model for the outcome, while we have assumed no unmeasured confounding and a linear model throughout our discussions. However, this general point still holds when considering unobserved factors that might be weak confounders of $Y^0$ but strong confound $Y^1$.

\subsection{Illustration}

We now illustrate this problem using our application, and consider the potential confounding role of Republican governance for our counterfactual estimate. Republican governance is a strong predictor of a state's decision to expand Medicaid (\cite{courtemanche2017early}). Moreover, existing evidence prior to Medicaid expansion showed that Medicaid take-up rates were lower in more conservative states \cite{sommers2012understanding}. However, when generating their synthetic control weights to estimate the ETT, \cite{courtemanche2017early} and \cite{kaestner2017effects} do not control for these factors. \footnote{\cite{courtemanche2017early} does control for Republican governor in their regression model and they find that it is a statistically significant predictor of 2013 uninsurance rates. One reason they may not control for this in the synthetic control model is practical: it is challenging to balance this covariate using control data without extrapolating from the data.} However, it is clear that if take-up rates depend on governance, we may expect this to be a strong confounder of $Y^1$  even if arguably it is not a confounder of $Y^0$.

We illustrate this by conducting a variable importance analysis to examine the confounding role of Republican governance on our estimate of $\bar{Y}_0^1$. Specifically, we remove the balance constraints from the Republican governance indicators and examine how our estimates of $\hat{Y}_0^1$ change. Letting $\hat{Y}^1_{0, s}$ be the estimate when removing the Republican governance indicators (or more generally, the covariate matrix $S$ where $X = (Z, S)$). We subtract our original point estimate $\hat{Y}^1_0$ from $\hat{Y}^1_{0, s}$ to generate the difference $\hat{\Delta}^1$. This difference tells us about the direction of the bias our estimate of $\hat{Y}^1_0$ would incur when we do attempt to constrain the imbalance in covariate $S$. Our hypothesis implies that we should expect $\hat{\Delta}_s^1 < 0$: that is, keeping all other covariates (roughly) fixed, we expect the predicted uninsurance rate will decrease when as the level of Republican governance decreases. In addition to the Republican governance indicators, we also examine four other covariate groups: pre-treatment uninsurance rates and pre-treatment unemployment rates, and three sets of different demographic indicators. We then calculate these differences when removing each state and provide the minimum and maximum differences in parentheses.\footnote{This quantity does not represent a fixed target of inference and so we therefore do not estimate confidence intervals for them.} We caution that our results do not imply that Republican governance is not an important confounder of $Y^0_{1, T}$ since we do not analyze this directly.

For the H-SBW estimator we calculate $\hat{\Delta}^1$ equal to -0.69 (min = -0.83, max = -0.42) and equal to -0.79 (min = -0.90, max = -0.66) on our unadjusted dataset. In other words, our primary estimated treatment effect moved 0.78 percentage points further away from zero when we excluded the Republican governance indicators. This reflects a 33 percent decrease in our point estimate, a not-unsubstantial difference. Moreover, all of these estimates were less than zero, regardless of whether we conditioned on the covariate adjustment or not, regardless of whether we remove the early expansion states or not, and when removing each state. Additional distributional results across all leave-one-state-out estimates for the primary dataset are available in Table~\ref{tab:rdiffc1} below. Additional results are available on request.

We also consider four other covariate sets. We find that our estimates are most sensitive to controlling for pre-treatment outcomes and unemployment rates. This is not unexpected: all else equal, the expansion region had much lower pre-treatment uninsurance rates. If we do not control for these covariates, the comparable region will likely have lower pre-treament uninsurance rates, causing the estimated counterfactual to be closer to zero. The effect estimates were less sensitive to the removal of other covariate groups, and all point estimates are available in Table~\ref{tab:ptests}. Overall these results highlight the importance of Republican governance to our counterfactual model of $Y^1$. 

\subsection{Effect heterogeneity}

If the models specified by \cite{kaestner2017effects} and \cite{courtemanche2017early} are correct (that is, they correctly omit Republican governance from their balancing weights for estimating $\bar{Y}^0_{1, T}$), our results imply treatment effect heterogeneity with respect to Republican governance. Moreover, because the expansion-state region is much more Democratic than the non-expansion region, this heterogeneity could potentially drive differences between the ETC and the ETT. Because this is a policy question of some interest, we directly investigate this by estimating the outcome model on the full data with treatment assignment interacted with each covariate.\footnote{For this analysis we calculate separate covariate adjustments on the untreated data.} We then examine how the estimated treatment effect would change if we decreased the interaction between treatment assignment and each Republican governance indicator -- Republican governor, Republican lower legislature control, and Republican total control -- by 50 percentage points (the original variables are either 0 or 100 and are measured at the state level). This linear combination of coefficients estimates how the treatment effect would change for any given collection of states against a set that is identical except for being 50 percentage points lower, on average, across the Republican governance indicators. We find that the linear combination is positive (0.21 percentage points) and statistically significant at the 5 percent level on the unadjusted dataset. In contrast to our previous results, this would indicate that the estimated treatment effect may be larger among Republican governed areas. However, this finding is not robust to any other specification that we run. We interpret these results as providing no evidence of treatment effect heterogeneity with respect to Republican governance. The full results are available in Table~\ref{tab:hte}.

\subsection{Variable importance result tables}

Table~\ref{tab:ptests} presents all point estimates from estimators that we calculated. The ``Var subset`` column indicates which variables were excluded from the estimation: 0 excludes no variables; 1 removes Republican governance indicators; 2 pre-treatment uninsurance and unemployment rates; 3 urban, age, education, citizenship, marital status, student, disability, or female; 4 race, ethnicity, income, foreign born; 5 children, population growth, and household to person ratio. We see that the largest changes generally occur when excluding the pre-treatment uninsurance and unemployment rates. This is not surprising: controlling for the other covariates, the pre-treatment uninsurance rate was substantially lower in the treated region compared to the control region. Given that pre-treatment uninsurance rates are highly correlated with post-treatment rates, we find that this comparison leads to a larger absolute magnitude point estimate, highlighting the need to control for these covariates. The results are qualitatively similar when excluding the early expansion states and are available on request.

%Wed Jan 13 15:24:43 2021
\begin{table}[h!]
\centering
\caption{Point estimates for all specifications}
\label{tab:ptests}
\begin{tabular}{llrrrr}
  \hline
Variable subset & Adjustment & H-SBW & BC-HSBW & SBW & BC-SBW \\ 
  \hline
0 & Homogeneous & -2.33 & -2.05 & -2.35 & -2.07 \\ 
  0 & Heterogeneous & -2.24 & -1.98 & -2.28 & -2.00 \\ 
  0 & None & -2.34 & -2.22 & -2.39 & -2.19 \\ 
  1 & Homogeneous & -3.02 & -2.98 & -3.03 & -2.76 \\ 
  1 & Heterogeneous & -3.02 & -2.98 & -3.01 & -2.76 \\ 
  1 & None & -3.13 & -3.17 & -3.14 & -2.93 \\ 
  2 & Homogeneous & -5.40 & -4.73 & -5.12 & -4.48 \\ 
  2 & Heterogeneous & -5.40 & -4.72 & -5.12 & -4.48 \\ 
  2 & None & -5.40 & -4.81 & -5.12 & -4.55 \\ 
  3 & Homogeneous & -2.21 & -1.94 & -2.24 & -1.91 \\ 
  3 & Heterogeneous & -2.14 & -1.86 & -2.18 & -1.85 \\ 
  3 & None & -2.33 & -2.03 & -2.38 & -2.02 \\ 
  4 & Homogeneous & -2.34 & -2.18 & -2.29 & -2.15 \\ 
  4 & Heterogeneous & -2.30 & -2.17 & -2.25 & -2.10 \\ 
  4 & None & -2.39 & -2.37 & -2.45 & -2.35 \\ 
  5 & Homogeneous & -2.33 & -2.12 & -2.37 & -2.15 \\ 
  5 & Heterogeneous & -2.24 & -2.05 & -2.30 & -2.09 \\ 
  5 & None & -2.34 & -2.25 & -2.38 & -2.26 \\ 
   \hline
\end{tabular}
\end{table}

Table~\ref{tab:rdiffc1} displays the quantiles of the distribution $\hat{\Delta}_v^1$ estimates when leaving out each state for the primary dataset and removing the early expansion states. The ``resample'' column indicates whether the entire adjustment procedure was recalculated (``Procedure'') or whether we left out each state conditional on the adjustment (``States''). These differences tend to be slightly larger in absolute magnitude when removing early expansion states and are available on request.

\begin{table}[h!]
\centering
\caption{$\hat{\Delta}^1_v$ leave-one-state-out estimates, primary dataset}
\label{tab:rdiffc1}
\begin{tabular}{lllrrrrrr}
  \hline
Inference & Adjustment & Weight type & Original & 0\% & 25\% & 50\% & 75\% & 100\% \\ 
  \hline
States & Heterogeneous & H-SBW & -0.78 & -0.88 & -0.80 & -0.78 & -0.73 & -0.59 \\ 
  Procedure & Heterogeneous & H-SBW & -0.78 & -0.89 & -0.82 & -0.74 & -0.69 & -0.46 \\ 
  States & Heterogeneous & BC-HSBW & -1.00 & -1.13 & -1.02 & -0.99 & -0.94 & -0.71 \\ 
  Procedure & Heterogeneous & BC-HSBW & -1.00 & -1.11 & -1.03 & -0.98 & -0.93 & -0.66 \\ 
  States & Heterogeneous & SBW & -0.74 & -0.93 & -0.77 & -0.73 & -0.70 & -0.52 \\ 
  Procedure & Heterogeneous & SBW & -0.74 & -0.96 & -0.78 & -0.73 & -0.69 & -0.53 \\ 
  States & Heterogeneous & BC-SBW & -0.77 & -0.97 & -0.79 & -0.75 & -0.72 & -0.56 \\ 
  Procedure & Heterogeneous & BC-SBW & -0.77 & -0.96 & -0.80 & -0.76 & -0.69 & -0.50 \\ 
  States & Homogeneous & H-SBW & -0.69 & -0.83 & -0.74 & -0.70 & -0.64 & -0.42 \\ 
  Procedure & Homogeneous & H-SBW & -0.69 & -0.86 & -0.76 & -0.71 & -0.65 & -0.36 \\ 
  States & Homogeneous & BC-HSBW & -0.93 & -1.07 & -0.95 & -0.94 & -0.88 & -0.64 \\ 
  Procedure & Homogeneous & BC-HSBW & -0.93 & -1.05 & -0.97 & -0.93 & -0.87 & -0.59 \\ 
  States & Homogeneous & SBW & -0.67 & -0.86 & -0.72 & -0.67 & -0.65 & -0.40 \\ 
  Procedure & Homogeneous & SBW & -0.67 & -0.90 & -0.73 & -0.69 & -0.65 & -0.47 \\ 
  States & Homogeneous & BC-SBW & -0.70 & -0.89 & -0.75 & -0.70 & -0.66 & -0.50 \\ 
  Procedure & Homogeneous & BC-SBW & -0.70 & -0.91 & -0.76 & -0.73 & -0.66 & -0.44 \\ 
  States & None & H-SBW & -0.79 & -0.90 & -0.81 & -0.80 & -0.76 & -0.66 \\ 
  Procedure & None & H-SBW & -0.79 & -0.90 & -0.81 & -0.80 & -0.76 & -0.66 \\ 
  States & None & BC-HSBW & -0.95 & -1.10 & -0.98 & -0.94 & -0.90 & -0.84 \\ 
  Procedure & None & BC-HSBW & -0.95 & -1.10 & -0.98 & -0.94 & -0.90 & -0.84 \\ 
  States & None & SBW & -0.75 & -0.89 & -0.77 & -0.75 & -0.73 & -0.58 \\ 
  Procedure & None & SBW & -0.75 & -0.89 & -0.77 & -0.75 & -0.73 & -0.58 \\ 
  States & None & BC-SBW & -0.74 & -0.88 & -0.75 & -0.72 & -0.70 & -0.57 \\ 
  Procedure & None & BC-SBW & -0.74 & -0.88 & -0.75 & -0.72 & -0.70 & -0.57 \\ 
   \hline
\end{tabular}
\end{table}

Table~\ref{tab:hte} displays the estimated linear combination of model coefficients on the Republican governance indicators with 95 percent confidence intervals (standard errors calculated using leave-one-state-out jackknife, repeating the entire covariate adjustment procedure each time). 

\begin{table}[h!]
\caption{Effect heterogeneity with respect to Republican governance}
\label{tab:hte}
\centering
\begin{tabular}{rlll}
  \hline
Estimate & Adjustment & CI (95\%) & Dataset \\ 
  \hline
-1.67 & Heterogeneous & (-5.27, 1.94) & Primary \\ 
  0.78 & Homogeneous & (-5.62, 7.18) & Primary \\ 
  0.21 & None & (0.02, 0.40) & Primary \\ 
  -1.80 & Heterogeneous & (-5.46, 1.85) & Early expansion excluded \\ 
  0.59 & Homogeneous & (-6.13, 7.31) & Early expansion excluded \\ 
  0.15 & None & (-0.12, 0.42) & Early expansion excluded \\ 
   \hline
\end{tabular}
\end{table}

\clearpage