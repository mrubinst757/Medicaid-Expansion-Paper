\section{Summary Statistics}\label{app:sumstats}

In this section we display summary statistics about the dataset. In Table~\ref{tab:txassign} we list the states that are in each group: the first two columns include the treatment states and control states in our primary analysis. The third column lists the treatment states for our sensitivity analysis that excludes ``early expansion'' states. The final column indicates states that were always excluded from the analysis. Table~\ref{tab:summarytab1} displays univariate summary statistics for the treated CPUMAs. We display the mean, interquartile range, and the range (as defined by the maximum value minus the minimum value) for the unadjusted dataset, the heterogeneous adjustment, and the homogeneous adjustment. We see that in general the covariate adjustments reduce the variability in the data. Table~\ref{tab:extreme1} displays the overall number of CPUMAs whose adjusted values fall outside of the range of the unadjusted data. We also calculate these adjustments when leaving out each state one at a time from the primary dataset to calculate our secondary variance estimates; these results are available on request. 


\begin{table}[h!]
\centering
\caption{Treatment assignment classification}\label{tab:txassign}
\begin{tabular}{llll}
\begin{tabularx}{\textwidth}{|X|X|X|X|}
  \hline
Treated states & Control states & Treated states (early expansion excluded) & Always excluded \\
  \hline
AR, AZ, CA, CO, CT, HI, IA, IL, KY, MD, MI, MN, ND, NJ, NM, NV, OH, OR, RI, WA, WV & AK, AL, FL, GA, ID, IN, KS, LA, ME, MO, MS, MT, NC, NE, OK, PA, SC, SD, TN, TX, UT, VA, WI, WY & AR, AZ, CO, HI, IA, IL, KY, MD, MI, ND, NM, NV, OH, OR, RI, WV & DE, MA, NH, NY, VT, DC \\ 
   \hline
\end{tabularx}
\end{tabular}
\end{table}

\begin{table}[h!]
\centering
\caption{Univariate summary statistics on adjusted data \\ (Mean, IQR, Range)}\label{tab:summarytab1}
\begin{tabular}{rllll}
  \hline
Variable & No adjustment & Heterogeneous & Homogeneous \\ 
  \hline
  Age: 19-29 Pct & (24.5, 6, 30.9) & (24.5, 5.9, 29) & (24.5, 5.9, 29) \\ 
  Age: 30-39 Pct & (20.9, 3.4, 20.9) & (20.9, 3.1, 19.1) & (20.9, 3.1, 19.4) \\ 
  Age: 40-49 Pct & (22.2, 2.5, 15.4) & (22.2, 2.3, 13.7) & (22.2, 2.2, 13.7) \\ 
  Avg Adult to Household Ratio & (151, 27.2, 174.3) & (151, 27.2, 173.8) & (151, 27.2, 173.3) \\ 
  Avg Pop Growth & (100.3, 1.9, 13.7) & (100.3, 1.2, 6.2) & (100.3, 1.2, 6.5) \\ 
  Children: Missing Pct & (10.5, 6.6, 41) & (10.5, 6.5, 40.8) & (10.5, 6.5, 40.5) \\ 
  Children: One Pct & (11.1, 3.1, 14.3) & (11.1, 2.8, 12.3) & (11.1, 2.8, 12.5) \\ 
  Children: Three or More Pct & (5.2, 2, 14.1) & (5.2, 1.7, 13.5) & (5.2, 1.7, 13.3) \\ 
  Children: Two Pct & (9.7, 3.5, 15) & (9.7, 3.3, 13.5) & (9.7, 3.2, 13.6) \\ 
  Citizenship Pct & (90, 11.9, 57.1) & (90, 11.8, 55.4) & (90, 11.7, 55.7) \\ 
  Disability Pct & (10.5, 5.3, 28.6) & (10.4, 5.3, 26.7) & (10.5, 5.4, 27.2) \\ 
  Educ: HS Degree Pct & (26.3, 10.7, 43.2) & (26.3, 10.6, 41.8) & (26.3, 10.6, 42) \\ 
  Educ: Less than HS Pct & (11.4, 7.7, 45.3) & (11.4, 7.5, 45) & (11.4, 7.4, 44.6) \\ 
  Educ: Some College Pct & (33.5, 7.9, 34.2) & (33.5, 7.5, 32.9) & (33.5, 7.4, 33) \\ 
  Female Pct & (50.1, 1.6, 15.4) & (50.1, 1.4, 14.2) & (50.1, 1.5, 14.2) \\ 
  Foreign Born Pct & (18.1, 22.4, 76) & (18.1, 22.2, 75.2) & (18.1, 22.2, 75.4) \\ 
  Hispanic Pct & (15.9, 17.7, 97.2) & (15.9, 17.7, 97) & (15.9, 17.7, 97) \\ 
  Inc Pov: $<$ 138 Pct & (20, 11.9, 45.6) & (20, 11.8, 44.8) & (20, 11.8, 43.9) \\ 
  Inc Pov: 139-299 Pct & (24.9, 8.4, 34.2) & (24.9, 7.9, 34.2) & (24.9, 7.8, 34.1) \\ 
  Inc Pov: 300-499 Pct & (23.6, 5.5, 23) & (23.6, 4.9, 22.2) & (23.6, 4.9, 22.2) \\ 
  Inc Pov: 500 + Pct & (29.3, 18.5, 69.1) & (29.3, 18.5, 68.1) & (29.3, 18.5, 68) \\ 
  Married Pct & (50.7, 9.4, 45.1) & (50.7, 9, 44.1) & (50.7, 9.1, 44.2) \\ 
  Race: White Pct & (73.8, 25.4, 91.9) & (73.8, 25.5, 91.7) & (73.8, 25.5, 91.8) \\ 
  Republican Governor 2013 & (31.1, 100, 100) & (31.1, 100, 100) & (31.1, 100, 100) \\ 
  Republican Lower Leg Control 2013 & (24.1, 0, 100) & (24.1, 0, 100) & (24.1, 0, 100) \\ 
  Republican Total Control 2013 & (19.8, 0, 100) & (19.8, 0, 100) & (19.8, 0, 100) \\ 
Student Pct & (11.7, 3.4, 29.5) & (11.7, 3.5, 28.1) & (11.7, 3.5, 28) \\ 
  Unemployed Pct 2011 & (10.2, 4.6, 25.5) & (10.2, 3.9, 23.8) & (10.2, 3.9, 22.5) \\ 
  Unemployed Pct 2012 & (9.4, 4.5, 28.3) & (9.4, 4.3, 23.6) & (9.4, 4.3, 23.5) \\ 
  Unemployed Pct 2013 & (8.4, 3.6, 23.4) & (8.4, 3.5, 20.1) & (8.4, 3.5, 20.5) \\ 
  Uninsured Pct 2011 & (19.6, 11.2, 59) & (19.7, 10.9, 51.8) & (19.6, 10.9, 52.5) \\ 
  Uninsured Pct 2012 & (19.4, 9.9, 50.6) & (19.4, 10.1, 49.7) & (19.4, 10.3, 50.2) \\ 
  Uninsured Pct 2013 & (19, 11.2, 49.9) & (19, 10.3, 48.2) & (19, 10.5, 48.7) \\ 
  Urban Pct & (82.9, 31.3, 91.3) & (82.9, 31.3, 91.3) & (82.9, 31.3, 91.3) \\ 
   \hline
\end{tabular}
\end{table}

Table~\ref{tab:extreme1} displays the frequency that the adjusted covariates fell outside of the support of the unadjusted dataset for the treatment data. We see that that frequency is comparable for either adjustment and the counts are low. We also note that these tables include 4 CPUMAs from New Hampshire that we include when conducting the covariate adjustment procedure; however, we exclude New Hampshire from the pool of treated states when computing our weights.

\begin{table}[h!]
\centering
    \caption{Frequency of ``extreme'' covariate adjustments}
    \label{tab:extreme1}
\begin{tabular}{lll}
  \hline
Variables & Heterogeneous & Homogeneous \\ 
  \hline
Age: 19-29 Pct & 0 & 0 \\ 
  Age: 30-39 Pct & 0 & 0 \\ 
  Age: 40-49 Pct & 0 & 0 \\ 
  Avg Adult to Household Ratio & 0 & 0 \\ 
  Citizenship Pct & 1 & 0 \\ 
  Disability Pct & 2 & 2 \\ 
  Educ: HS Degree Pct & 0 & 0 \\ 
  Educ: Less than HS Pct & 1 & 1 \\ 
  Educ: Some College Pct & 0 & 0 \\ 
  Female Pct & 0 & 0 \\ 
  Foreign Born Pct & 1 & 1 \\ 
  Uninsured Pct 2011 & 0 & 0 \\ 
  Uninsured Pct 2012 & 1 & 1 \\ 
  Uninsured Pct 2013 & 0 & 1 \\ 
  Hispanic Pct & 0 & 0 \\ 
  Inc Pov: $<$ 138 Pct & 1 & 1 \\ 
  Inc Pov: 139-299 Pct & 1 & 1 \\ 
  Inc Pov: 300-499 Pct & 0 & 0 \\ 
  Inc Pov: 500 + Pct & 0 & 0 \\ 
  Married Pct & 0 & 0 \\ 
  Children: Missing Pct & 1 & 1 \\ 
  Children: One Pct & 0 & 0 \\ 
  Avg Pop Growth & 0 & 0 \\ 
  Race: White Pct & 1 & 1 \\ 
  Student Pct & 0 & 0 \\ 
  Children: Three or More Pct & 2 & 1 \\ 
  Children: Two Pct & 1 & 1 \\ 
  Unemployed Pct 2011 & 0 & 0 \\ 
  Unemployed Pct 2012 & 0 & 0 \\ 
  Unemployed Pct 2013 & 0 & 0 \\ 
   \hline
\end{tabular}
\end{table}

Figure~\ref{fig:corrmatrix} displays the Pearson's correlation coefficients for the bivariate relationships between the covariates on the unadjusted dataset (including both treated and untreated units). We caution that these point estimates may be biased due to the measurement error in the covariates. Nevertheless, this matrix is useful for at least two reasons: first, assuming the correlations among the treated and untreated units are similar, the more heavily correlated the data the easier it should be to attain covariate balance (see, e.g., \cite{d2021overlap}). This matrix helps give a general sense of how correlated the data are, even if the estimates are biased. Second, these correlations can suggest potential confounders by revealing which variables are most heavily associated with treatment assignment and the pre-treatment outcomes. For example, we see that Republican governance is strongly associated with treatment, and somewhat associated with pre-treatment outcomes. We can also examine associations between the variables and pre-treatment outcomes to get a sense of which variables may be important to the outcome model: for example, we see strong associations between the pre-treatment uninsurance rates, though they are more weakly associated with treatment assignment. 

\begin{figure}[h!]
\begin{center}
    \caption{Correlation matrix: full data, unadjusted covariates}
    \label{fig:corrmatrix}
    \includegraphics[scale=0.25]{01_Plots/correlation-plot-c1-sigma-zero.png}
\end{center}
\end{figure}

\clearpage